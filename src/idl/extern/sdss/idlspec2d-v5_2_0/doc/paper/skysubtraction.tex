%%%%%%%%%%%%%%%%%%%%%%%%%%%%%%%%%%%%%%%%%%%%%%%%%%%%%%%%%%%%%%%%%%%%%%%%%

% Definitions
%\pagestyle{empty}
\documentclass[12pt,preprint]{aastex}
%\documentstyle[emulateapj,danonecolfloat]{article}
%\usepackage{emulateapj,danonecolfloat}
\usepackage{rotating}

\begin{document}


\subsection{Extraction}

We utilize an optimal extraction procedure to transform a 2-dimensional image of
the fiber spectrograms to a set of 320 1-dimensional spectra (per spectrograph).
We use the same procedure on all processed frames, 
and will describe the details in processing the flat-field frames, the
arc lamp frames, and finally the science frames.  
The extraction is performed row-by-row, so each frame requires an 
outside loop over 2048 individual rows.
The counts in each row of 2048 columns are modeled with a linear 
combination of 320 profiles plus a low-order
polynomial background.  In addition, deviations to the trace centers and
the profile widths are calculated as linear basis modes (reflecting the first
and second spatial derivatives, respectively).  The compiled routine
we implement to fit the profiles in a single row (extract\_row.c) can
be invoked with any combination of these parameters.

We decided to perform profile fitting to the 
SDSS spectral images for the following reasons:  

\begin{enumerate}

\item{The fiber profiles overlap significantly with their nearest neighbors.}

\item{The profiles are relatively stable and smoothly vary over 
location on the CCDs.  They result from the convolution of the fiber 
illumination pattern + the point-spread function
of the optics in the SDSS spectrographs + the window funciton of the 
Site CCD 24 micron square pixels. }

\item{A least-squares fit with the true profile delivers an 
unbiased and minimum variance estimate of the counts (e.g. Horne et al. 1995).}

\item{A model profile fit leads to a model image of the SDSS data,
and iteration allows cosmic-ray rejection and goodness-of-fit tests.}

\end{enumerate}

The major obstacle to actual optimal extraction is the determination of 
the exact shape and position of the underlying profile.  
We investigated a range of symmetric, analytic profiles with a 
minimum number of free parameters.  In the end, we found the best 
results using a simple Gaussian or a slightly steeper profile 
characterized by $e^{-|x|^3/\sigma^3}$.  The profiles are defined to be 
normalized to unity, so the amplitude of the profile fit is a 
direct estimator of the counts at a single spectral position. 

Extraction is first carried out on all flat-field exposures 
(typically 10 second observations of 4 quartz lamps projected on petals covering the entrance aperture of the telescope) which were observed
with a common plug plate configuration (a spectroscopic sequence). 
The flat-field extraction is used to determine the shape of
the profile as a function of wavelength and position on the CCD.  
The resulting spectral extractions are also used to 
determine the empirical fiber-to-fiber variations in throughput 
as a function of wavelength.   
A flat-field frame is rejected by the pipeline if one of the
following conditions is met:  1) More the 2\% of the CCD pixels are labeled
bad by sdssproc, 2) More than 100 rows of the image show saturation, 
3) The 80th percentile count level is less than 1000 counts, 
4) Fewer than all 8 flat-field telescope petals were closed, 
5) Fewer than all 4 flat-field lamps were on,
6) More than 10000 pixels were rejected during fiber tracing, 
7) Fiber tracing fails, or traces are seperated by less than 3 pixels anywhere
on the flat-field image.

The first very critical step is to robustly determine the fiber positions 
in CCD pixel coordinates.
This is performed by identifying peaks in a cross-sectional cut of across 
the center row (actually a median cross-section of the central 9 rows).  
The correct identification of the 320 fibers per frame can be 
uniquely determined by demanding a wider spacing between the bundles 
of 20 fibers (spaced by about 10 pixels) relative to the nominal inter-fiber 
spacing (6.1 pixels on average).  Once the starting peaks and 
identification have been determined, we trace the fiber centers 
by performing flux-weighted centroids on subsequent rows
using the previous row's center as the intial guess.  This has proven to 
work well on SDSS flat-fields, and we have yet to uncover a failure 
mode in over 10000 flat-field frames.
The flux-weighted centroids do have scatter from shot noise, 
so we fit a 7th order polynomial to each set of fiber centroids, 
and the best-fit coefficients are stored to represent the fiber positions 
on each flat-field frame.


Each flat-field frame is extracted twice, once assuming a Gaussian
profile and again assuming the steeper $|x|^3$ profile.  In both cases,
the width is allowed to vary freely.  The total number of free parameters
per row in the flat-field extractions is equal to 2 per fiber plus 10
parameters to represent a smooth polynomial background.  After the extractions
are complete, $\chi^2$ is computed by directly comparing the model fit
to the flat-field image.  The median $\chi^2$ for each case is compared
directly, and the better fit and profile (lower median $\chi^2$) are retained
and implemented for the extraction of the remaining frames.  Also retained
is a smooth polynomial fit the average profile width in each fiber bundle
as a function of CCD row number.

After all flat-field frames for a given sequence have been extracted
and stored, the arc-lamp frames in the same sequence are extracted.  
Each arc-lamp frame is matched with the nearest (in time) 
non-rejected flat-field frame.  The matched flat-field gives the 
initial trace centers to
be assumed, as well as the profile shape and width determined from the
flat-field.  To account for any flexure between the frames, the trace
centers are allowed to shift/stretch to the arc-lamp image given a 
3x3 polynomial distortion kernel over the entire extent of the image
(match\_trace.pro).

The arc-lamp frames are extracted row-by-row, in the same
manner as the flat-fields, the number of free parameters assumed
for the model fit is 2 per fiber (profile area and width) and 1 parameter
to fit for a constant background per row.  The restricted freedom in the
background parameterization reflects the lower scattered light levels in
the images of the arc-lamps.

\subsection{(or arc-lamp spectra -> wavelength map)}

We determine an initial wavelength solution for every arc-lamp frame
successfully extracted in the following manner:

For the appropriate side of each spectrograph (red or blue), we 
construct a list of suitable unblended lines from a parent list including
known lines of Mercury, Cadmium, Neon and (a trace of) Argon 
that are readily observable in the standard set of SDSS arc-lamps. 
A representative spectrum is calculated as the median extracted spectrum
of the 5 closest "unflagged" spectra falling near the central columns 
of the arc-lamp image.  This median arc-spectrum is matched with 
cross-correlation to a simulated spectrum of suitable arc-line positions and 
strengths by systematically searching a small volume of 
parameter space which conservatively covers
the polynomial coefficients of the SDSS wavelength solution.
In the first iteration, we search a set of 
100 coefficient vectors allowing for quadratic changes and return 
the best match.  In the second iteration, we search in a smaller region
about the previous best match, and search a list of 125 coefficient vector
allowing for cubic deviations.  The coefficient vector with the best
match from the second iteration is returned as an initial guess along with
the cross-correlation coefficient.  

We apply the best guess coefficients to predict the approximate pixel 
positions of the suitable list of arc lines.  The arc line centroids
are traced across extracted spectra (along the axis of 320 fibers).
The centroids are fit to a smooth polynomial as a function of fiber
position on the arc-lamp image, and subsequently fit through 6 iterations
with simple flux weighting to arrive at a converged measure of line 
centroids in each extracted spectrum.  Lines are excluded from the resulting
list if the flux-weighted centroids are bad in more than 10\% of the
spectra or if the RMS residuals are larger than 20\% of a pixel.
We perform an iterative 4th order Legendre polynomial fit of 
logarithmic wavelength as a function of pixel in each spectrum, 
and retain a list of rejected centroids, 
which have deviations relative to the model of
10\% of a pixel width.  An additional requirement is applied to each
suitable arc line, wherein the line is rejected in no more than 3 fibers
per bundle of 20.  Rejected centroids are replaced with the appropriate
positions based on a linear interpolation of the differences of non-masked
centroids relative to updated polynomial fits per spectrum.
The final set of positions, where the same number of line positions have
reliable centroids per spectrum, are fit with equal weighting per line
position to produce the final wavelength
coefficients which give the logarithmic (base 10) wavelengths as a function
of spectral pixel.

An arc-lamp image is declared unacceptable using the first 4 tests as employed
in tests of the flat-field images.  In addition, an arc-lamp exposure
is declared unacceptable ("BAD") if any of these following criteria are met:
5) All 8 arc-lamps must be on, 4 Neon lamps and 5 Mercury-Cadmium lamps,
6) The initial cross-correlation coefficient is less than 0.5,
7) Less than 6 arc lines are acceptable after quality assurance checks.

The width of unblended arc lines are determined as an average over
each bundle of 20 spectra.  In short, the arc-lamp emission lines 
are extracted (with a Gaussian spectral profile) 
allowing free parameters for the area and width of each suitable line.
The resulting Gaussian widths per fiber bundle are fit 
with a low order polynomial as a function of average line position
(the fit is quadratic and cubic for the blue and red side respectively).
The fitted quantity is referred to as the arc-lamp dispersion,
but the values reflect the standard deviation of a Gaussian 
profile in pixel units and represents the line-spread function of the 
arc-lamp spectra.

(or just refer to data model appendix instead of the paragraph below:)

The arc-lamp reductions are written to disk at this stage, and
stored in an extended FITS file:  HDU \#0 contains the processed
and extracted arc-lamp spectra as counts per pixel, HDU \#1 contains
the identified arc line wavelengths and the measured line centroids
(including replaced positions of rejections), HDU \#2 contains a single
structure with the Legendre polynomial coefficients applicable to each
of the 320 arc-lamp spectra, HDU \#3 contains the maskbits for each spectrum,
and HDU \#4 contains the polynomial coefficients to describe the arc-lamp
Gaussian widths as a function of spectral pixel.

\subsection{Constructing fiber flats}

The final step in the reduction of the calibration frames in a given
sequence is the construction of the fiber flat vectors.  Each flat-field
frame is matched to the nearest arc-lamp frame as before, and the
final flat-field fiber spectra are extracted in two steps.  The first
step uses the preferred spatial profile and width determined above,
and the extraction procedure produces a model of the flat-field frame.
The model is used to subtract a diffuse scattered background given
by an exponential point-spread function of :  (CCD halo PSF).
We apply a 1-dimensional version of this 2-d PSF as an approximation,
and allow the PSF to vary appropriately with wavelength.
The final extraction is applied to the scattered light subtracted flat-field
image allowing for 2 free paramters per fiber per row and 
5 background parameters per row.

The final flat-field spectra are stored as 
detected counts per fiber per pixel with wavelengths of each pixel
given by the associated solution of the arc-lamp frame.  
We model the flat-field spectra as the product:


\begin{equation} 
{\rm Flat Model}_{ij} = A * f_{avg}(\lambda) \, {\rm fflat_i} (\lambda) 
\end{equation} 

where $A$ is a overall normalization in counts, 
$f_{avg}$ is the average response per constant logarithmic wavelength
interval, $fflat_i$ is the fiber-to-fiber response as a function of wavelength.
This model of the extracted flat-field spectra provides a transformation to
compare detected counts from one fiber to another.  

This should remove the differential response due to the individual fiber
throughput, as well as the small relative differences in the size of the
spectral pixels at a constant wavelength.  ??Why do we correct for pixel
sizes later on if the differential wavelength dispersion is included in fflat??

The common normalized response function, $f_{avg}$, is specified as a B-spline
function with breakpoints spaced approximately every native pixels.
An example of normalized individual fiber spectra 
and the resulting B-spline fit are shown in Figure ?.  The individual fiber
flat vectors are fits to the ratio of the extracted spectra to the superflat
(define above).  The individual vector fits are also B-splines with breakpoints
spaced approximately every 5 native pixels.  The final set of flat vectors are
stored as an image with dimensions Nfiber x Npixels.


\subsection{Object Extraction}

With the extracted calibration frames for a given spectroscopic sequence
reduced, we proceed to extract the remaining science frames.
%To minimize the differences between individual science frames
We first select a set of arc-lamp and flat-field frames as the common
choice of initial calibrations for all science frame extractions in a given
observing sequence.  We loop through the science exposures, processing the
raw images with bias and pixel-to-pixel flat images (see section ??).
We reject a science exposure outright if 1) more than 10\% of the pixels
are flagged as bad, 2) more than 100 rows are saturated, 3) the 25th
percentile count level is greater than 1000 counts (the nominal level
is of order 10 counts), 4) any of the flat-field petals are closed, or 5)
any of the calibration lamps are recorded as on.

Inputs to object extraction include the wavelength coefficients derived
from the associated arc-lamp exposure, both the superflat and fiber flat
vectors from the selected flat-field exposure, the profile shape and width
as a function of position on the CCD image.

The first step in object extraction is a simple tophat aperture (boxcar) 
extraction centered at the fiber positions determined from the flat-field 
and with a full spatial width of 6 CCD pixels.  The median counts per fiber 
are passed directly to a test for overly bright fibers. Fiber spectra are 
identified as bright (a.k.a. "WHOPPING") if the median counts per pixel in the 
boxcar extraction are greater than 10000 counts above the median in the 
frame.  In addition, only the brightest fiber in a spatially contiguous set of 
bright fibers is labelled "WHOPPING".  

A pixelmask vector is created for each fiber spectrum to be extracted.
See Appendix ?? for a description of the mask bits and their use.

We fit for a slight distortion of the stored flat-field fiber traces to 
best match the fiber positions of the science image.  Nominal shifts and
distortions are usually less than a native CCD pixel.  The science frame
is now extracted in a series of 6 steps which we detail below.

\begin{enumerate}
\item{Initial extraction including broad wings of bright (WHOPPING) fibers.}
\item{Reconstruct a smooth background light model based on fitted parameters.}
\item{Calculate a scattered halo image using the model of the fiber spectra less
the aforementioned smooth background.}
\item{Re-extract fiber spectra on the scattered halo subtracted image.}
\item{Reconstruct smooth background and subtract.}
\item{Final extraction with no free parameters in the background and
no free parameters in the profile widths.}
\end{enumerate}

The extracted science spectra are first normalized by the fiber flat vectors.
Strong atmospheric emission lines (sky lines) are identified in the the 
set of spectra, and the wavelength solution is shifted to match the 
identified lines.  Examples of the shift corrections are shown in Figure ??.  
The widths of the identified sky lines are also measured with the same method
as the arc-line width measurements.  All spectra are also normalized by
the average flat-field spectrum (the superflat) to remove small scale
features likely attributed to the chromatic response of the dichroic.
The wavelength solution is corrected to both vacuum wavelengths and the
heliocentric velocity frame.

\subsection{Sky Subtraction}

Sky subtraction is performed on a frame by frame basis using the 
"flattened" wavelength calibrated extractions.  The goal of sky subtraction
is to estimate the expected background counts contributing to each of 
the native 320x2048 pixels in each extracted frame.  A "supersky" model 
at a fiducial airmass of 1 is constructed by performing iterative fits to
the extracted spectra associated with blank sky positions (so-called
"sky" fibers).   The following steps are taken in 3 sequential iterations
in order to produce the sky-subtracted spectra:

\begin{enumerate}

\item{Identify the "sky" fibers which have no bits flagged 
in the fiber mask array.}

\item{Divide each sky spectrum by the airmass of the fiber position appropriate
for the midpoint of the observation to produce respresentative sky spectra for 
an airmass of 1.  The inverse variance is scaled accordingly.}

\item{Group and sort the sky spectra as a function of wavelength.  This step
transforms a 2-dimensional set of $N_{\rm sky}$ x 2048 pixels into a sorted 1-dimensional array.  
Each pixel maintains its associated central wavelength and inverse variance. }

\item{Perform a single cubic b-spline fit as a function of wavelength.  
The number of breakpoints (number of free parameters is 4 greater) is set 
at 2048 on the red side and 1536 on the blue side.  The breakpoint spacing 
is set automatically to maintain approximately constant S/N between breakpoint 
positions. The b-spline fit itself is iterative, with upper and lower rejection 
thresholds set to mask bad or deviant pixels. }

\item{The b-spline fit is evaluated at all 320x2048 wavelengths representing all
spectral pixels in the frame.  The airmass correction is applied for each individual 
fiber, and the rescaled super-sky is subtracted from each spectrum.}

\item{Estimate the significance of the residuals in the sky fibers used for the
fit in wavelength bins separated by breakpoints determined above.  for each bin
where inverse the 67th percentile of $\chi^2$ is greater than 1, we rescale 
all the inverse variance values down by the 67th percentile value of $\chi^2$.
The rescaling is actually done with a smooth interpolation as a function of 
redshift with a constraint that the inverse variances cannot increase. }

\end{enumerate}

%The b-spline routines from the numerical fortran library: {\bf slatec}
%were ported to IDL to be easily incorporated into the pipeline.

The sequence followed in the determination of the sky subtracted spectrum
and re-scaled errors is called 3 times successively.  After the 1st call,
any sky fibers which have a median $\chi^2$ per pixel greater than 2 is 
masked as a 'BADSKYFIBER' and the corresponding bit is set in the fibermask
array.  If any fibers are masked, the sky subtraction is iterated a 2nd time.
The sky subtraction routine is invoked for one final iteration, but with extra 
freedom allowed in the fit.  The same freedom is allowed in wavelength, by
keeping the same number of breakpoints, but a quadratic polynomial fit is
allowed as a function of wavelength.  This triples the number of free 
parameters in the linear least-squares fit, but the values are still 
overconstrained due to the ample number of sky fibers.  The extra freedom
in the model can accomodate variation in the PSF of sky features as a function 
of position on the CCD.

The sky subtraction method which utilizes the b-spline model takes advantage 
of the inherent over-sampling in wavelength due to the large number of 
sky fibers assigned per plate, without the requirement of resampling the 
pixel value to a fixed linear wavelength scale.  The least squares fit
is carried out for native spectral pixels where the errors are very nearly 
uncorrelated and the fit is evaluated for all the native pixels (320x2048)  
at the central wavelengths of the single spectral frames.
    
The accuracy of the SDSS spectral sky subtraction has been tested in recent
scientific analyses using large numbers of background-limited spectra.   
The distribution of flux residuals between spectral observations 
on separate nights has been shown to be 8\% larger in width than the average expected
from the reported errors (McDonald et al. 2004, Burgess \& Burles 2005).
The spectra of faint, background-limited galaxies observed with SDSS exhibit a 
distribution of residuals about the best fit spectral models described by the sum of a
Gaussian and Laplace distribution (Bolton et al. 2004).  
The influence of the exponential tail is greatly enhanced in the vicinity of strong sky 
emission lines representative of systematic errors in the brightest lines, but the gaussian 
width closely is closely represented by the reported variance array.

\end{document}

