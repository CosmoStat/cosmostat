
\section{IDL Routines}
\label{ch_mrs_idl}
%\chapterhead{IDL Routines}
\markright{IDL Routines}

\subsection{Introduction}
A set of routines has been developed in IDL. Starting IDL using
the script program {\em mre} allows the user to get the multiresolution
environment, and all routines
described in the following can be called. An online help facility 
is also available by
invoking the {\em mrh} program under IDL.

\subsection{mw1d\_filter}
Filter a 1D signal by the the multiscale entropy.
{\bf
\begin{center}
     USAGE: mw1d\_filter, Signal, Result, Opt=Opt
\end{center}}
where 
\begin{itemize}
\item {\em Signal}: input  one-dimensional IDL array.
\item {\em Result}: output one-dimensional IDL array (filtered signal).
\item {\em Opt}:  string which contains the different options 
(see the {\em mw1d\_filter} C++  program).
\end{itemize}

\subsection{mw1d\_predict}
Considering a temporal signal $S$ with $n$ measures ($S[0..n-1]$), 
 {\em mr1d\_predict} estimates (or predicts) the next values $S[n .. n+dt]$.
A multiscale transformation is first applied, followed by filtering
in the wavelet space. At each scale, a predictor is then applied,
and the predicted signal is obtained from the reconstruction of
the predicted signal. 

{\bf
\begin{center}
     USAGE: Result = MW1D\_PREDICT(Signal,wave=wave,PredWave=PredWave,
                      NPredict=NPredict, Nscale=Nscale,  OPT=OPT, NCoef=NCoef)
\end{center}}
where 
\begin{itemize}
\item {\em wave}: output 2D IDL array; filtered wavelet coefficient
\item {\em PredWave}: output 2D IDL array; Predicted wavelet coefficient
\item {\em NPredict}: number of values to predict. Default is one.
\item {\em Nscale}: number of scales used for the wavelet transform
\item {\em NCoef:} Number of coefficients used for the prediction. Default is 3.
\item {\em OPT}: string which contains the differents options accepted by the
mw1d\_filter C++ program.
\end{itemize}

\subsection{mw\_filter}
Filter  an image by the multiscale entropy. 
This routine is calling the C++ executable {mw\_filter}. The keyword 
``OPT" allows 
all options described in the section corresponding to the 
 {\em mw\_filter} program.

{\bf
\begin{center}
     USAGE: mw\_filter, Data, FilterData, opt=opt
\end{center}}
{\em Data} is an image (2D IDL array), and {\em FilterData} is the result
of the filtering.

\subsection{mw\_deconv}
Deconvolve an image by  the  multiscale entropy. This routine calls 
the C++ executable {mw\_deconv}. The keyword ``OPT" 
allows 
all options described in the section corresponding to the 
 {\em mw\_deconv} program.

{\bf
\begin{center}
     USAGE: mw\_deconv, Data, PSF, DeconvData, opt=opt
\end{center}}
\subsubsection*{Examples:} 
\begin{itemize}
\item mw\_deconv, Imag, Psf, Result \\
deconvolve an image with all default options.
\item  mw\_deconv, Imag, Psf, Result, OPT='-i 30 -e 0'  \\
same example, but impose the number of iterations to be 30.
\end{itemize}

 
