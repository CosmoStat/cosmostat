

\chapter{\projmr3 \ Three Dimensional Data Set}

\section{\projmr3 \ Multiresolution}

\subsection{Introduction}
\subsubsection{The 3D \`a trous Wavelet Transform}
The 3D \`a trous WT of a cube produces $J$ bands, each one being
a cube with the same dimension as the input cube.
The 3D WT of a cube $C(*,*,*)$ is therefore a 4D data set
$W(*,*,*,0:J-1)$, where $J$ is the number of 
resolutions used in the decomposition.

\subsubsection{The 3D bi-orthogonal Wavelet Transform}

A one-level 3D sub-band decomposition transforms a cube of $N^3$ pixels into 
eight sub-cubes, also called bands, of  $({N\over 2})^3$ pixels. 
One of these bands corresponds to the original cube at a lower resolution
(smoothed cube), while the seven other bands are the detail 
information, i.e., 
information lost between the two resolution levels, original
and smoothed data resolution level.
A reconstruction of the original input cube can be performed from the 
eight sub-bands. The 3D wavelet transform (WT) iterates the decomposition 
process on the smoothed cube. For a 3D WT with $P$ scales, 
the final number of bands is equal to $7(P-1) + 1$, i.e., seven  
detail bands for the first $P-1$ scales, plus the last smoothed array.
For an $N^3$ pixels cube $D(0..N-1, 0..N-1, 0..N-1)$, the eight bands 
produced by 
one-level decomposition (two scales) are stored in the following way:
\begin{enumerate}
\item Band 1: $B_1({N\over 2}:N-1,0:{N\over 2}-1,0:{N\over 2}-1)$ is obtained
by 1D convolution with filters $ g_x, h_y, h_z$.
\item Band 2: $B_2(0:{N\over 2}-1, {N\over 2}:N-1,0:{N\over 2}-1)$ is obtained
by  1D convolution with filters $ h_x, g_y, h_z$.
\item Band 3: $B_3({N\over 2}:N-1, {N\over 2}:N-1,0:{N\over 2}-1)$ is obtained
by  1D convolution with filters $ g_x, g_y, h_z$.
\item Band 4: $B_4({N\over 2}:N-1,0:{N\over 2}-1, {N\over 2}:N-1)$ is obtained
by  1D convolution with filters $ g_x, h_y, g_z$.
\item Band 5: $B_5(0:{N\over 2}-1, {N\over 2}:N-1,{N\over 2}:N-1)$ is obtained
by  1D convolution with filters $ h_x, g_y, g_z$.
\item Band 6: $B_6({N\over 2}:N-1, {N\over 2}:N-1,{N\over 2}:N-1)$ is obtained
by  1D convolution with filters $ g_x, g_y, g_z$.
\item Band 7: $B_7( 0:{N\over 2}-1, 0:{N\over 2}-1,{N\over 2}:N-1)$ is obtained
by  1D convolution with filters $ h_x h_y g_z $.
\item Band 8: $B_8( 0:{N\over 2}-1, 0:{N\over 2}-1,0:{N\over 2}-1)$ is obtained
by  1D convolution with filters $ h_x, h_y, h_z$.
\end{enumerate}
For the three-scale decomposition, $B_8$ pixels of the 2-scale 
decomposition are replaced by bands 8 to 15, each having a size of 
 $({N\over 4})^3$  pixels.

\subsection{Multiresolution transform of cube: mr3d\_trans}
\index{mr3d\_trans}
\label{sect_trans3d}
The program 
{\em mr3d\_trans} computes the multiresolution transform of a cube by
the \`a trous WT or a non-redundant transform. 
The output file which contains the transformation has a 
suffix, .mr. If the output file name
given by the user does not contain this suffix, it is automatically
added. The ``.mr'' file is a FITS format file, and can be manipulated by
any package dealing with FITS format, or using the 
{\em mr3d\_extract} program.
{\bf 
\begin{center}
 USAGE: mr3d\_trans option cube\_in multiresolution\_transform\_out
\end{center}}
where options are: 
\begin{itemize} 
\baselineskip=0.4truecm
\itemsep=0.1truecm
\item {\bf [-t type\_of\_multiresolution\_transform]}
{\small 
\begin{enumerate}
\item  (bi-) orthogonal wavelet transform.  
Antonini 7/9 filters ~\cite{wave:antonini92} are used by default, with an 
$L_1$ normalization. The filters can be changed using the ``-T'' option, and
an $L_2$ normalization is obtained by the ``-L'' option.
\item Wavelet transform via lifting scheme 
\item A trous wavelet transform
\end{enumerate}}
Default is 1.
\item {\bf [-n number\_of\_scales]} \\
 Number of scales used in the multiresolution transform. \\
 Default is 4.
\item {\bf [-T type\_of\_filters]}  
{\small
\begin{enumerate}
\itemsep=0.1truecm
\item Antonini 7/9 filters. 
\item Daubechies filter 4. 
\item Biorthogonal 2/6 Haar filters.
\item Biorthogonal 2/10 Haar filters.
\item Odegard 7/9 filters.
\item User's filters.
\end{enumerate}}
Default is Antonini 7/9 filters.
\item {\bf [-L]} \\
Use an $L_2$ normalization. Default is $L_1$.
\item {\bf [-l type\_of\_lifting\_transform]}  
{\small 
\begin{enumerate}
\baselineskip=0.4truecm
\item Lifting scheme: CDF WT. 
\item Lifting scheme: median prediction.
\item Lifting scheme: integer Haar WT. 
\item Lifting scheme: integer CDF WT. 
\item Lifting scheme: integer (4,2) interpolating transform. 
\item Lifting scheme:  Antonini 7/9 filters.
\item Lifting scheme: integer Antonini 7/9 filters. 
\end{enumerate}}
 Default is Lifting scheme: integer Haar WT.  
\item {\bf [-i]} \\
Print statistical information about each band.
\end{itemize}
The result is stored in a file (suffix ``.mr''),
and cube (or scales) of the transformation can be extracted
by using the {\em mr3d\_extract} program. \\
\subsubsection*{Examples:}
\begin{itemize}
\itemsep=0.1truecm
\item mr3d\_trans cube.fits cube\_trans.mr \\
Apply the wavelet transform transform to a cube, and store the
result in cube\_trans.mr. 
\item mr3d\_trans -T 3 -i cube.fits cube\_trans.mr \\
Same as before, but use Haar filters, and
  print statistical information.
\item mr3d\_trans -t 2 -l 1 cube.fits cube\_trans.mr \\
Apply a wavelet transform via the lifting scheme.
\end{itemize}


\subsection{Extraction of a scale: mr3d\_extract}
\index{mr3d\_extract}
\label{sect_extr3d}

The program
{\em mr3d\_extract} allows the user to extract a  band from
a multiresolution transform file (suffix ``.mr").
{\bf
\begin{center}
 USAGE: mr3d\_extract options multiresolution\_file  output\_cube
\end{center}}
where options are:
\begin{itemize}
\itemsep=0.1truecm
 \item {\bf  [-b band\_number] } \\
 Band number to extract. Default is 1.
\end{itemize}
\subsubsection*{Example:}
\begin{itemize}
\item mr3d\_extract -b 2 mr\_file.mr cube\_band\_2.fits \\
Extract the second band of the wavelet transform, and write as a .fits
file of name ``cube\_band\_2.d''.  
\end{itemize}

\subsection{Insertion of an image: mr3d\_insert}
\index{mr3d\_insert}
The program
{\em mr3d\_insert} replaces a band by some cube, by inserting it
in the multiresolution transform file. The band and the cube
must have the same size. 
{\bf
\begin{center}
 USAGE: mr3d\_insert options multiresolution\_file input\_cube
\end{center}}
where options are:
\begin{itemize}
\itemsep=0.1truecm
\item  {\bf  [-b band\_number] } \\
 Band number to insert. Default is 1.
\end{itemize}
{\em multiresolution\_file} is the file (.mr) which contains 
the multiresolution transformation, {\em input\_cube} is the cube
which must replace the band cube. {\em band \_number} specifies
the band  number to be replaced. Default is 1. \\
The multiresolution transform is updated. \\
\subsubsection*{Example:}
\begin{itemize}
\item mr3d\_insert -b 3 mr\_file.mr cube.fits \\
Insert a cube at the third band of the multiresolution file. 
\end{itemize}

\subsection{Reconstruction: mr3d\_recons}
\index{mr3d\_recons}
The program 
{\em mr3d\_recons}  reconstructs a cube from its multiresolution
transform.
{\bf
\begin{center}
 USAGE: mr3d\_recons  multiresolution\_file  cube\_out
\end{center}}
{\em multiresolution\_file} is the file (.mr) which contains 
the multiresolution transformation, {\em output\_cube} is  the
output reconstructed cube.


\section{\projmr3 Denoising: mr3d\_filter}
\index{filtering}
\subsection{Gaussian noise filtering: mr3d\_filter}
\index{mr3d\_filter}
Program {\em mr\_filter} filters a cube. Only Gaussian noise 
is considered.
{\bf
\begin{center}
 USAGE: mr3d\_filter option cube\_in cube\_out
\end{center}}
where options are 
\begin{itemize}
\baselineskip=0.4truecm
\itemsep=0.1truecm
\item {\bf [-t type\_of\_multiresolution\_transform]}
{\small 
\begin{enumerate}
\item  (bi-) orthogonal wavelet transform.  
Antonini 7/9 filters ~\cite{wave:antonini92} are used by default, with an 
$L_1$ normalization. The filters can be changed using the ``-T'' option, and
an $L_2$ normalization is obtained by the ``-L'' option.
\item A trous wavelet transform
\end{enumerate}}
\item {\bf [-n number\_of\_scales]} \\
Number of scales used in the multiresolution transform. Default is 4.
\item {\bf [-T type\_of\_filters]}  \\
see~\ref{sect_trans3d}.
\item {\bf [-g sigma]} \\
 The image contains Gaussian noise, and the standard deviation is
given by {\em sigma }. The option should be set only if the user
knows the standard deviation of the noise. 
\item {\bf [-s NSigma]} \\
The detection level at each scale is determined by the product
of the standard deviation of the noise by the {\em NSigma}.
{\em NSigma} fixes the confidence interval we want. By default,
{\em NSigma} is equal to 3.
\item {\bf [-C]} \\
Correlated noise.
\end{itemize}
\subsubsection*{Examples:}
\begin{itemize}
\item mr3d\_filter cube\_in.fits cube\_out.fits \\
Filters a cube by multiresolution  thresholding, assuming
Gaussian noise (its standard deviation is automatically estimated).
\item mr3d\_filter -s 4 -g 10. cube\_in.fits cube\_out.fits \\
Same as before, but the noise standard deviation is fixed to 10, and
a  $4\sigma$ thresholding is performed. 
\item mr3d\_filter -C cube\_in.fits cube\_out.fits \\
Consider correlated noise instead of Gaussian noise. The noise standard
deviation is estimated at each scale using the MAD (median absolute
deviation) estimator.
\end{itemize}


\newpage

\section{3D Point Clouds Analysis}
We will call {\em Catalog} a set of points in the 3D space, each point 
being represented by its three coordinates $X,Y,Z$.

\subsubsection*{ASCII Catalog}
The catalog format is the following: 
\subsubsection*{Catalogue format}
\begin{itemize}
\item the first line must contain the number of points $N$, the
  dimension $D$ (i.e. 3), and the coordinate system $S$ ($S$ = 1 or
  2).  The recognized coordinate systems are:
  \begin{enumerate}
  \item $X$ and/or $Y$ and/or $Z$ Euclidien system.
  \item Angular coordinate system (longitude, latitude and/or
    distance; that could be the equatorial system -- right ascension
    $\alpha$, declination $\delta$, the galactic coordinate system
    with galactic longitude $l$, galactic latitude $b$, the
    supergalactic coordinate system with $SL$ and $SB$).
  \end{enumerate}
\item the $D$ following lines must contains three values: the range of
  variation of the $i-th$ coordinate (min,max) and a flag indicating
  the generation model for the corresponding coordinate: 0 -- uniform
  between [min,max], 1 -- bootstrapping the coordinate and 2 -- uniform
  on sphere between [min,max] in degrees. When the user have supplied
  its own generated random catalogs, then those three lines are
  ignored. This last value is used only by programs which need to 
  simulate data from the input data, and can generally be set to zero.
\item the $N$ following lines contains the coordinates of the points.
\end{itemize}
An example of a 3D catalogue, with 10 points in the Euclidien system,
each coordinate being defined in the interval $[0,100[$ and random
catalog coordinates uniform in $[0,100[$ for each axis, is:
\begin{verbatim}
       10       3           1
       0      100.000       0
       0      100.000       0
       0      100.000       0
      42.1782      13.4610      73.7444
      41.6855      9.82727      75.3605
      42.3580      14.7867      73.1548
      42.0255      12.3347      74.2453
      42.7474      17.6581      71.8777
      41.9410      11.7113      74.5226
      65.5637      9.84036      71.3585
      65.7140      12.6019      70.5645
      65.5843      10.2196      71.2495
      65.4521      7.79074      71.9479
\end{verbatim}
The file name must have the suffixe ``\.cat''.

\subsection{ASCII Catalog to 3D Cube in FITS Format: mr3d\_cat2fits}
\index{mr3d\_cat2fits}
\label{sect_cat2fits}

The program {\em mr3d\_cat2fits} allows the user to convert a catalog
into a 3D cube in the FITS format using a $B_3$ spline scaling function.
When using the \`a trous Wavelet Transform, this step corresponds to
the projection in the $V_0$ space, and garanties that the obtained 
wavelet coefficients are exactly the scalar products of the input 
irregularly spaced data with the wavelet function. Assuming Poisson 
noise, very robust threshold levels can be derived 
in the wavelet space (program {\em mr\_phisto}), which can be used for the 
detection of clusters (program {\em mr\_psupport}) or for the
filtering (program {\em mr\_pfilter}). See \cite{starck:book02} for
more details.

{\bf
\begin{center}
 USAGE: mr3d\_cat2fits options catalog cube\_out
\end{center}}
where options are:
\begin{itemize}
\itemsep=0.1truecm
\item {\bf  [-B Bin] } \\
 Bin gives the resolution. Default is 1. 
\item {\bf  [-p] } \\
Do not interpolate the data. By default, a interpolation by a spline
a degree 3 is performed. It corresponds to the projection in the 
space $V_0$ in the wavelet theory.
\end{itemize}

\subsubsection*{Example:}
\begin{itemize}
\item mr3d\_cat2fits data.cat cube\_out.fits \\
Convert a catalog into a 3D fits file.
\item mr3d\_cat2fits -p -b5 data.cat cube\_out.fits \\
Ditto, but do not use an interpolation, and take a pixel resolution of 5
(units are the same as in the input catalog).
\item mr3d\_transform -t3 cube\_out.fits trans.mr\\
Wavelet transform of the catalog. The wavelet coefficients in {\em trans.mr}
are exactly the scalar products of the irregularly spaced data with
the wavelet function. 
\end{itemize}


\subsection{Wavelet Histogram Autoconvolution: mr3d\_phisto}
\index{mr3d\_phisto}
Program {\em mr3d\_phisto} precomputes a set tables which 
is used by {\em mr3d\_psupport} and
{\em mr3d\_pfilter}. The tables are saved 
in the FITS table format. They allows the estimation of the detection
levels for a wavelet coefficient caculated from $n$ evenements.
To derive these thresholds, the wavelet funciton histogram must be
autoconvolved $n$ times.
The filenames  are:
\begin{itemize}
\item \_Aba\_histo.fits: 2D array $H[n, 2049]$, which contains
$n$ autoconvolutions.   
\item \_histobin.fits: 2D array $B_1[n, 2]$, where $B_2[n,0]$ is the bin
for $H[n,*]$ and $B_1[n,1]$ is the number of points where the $H[n,*]$ is 
 different from zero.
\item \_histobound.fits: 2D array $B_2[n, 2]$, where 
$B_2[n,0]$ and $B_2[n,1]$  are respectively 
the min and the max of the nth histogram.
\item \_histodistrib.fits: 2D array $D[n, 2049]$,  where $D[n,*]$ is 
the repartition function of the histogram distribution $H[n,*]$.
\item \_param.fits: 1D array $P[2]$, where $P[0]$ and $P[0]$
are respectively the $r$ (rythm variable) and the $n$ 
(number of calculated histograms) parameters,
\end{itemize}
% h = rim('_Aba_histo.fits')
% b = rim('_histobin.fits')
% m = rim('_histobound.fits')
% x = indgen(2049,/float)
% x = x * b(10,0) + m(10,0)
% plot,x,h(10,*)
% d = rim('_histodistrib.fits')
% plot, x, d(10,*)


{\bf
\begin{center}
 USAGE: mr3d\_phisto option  
\end{center}}
where options are 
\begin{itemize}
\baselineskip=0.4truecm
\itemsep=0.1truecm
\item {\bf [-n number\_of\_convolution]} \\
Total number of calculated histograms. Default is 30
(the total of the cube must be lower than $2^{n}$).
\item {\bf [-r rythm]}  \\
 Parameter relative to the rythm of autoconvolution (default=0).
The program computes the $2^n$ autoconvolutions of the wavelet histogram. 
It computes also $2^r$ convolutions between $2^n$ and $2^(n+1)$.
\item {\bf [-d]}  \\
Use all default options.
\item {\bf [-v]}  \\
Verbose. Default is no.
\end{itemize}

\subsubsection*{Examples:}
\begin{itemize}
\item mr3d\_phisto -d \\
Estimates the autoconvoled histograms, and calculates the threshold levels.
\end{itemize}


\subsection{Multiresolution Support Calculation: mr3d\_psupport}
\label{sect_event3D}
\index{mr3d\_psupport}
Program 
{\em mr3d\_psupport} applies a wavelet transform using the \`a trous
algorithm to 3D data, assuming that the noise follows a Poisson distribution,
and in the case where we have only few events per pixel \cite{starck:book02}.
Data can either be given by an ASCII table of coordinates in real values, or
by a cube in the fits format.  
{\bf
\begin{center}
 USAGE: mr3d\_psupport options  in\_cube  out\_file
\end{center}}
where options are 
\begin{itemize}
\baselineskip=0.4truecm
\itemsep=0.1truecm
\item {\bf [-F first\_detection\_scale]} \\
First scale used for the detection. Default is 1.
\item {\bf [-e minimum\_of\_events]}  \\
Minimum number of events for a detection. Default is 5. If a wavelet 
coefficient has been calculated from fewer events than this minimum
value, it is not considered as significant.
% \item {\bf [-s SignifStructureAnalysis\_FileName]} \\
% Analyze the detected wavelet coefficients, and write in the file:
% \begin{itemize}
% \baselineskip=0.4truecm
% \item Number of detected structures per scale
% \item Percentage of significant wavelet coefficients
% \item Mean deviation of shape from sphericity
% \item For each detected structure, its volume, its surface, and
% \item its deviation of shape from sphericity, 
% \item its angles, its elongation in the three axis directions
% \end{itemize}
% \item {\bf [-t SignifStructureAnalysis\_FileName]} \\
% Same as -s option, but results are stored in an ASCII table format.
% The table contains: scale number, structure number, 
% Max\_x, Max\_y, Max\_z, Volume, Surface, Morpho, 
% Angle\_Theta, Angle\_Phi, Sigma\_1, Sigma\_2, Sigma\_3. 
\item {\bf [-p]}  \\
Detect only positive structure.
% \item {\bf [-f filename\_filtered\_data]}  \\
% File with filtered input data.
\item {\bf [-n number\_of\_scales]}  \\
Number of scales used in the multiresolution transform.
Default is 6. The maximum number of scales depends on the 
cube size. For a $32 \times 32 \times 32$ cube, three scales are allowed (four
scale for a $64 \times 64 \times 64$ cube, 
five for a $128 \times 128 \times 128$ cube, \dots).
\item {\bf [-E epsilon]}  \\
Value of epsilon used for the threshold (default=1e-3 ).
\item {\bf [-k]}  \\
Suppress isolated pixels in the support. Default is no.
\item {\bf [-R]}  \\
  Remove the structures near the borders.
\item {\bf [-B bin]}  \\
 If the input data is a catalogue, it gives the bin of it (default=1).
% \item {\bf [-w]]}  \\
% Write the multiscale support to the disk.
% \item {\bf [-S]}  \\
% Read the support already computed.
\end{itemize}


\subsubsection*{Examples:}
\begin{itemize}
\baselineskip=0.4truecm
\itemsep=0.1truecm
\item mr3d\_phisto -d \\
Compute a set a table which will be used by {\em mr3d\_psupport}.
\item mr3d\_psupport -v simcub32.fits w1   \\
Multiresolution support estimation. \\
Creates the two files
w1\_support\_1.fits and w1\_support\_2.fits cor\-res\-pon\-ding
the support of two wavelet scales. 
\item mr3d\_psupport -R -k -E1e-4 -v simcub32.fits w1   \\
Ditto, but remove isolated pixels, structures at the border,
and increase the detection the level.
\end{itemize}

\subsection{Data Cube Filtering: mr3d\_pfilter}
\index{mr3d\_pfilter}
Program {\em mr3d\_pfilter} filters an image (as does {\em mr3d\_filter})
assuming  that the noise follows a Poisson distribution and in the case 
where we have only few events per pixel.
Data can either be given by an image or by 
an ASCII table of coordinates in real values.
As for {\em mr3d\_psupport}, the pre-computed table must exist (i.e.
the program {\em mr3d\_phisto must be executed.}
If the ``-c'' option is used, the user gives a second input image file name.
Then the multiresolution support is estimated from the first image
(given by option ``-a'' or ``-I''), and the second image is filtered
using the multiresolution support of the first.
{\bf
\begin{center}
 USAGE: mr3d\_pfilter option image\_out
\end{center}}
where options are:
\begin{itemize}
\baselineskip=0.4truecm
\itemsep=0.1truecm
\item {\bf [-n number\_of\_scales]}  \\
Number of scales used in the multiresolution transform.
Default is 6.
\item {\bf [-F first\_detection\_scale]} \\
First scale used for the detection. Default is 1.
\item {\bf [-E epsilon]}  \\
Value of epsilon used for the threshold (default=1e-3 ).
\item {\bf [-m minimum\_of\_events]}  \\
Minimum number of events for a detection. Default is 5. If a wavelet 
coefficient has been calculated from fewer events than this minimum
value, it is not considered as significant.
\item {\bf [-i number\_of\_iterations]}  \\
 Maximum number of iterations. Default is 0.
 If this option is set, then an iterative reconstruction method is used to
 restore the cube.
\item {\bf [-B bin]}  \\
 If the input data is a catalogue, it gives the bin of it (default=1).
\item {\bf [-R]}  \\
  Remove the structures near the borders.
\item {\bf [-C]}  \\
Smoothness constraint. 
\item {\bf [-p]}  \\
Detect only positive structure.
\item {\bf [-w]}  \\
Write the multiscale support to the disk. Each scale of the multiresolution
support is written separately in a FITS file, 
with name ``x\_support\_j.fits'', where ``x'' is the output file name
and $j$ is the scale number.
\item {\bf [-S]}  \\
Read the multiresolution support already computed. When we want to 
apply the filtering several times with different options, the
multiresolution can calculated the first time ans saved on the disk
with the ``-w'' option, and for the other experiments can read the
pre-computed multiresolution support and do not have to recalculate it.
\item {\bf [-k]}  \\
Suppress isolated pixels in the support. Default is no
\item {\bf [-K]} \\       
Suppress the last scale.
\end{itemize}

\noindent

\subsubsection*{Examples:}
\begin{itemize}
\baselineskip=0.4truecm
\itemsep=0.1truecm
\item mr3d\_phisto -d \\
Compute a set a table which will be used by {\em mr3d\_pfilter}.
\item mr3d\_pfilter cube\_in cube\_out \\
Filtering with all default options.
\item mr3d\_pfilter -R -k -E1e-4 cube\_in cube\_out \\
Ditto, but remove isolated pixels and in structures at the border
the multiresolution support, and increase the detection the level.
\end{itemize}
