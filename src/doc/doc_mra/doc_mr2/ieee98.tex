 
\documentclass[11pt,a4paper]{article}
% \renewcommand{\baselinestretch}{2}

% Psfig/TeX 
\def\PsfigVersion{1.9}
% dvips version
%
% All psfig/tex software, documentation, and related files
% in this distribution of psfig/tex are 
% Copyright 1987, 1988, 1991 Trevor J. Darrell
%
% Permission is granted for use and non-profit distribution of psfig/tex 
% providing that this notice is clearly maintained. The right to
% distribute any portion of psfig/tex for profit or as part of any commercial
% product is specifically reserved for the author(s) of that portion.
%
% *** Feel free to make local modifications of psfig as you wish,
% *** but DO NOT post any changed or modified versions of ``psfig''
% *** directly to the net. Send them to me and I'll try to incorporate
% *** them into future versions. If you want to take the psfig code 
% *** and make a new program (subject to the copyright above), distribute it, 
% *** (and maintain it) that's fine, just don't call it psfig.
%
% Bugs and improvements to trevor@media.mit.edu.
%
% Thanks to Greg Hager (GDH) and Ned Batchelder for their contributions
% to the original version of this project.
%
% Modified by J. Daniel Smith on 9 October 1990 to accept the
% %%BoundingBox: comment with or without a space after the colon.  Stole
% file reading code from Tom Rokicki's EPSF.TEX file (see below).
%
% More modifications by J. Daniel Smith on 29 March 1991 to allow the
% the included PostScript figure to be rotated.  The amount of
% rotation is specified by the "angle=" parameter of the \psfig command.
%
% Modified by Robert Russell on June 25, 1991 to allow users to specify
% .ps filenames which don't yet exist, provided they explicitly provide
% boundingbox information via the \psfig command. Note: This will only work
% if the "file=" parameter follows all four "bb???=" parameters in the
% command. This is due to the order in which psfig interprets these params.
%
%  3 Jul 1991	JDS	check if file already read in once
%  4 Sep 1991	JDS	fixed incorrect computation of rotated
%			bounding box
% 25 Sep 1991	GVR	expanded synopsis of \psfig
% 14 Oct 1991	JDS	\fbox code from LaTeX so \psdraft works with TeX
%			changed \typeout to \ps@typeout
% 17 Oct 1991	JDS	added \psscalefirst and \psrotatefirst
%

% From: gvr@cs.brown.edu (George V. Reilly)
%
% \psdraft	draws an outline box, but doesn't include the figure
%		in the DVI file.  Useful for previewing.
%
% \psfull	includes the figure in the DVI file (default).
%
% \psscalefirst width= or height= specifies the size of the figure
% 		before rotation.
% \psrotatefirst (default) width= or height= specifies the size of the
% 		 figure after rotation.  Asymetric figures will
% 		 appear to shrink.
%
% \psfigurepath#1	sets the path to search for the figure
%
% \psfig
% usage: \psfig{file=, figure=, height=, width=,
%			bbllx=, bblly=, bburx=, bbury=,
%			rheight=, rwidth=, clip=, angle=, silent=}
%
%	"file" is the filename.  If no path name is specified and the
%		file is not found in the current directory,
%		it will be looked for in directory \psfigurepath.
%	"figure" is a synonym for "file".
%	By default, the width and height of the figure are taken from
%		the BoundingBox of the figure.
%	If "width" is specified, the figure is scaled so that it has
%		the specified width.  Its height changes proportionately.
%	If "height" is specified, the figure is scaled so that it has
%		the specified height.  Its width changes proportionately.
%	If both "width" and "height" are specified, the figure is scaled
%		anamorphically.
%	"bbllx", "bblly", "bburx", and "bbury" control the PostScript
%		BoundingBox.  If these four values are specified
%               *before* the "file" option, the PSFIG will not try to
%               open the PostScript file.
%	"rheight" and "rwidth" are the reserved height and width
%		of the figure, i.e., how big TeX actually thinks
%		the figure is.  They default to "width" and "height".
%	The "clip" option ensures that no portion of the figure will
%		appear outside its BoundingBox.  "clip=" is a switch and
%		takes no value, but the `=' must be present.
%	The "angle" option specifies the angle of rotation (degrees, ccw).
%	The "silent" option makes \psfig work silently.
%

% check to see if macros already loaded in (maybe some other file says
% "\input psfig") ...
\ifx\undefined\psfig\else\endinput\fi

%
% from a suggestion by eijkhout@csrd.uiuc.edu to allow
% loading as a style file. Changed to avoid problems
% with amstex per suggestion by jbence@math.ucla.edu

\let\LaTeXAtSign=\@
\let\@=\relax
\edef\psfigRestoreAt{\catcode`\@=\number\catcode`@\relax}
%\edef\psfigRestoreAt{\catcode`@=\number\catcode`@\relax}
\catcode`\@=11\relax
\newwrite\@unused
\def\ps@typeout#1{{\let\protect\string\immediate\write\@unused{#1}}}
\ps@typeout{psfig/tex \PsfigVersion}

%% Here's how you define your figure path.  Should be set up with null
%% default and a user useable definition.

\def\figurepath{./}
\def\psfigurepath#1{\edef\figurepath{#1}}

%
% @psdo control structure -- similar to Latex @for.
% I redefined these with different names so that psfig can
% be used with TeX as well as LaTeX, and so that it will not 
% be vunerable to future changes in LaTeX's internal
% control structure,
%
\def\@nnil{\@nil}
\def\@empty{}
\def\@psdonoop#1\@@#2#3{}
\def\@psdo#1:=#2\do#3{\edef\@psdotmp{#2}\ifx\@psdotmp\@empty \else
    \expandafter\@psdoloop#2,\@nil,\@nil\@@#1{#3}\fi}
\def\@psdoloop#1,#2,#3\@@#4#5{\def#4{#1}\ifx #4\@nnil \else
       #5\def#4{#2}\ifx #4\@nnil \else#5\@ipsdoloop #3\@@#4{#5}\fi\fi}
\def\@ipsdoloop#1,#2\@@#3#4{\def#3{#1}\ifx #3\@nnil 
       \let\@nextwhile=\@psdonoop \else
      #4\relax\let\@nextwhile=\@ipsdoloop\fi\@nextwhile#2\@@#3{#4}}
\def\@tpsdo#1:=#2\do#3{\xdef\@psdotmp{#2}\ifx\@psdotmp\@empty \else
    \@tpsdoloop#2\@nil\@nil\@@#1{#3}\fi}
\def\@tpsdoloop#1#2\@@#3#4{\def#3{#1}\ifx #3\@nnil 
       \let\@nextwhile=\@psdonoop \else
      #4\relax\let\@nextwhile=\@tpsdoloop\fi\@nextwhile#2\@@#3{#4}}
% 
% \fbox is defined in latex.tex; so if \fbox is undefined, assume that
% we are not in LaTeX.
% Perhaps this could be done better???
\ifx\undefined\fbox
% \fbox code from modified slightly from LaTeX
\newdimen\fboxrule
\newdimen\fboxsep
\newdimen\ps@tempdima
\newbox\ps@tempboxa
\fboxsep = 3pt
\fboxrule = .4pt
\long\def\fbox#1{\leavevmode\setbox\ps@tempboxa\hbox{#1}\ps@tempdima\fboxrule
    \advance\ps@tempdima \fboxsep \advance\ps@tempdima \dp\ps@tempboxa
   \hbox{\lower \ps@tempdima\hbox
  {\vbox{\hrule height \fboxrule
          \hbox{\vrule width \fboxrule \hskip\fboxsep
          \vbox{\vskip\fboxsep \box\ps@tempboxa\vskip\fboxsep}\hskip 
                 \fboxsep\vrule width \fboxrule}
                 \hrule height \fboxrule}}}}
\fi
%
%%%%%%%%%%%%%%%%%%%%%%%%%%%%%%%%%%%%%%%%%%%%%%%%%%%%%%%%%%%%%%%%%%%
% file reading stuff from epsf.tex
%   EPSF.TEX macro file:
%   Written by Tomas Rokicki of Radical Eye Software, 29 Mar 1989.
%   Revised by Don Knuth, 3 Jan 1990.
%   Revised by Tomas Rokicki to accept bounding boxes with no
%      space after the colon, 18 Jul 1990.
%   Portions modified/removed for use in PSFIG package by
%      J. Daniel Smith, 9 October 1990.
%
\newread\ps@stream
\newif\ifnot@eof       % continue looking for the bounding box?
\newif\if@noisy        % report what you're making?
\newif\if@atend        % %%BoundingBox: has (at end) specification
\newif\if@psfile       % does this look like a PostScript file?
%
% PostScript files should start with `%!'
%
{\catcode`\%=12\global\gdef\epsf@start{%!}}
\def\epsf@PS{PS}
%
\def\epsf@getbb#1{%
%
%   The first thing we need to do is to open the
%   PostScript file, if possible.
%
\openin\ps@stream=#1
\ifeof\ps@stream\ps@typeout{Error, File #1 not found}\else
%
%   Okay, we got it. Now we'll scan lines until we find one that doesn't
%   start with %. We're looking for the bounding box comment.
%
   {\not@eoftrue \chardef\other=12
    \def\do##1{\catcode`##1=\other}\dospecials \catcode`\ =10
    \loop
       \if@psfile
	  \read\ps@stream to \epsf@fileline
       \else{
	  \obeyspaces
          \read\ps@stream to \epsf@tmp\global\let\epsf@fileline\epsf@tmp}
       \fi
       \ifeof\ps@stream\not@eoffalse\else
%
%   Check the first line for `%!'.  Issue a warning message if its not
%   there, since the file might not be a PostScript file.
%
       \if@psfile\else
       \expandafter\epsf@test\epsf@fileline:. \\%
       \fi
%
%   We check to see if the first character is a % sign;
%   if so, we look further and stop only if the line begins with
%   `%%BoundingBox:' and the `(atend)' specification was not found.
%   That is, the only way to stop is when the end of file is reached,
%   or a `%%BoundingBox: llx lly urx ury' line is found.
%
          \expandafter\epsf@aux\epsf@fileline:. \\%
       \fi
   \ifnot@eof\repeat
   }\closein\ps@stream\fi}%
%
% This tests if the file we are reading looks like a PostScript file.
%
\long\def\epsf@test#1#2#3:#4\\{\def\epsf@testit{#1#2}
			\ifx\epsf@testit\epsf@start\else
\ps@typeout{Warning! File does not start with `\epsf@start'.  It may not be a PostScript file.}
			\fi
			\@psfiletrue} % don't test after 1st line
%
%   We still need to define the tricky \epsf@aux macro. This requires
%   a couple of magic constants for comparison purposes.
%
{\catcode`\%=12\global\let\epsf@percent=%\global\def\epsf@bblit{%BoundingBox}}
%
%
%   So we're ready to check for `%BoundingBox:' and to grab the
%   values if they are found.  We continue searching if `(at end)'
%   was found after the `%BoundingBox:'.
%
\long\def\epsf@aux#1#2:#3\\{\ifx#1\epsf@percent
   \def\epsf@testit{#2}\ifx\epsf@testit\epsf@bblit
	\@atendfalse
        \epsf@atend #3 . \\%
	\if@atend	
	   \if@verbose{
		\ps@typeout{psfig: found `(atend)'; continuing search}
	   }\fi
        \else
        \epsf@grab #3 . . . \\%
        \not@eoffalse
        \global\no@bbfalse
        \fi
   \fi\fi}%
%
%   Here we grab the values and stuff them in the appropriate definitions.
%
\def\epsf@grab #1 #2 #3 #4 #5\\{%
   \global\def\epsf@llx{#1}\ifx\epsf@llx\empty
      \epsf@grab #2 #3 #4 #5 .\\\else
   \global\def\epsf@lly{#2}%
   \global\def\epsf@urx{#3}\global\def\epsf@ury{#4}\fi}%
%
% Determine if the stuff following the %%BoundingBox is `(atend)'
% J. Daniel Smith.  Copied from \epsf@grab above.
%
\def\epsf@atendlit{(atend)} 
\def\epsf@atend #1 #2 #3\\{%
   \def\epsf@tmp{#1}\ifx\epsf@tmp\empty
      \epsf@atend #2 #3 .\\\else
   \ifx\epsf@tmp\epsf@atendlit\@atendtrue\fi\fi}


% End of file reading stuff from epsf.tex
%%%%%%%%%%%%%%%%%%%%%%%%%%%%%%%%%%%%%%%%%%%%%%%%%%%%%%%%%%%%%%%%%%%

%%%%%%%%%%%%%%%%%%%%%%%%%%%%%%%%%%%%%%%%%%%%%%%%%%%%%%%%%%%%%%%%%%%
% trigonometry stuff from "trig.tex"
\chardef\psletter = 11 % won't conflict with \begin{letter} now...
\chardef\other = 12

\newif \ifdebug %%% turn me on to see TeX hard at work ...
\newif\ifc@mpute %%% don't need to compute some values
\c@mputetrue % but assume that we do

\let\then = \relax
\def\r@dian{pt }
\let\r@dians = \r@dian
\let\dimensionless@nit = \r@dian
\let\dimensionless@nits = \dimensionless@nit
\def\internal@nit{sp }
\let\internal@nits = \internal@nit
\newif\ifstillc@nverging
\def \Mess@ge #1{\ifdebug \then \message {#1} \fi}

{ %%% Things that need abnormal catcodes %%%
	\catcode `\@ = \psletter
	\gdef \nodimen {\expandafter \n@dimen \the \dimen}
	\gdef \term #1 #2 #3%
	       {\edef \t@ {\the #1}%%% freeze parameter 1 (count, by value)
		\edef \t@@ {\expandafter \n@dimen \the #2\r@dian}%
				   %%% freeze parameter 2 (dimen, by value)
		\t@rm {\t@} {\t@@} {#3}%
	       }
	\gdef \t@rm #1 #2 #3%
	       {{%
		\count 0 = 0
		\dimen 0 = 1 \dimensionless@nit
		\dimen 2 = #2\relax
		\Mess@ge {Calculating term #1 of \nodimen 2}%
		\loop
		\ifnum	\count 0 < #1
		\then	\advance \count 0 by 1
			\Mess@ge {Iteration \the \count 0 \space}%
			\Multiply \dimen 0 by {\dimen 2}%
			\Mess@ge {After multiplication, term = \nodimen 0}%
			\Divide \dimen 0 by {\count 0}%
			\Mess@ge {After division, term = \nodimen 0}%
		\repeat
		\Mess@ge {Final value for term #1 of 
				\nodimen 2 \space is \nodimen 0}%
		\xdef \Term {#3 = \nodimen 0 \r@dians}%
		\aftergroup \Term
	       }}
	\catcode `\p = \other
	\catcode `\t = \other
	\gdef \n@dimen #1pt{#1} %%% throw away the ``pt''
}

\def \Divide #1by #2{\divide #1 by #2} %%% just a synonym

\def \Multiply #1by #2%%% allows division of a dimen by a dimen
       {{%%% should really freeze parameter 2 (dimen, passed by value)
	\count 0 = #1\relax
	\count 2 = #2\relax
	\count 4 = 65536
	\Mess@ge {Before scaling, count 0 = \the \count 0 \space and
			count 2 = \the \count 2}%
	\ifnum	\count 0 > 32767 %%% do our best to avoid overflow
	\then	\divide \count 0 by 4
		\divide \count 4 by 4
	\else	\ifnum	\count 0 < -32767
		\then	\divide \count 0 by 4
			\divide \count 4 by 4
		\else
		\fi
	\fi
	\ifnum	\count 2 > 32767 %%% while retaining reasonable accuracy
	\then	\divide \count 2 by 4
		\divide \count 4 by 4
	\else	\ifnum	\count 2 < -32767
		\then	\divide \count 2 by 4
			\divide \count 4 by 4
		\else
		\fi
	\fi
	\multiply \count 0 by \count 2
	\divide \count 0 by \count 4
	\xdef \product {#1 = \the \count 0 \internal@nits}%
	\aftergroup \product
       }}

\def\r@duce{\ifdim\dimen0 > 90\r@dian \then   % sin(x+90) = sin(180-x)
		\multiply\dimen0 by -1
		\advance\dimen0 by 180\r@dian
		\r@duce
	    \else \ifdim\dimen0 < -90\r@dian \then  % sin(-x) = sin(360+x)
		\advance\dimen0 by 360\r@dian
		\r@duce
		\fi
	    \fi}

\def\Sine#1%
       {{%
	\dimen 0 = #1 \r@dian
	\r@duce
	\ifdim\dimen0 = -90\r@dian \then
	   \dimen4 = -1\r@dian
	   \c@mputefalse
	\fi
	\ifdim\dimen0 = 90\r@dian \then
	   \dimen4 = 1\r@dian
	   \c@mputefalse
	\fi
	\ifdim\dimen0 = 0\r@dian \then
	   \dimen4 = 0\r@dian
	   \c@mputefalse
	\fi
%
	\ifc@mpute \then
        	% convert degrees to radians
		\divide\dimen0 by 180
		\dimen0=3.141592654\dimen0
%
		\dimen 2 = 3.1415926535897963\r@dian %%% a well-known constant
		\divide\dimen 2 by 2 %%% we only deal with -pi/2 : pi/2
		\Mess@ge {Sin: calculating Sin of \nodimen 0}%
		\count 0 = 1 %%% see power-series expansion for sine
		\dimen 2 = 1 \r@dian %%% ditto
		\dimen 4 = 0 \r@dian %%% ditto
		\loop
			\ifnum	\dimen 2 = 0 %%% then we've done
			\then	\stillc@nvergingfalse 
			\else	\stillc@nvergingtrue
			\fi
			\ifstillc@nverging %%% then calculate next term
			\then	\term {\count 0} {\dimen 0} {\dimen 2}%
				\advance \count 0 by 2
				\count 2 = \count 0
				\divide \count 2 by 2
				\ifodd	\count 2 %%% signs alternate
				\then	\advance \dimen 4 by \dimen 2
				\else	\advance \dimen 4 by -\dimen 2
				\fi
		\repeat
	\fi		
			\xdef \sine {\nodimen 4}%
       }}

% Now the Cosine can be calculated easily by calling \Sine
\def\Cosine#1{\ifx\sine\UnDefined\edef\Savesine{\relax}\else
		             \edef\Savesine{\sine}\fi
	{\dimen0=#1\r@dian\advance\dimen0 by 90\r@dian
	 \Sine{\nodimen 0}
	 \xdef\cosine{\sine}
	 \xdef\sine{\Savesine}}}	      
% end of trig stuff
%%%%%%%%%%%%%%%%%%%%%%%%%%%%%%%%%%%%%%%%%%%%%%%%%%%%%%%%%%%%%%%%%%%%

\def\psdraft{
	\def\@psdraft{0}
	%\ps@typeout{draft level now is \@psdraft \space . }
}
\def\psfull{
	\def\@psdraft{100}
	%\ps@typeout{draft level now is \@psdraft \space . }
}

\psfull

\newif\if@scalefirst
\def\psscalefirst{\@scalefirsttrue}
\def\psrotatefirst{\@scalefirstfalse}
\psrotatefirst

\newif\if@draftbox
\def\psnodraftbox{
	\@draftboxfalse
}
\def\psdraftbox{
	\@draftboxtrue
}
\@draftboxtrue

\newif\if@prologfile
\newif\if@postlogfile
\def\pssilent{
	\@noisyfalse
}
\def\psnoisy{
	\@noisytrue
}
\psnoisy
%%% These are for the option list.
%%% A specification of the form a = b maps to calling \@p@@sa{b}
\newif\if@bbllx
\newif\if@bblly
\newif\if@bburx
\newif\if@bbury
\newif\if@height
\newif\if@width
\newif\if@rheight
\newif\if@rwidth
\newif\if@angle
\newif\if@clip
\newif\if@verbose
\def\@p@@sclip#1{\@cliptrue}


\newif\if@decmpr

%%% GDH 7/26/87 -- changed so that it first looks in the local directory,
%%% then in a specified global directory for the ps file.
%%% RPR 6/25/91 -- changed so that it defaults to user-supplied name if
%%% boundingbox info is specified, assuming graphic will be created by
%%% print time.
%%% TJD 10/19/91 -- added bbfile vs. file distinction, and @decmpr flag

\def\@p@@sfigure#1{\def\@p@sfile{null}\def\@p@sbbfile{null}
	        \openin1=#1.bb
		\ifeof1\closein1
	        	\openin1=\figurepath#1.bb
			\ifeof1\closein1
			        \openin1=#1
				\ifeof1\closein1%
				       \openin1=\figurepath#1
					\ifeof1
					   \ps@typeout{Error, File #1 not found}
						\if@bbllx\if@bblly
				   		\if@bburx\if@bbury
			      				\def\@p@sfile{#1}%
			      				\def\@p@sbbfile{#1}%
							\@decmprfalse
				  	   	\fi\fi\fi\fi
					\else\closein1
				    		\def\@p@sfile{\figurepath#1}%
				    		\def\@p@sbbfile{\figurepath#1}%
						\@decmprfalse
	                       		\fi%
			 	\else\closein1%
					\def\@p@sfile{#1}
					\def\@p@sbbfile{#1}
					\@decmprfalse
			 	\fi
			\else
				\def\@p@sfile{\figurepath#1}
				\def\@p@sbbfile{\figurepath#1.bb}
				\@decmprtrue
			\fi
		\else
			\def\@p@sfile{#1}
			\def\@p@sbbfile{#1.bb}
			\@decmprtrue
		\fi}

\def\@p@@sfile#1{\@p@@sfigure{#1}}

\def\@p@@sbbllx#1{
		%\ps@typeout{bbllx is #1}
		\@bbllxtrue
		\dimen100=#1
		\edef\@p@sbbllx{\number\dimen100}
}
\def\@p@@sbblly#1{
		%\ps@typeout{bblly is #1}
		\@bbllytrue
		\dimen100=#1
		\edef\@p@sbblly{\number\dimen100}
}
\def\@p@@sbburx#1{
		%\ps@typeout{bburx is #1}
		\@bburxtrue
		\dimen100=#1
		\edef\@p@sbburx{\number\dimen100}
}
\def\@p@@sbbury#1{
		%\ps@typeout{bbury is #1}
		\@bburytrue
		\dimen100=#1
		\edef\@p@sbbury{\number\dimen100}
}
\def\@p@@sheight#1{
		\@heighttrue
		\dimen100=#1
   		\edef\@p@sheight{\number\dimen100}
		%\ps@typeout{Height is \@p@sheight}
}
\def\@p@@swidth#1{
		%\ps@typeout{Width is #1}
		\@widthtrue
		\dimen100=#1
		\edef\@p@swidth{\number\dimen100}
}
\def\@p@@srheight#1{
		%\ps@typeout{Reserved height is #1}
		\@rheighttrue
		\dimen100=#1
		\edef\@p@srheight{\number\dimen100}
}
\def\@p@@srwidth#1{
		%\ps@typeout{Reserved width is #1}
		\@rwidthtrue
		\dimen100=#1
		\edef\@p@srwidth{\number\dimen100}
}
\def\@p@@sangle#1{
		%\ps@typeout{Rotation is #1}
		\@angletrue
%		\dimen100=#1
		\edef\@p@sangle{#1} %\number\dimen100}
}
\def\@p@@ssilent#1{ 
		\@verbosefalse
}
\def\@p@@sprolog#1{\@prologfiletrue\def\@prologfileval{#1}}
\def\@p@@spostlog#1{\@postlogfiletrue\def\@postlogfileval{#1}}
\def\@cs@name#1{\csname #1\endcsname}
\def\@setparms#1=#2,{\@cs@name{@p@@s#1}{#2}}
%
% initialize the defaults (size the size of the figure)
%
\def\ps@init@parms{
		\@bbllxfalse \@bbllyfalse
		\@bburxfalse \@bburyfalse
		\@heightfalse \@widthfalse
		\@rheightfalse \@rwidthfalse
		\def\@p@sbbllx{}\def\@p@sbblly{}
		\def\@p@sbburx{}\def\@p@sbbury{}
		\def\@p@sheight{}\def\@p@swidth{}
		\def\@p@srheight{}\def\@p@srwidth{}
		\def\@p@sangle{0}
		\def\@p@sfile{} \def\@p@sbbfile{}
		\def\@p@scost{10}
		\def\@sc{}
		\@prologfilefalse
		\@postlogfilefalse
		\@clipfalse
		\if@noisy
			\@verbosetrue
		\else
			\@verbosefalse
		\fi
}
%
% Go through the options setting things up.
%
\def\parse@ps@parms#1{
	 	\@psdo\@psfiga:=#1\do
		   {\expandafter\@setparms\@psfiga,}}
%
% Compute bb height and width
%
\newif\ifno@bb
\def\bb@missing{
	\if@verbose{
		\ps@typeout{psfig: searching \@p@sbbfile \space  for bounding box}
	}\fi
	\no@bbtrue
	\epsf@getbb{\@p@sbbfile}
        \ifno@bb \else \bb@cull\epsf@llx\epsf@lly\epsf@urx\epsf@ury\fi
}	
\def\bb@cull#1#2#3#4{
	\dimen100=#1 bp\edef\@p@sbbllx{\number\dimen100}
	\dimen100=#2 bp\edef\@p@sbblly{\number\dimen100}
	\dimen100=#3 bp\edef\@p@sbburx{\number\dimen100}
	\dimen100=#4 bp\edef\@p@sbbury{\number\dimen100}
	\no@bbfalse
}
% rotate point (#1,#2) about (0,0).
% The sine and cosine of the angle are already stored in \sine and
% \cosine.  The result is placed in (\p@intvaluex, \p@intvaluey).
\newdimen\p@intvaluex
\newdimen\p@intvaluey
\def\rotate@#1#2{{\dimen0=#1 sp\dimen1=#2 sp
%            	calculate x' = x \cos\theta - y \sin\theta
		  \global\p@intvaluex=\cosine\dimen0
		  \dimen3=\sine\dimen1
		  \global\advance\p@intvaluex by -\dimen3
% 		calculate y' = x \sin\theta + y \cos\theta
		  \global\p@intvaluey=\sine\dimen0
		  \dimen3=\cosine\dimen1
		  \global\advance\p@intvaluey by \dimen3
		  }}
\def\compute@bb{
		\no@bbfalse
		\if@bbllx \else \no@bbtrue \fi
		\if@bblly \else \no@bbtrue \fi
		\if@bburx \else \no@bbtrue \fi
		\if@bbury \else \no@bbtrue \fi
		\ifno@bb \bb@missing \fi
		\ifno@bb \ps@typeout{FATAL ERROR: no bb supplied or found}
			\no-bb-error
		\fi
		%
%\ps@typeout{BB: \@p@sbbllx, \@p@sbblly, \@p@sbburx, \@p@sbbury} 
%
% store height/width of original (unrotated) bounding box
		\count203=\@p@sbburx
		\count204=\@p@sbbury
		\advance\count203 by -\@p@sbbllx
		\advance\count204 by -\@p@sbblly
		\edef\ps@bbw{\number\count203}
		\edef\ps@bbh{\number\count204}
		%\ps@typeout{ psbbh = \ps@bbh, psbbw = \ps@bbw }
		\if@angle 
			\Sine{\@p@sangle}\Cosine{\@p@sangle}
	        	{\dimen100=\maxdimen\xdef\r@p@sbbllx{\number\dimen100}
					    \xdef\r@p@sbblly{\number\dimen100}
			                    \xdef\r@p@sbburx{-\number\dimen100}
					    \xdef\r@p@sbbury{-\number\dimen100}}
%
% Need to rotate all four points and take the X-Y extremes of the new
% points as the new bounding box.
                        \def\minmaxtest{
			   \ifnum\number\p@intvaluex<\r@p@sbbllx
			      \xdef\r@p@sbbllx{\number\p@intvaluex}\fi
			   \ifnum\number\p@intvaluex>\r@p@sbburx
			      \xdef\r@p@sbburx{\number\p@intvaluex}\fi
			   \ifnum\number\p@intvaluey<\r@p@sbblly
			      \xdef\r@p@sbblly{\number\p@intvaluey}\fi
			   \ifnum\number\p@intvaluey>\r@p@sbbury
			      \xdef\r@p@sbbury{\number\p@intvaluey}\fi
			   }
%			lower left
			\rotate@{\@p@sbbllx}{\@p@sbblly}
			\minmaxtest
%			upper left
			\rotate@{\@p@sbbllx}{\@p@sbbury}
			\minmaxtest
%			lower right
			\rotate@{\@p@sbburx}{\@p@sbblly}
			\minmaxtest
%			upper right
			\rotate@{\@p@sbburx}{\@p@sbbury}
			\minmaxtest
			\edef\@p@sbbllx{\r@p@sbbllx}\edef\@p@sbblly{\r@p@sbblly}
			\edef\@p@sbburx{\r@p@sbburx}\edef\@p@sbbury{\r@p@sbbury}
%\ps@typeout{rotated BB: \r@p@sbbllx, \r@p@sbblly, \r@p@sbburx, \r@p@sbbury}
		\fi
		\count203=\@p@sbburx
		\count204=\@p@sbbury
		\advance\count203 by -\@p@sbbllx
		\advance\count204 by -\@p@sbblly
		\edef\@bbw{\number\count203}
		\edef\@bbh{\number\count204}
		%\ps@typeout{ bbh = \@bbh, bbw = \@bbw }
}
%
% \in@hundreds performs #1 * (#2 / #3) correct to the hundreds,
%	then leaves the result in @result
%
\def\in@hundreds#1#2#3{\count240=#2 \count241=#3
		     \count100=\count240	% 100 is first digit #2/#3
		     \divide\count100 by \count241
		     \count101=\count100
		     \multiply\count101 by \count241
		     \advance\count240 by -\count101
		     \multiply\count240 by 10
		     \count101=\count240	%101 is second digit of #2/#3
		     \divide\count101 by \count241
		     \count102=\count101
		     \multiply\count102 by \count241
		     \advance\count240 by -\count102
		     \multiply\count240 by 10
		     \count102=\count240	% 102 is the third digit
		     \divide\count102 by \count241
		     \count200=#1\count205=0
		     \count201=\count200
			\multiply\count201 by \count100
		 	\advance\count205 by \count201
		     \count201=\count200
			\divide\count201 by 10
			\multiply\count201 by \count101
			\advance\count205 by \count201
			%
		     \count201=\count200
			\divide\count201 by 100
			\multiply\count201 by \count102
			\advance\count205 by \count201
			%
		     \edef\@result{\number\count205}
}
\def\compute@wfromh{
		% computing : width = height * (bbw / bbh)
		\in@hundreds{\@p@sheight}{\@bbw}{\@bbh}
		%\ps@typeout{ \@p@sheight * \@bbw / \@bbh, = \@result }
		\edef\@p@swidth{\@result}
		%\ps@typeout{w from h: width is \@p@swidth}
}
\def\compute@hfromw{
		% computing : height = width * (bbh / bbw)
	        \in@hundreds{\@p@swidth}{\@bbh}{\@bbw}
		%\ps@typeout{ \@p@swidth * \@bbh / \@bbw = \@result }
		\edef\@p@sheight{\@result}
		%\ps@typeout{h from w : height is \@p@sheight}
}
\def\compute@handw{
		\if@height 
			\if@width
			\else
				\compute@wfromh
			\fi
		\else 
			\if@width
				\compute@hfromw
			\else
				\edef\@p@sheight{\@bbh}
				\edef\@p@swidth{\@bbw}
			\fi
		\fi
}
\def\compute@resv{
		\if@rheight \else \edef\@p@srheight{\@p@sheight} \fi
		\if@rwidth \else \edef\@p@srwidth{\@p@swidth} \fi
		%\ps@typeout{rheight = \@p@srheight, rwidth = \@p@srwidth}
}
%		
% Compute any missing values
\def\compute@sizes{
	\compute@bb
	\if@scalefirst\if@angle
% at this point the bounding box has been adjsuted correctly for
% rotation.  PSFIG does all of its scaling using \@bbh and \@bbw.  If
% a width= or height= was specified along with \psscalefirst, then the
% width=/height= value needs to be adjusted to match the new (rotated)
% bounding box size (specifed in \@bbw and \@bbh).
%    \ps@bbw       width=
%    -------  =  ---------- 
%    \@bbw       new width=
% so `new width=' = (width= * \@bbw) / \ps@bbw; where \ps@bbw is the
% width of the original (unrotated) bounding box.
	\if@width
	   \in@hundreds{\@p@swidth}{\@bbw}{\ps@bbw}
	   \edef\@p@swidth{\@result}
	\fi
	\if@height
	   \in@hundreds{\@p@sheight}{\@bbh}{\ps@bbh}
	   \edef\@p@sheight{\@result}
	\fi
	\fi\fi
	\compute@handw
	\compute@resv}

%
% \psfig
% usage : \psfig{file=, height=, width=, bbllx=, bblly=, bburx=, bbury=,
%			rheight=, rwidth=, clip=}
%
% "clip=" is a switch and takes no value, but the `=' must be present.
\def\psfig#1{\vbox {
	% do a zero width hard space so that a single
	% \psfig in a centering enviornment will behave nicely
	%{\setbox0=\hbox{\ }\ \hskip-\wd0}
	%
	\ps@init@parms
	\parse@ps@parms{#1}
	\compute@sizes
	%
	\ifnum\@p@scost<\@psdraft{
		%
		\special{ps::[begin] 	\@p@swidth \space \@p@sheight \space
				\@p@sbbllx \space \@p@sbblly \space
				\@p@sbburx \space \@p@sbbury \space
				startTexFig \space }
		\if@angle
			\special {ps:: \@p@sangle \space rotate \space} 
		\fi
		\if@clip{
			\if@verbose{
				\ps@typeout{(clip)}
			}\fi
			\special{ps:: doclip \space }
		}\fi
		\if@prologfile
		    \special{ps: plotfile \@prologfileval \space } \fi
		\if@decmpr{
			\if@verbose{
				\ps@typeout{psfig: including \@p@sfile.Z \space }
			}\fi
			\special{ps: plotfile "`zcat \@p@sfile.Z" \space }
		}\else{
			\if@verbose{
				\ps@typeout{psfig: including \@p@sfile \space }
			}\fi
			\special{ps: plotfile \@p@sfile \space }
		}\fi
		\if@postlogfile
		    \special{ps: plotfile \@postlogfileval \space } \fi
		\special{ps::[end] endTexFig \space }
		% Create the vbox to reserve the space for the figure.
		\vbox to \@p@srheight sp{
		% 1/92 TJD Changed from "true sp" to "sp" for magnification.
			\hbox to \@p@srwidth sp{
				\hss
			}
		\vss
		}
	}\else{
		% draft figure, just reserve the space and print the
		% path name.
		\if@draftbox{		
			% Verbose draft: print file name in box
			\hbox{\frame{\vbox to \@p@srheight sp{
			\vss
			\hbox to \@p@srwidth sp{ \hss \@p@sfile \hss }
			\vss
			}}}
		}\else{
			% Non-verbose draft
			\vbox to \@p@srheight sp{
			\vss
			\hbox to \@p@srwidth sp{\hss}
			\vss
			}
		}\fi	



	}\fi
}}
\psfigRestoreAt
\let\@=\LaTeXAtSign




\textwidth 17truecm
\textheight 24.2truecm
\hoffset -1.5truecm
\voffset -1.5truecm
 
\title{A New Entropy Measure Based on the Wavelet Transform and Noise Modeling}
\author{ J.-L. Starck \\
CEA/DSM/DAPNIA, CE-SACLAY, F-91191 Gif sur Yvette Cedex \\ [12pt]
F. Murtagh \\
Faculty of Informatics, University of Ulster, Magee College, \\
Londonderry BT48 7JL, Northern Ireland \\ [12pt]
R. Gastaud \\
CEA/DSM/DAPNIA, CE-SACLAY, F-91191 Gif sur Yvette Cedex}


\begin{document}
\maketitle

\abstract{
We present in this paper a new way to measure the information
in a signal, based on noise modeling. We show that the use of such 
an entropy-related measure leads to good results for signal restoration.}
  
\section{Introduction}
The term ``entropy'' term is due to Clausius (1865), and the concept of 
entropy was introduced by Boltzmann into statistical mechanics,
in order to measure the number of microscopic ways that a given macroscopic
state can be realized. Shannon \cite{shannon48} 
founded the mathematical theory of
communication when he suggested that the information gained in a 
measurement depends on the number of possible outcomes out of 
which one is realized. Shannon also suggested that 
the entropy can be used for maximization of the bits transferred under
a quality constraint. Jaynes \cite{jaynes57} proposed to use 
the entropy measure
for radio interometric image deconvolution,  in
order to select in a set of possible solutions that which contains the
minimum of information, or following his entropy definition, that 
which has a maximum entropy. In principle, the solution verifying such 
a condition should be the most reliable. A lot of work has been carried out
in the last 30 years on the use of entropy for the general problem
of data filtering and deconvolution 
\cite{ables74,bontekoe94,burg67,frieden75,gull91,narrayan86,pantin96,skilling84_1,
weir92}. 

Traditionally information and entropy are determined from events and the
probability of their occurrence.  Signal and noise are basic building-blocks 
of signal and data analysis in the physical sciences.  Instead of the 
probability of an event, in this work
 we are led to consider the probabilities of our
data being either signal or noise. 

Observed data $Y$ in the physical sciences 
are generally corrupted by noise, which is often additive and which 
follows in many cases a Gaussian distribution, a Poisson distribution, or
a combination of both.
 Using Bayes' theorem to evaluate the probability of the 
realization of the original signal $X$,
knowing the data $Y$, we have

\begin{eqnarray}
 \mathrm{Prob}(X|Y) = \frac{\mathrm{Prob}(Y|X).\mathrm{Prob}(X)}{\mathrm{Prob}(Y)}
\label{eqn_bayes}
\end{eqnarray}
$\mathrm{Prob}(Y|X)$ is the conditional probability of getting the data 
$Y$ given an original signal $X$, i.e.\ it represents the distribution 
of the noise. It is given, in the case of uncorrelated Gaussian 
noise with variance $\sigma^2$, by:
\begin{eqnarray}
 \mathrm{Prob}(Y|X) = \mathrm{exp}(-\sum_{pixels} \frac{ (Y-X)^2}{2{\sigma}^2} )
\label{eqn_proba}
\end{eqnarray}
The denominator in  equation \ref{eqn_bayes} is independent of $X$ and
 is considered as a constant (stationary noise). 
$\mathrm{Prob(X)}$ is the a priori distribution 
of the solution $X$. In the absence of any information on the solution 
$X$ except its positivity, a possible course of action 
is to derive the probability
of $X$ from its entropy, which is defined from information theory.

The main idea of information theory \cite{shannon48} is to establish
a relation between the received information and the probability of
the observed event \cite{bijaoui84}. If we note ${\cal I}(E)$ the information
related to the event $E$, and $p$ the probability of this
event happening, then we consider that
\begin{eqnarray}
%\cal{I}(E) =    f(p)       
{\cal I}(E) = f(p)
\end{eqnarray}

Then we assume the two following principles:
\begin{itemize}
\item The information is a decreasing function of the probability. This 
implies that the more information we have, the  less will be the probability 
associated with one event.
\item Additivity of the information. If we have two independent events 
$E_1$ and $E_2$, the information ${\cal I}(E)$ associated with the happening
of both is equal
to the addition of the information of each of them.
\begin{eqnarray}
{\cal I}(E) = {\cal I}(E_1) + {\cal I}(E_2)
\end{eqnarray}
\end{itemize}

Since $E_1$ (of probability $p_1$) and $E_2$ (of probability $p_2$) are 
independent, then the probability of both happening is equal to the
product of $p_1$ and $p_2$.  Hence
\begin{eqnarray}
f(p_1 p_2) = f(p_1) + f(p_2) 
\end{eqnarray}

Then we can say that the information measure is
\begin{eqnarray}
{\cal I}(E) = k \ln(p)
\end{eqnarray}
where k is a constant. Information must be positive, and $k$
is generally fixed at $-1$.

Another interesting measure is the mean information which is denoted
\begin{eqnarray}
H = - \sum_i p_i \ln(p_i)
\end{eqnarray}
This quantity is called the entropy of the system and was established by 
Shannon in 1948 \cite{shannon48}.

This measure has several properties:
\begin{itemize}
\item It is maximal when all events have the same probability 
$p_i = 1/ N_e$ ($N_e$ being the number of events), and is equal to 
$\ln(N_e)$. It is in this 
configuration that the system is the most undefined.
\item It is minimal when one event is sure. In this case, the system is 
perfectly known, and no information can be added.
\item The entropy is a positive, continuous, and symmetric function.
%\item The mean information, obtained in two steps, can be added.
\end{itemize}

Then if we know the entropy $H$ of the solution (the next section 
describes different ways to calculate it), 
we derive its probability by
\begin{eqnarray}
\mathrm{Prob}(X) = \mathrm{exp}(- \alpha H(X))
\label{info_prop}
\end{eqnarray}

Given the data, the most probable image is obtained by maximizing
$\mathrm{Prob}(X|Y)$. Taking the logarithm of equation \ref{eqn_bayes}, we 
thus need to maximize
\begin{eqnarray}
 \ln (\mathrm{Prob}(X|Y))  = - \alpha  H(X) + \ln(\mathrm{Prob}(Y|X)) - 
\ln(\mathrm{Prob}(Y))
\end{eqnarray}
The last term is a constant and can be omitted.
Then, in the case of Gaussian noise, the solution is found by minimizing 
\begin{eqnarray}
J(X) = \sum_{pixels} \frac{{(Y-X)}^{2}}{2 {\sigma}^{2}} + {\alpha} H(X)
= \frac{{\chi}^2}{2} + {\alpha} H(X)
\label{eqn_j1}
\end{eqnarray}
which is a linear combination of two terms: the entropy of the signal,
and a quantity corresponding to ${\chi}^2$ in statistics measuring the
discrepancy between the data and the predictions of the model.
$\alpha$ is a parameter that can be viewed alternatively as 
a Lagrangian parameter or a value fixing the relative weight between 
the goodness-of-fit and the entropy H. 

For the deconvolution problem, the object-data relation is given by the
convolution
\begin{eqnarray}
Y = P * X
\end{eqnarray}
where $P$ is the point spread function, and the solution is found (in the case
of Gaussian noise) by minimizing
\begin{eqnarray}
J(X) = \sum_{pixels} \frac{{(Y-P*X)}^{2}}{2 {\sigma}^{2}} + {\alpha} H(X)
\end{eqnarray}

The way the entropy is defined is fundamental, because from its definition
will depend the solution. The next section discusses the different approaches 
which have been proposed in the past.

\section{The concept of entropy}
\label{sect_entr}

We wish to estimate an unknown probability density $p(x)$ of the data.
A direct approach would be to build up the histogram of values $X(i)$, using
a suitable interval $\Delta x$, counting up how many times $m_k$ each interval
$(x_k, x_k + \Delta x)$ occurs among the N occurrences. Then the probability
that a data value belongs to an interval $k$ is $p_k = \frac{m_k}{N}$, and
each data value has a probability $p_k$. The entropy is defined by
\begin{eqnarray}
H_s(X) = - \sum_{k=1}^{m} p_k \ln(p_k) 
\end{eqnarray}
where $m$ is the number of intervals.
The entropy is minimum and equal to zero when the signal is flat, and
increases when we have some fluctuations. Using this entropy in 
equation~\ref{eqn_j1} for restoration leads to a minimum entropy restoration
 method.

The trouble with this approach is that, because the number of occurrences is
finite, the estimate $p_k$ will be in error by an amount proportional
to $m_k^{-\frac{1}{2}}$~\cite{frieden}. The error becomes significant when
$m_k$ is small. Furthermore this kind of entropy definition is not
easy to use for signal restoration, because the gradient of 
equation~\ref{eqn_j1} is not easy to compute. For these reasons, other
entropy functions are generally used. The main ones are:
\begin{itemize}
\item Burg \cite{burg67}:
\begin{eqnarray}
H_b(X) = -\sum_{pixels} \ln(X) 
\end{eqnarray}
\item Frieden \cite{frieden75}:
\begin{eqnarray}
H_f(X) = -\sum_{pixels}X \ln(X)
\end{eqnarray}
\item Gull and Skilling \cite{gull91}:
\begin{eqnarray}
H_g(X) = \sum_{pixels} X - M - X \ln(X|M)
\end{eqnarray}
\end{itemize}
Each of these entropies can be used, and they correspond to different
probability distributions that one can associate with 
an image \cite{narrayan86}.
(See \cite{frieden75,skilling84_1,skilling89} for descriptions).
The last definition of the entropy has the advantage of having a zero
 maximum when $X$ equals the model $M$, usually taken as a flat image. 
All of these entropy measures are negative, and maximum when the image is flat.
They are negative because an offset term is omitted which has no importance
for the minimization of the functional. The fact that we consider that
a signal has maximum information value when it is flat is evidently
a curious way to measure information. The probability of $X$ 
must be defined by $\mathrm{Prob}(X) = \mathrm{exp}(\alpha H(X))$. The sign has 
been inverted
(see equation~\ref{info_prop}), which is natural if we want the best
solution to be  the smoothest. These three entropies lead to the 
maximum entropy restoration method, for which the 
solution is found by minimizing 
(for Gaussian noise)
\begin{eqnarray}
J(X) = \sum_{pixels} \frac{{(Y-X)}^{2}}{2 {\sigma}^{2}} - {\alpha} H(X)
\label{eqn_j3}
\end{eqnarray}

In 1986, Narayan and Nityanda \cite{narrayan86} 
compared several entropy functions,
 and finally 
concluded by saying that all were comparable if they have good
properties, i.e.\ they enforce positivity, and they have a negative 
second derivative which discourages ripple. They showed also that 
results varied strongly with the background level, and
that these entropy functions produced poor results
for negative structures, i.e.\ structures under the background level
(absorption area in an image, absorption band in a spectrum, etc.), and 
compact structures in the signal.
The Gull and Skilling entropy gives rise to  
the difficulty of estimating a model.
Furthermore it has been shown \cite{bontekoe94} 
that the solution was dependent on this choice.

Many studies \cite{weir92,bontekoe94,pantin96} 
have been carried out in order to improve the functional to be minimized.
But the question which should be raised is: what is a good entropy for 
signal restoration?

Trying to answer this corresponds to asking what is the information
in the signal. The entropy should verify the following criteria:
\begin{enumerate}
\item The information in a flat signal is zero.
\item The amount of information in a signal is independent of the background.
\item The amount of information is dependent on the noise. A given 
signal $Y$ ($Y = X + Noise$) doesn't furnish the  same information if 
the noise is high or small.
\item The entropy must work in the same way for a pixel which
has a value $B + \epsilon$ ($B$ being the background), and
for a pixel which has a value $B - \epsilon$.
\item The amount of information is dependent on the correlation in the signal.
If a signal $S$  presents large features above the noise, it contains
a lot of information. By generating a new set of  data from $S$, by 
randomly taking the pixel values in $S$, the large features will
evidently disappear, and this new signal will contain less information.
But the pixel values will be the same as in $S$.
\end{enumerate}

\begin{figure}[htb]
\centerline{
\hbox{
\psfig{figure=lenna256.ps,bbllx=1.8cm,bblly=12.9cm,bburx=14.5cm,bbury=25.5cm,width=8cm,height=8cm,clip=}
\psfig{figure=scrambled_lenna.ps,bbllx=1.8cm,bblly=12.9cm,bburx=14.5cm,bbury=25.5cm,width=8cm,height=8cm,clip=}
}}
\caption{Lena image (left) and the same data distributed differently (right). 
These
two images have the same entropy, using any of the standard entropy methods.}
\label{fig_lenna}
\end{figure}

Figure~\ref{fig_lenna} shows the Lena image (left), and an image obtained 
by distributing randomly the Lena image pixel values. For someone, who is
not involved in image processing, the second image contains less information
than the first one. For someone working on image transmission, it is clear
that the second image will require more bits for a lossless transmission,
and from this point of view, he will consider that the second one contains
more information. The standard entropy methods produce exactly 
the same value for both images and, for such methods, both images contain
the same amount of information. For data restoration, all fluctuations
due to noise are not of interest, and do not contain relevant 
information.  From this physical point of view, that is the reason why 
the standard 
definition of entropy seems badly adapted to information measurement 
in signal restoration.

\section{Entropy from noise modeling}
 
In the case of signal restoration, the noise is the main problem. This 
means that we should not consider the probability of appearance of 
a pixel value in an image, but rather its probability of being due to the
signal (or to the noise). 
If we consider a variable $x$ which follows a probability distribution 
$p(x)$, we
can define the information in $x$ by $- \ln(p(x))$, and a signal $S$ can
be considered as a set of individual variables $x_k$ (pixels), each of which 
follows the same probability distribution. 
Then the information contained in the data 
can be measured by $- \sum_{pixel} \ln(p(x))$. If $x$ follows a Gaussian 
distribution with zero mean, we have
\begin{eqnarray}
H(X) = \sum_{pixel} \frac{x^2}{2 \sigma^2}
\end{eqnarray}
The energy gives a good measurement of information. But many of the required
criteria are not fulfilled by using such an entropy (correlation between 
pixels, background-independent, etc.). It seems difficult to derive
a good probability distribution from the pixel values which fulfill the 
entropy requirements.

This is not so for transformed data, especially when using  
the wavelet transform. 
This has  already been done, in fact,
for finding threshold levels in filtering 
methods by means of wavelet coefficient thresholding \cite{starck95}. Thus we 
must introduce the concept of multiresolution into our entropy.
We will now consider that the information contained in some dataset
is the sum of the information at different resolution levels $j$.
Choosing the ``\`a trous'' wavelet transform (see \cite{starck95} 
for a description of this wavelet transform algorithm), a signal $S$ can 
be represented by:
\begin{eqnarray}
S(k) = \sum_{j=1}^{l} w_j(k) + c_l(k)
\end{eqnarray}
where $k$ is the pixel index, $w_j$ are the wavelet coefficients of $S$, $j$ the 
resolution
level, and $c_l$ the smoothed version of $S$. Due to the properties of
the wavelet transform, the set $w_j(x)$ for all $x$ has a zero mean. From 
noise modeling, we can derive the probability distribution in the 
wavelet space of a wavelet coefficient, assuming it is due to the noise. 
The entropy becomes
\begin{eqnarray}
H(X) = - \sum_{j=1}^{l}  \sum_{k=1}^{N} \ln ( p(w_j(k)))
\end{eqnarray}
For Gaussian noise, we get
\begin{eqnarray}
H(X) =  \sum_{j=1}^{l}  \sum_{k=1}^{N} \frac{w_j(k)^2}{2 \sigma_j^2}
\end{eqnarray}
where $\sigma_j$ is the noise at scale $j$. We see that 
the information is proportional
to the energy of the wavelet coefficients.
The higher a wavelet coefficient, then the lower will be the  probability, and the 
higher will
be the information furnished by this wavelet coefficient. We can see
easily that this entropy fulfills all the requirements of 
section~\ref{sect_entr}.
If we consider two signals $S_1$, $S_2$, derived from a third one $S_0$ by 
adding noise:
\begin{eqnarray}
S_1 & = & S_0 + N_1(\sigma_1) \\ \nonumber
S_2 & = & S_0 + N_2(\sigma_2)
\end{eqnarray}
then we have:
\begin{eqnarray}
\mbox{if } \sigma_1 < \sigma_2 \mbox{ then } H(S_1) > H(S_2)
\end{eqnarray}
and a flat image has zero entropy. 

Our entropy definition is completely dependent on the noise modeling.
If we consider a signal $S$, and we assume that the noise is Gaussian, with 
a  standard deviation equal to $\sigma$, we won't measure the same
information compared to  the case when we consider that the noise has 
another standard deviation
value, or if the noise follows another distribution.

\begin{figure}[htb]
\centerline{
\hbox{
\psfig{figure=fig_multi_memscale.ps,bbllx=2.5cm,bblly=13cm,bburx=19.5cm,bbury=25.5cm,width=14cm,height=7cm,clip=}
}}
\caption{Multiscale entropy of the Lena image (continuous curve), 
and multiscale
entropy of the scrambled image (dashed curve).}
\label{fig_multi_memscale}
\end{figure}
Figure~\ref{fig_multi_memscale} shows the information measure at each
scale for both the Lena image and its scrambled version. The global 
information is the addition of the information at each scale. We see
that for the scrambled image (dashed curve), the 
information-versus-scale 
curve is flat, while for the unscrambled Lena image,
it increases with the scale.  

\section{Signal information and noise information}
\subsection{Definition}
In the previous section, we have seen how it was possible to measure
the information related to a wavelet coefficient. Since the data
is composed of an original signal and noise, our information measure
is corrupted by noise. Trying to decompose our information measure
into two components, one ($H_S$) corresponding to the non-corrupted part, and
another  ($H_N$) to the corrupted part, we have
\begin{eqnarray}
H(X) = H_S(X) + H_N(X)
\end{eqnarray}
We will define in the following $H_S$ as the signal information, and $H_N$
as the noise information. It must be clear that noise does not 
contain any information, and what we call noise information is a quantity
which is measured as information by the multiscale entropy, and which is 
probably not informative to us.

As described in the previous section, the information $h$ relative to 
a wavelet coefficient $w_j$ is $- \ln(p(w_j))$. If the wavelet coefficient
is small, its value can be due to the noise, 
and  $h$ should be assigned to $H_N$.
If the wavelet coefficient is high, compared to the noise standard
deviation, $h$ cannot be due to the noise, and $h$ should be assigned to $H_S$.
$h$ can be distributed as $H_N$ or $H_S$ based on  the probability $p_n(w_j)$
that the wavelet coefficient is due to noise, or the probability 
$p_s(w_j)$ that it is due to 
signal.  We have $p_s(w_j) = 1 - p_n(w_j)$. We consider 
that  $h_n(w_j) = - p_n(w_j) \ln(p(w_j))$ is
the noise information, and 
$h_s(w_j) = - p_s(w_j) \ln(p(w_j))$ is the signal information. 
Hence  signal information and  noise information are defined by
\begin{eqnarray}
H_s(X) & = & \sum_{j=1}^{l}  \sum_{k=1}^{N} h_s(w_j(k)) = - \sum_{j=1}^{l}  
\sum_{k=1}^{N} p_s(w_j(k)) \ln ( p(w_j(k))) \\ \nonumber
H_n(X) & = & \sum_{j=1}^{l}  \sum_{k=1}^{N} h_n(w_j(k)) = - \sum_{j=1}^{l}  
\sum_{k=1}^{N} p_n(w_j(k)) \ln ( p(w_j(k)))
\end{eqnarray}
For the Gaussian noise case, we estimate $p_n(w_j)$ that a wavelet 
coefficient is due to the noise by
\begin{eqnarray}
p_n(w_j) = \mathrm{Prob}(W > \mid w_j \mid)  & =  & \frac{2}{\sqrt{2 \pi} 
\sigma_j} \int_{\mid w_j \mid}^{+\infty} \exp(-W^2/2\sigma^2_j) dW\\ \nonumber
 & = & \mbox{erfc}(\frac{\mid w_j \mid }{\sqrt{2}\sigma_j})
\end{eqnarray}
and
\begin{eqnarray}
H_s(X) &= & \sum_{j=1}^{l}  \sum_{k=1}^{N}  \frac{w_j^2}{2\sigma_j^2} 
\mbox{erf}(\frac{\mid w_j \mid }{\sqrt{2}\sigma_j}) \\ \nonumber
H_n(X) &= & \sum_{j=1}^{l}  \sum_{k=1}^{N}  \frac{w_j^2}{2\sigma_j^2} 
\mbox{erfc}(\frac{\mid w_j \mid }{\sqrt{2}\sigma_j}) 
\label{eq_entrop_result_1} 
\end{eqnarray}

Note that $H_s(X) + H_n(X)$ is always equal to $H(X)$. 
For Gaussian noise, the functional to minimize becomes
\begin{eqnarray}
J(X) = \sum_{pixels} \frac{{(Y-X)}^{2}}{2 {\sigma}^{2}} + {\alpha} (H_s(X)+H_n(X))
\end{eqnarray}
If we want to preserve features with high signal-to-noise ratio from the
regularization, we just omit $H_s(X)$ and we get
\begin{eqnarray}
J(X) = \sum_{pixels} \frac{{(Y-X)}^{2}}{2 {\sigma}^{2}} + {\alpha} H_n(X)
\end{eqnarray}
We seek a solution which minimizes the amount of information which could
be due to the noise.

By this measure, information relative to high wavelet coefficients is  
completely assigned to the signal. This allows us also to exclude
wavelet coefficients with high signal-to-noise ratio (SNR)
from the regularization.
It leads to perfect fit of the solution with the data at scales and
 positions with high SNR. If we want to consider the information due
to noise, even for significant wavelet coefficients, the noise information
relative to a wavelet coefficient is 
\begin{eqnarray}
h_n(w_j) =  \int_{0}^{\mid w_j \mid } p_n(u|w_j) (\frac{\partial H(x)}{\partial 
x})_{x=u} du
\end{eqnarray}
which gives for Gaussian noise
\begin{eqnarray}
h_n(w_j) =  \frac{1}{\sigma_j^2} \int_{0}^{\mid w_j \mid} u \mbox{ 
erfc}(\frac{\mid w_j \mid -u}{\sqrt{2} \sigma_j}) du
\label{eqn_hn2}
\end{eqnarray}
and the noise and signal information in a signal are
\begin{eqnarray}
H_s(X) & = & \sum_{j=1}^{l}  \sum_{k=1}^{N}  \frac{1}{\sigma_j^2} \int_{0}^{\mid 
w_j \mid} u \mbox{ erf}(\frac{\mid w_j \mid -u}{\sqrt{2} \sigma_j}) du \\ 
\nonumber 
H_n(X) & = & \sum_{j=1}^{l}  \sum_{k=1}^{N}   \frac{1}{\sigma_j^2} \int_{0}^{\mid 
w_j \mid} u \mbox{ erfc}(\frac{\mid w_j \mid -u}{\sqrt{2} \sigma_j}) du  
\label{eq_entrop_result_2}
\end{eqnarray}

Equations 27 and 32 lead to two
different ways to regularize a signal. The first requires that we use
all the information which is furnished in high wavelet coefficients, and 
leads to an exact preservation of the flux in a structure. If the signal
presents high discontinuities, artifacts can appear in the solution 
due to the fact that the wavelet coefficients located at the discontinuities
are not noisy, but have been modified like noise. The second equation 
doesn't have this drawback, but a part of the flux of a structure
(compatible with noise amplitude) can be lost in the restoration process. 
It is however not as effective as in the standard maximum entropy methods.

\subsection{A new approach for signal restoration}
The new  definition of the information contained
in noisy data can easily lead to a new approach for restoration of images.

The problem of filtering or restoring data $D$ can be expressed by the 
following: 
We search for a solution $\tilde D$ such that the difference between
$D$ and $\tilde D$ minimizes the information due to the signal, and 
such that  $\tilde D$ minimizes the information due to the noise. 
\begin{eqnarray}
J(\tilde D) = H_s(D-\tilde D) + H_n(\tilde D)
\label{eqn_func1}
\end{eqnarray}
Furthermore, the smoothness of the solution can be controlled by adding
a parameter:
\begin{eqnarray}
J(\tilde D) = H_s(D-\tilde D) + \alpha H_n(\tilde D)
\label{eqn_func2}
\end{eqnarray}
Here, $\alpha$ is considered as a constant value, but we can easely 
imagine to have a regularization parameter per scale, or even per wavelet
coefficient, depending on the Signal to Noise Ratio of the data. This
direction will be investigated in the future.
   
Three points must be noted:
\begin{itemize}
\item The positivity of the solution is not enforced.
\item There is no constraint on the flux.
\item The last scale of the wavelet transform is not taken into 
account in this entropy.
\end{itemize}
The first two points can be easily resolved by introducing strict a priori
 constraints on the solution \cite{thiebault95}:
\begin{eqnarray}
J(Z) = H_s(D-{\cal C}(Z)) + \alpha H_s({\cal C}(Z))
\end{eqnarray}
And the real solution is evidently $\tilde D = {\cal C}(Z)$. 
Positivity and total flux conservation impose
\begin{eqnarray}
{\cal C}(Z)(x) = \frac{\sum_x I(x)}{\sum_x Z(x)^2} Z(x)^2
\end{eqnarray}
Any other constraint can evidently be introduced into the function ${\cal C}$.

There is no constraint to be introduced to cater for the third point above, 
but this should not be a problem
if the number of scales we use for the entropy is high enough. Indeed, in
this case, the last scale becomes flat, and flux normalization should 
correctly fix this level.

\subsection{Example}
\begin{figure}[htb]
\centerline{
\vbox{
\psfig{figure=fig_lit6_mem1.ps,bbllx=2.5cm,bblly=13cm,bburx=19.5cm,bbury=25.5cm,width=16cm,height=9cm}
}}
\caption{Spectrum and filtered spectrum superimposed.}
\label{fig_lit6_mem1}
\end{figure}

\begin{figure}[htb]
\centerline{
\vbox{ 
\psfig{figure=fig_lit6_mem2.ps,bbllx=2.5cm,bblly=13cm,bburx=19.5cm,bbury=25.5cm,width=16cm,height=10cm}
}}
\caption{Difference (upper part) 
between the real spectrum and its smoothed version.
Part (pixels 400 to 500) of the spectrum (continuous curve), 
with the filtered spectrum
overplotted (dashed).}
\label{fig_lit6_mem2}
\end{figure}
Figure~\ref{fig_lit6_mem1} presents a spectrum and the result (overplotted)
after filtering using the multiscale entropy. The difference between the
spectrum and its smoothed version is plotted in 
Figure~\ref{fig_lit6_mem2} (upper part). 
As we can see, the residual contains only noise.
In order to better see the quality of the smoothing, we have plotted
only a part of the spectrum (see lower part of Figure~\ref{fig_lit6_mem2}),
and the filtered spectrum superimposed. The absorption lines are not 
modified using our filtering technique. 

\begin{figure}[h]
\centerline{
\hbox{
\psfig{figure=fig_lenna_g10.ps,bbllx=1.8cm,bblly=12.8cm,bburx=14.5cm,bbury=25.5cm,width=8cm,height=8cm,clip=}
\psfig{figure=fig_filmem_g2_lenna.ps,bbllx=1.8cm,bblly=12.8cm,bburx=14.5cm,bbury=25.5cm,width=8cm,height=8cm,clip=}
}}
\caption{Left, Lena + Gaussian noise, right, filtered image .}
\label{fig_filter_gauss_noise}
\end{figure}
Figure~\ref{fig_filter_gauss_noise} (left) shows the Lena image 
(see Figure~\ref{fig_lenna}) to which Gaussian 
noise of standard deviation 10 has been added.  Figure~\ref{fig_filter_gauss_noise} (right) shows 
the result using equation 32 with a regularization parameter value of 2.  

\section{Conclusion}

We have seen that information must be measured from the transformed
data, and not from the data itself. This approach has been used in fact
for several years in the domain of image compression. Indeed, modern
image compression methods consist firstly of  applying a transformation 
(cosine transform for JPEG, wavelet transform, etc.) to the image, and
then coding the coefficients obtained. A good transform for
image compression is obviously an orthogonal transform because there
is no redundancy, and the number of pixels is the same as in the original
image. The exact number of bits necessary to code the coefficients is
given by the Shannon entropy. For signal restoration, the problem is
not to reduce the number of bits in the representation of the data, and we
prefer to use a non-orthogonal wavelet transform, which avoids
artifacts in reconstruction due to undersampling. 

We could
have used the Shannon entropy to measure the information at a given scale,
and derive the bins of the histogram from the standard deviation of the noise,
but for several reasons we thought it better to directly introduce 
noise probability into our information measure. 
Firstly, we have seen that this leads, for Gaussian noise, to
a very physical relation between the information and the wavelet coefficients:
information is proportional to the energy of the wavelet coefficients
normalized by the standard deviation of the noise. 
Secondly, it works even in the case
of images with few photons/events (the histograms in this case present a bias).
We have seen that the equations are easy to manipulate. 
Finally, experiments have 
confirmed that this approach gives good results.
We have seen also that our new information
measure leads naturally to a new method for signal restoration. 
We are now experimenting with this method, and working on generalizations to
other classes of noise.

\newpage
% \chapter*{Appendix A: The ``\`A Trous'' Wavelet Transform Algorithm}
\addcontentsline{toc}{chapter}{Appendix A: The ``\`A Trous'' Wavelet Transform Algorithm}

In a wavelet transform, a series of transformations of a signal is 
generated, providing a resolution-related set of  ``views'' of the signal.  
The properties satisfied by a wavelet transform, and in particular by the
{\it \`a trous} wavelet transform, are further discussed by Bijaoui et al. \cite{starck:bij94_1}. 
Extensive literature exists on the wavelet transform
and its applications (\cite{wave:daube88,wave:chui92,wave:ruskai92,starck:book98}). 
The discrete {\it \`a trous} algorithm is described in \cite{wave:hol89,wave:shensa92}.

We consider spectra, 
$\{c_0(k)\}$, defined as  the scalar product at 
samples $k$ of the function $f(x)$ with a scaling function $\phi(x)$
which corresponds to a low pass filter:
\begin{eqnarray}
c_0(k) = < f(x), \phi(x-k)>
\end{eqnarray}
The scaling function is chosen to satisfy the dilation equation:
\begin{eqnarray}
\frac{1}{2}\phi(\frac{x}{2}) = \sum_l h(l)\phi(x-l)
\end{eqnarray}
where $h$ is a discrete low-pass filter associated with the scaling function
$\phi$.  This means that a low-pass filtering
of the signal is, by definition, closely linked to another resolution level
of the signal.  The distance between levels increases by a factor 2 from one
scale to the next.


The smoothed data $c_j(k)$ at a given resolution $j$ and at a position
$k$  is the scalar product 

\begin{eqnarray}
c_j(k)= \frac{1}{2^j}< f(x), \phi(\frac{x-k}{2^j})>
\end{eqnarray}

This is consequently obtained by the convolution:
\begin{eqnarray}
c_j(k) = \sum_l h(l) \ \ c_{j-1} (k+2^{j-1}l)
\end{eqnarray}
The signal difference $w_j$ between two consecutive resolutions is:
\begin{eqnarray}
w_j(k) = c_{j-1}(k) - c_j(k) 
\end{eqnarray}
or:
\begin{eqnarray}
w_j(k) = \frac{1}{2^j}< f(x), \psi(\frac{x-k}{2^j})>  
\label{wj}
\end{eqnarray}
Here, the wavelet function $\psi$ is defined by:
\begin{eqnarray}
\frac{1}{2}\psi(\frac{x}{2})  = \phi(x) - \frac{1}{2}\phi(\frac{x}{2})
\end{eqnarray}

Equation~\ref{wj} defines the discrete wavelet transform, for a  
resolution level $j$.

For the scaling function, $\phi(x)$, the B-spline of degree 3 
was used in our calculations. As a filter we use 
$h = ( \frac{1}{16}, \frac{1}{4}, \frac{3}{8}, \frac{1}{4}, \frac{1}{16})$. 
See  Starck (1993)
for discussion of 
linear and other scaling functions.  Here  
we have derived a simple algorithm in order to compute the 
associated wavelet transform:
\begin{enumerate}
\item We initialize $j$ to 0 and we start with the data $c_j(k)$.
\item We increment $j$, and carry out a discrete convolution of the data
$c_{j-1}(k)$ using  the filter $h$. The distance between the central sample
and the adjacent ones is $2^{j-1}$.
\item After this smoothing, we obtain the discrete wavelet transform
from the difference $c_{j-1}(k) - c_j(k)$.
\item If $j$ is less than the number $p$ of resolutions we want to
compute, then we go to step 2.
\item The set ${\cal W} = \{ w_1, ..., w_p, c_p \}$ represents the
wavelet transform of the data.
\end{enumerate}

A series expansion of the original signal, $c_0$, 
in terms of
the wavelet coefficients is now given as follows. 
The final smoothed array $c_{p}(x)$ is added to all the differences $w_j$:
\begin{eqnarray}
c_0(k) = c_{p} + \sum_{j=1}^{p} w_j(k)
\label{eqn_rec}
\end{eqnarray}
 This equation provides a reconstruction formula for the original signal. At each scale $j$, we obtain a set $\{w_j\}$ which 
we call a wavelet scale.  The  wavelet scale 
has the same number of samples as the signal. 

% % \chapter*{Appendix B: The Combined Filtering Method}
% \addcontentsline{toc}{chapter}{Appendix B:The Combined Filtering Method}

\chapter{The Combined Filtering Method}
 
In general, suppose that we are given $K$ linear transforms $T_1,
\ldots, T_K$ and let $\alpha_k$ be the coefficient sequence of an
object $x$ after applying the transform $T_k$, i.e. $\alpha_k = T_k
x$. We will assume that for each transform $T_k$ we have available a
reconstruction rule that we will denote by $T^{-1}_k$ although this is
clearly an abuse of notation.  Finally, $T$ will denote the block
diagonal matrix with the $T_k$'s as building blocks and $\alpha$ the
amalgamation of the $\alpha_k$'s.

A hard thresholding rule associated with the transform $T_k$ synthesizes 
an estimate $\tilde{s}_k$ via the formula 
\begin{equation}
\label{eq:ht}
\tilde{s}_k = T_k^{-1} \delta(\alpha_k)
\end{equation}
where $\delta$ is a rule that sets to zero all the coordinates of
$\alpha_k$ whose absolute value falls below a given sequence of
thresholds (such coordinates are said to be non-significant).
 
Given data $y$ of the form $y = s + \sigma z$, where $s$ is the image
we wish to recover and $z$ is standard white noise, we propose solving
the following optimization problem \cite{starck:spie01a}:
\begin{equation}
  \label{eq:l1-min}
  \min \|T\tilde{s}\|_{\ell_1}, \quad \mbox{subject to} \quad s \in C,  
\end{equation}
where $C$ is the set of vectors $\tilde{s}$ 
which obey the linear constraints
\begin{equation}
\label{eq:constraints}
\left\{  \begin{array}{ll}
  \tilde{s} \ge 0, \\
  |T\tilde{s} - Ty| \le e; 
  \end{array}
  \right. 
\end{equation}
here, the second inequality constraint 
only concerns the set of significant coefficients, 
i.e. those indices $\mu$ such that $\alpha_\mu =
(Ty)_\mu$ exceeds (in absolute value) a threshold $t_\mu$. Given a
vector of tolerance $(e_\mu)$, we seek a solution whose coefficients
  $(T\tilde{s})_\mu$ are within $e_\mu$ of the noisy
empirical $\alpha_\mu$'s.  Think of $\alpha_\mu$ as being given by
\[
y = \langle y, \varphi_\mu \rangle, 
\]
so that $\alpha_\mu$ is normally distributed with mean $\langle f,
\varphi_\mu \rangle$ and variance $\sigma^2_\mu = \sigma^2
\|\varphi_\mu\|^2_2$. In practice, the threshold values range
typically between three and four times the noise level $\sigma_\mu$
and in our experiments we will put $e_\mu = \sigma_\mu/2$. In short,
our constraints guarantee that the reconstruction will take into
account any pattern which is detected as significant by  any of the
$K$ transforms.
   
\subsubsection*{The Minimization Method}

We propose solving (\ref{eq:l1-min}) using the method of hybrid
steepest descent (HSD) \cite{wave:yamada01}. HSD consists of building
the sequence
\begin{eqnarray}
 s^{n+1} = P(s^{n}) - \lambda_{n+1} \nabla_J(P(s^{n})); 
\end{eqnarray}
Here, $P$ is the $\ell_2$ projection operator onto the feasible set
$C$, $\nabla_J$ is the gradient of equation~\ref{eq:l1-min}, and
$(\lambda_{n})_{n \ge 1}$ is a sequence obeying $(\lambda_{n})_{n\ge
  1} \in [0,1] $ and $\lim_{ n \rightarrow + \infty } \lambda_{n} = 0$.

The combined filtering algorithm is:
\begin{enumerate}
\baselineskip=0.4truecm
\itemsep=0.1truecm
\item Initialize $L_{\max} = 1$, the number of iterations $N_i$, and
  $\delta_{\lambda} = \frac{L_{\max}}{N_i}$.
\item Estimate the noise standard deviation $\sigma$, and set $e_k =
  \frac{\sigma}{2}$.
\item For k = 1, .., $K$ calculate the transform: $\alpha^{(s)}_k
  = T_k s$.
\item Set $\lambda = L_{\max}$, $n = 0$, and $\tilde s^{n}$ to 0.
\item While $\lambda >= 0$ do
\begin{itemize}
\item $u = \tilde s^{n}$.
\item For k = 1, .., $K$ do
  \begin{itemize}
  \item Calculate the transform $\alpha_{k} = T_k u$.
  \item For all coefficients $\alpha_{k,l}$ do
     \begin{itemize}
     \item Calculate the residual $r_{k,l} = \alpha^{(s)}_{k,l} -
       \alpha_{k,l}$
       
     \item if $\alpha^{(s)}_{k,l}$ is significant and $ \mid r_{k,l}
       \mid > e_{k,l}$ then $\alpha_{k,l} = \alpha^{(s)}_{k,l}$
     \item $\alpha_{k,l} = sgn(\alpha_{k,l}) ( \mid \alpha_{k,l} \mid - \lambda)_{+}$.
     \end{itemize}
   \item $u = T_k^{-1} \alpha_{k}$
  \end{itemize}
\item Threshold negative values in $u$ and $\tilde s^{n+1} = u$.
\item $n = n + 1$, $\lambda = \lambda - \delta_{\lambda} $, and goto 5.
\end{itemize}
\end{enumerate}
 


\begin{thebibliography}{99}
\bibitem{ables74} J.G. Ables, {\it Astronomy and Astrophysics Supplement Series},
15, 383--393, 1974.
 
\bibitem{bijaoui84} A. Bijaoui, {\it Introduction au 
Traitement Num\'erique des Images}, Masson, Paris, 1984.

% \bibitem{bijaoui94} A. Bijaoui, J.L. Starck and 
% F. Murtagh, ``Restauration des 
% images
% multi-\'echelles par l'algorithme \`a trous'', {\it Traitement du Signal},
% 11, 229--243, 1994.

\bibitem{bontekoe94} Tj.R. Bontekoe, E. Koper and D.J.M. Kester,   
``Pyramid maximum entropy images of IRAS survey data'', {\it Astronomy and
 Astrophysics}, 294, 1037--1053, 1994.

\bibitem{burg67} J.P. Burg, {\it 
Annual Meeting International Society Exploratory Geophysics}, 1967. 
Reprinted in {\it Modern Spectral Analysis}, 1978, D.G. Childers, ed., 
IEEE Press, New York, 34--41, 1978.

% \bibitem{chui92} C.H. Chui, {\it Wavelet Analysis and Its 
% Applications}, Academic Press, New York, 1992.

% \bibitem{daubechies88} I. Daubechies, 
% ``Orthogonal bases of compactly supported 
% wavelets'', {\it Communications on Pure and Applied Mathematics}, 
% 41, 909--996, 1988.

\bibitem {frieden75} B.R. Frieden, ``Image enhancement and restoration'', 
 {\it Topics in Applied Physics}, 6,  Springer-Verlag Berlin, 177--249, 1975.

\bibitem{frieden} B.R Frieden, {\it Probability, Statistical Optics, and
Data Testing: a problem solving approach}, 2nd edition, Springer-Verlag Berlin, 
1991.

\bibitem{jaynes57} E.T. Jaynes, {\it Phys. Rev.}, 106, 620--630, 1957.

% \bibitem{holschneider89} M. Holschneider, R. Kronland-Martinet, J. Morlet
% and Ph. Tchamitchian, ``A real-time algorithm for signal analysis with the
% help of the wavelet transform'', in 
% {\it Wavelets: Time-Frequency Methods and Phase Space}, 
% Springer-Verlag,  Berlin, 286--297, 1989.

\bibitem{gull91} S.F. Gull, and J. Skilling, MEMSYS5 User's Manual, 1991.

\bibitem{narrayan86} R. Narayan, and R. Nityananda, ``Maximum 
Entropy Image Restoration in Astronomy'', Ann. Rev. Astron. Astrophys., 
24, pp 127--170, 1986.

\bibitem{pantin96} E. Pantin and J.L. Starck, ``Deconvolution of 
astronomical images using the multiresolution maximum entropy method'', 
 {\it Astronomy and Astrophysics Supplement Series}, 118, 575--585, 1996. 

% \bibitem {ruskai92} M.M. Ruskai, G. Beylkin, R. Coifman, I. Daubechies,
% S. Mallat, Y. Meyer and  L. Raphael, {\it Wavelets and Their Applications},
% Jones and Barlett, 1992.

\bibitem{shannon48} C.E. Shannon,  
``A Mathematical Theory for Communication'', {\it Bell System Tech J.} 27 
379--423, 1948.

% \bibitem{shensa92} M.J. Shensa, ``Discrete wavelet transforms:  
% wedding  the \`a trous and Mallat algorithms", {\it IEEE Transactions on 
% Signal
% Processing}, 40, 2464--2482, 1992.

\bibitem{skilling84_1} J. Skilling and S.F. Gull, {\it Proceedings of 
the American Mathematical Society}, SIAM, 14, 167, 1984.

\bibitem{skilling89} J. Skilling, ``Classic maximum entropy'', 
in: Skilling. J, ed., {\it Maximum 
Entropy and
Bayesian Methods}, Kluwer, Dordrecht, 45--52, 1989. 

\bibitem{starck95}  J.L. Starck, F. Murtagh and  A. Bijaoui, 
``Multiresolution support
 applied to image filtering and deconvolution'', {\it Graphical
Models and Image Processing}, 57, 420--431, 1995.

\bibitem{starck97} J.L. Starck, F. Murtagh  and A. Bijaoui,  
{\it Image Processing and Data Analysis: The
Multiscale Approach},  Cambridge University Press, forthcoming, 1997.

\bibitem{weir92} N. Weir, ``A multi-channel method of maximum entropy image
restoration'', in: D.M. Worral, C. Biemesderfer and J. Barnes, 
eds., {\it Astronomical Data Analysis Software and System 1}, Astronomical 
Society of the Pacific, San Francisco, 186--190, 1992.

\bibitem{thiebault95} E. Thiebault and J.-M. Conan, ``Strict a priori
constraints for maximum-likelihood blind deconvolution'', {\it JOSA}, 
12, 485--492, 1995.

\end{thebibliography}


\end{document}
\subsection{Spectrum filtering}

\subsection{Image filtering}
\begin{figure}[h]
\centerline{
\hbox{
\psfig{figure=fig_simu4_bw.ps,bbllx=1.8cm,bblly=12.8cm,bburx=14.5cm,bbury=25.5cm,width=16cm,height=16cm,clip=}
}}
\caption{(a) Simulated image, (b) simulated image and 
Gaussian noise, (c) filtered image, and (d)
residual image.}
\label{fig_filter_gauss_noise}
\end{figure}

A simulated image containing stars and galaxies is shown in Fig.\ 
\ref{fig_filter_gauss_noise} (top left). The simulated noisy
image, the filtered image and the residual image are respectively shown in
Fig.\ \ref{fig_filter_gauss_noise} top right, bottom left, and
bottom right. We can see that there is no structure in the residual 
image. The filtering was carried out using the multiresolution support.


