\chapter{Multiscale Entropy applied to Background Fluctuation Analysis}

\section*{The mean entropy vector}

The multiscale entropy has been defined by:
\begin{eqnarray}
H(X) = \sum_{j=1}^{l}  \sum_{k=1}^{N} h(w_j)  
\end{eqnarray}
with $h(w_j = \ln ( p(w_j(k))))$.
In order to study the behavior of the information at a given scale, we 
prefer to calculate the mean entropy vector $E$ defined by:
\begin{eqnarray}
E(j) = \frac{1}{N}\sum_{k=1}^{N} h(w_j)
\end{eqnarray}
$E(j)$ gives the mean entropy at the scale $j$. From the mean entropy vector,
we have  statistical information on each scale separately. 
Having a noise model, we are able to calculate (generally from simulations)
the mean entropy vector $E^{(noise)}(j)$ resulting from a pure noise.
Then we define the normalized mean entropy vector by
\begin{eqnarray}
E_n(j) = \frac{ E(j) }{ E^{(noise)}(j) }
\end{eqnarray}


Figure~\ref{fig_source1} shows the result of a simulation. 
Five simulated images were created by adding $n$ sources (i.e., point
sources, or idealized stars) to an 1024
$\times$ 1024 image 
containing Gaussian noise of standard deviation equal to 1. The $n$ sources
are identical, with a maximum equal to 1, and standard deviation equal to 2.
Defining the signal to noise ratio (SNR) as the ratio between
the standard deviation in the smallest box which contains at least 90\% 
of the flux of the source, and the noise standard deviation, we have a SNR
equal to $0.25$. The sources  are not detectable in the simulated
image, nor in its wavelet transform. Figure~\ref{fig_source2} shows a region
which contains a source at the center. It is clear there is no way to find
this kind of noisy signal. The five images were created using a number 
of sources
respectively equal to 0,50,100,200 and 400, and the simulation was
repeated ten times with different noise maps in order to have an error bar on
each entropy measurement. For the image which contains 400
sources, the number of pixels affected by a source is less  than $2.5$\%.

When the number of sources increases, the difference between the multiscale
entropy curves increases. Even if the sources are very faint, the presence
of signal can be clearly detected using the mean entropy vector. But it is 
obvious that the positions of these sources remain unknown.
 
\begin{figure}[htb]
\centerline{
\vbox{
\psfig{figure=fig_source.ps,bbllx=3.5cm,bblly=13cm,bburx=19.5cm,bbury=25.5cm,width=16cm,height=12.5cm}
}}
\caption{Mean entropy versus the scale of 5 simulated images containing undetectable
sources and noise. Each curve corresponds to the multiscale transform
of one image. From
top to bottom, the image contains respectively 400,200,100, 50 and 0 sources.}
\label{fig_source1}
\end{figure}

\begin{figure}[htb]
\centerline{
\vbox{
\psfig{figure=fig_1source.ps,bbllx=1.8cm,bblly=12.9cm,bburx=14.5cm,bbury=25.5cm,width=8cm,height=8cm,clip=}
}}
\caption{Region of a simulated image containing an undetectable source at the
center.}
\label{fig_source2}
\end{figure}

