\chapter*{Appendix B: Noise Modeling the Wavelet Space}
\addcontentsline{toc}{chapter}{Appendix B: Noise Modeling the Wavelet Space}
 

\subsection*{Gaussian noise}
\index{noise}
We have the probability density
\begin{eqnarray}
p(w_j(x,y)) = \frac{1}{\sqrt{2\pi} \sigma_j} e^{{-w_j(x,y)^2}/2\sigma^2_j} 
\end{eqnarray}

So we need to estimate, in the case of Gaussian noise models,
 the noise standard deviation at each scale.  These standard deviations can 
be determined analytically in the case of some transforms, including the 
\`a trous transform, but the calculations can become complicated.  

The appropriate value of $\sigma_j$ 
in the succession of wavelet planes is assessed 
from the standard deviation of the noise $\sigma_I$ in the original image
$I$, 
and from study of the noise in the wavelet space.  This study consists of 
simulating an image containing Gaussian noise with a standard deviation 
equal to 1, and taking the wavelet transform of this image.  Then we
compute the standard deviation $\sigma^e_j$ at each scale.  We get a curve 
$\sigma^e_j$ as a function of $j$, giving the behavior of the noise in the 
wavelet space.
(Note that if we had used an orthogonal wavelet transform, this curve would
\index{wavelet transform}
be linear.)  Due to the properties of the wavelet transform, we have 
$ \sigma_j = \sigma_I \sigma^e_j $.
The standard deviation of the noise at a scale $j$ of the image is equal to
the standard deviation of the noise of the image multiplied by the 
standard deviation of the noise of  scale $j$ of the wavelet transform.
 
\subsection*{Poisson noise}
\index{noise}

If the noise in the data $I$ is Poisson, the Anscombe transform 
\begin{eqnarray}
t(I(x,y)) = 2\sqrt{I(x,y) + \frac{3}{8}}
\label{eqn_noise_anscombe}
\end{eqnarray}
acts as if the data arose from a
Gaussian white noise model (Anscombe, \cite{rest:anscombe48}), 
with $\sigma = 1$, under the
assumption that the mean value of $I$ is large.
\index{Anscombe transformation}

For Poisson parameter values under about 20, the Anscombe transformation 
looses control over the bias.  In this case, an alternative approach to 
variance stabilization is needed. An approach for very small numbers of 
counts, including frequent zero cases, has been described in 
in \cite{astro:slezak93}, \cite{astro:bijaoui94} and \cite{astro:bury95},
and  will be described below. 
Small numbers of detector counts will most likely be associated  with the 
image background. 
Note that errors related to small values carry the risk of 
removing real objects, but not of amplifying noise.   

\subsection*{Gaussian and Poisson Noise}
\index{variance stabilization}
\index{noise}

The arrival of photons, and their expression by electron counts, on CCD
detectors may be modeled by a Poisson distribution.  In addition, there is 
additive Gaussian read-out noise.  
The Anscombe 
transformation (eqn.\ \ref{eqn_noise_anscombe}) has been extended \cite{starck:mur95_2}
to take this combined noise into account.
As an
approximation, consider the signal's value, $I(x,y)$, as a sum of a Gaussian
variable, $\gamma$, of mean $g$ and standard-deviation $\sigma$; and a
Poisson variable, $n$, of mean $m_0$: we set 
$I(x,y) = \gamma + \alpha n $ where $\alpha$
is the gain.  

The generalization of the variance stabilizing Anscombe formula is:
\begin{eqnarray}
t = \frac{2}{\alpha} \sqrt{\alpha I(x,y) + \frac{3}{8} \alpha^2 + \sigma^2 -
\alpha g}
\label{eqn_noise_bijaoui}
\end{eqnarray}
With appropriate values of $\alpha$, $\sigma$ and $g$, this reduces to 
Anscombe's transformation (eqn.\ \ref{eqn_noise_anscombe}).  
\index{Anscombe transformation}


\subsection*{Poisson Noise with Few Photons or Counts}
\index{noise}
\label{noise_few_photons}

A wavelet coefficient at a given position and at a given scale $j$ is
\begin{eqnarray}
w_j(x,y) =  \sum_{k \in K} n_k \psi(\frac{x_k - x}{2^j} , \frac{y_k - y}{2^j})
\end{eqnarray}
where $K$ is the support of the wavelet function  $\psi$ and $n_k$ is the 
number  of events which contribute to the calculation of $w_j(x,y)$ (i.e.\ the 
number of 
photons included in the support of the dilated wavelet centered at ($x$,$y$)).
\index{wavelet transform}

If a wavelet coefficient $w_j(x,y)$ is due to the noise, it can be considered
as a realization of the sum $\sum_{k \in K} n_k$ of 
independent random variables 
with the same distribution as that of the wavelet function ($n_k$
being the number of photons or events used for the calculation of $w_j(x,y)$).
Then we compare the wavelet coefficient of the data to the values 
which can be taken by the sum of $n$ independent variables.

The distribution of one event in the wavelet space is directly 
given by the histogram $H_1$ of the wavelet $\psi$. Since 
independent events are considered, the distribution of the random variable 
$W_n$ (to be associated with a wavelet coefficient) related to $n$
events is given by $n$ autoconvolutions of $H_1$
\begin{eqnarray}
H_n = H_1 \otimes  H_1 \otimes ... \otimes H_1
\end{eqnarray}
For a large number of events, $H_n$ converges to a Gaussian. 

In order to facilitate the comparisons, the variable $W_n$ of distribution
 $H_n$ 
 is reduced by
\begin{eqnarray}
c = \frac{W_n - E(W_n)}{\sigma(W_n)}
\end{eqnarray}
and the cumulative distribution  function is 
\begin{eqnarray}
F_n(c) = \int_{-\infty}^{c} H_n(u) du
\end{eqnarray}

From $F_n$, we derive $c_{min}$ and $c_{max}$ such 
that $F(c_{min}) = \epsilon$
and $F(c_{max}) = 1 - \epsilon$.

Therefore a reduced wavelet coefficient $w^r_j(x,y)$, calculated from
$w_j(x,y)$, and  resulting 
from $n$ photons or counts is
significant if:
\begin{eqnarray}
F(w^r) > c_{max}
\end{eqnarray}
or
\begin{eqnarray}
F(w^r) < c_{min}
\end{eqnarray}
and $w^r_j(x,y)$ is obtained by
\begin{eqnarray}
w^r_j(x,y)  & = &   \frac{w_j(x,y)}{\sqrt{n} \sigma_{\psi_j}} \\
             & =  & \frac{w_j(x,y)}{\sqrt{n} \sigma_{\psi}} 4^j
\end{eqnarray}
where $\sigma_{\psi}$ is the standard deviation of the wavelet function, and
$\sigma_{\psi_j}$ is the standard deviation of the dilated wavelet function
($\sigma_{\psi_j} = \sigma_{\psi}/4^j$).

\subsection*{Root mean square map}
If, associated to the data $I(x,y)$, we have the root mean square map  
$R_{\sigma}(x,y)$, the noise in $I$  is non homogeneous.
For each wavelet coefficient $w_j(x,y)$ of $I$, the 
exact standard deviation $\sigma_j(x,y)$ have to be calculated from 
 $R_{\sigma}$. A wavelet coefficient $w_j(x,y)$ is obtained by
the correlation product between the image $I$ and a function 
$g_j$:
\begin{eqnarray}
 w_j(x,y) = \sum_k \sum_l I(x,y)  g_j(x+k,y+l)
\end{eqnarray}

 
Then we have
\begin{eqnarray}
\sigma_j^2(x,y) =  \sum_k \sum_l R_{\sigma}^2(x,y) g_j^2(x+k,y+l).
\end{eqnarray}

In the case of the \`a trous algorithm, the coefficients $g_j(x,y)$
 are not known exactly, but they can easily be computed by taking the
wavelet transform of a Dirac $w^{\delta}$. The map $\sigma_j^2$ is calculated 
by correlating the square of the wavelet scale $j$ of  $w^{\delta}$
by $R^2_\sigma(x,y)$.
 
\subsection*{Speckle Noise}
Speckle occurs in all types of coherent imagery such as synthetic aperture
 radar (SAR) imagery,  acoustic imagery and laser illuminated imagery. 
 The probability density function (pdf) of the modulus of a homogeneous
 scene is a Rayleigh distribution:
\begin{eqnarray*}
p(\rho)={\rho\over \sigma^2}e^{-{\rho^2\over 2\sigma^2}} 
\quad  M_{\rho} = \sqrt{ \frac{\pi}{2} } \sigma \quad \sigma_{\rho}=\sqrt {\frac{4-\pi}{2}}\sigma
 \end{eqnarray*}
The ratio $\sigma_{\rho}/M_{\rho}$ is a constant of value $\sqrt{\frac{4-\pi}{\pi}}$.
This 
means that the speckle is a multiplicative noise. The pdf of the  modulus of a log-transformed 
 speckle noise is:
\begin{eqnarray*}
p(\ell) =  \frac{e^{2\ell}}{ \sigma^2} e^{-\frac{e^{2\ell}}{2 \sigma^2} } \quad 
M_{\ell} = 0.058 + \log(\sigma) \quad \sigma_{\ell}= \sqrt{\frac{\pi^2}{24}} = 0.641
\end{eqnarray*}

A better estimator for Rayleigh distribution is the  energy ($I=\rho^2$)
which is known to have an exponential ({\it i.e.} Laplace) 
distribution of parameter $a=2\sigma^2$
\begin{eqnarray*}
p(I)={1\over a}e^{-\frac{I}{a}}
\end{eqnarray*}

\subsection*{Other Types of Noise}
For any type of noise, an analogous study can be carried out in order to find
the detection level at each scale and at each position. The 
types of noise considered so far in this chapter correspond to the general 
cases in astronomical imagery. 
We now describe briefly methods which can be used for non-uniform and
multiplicative noise. 

\subsubsection*{Additive Non-Uniform Noise}
\label{noise_addi_uni}
If the noise is additive, but non-uniform, we cannot estimate a standard
deviation for the whole image. However, we can often assume that the noise
is locally Gaussian, and we can compute a local standard deviation of the 
noise for each pixel. In this way, we obtain a standard deviation 
map of the noise, $I_{\sigma}(x,y)$. 
A given wavelet coefficient $w_j(x,y)$ is calculated
from the pixels of the input image $I$ in the range $I(x-l \dots x+l, y-l 
\dots y+l)$
where $l$ is dependent on the wavelet transform algorithm, the wavelet 
function,
and the scale $j$. An upper limit $u_j(x,y)$ for the noise associated with
$w_j(x,y)$ is 
found by just considering the maximum value in 
$I_{\sigma}(x-l \dots x+l,y-l \dots y+l)$ and
 by multiplying this value by the constant $\sigma_j^e$ (defined in the 
subsection ``Gaussian noise'' at the beginning of this Appendix).
\begin{eqnarray}
u_j(x,y) = \max(I_{\sigma}(x-l \dots x+l, y-l \dots y+l)) \sigma_j^e
\end{eqnarray}
The detection level is not constant over each scale. 
 
\subsubsection*{Multiplicative Noise}
If the noise is multiplicative, the image can be transformed by taking
its logarithm. In the resulting image, the noise is additive, and 
a hypothesis of Gaussian noise can be used in order to find the 
detection level at each scale.

\subsubsection*{Multiplicative Non-Uniform Noise}
In this case, we take the logarithm of the image, and the resulting image
is treated as for additive non-uniform noise above.

\subsubsection*{Unknown Noise}
If the noise does not follow any known distribution, 
we can consider as significant
only wavelet coefficients which are greater than their local standard deviation
multiplied by a constant: $w_j(x,y)$ is significant if 
\begin{eqnarray}
\mid w_j(x,y) \mid \ > \ k \sigma(w_j(x-l \dots x+l, y-l \dots y+l)) 
\end{eqnarray}


 
 
