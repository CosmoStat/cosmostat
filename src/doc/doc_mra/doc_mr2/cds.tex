\chapter{Multiscale Entropy as a Measure of Relevant Information in an Image}
 
Since the multiscale entropy extracts the information from the signal only, 
it was a  challenge to see if the  astronomical content of  an image
was related to its multiscale entropy.

For this purpose, we studied the astronomical content of 200 images 
of 1024 $\times$ 
1024 pixels extracted from scans of 8 different plates carried out  
by the MAMA facility (Paris, France) 
\cite{astro:guibert92} and stored at CDS (Strasbourg, France) 
in the Aladin
archive \cite{compress:bonnarel99}. We estimated the content of these images 
in three different ways:
\begin{enumerate}     
\item By counting the number of objects in an astronomical catalog
(USNO A2.0 catalog)   
within the image. The
 USNO (United States Naval Observatory) 
catalog was obtained by source extraction from the same survey 
 plates as we used in our study.
\item By counting the number of objects estimated in the image by the
 Sextractor object detection 
package \cite{astro:bertin96}. As in the case of USNO 
these detections are mainly point sources (stars, as opposed to 
spatially extended objects like galaxies).
\item  By counting the number of structures detected at several scales using 
 the MR/1 multiresolution analysis package \cite{starck:mr1_99}.
\end{enumerate}

\begin{figure}[htb]
\centerline{
\vbox{
\psfig{figure=fig_cds_pmm_entropy.ps,bbllx=2.8cm,bblly=2.8cm,bburx=20.5cm,bbury=15.4cm,width=10cm,height=6cm,clip=}
\psfig{figure=fig_cds_sext_entropy.ps,bbllx=2.8cm,bblly=2.8cm,bburx=20.5cm,bbury=15.4cm,width=10cm,height=6cm,clip=}
\psfig{figure=fig_cds_support_entropie.ps,bbllx=2.8cm,bblly=2.8cm,bburx=20.5cm,bbury=15.4cm,width=10cm,height=6cm,clip=}
}}
\caption{Multiscale entropy versus the number of objects: the number
of objects is, respectively, obtained from (top) the USNO catalog, (middle)
the Sextractor package, and (bottom) the MR/1 package.}
\label{fig_cds_entropy}
\end{figure}

Figs.~\ref{fig_cds_entropy} show the results of plotting these numbers 
for each image against the multiscale signal entropy of the image. 
The best results are obtained using the MR/1 package, 
 followed by Sextractor and then by the number of sources extracted
 from USNO. Of course the latter two basically miss the content
at large scales, which is taken into account by MR/1.

Sextractor and multiresolution methods were also applied to a set of CCD 
images from CFH UH8K, 2MASS and DENIS near infrared surveys.
Results obtained were very similar to what was obtained above.  This seems
to point to multiscale entropy as being a universal measurement of image 
content.

Subsequently we looked for the relation between the multiscale entropy and 
the optimal compression
rate of an image which we can obtain by multiresolution 
techniques \cite{starck:book98}. 
 By optimal compression rate we mean 
a compression rate which allows all the sources to be preserved, and which 
does not
degrade the astrometry and photometry.
Louys et al.\ \cite{starck:louys99}  and Couvidat \cite{compress:couvidat99}
have  estimated   this optimal
compression rate using the compression program of the 
MR/1 package  \cite{starck:mr1_99}.

\begin{figure}[htb]
\centerline{
\hbox{
\psfig{figure=fig_cds_taux.ps,bbllx=2.8cm,bblly=2.8cm,bburx=20.5cm,bbury=15.4cm,width=10cm,height=6cm,clip=}
}}
\caption{Multiscale entropy of astronomical images versus the optimal
compression ratio. Images which contain a high number of sources have
a small ratio and a high multiscale entropy value. The relation 
is almost linear.}
\label{fig_cds_taux}
\end{figure}

Fig.~\ref{fig_cds_taux} shows the relation obtained 
between the multiscale entropy and the optimal compression rate
for all the images used in our previous tests including CCD ones. 
The power law  relation is obvious thus allowing us to conclude that:
\begin{itemize}
\item  The compression rate depends strongly on the astronomical content 
of the image. We can then say that compressibility is 
also an estimator of the content of the image.
\item The multiscale entropy allows us to predict the optimal 
compression rate of the image.
\end{itemize}

