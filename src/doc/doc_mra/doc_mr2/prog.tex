
\chapter{\projmw Programs}
\label{ch_prog_mw}
\markright{Programs}

\section{Probability in wavelet space: mw\_proba}
\index{mw\_proba}
The program 
{\em mw\_proba} computes the probability of each wavelet coefficient to be
due to signal (i.e.\ not due to noise).
{\bf 
\begin{center}
 USAGE: mw\_proba options image\_in mr\_file\_out
\end{center}}
where options are~:
\begin{itemize}
\baselineskip=0.4truecm
\itemsep=0.1truecm
\item {\bf [-t type\_of\_multiresolution\_transform]} 
\item {\bf [-m type\_of\_noise]}
\item {\bf [-g sigma]}
\item {\bf [-c gain,sigma,mean]}
\item {\bf [-n number\_of\_scales]}
\item {\bf [-S SizeBlock]} \\
\item {\bf [-N NiterSigmaClip]} \\
\item {\bf [-R RMS\_Map\_File\_Name]} \\
\end{itemize}

\subsubsection*{Examples:}
\begin{itemize}
\item mw\_prob image\_in.d MR\_File\_out.mr \\
Compute the probability of each wavelet coefficient of an image
to be due to signal (i.e. not due to noise).
\item mr\_extract -s 1 MR\_File\_out.mr  scale1.d \\
Create an image from the first scale of the multiresolution file.
\end{itemize}

\section{Entropy of an image: mw\_entrop}
\index{mw\_entrop}
The program 
{\em mw\_entrop} computes the information (entropy) and the signal information
of an image (N1-MSE approach). The two output files mr\_file1\_out and mr\_file2\_out 
are multiresolution files (``.mr'') which contain respectively 
the information and the signal information 
relative to the wavelet coefficients. The program calculates and prints to
the screen the mean entropy and signal entropy per band.
{\bf 
\begin{center}
 USAGE: mw\_entrop options image\_in  mr\_file1\_out mr\_file2\_out
\end{center}}
where options are:
\begin{itemize}
\baselineskip=0.4truecm
\itemsep=0.1truecm
\item {\bf [-t type\_of\_multiresolution\_transform]} 
\item {\bf [-m type\_of\_noise]}
\item {\bf [-g sigma]}
\item {\bf [-c gain,sigma,mean]}
\item {\bf [-n number\_of\_scales]}
\item {\bf [-S SizeBlock]} 
\item {\bf [-N NiterSigmaClip]}
\item {\bf [-R RMS\_Map\_File\_Name]}
\end{itemize}
\subsubsection*{Example:}
\begin{itemize}
\item mw\_entrop image\_in.d MR\_File1\_out.mr MR\_File2\_out.mr \\
Compute the multiscale entropy of an image
\end{itemize}

\section{Filtering}
\subsection{1D filtering: mw1d\_filter}
\index{mw1d\_filter}
The program {\em mw1d\_filter} filters a one dimensional signal using the 
multiscale entropy method (N2-MSE approach). 
{\bf 
\begin{center}
 USAGE: mw1d\_filter options signal\_in  signal\_out
\end{center}}
where options are~:
\begin{itemize}
\baselineskip=0.4truecm
\itemsep=0.1truecm
\item {\bf [-t type\_of\_multiresolution\_transform]} 
\item {\bf [-m type\_of\_noise]}
\item {\bf [-g sigma]} 
\item {\bf [-c gain,sigma,mean]}
\item {\bf [-s NSigma]} 
\item {\bf [-n number\_of\_scales]} 
\item {\bf [-e epsilon]} \\
Convergence parameter. Default is $1e^{-4}$.
\item {\bf [-i number\_of\_iterations]} \\
Maximum number of iterations. Default is 10.
\item {\bf [-G RegulParam]} \\
Regularization parameter. Default is 1.
\item {\bf [-D]} \\
The regularization parameter is a function of the SNR in the data.
Default is no.
\item {\bf [-w FilterCoefFileName]} \\
Write to  disk the filtered wavelet coefficient.
\item {\bf [-v]} \\
Verbose. Default is no.
\end{itemize}

\subsubsection*{Examples:}
\begin{itemize}
\baselineskip=0.4truecm
\itemsep=0.1truecm
\item mw1d\_filter sig\_in.fits sig\_out.d \\
filters a signal by the multiscale entropy method, assuming
Gaussian noise (its standard deviation is automatically estimated).
\item mw1d\_filter -G 2 sig\_in.fits sig\_out.fits \\
Same as before, but the regularization will be stronger, and the solution
more smooth.  
\item mw1d\_filter -G 2 -D sig\_in.fits sig\_out.fits \\
The regularization is adaptive, depending on the wavelet SNR.
\item mw\_filter -G 2 -D -s 5 sig\_in.fits sig\_out.fits \\ 
Same as before, preserving  feature in the wavelet space greater than
$5\sigma$ instead of the default $3\sigma$ value.
\end{itemize}


\subsection{2D filtering: mw\_filter}
\index{mw\_filter}
The program {\em mw\_filter} filters an image using the 
multiscale entropy method (N2-MSE approach). 
{\bf 
\begin{center}
 USAGE: mw\_filter options image\_in image\_out
\end{center}}
where options are:
\begin{itemize}
\baselineskip=0.4truecm
\itemsep=0.1truecm
\item {\bf[-T Type\_of\_Regularization]}
\begin{enumerate}
\baselineskip=0.4truecm
\itemsep=0.1truecm
\item Use a fixed user Alpha value.
\item Estimate the optimal Alpha.
\item Estimate one  Alpha value per band.
\end{enumerate}
Default is 1.
\item {\bf [-D]} \\
Wavelet coefficients with a high signal to noise ratio are not
regularized. For a lower SNR, the Alpha parameter is modified using the SNR.
Default is no regularization.
\item {\bf [-t type\_of\_multiresolution\_transform]} 
\item {\bf [-m type\_of\_noise]} \\
Noise models 1 to 9 are available.
\item {\bf [-g sigma]} 
\item {\bf [-c gain,sigma,mean]}
\item {\bf [-s NSigma]} 
\item {\bf [-S SizeBlock]} 
\item {\bf [-N NiterSigmaClip]} 
\item {\bf [-R RMS\_Map\_File\_Name]} 
\item {\bf [-n number\_of\_scales]} 
\item {\bf [-e epsilon]} \\
Convergence parameter. Default is $1e^{-4}$.
\item {\bf [-i MaxIter]} \\
Maximum number of iterations. Default is 20.
\item {\bf [-G RegulParam]} \\
Regularization parameter. Default is 1.
\item {\bf [-C]} \\
Convergence parameter. Only used when regularization type is equal to
2 or 3.
Default is $0.01$.
\item {\bf [-P]} \\
Apply the positivity constraint. If set, the solution cannot have negative 
values.
\item {\bf [-v]} \\
Verbose. Default is no.
\end{itemize}

\subsubsection*{Examples:}
\begin{itemize}
\baselineskip=0.4truecm
\itemsep=0.1truecm
\item mw\_filter image\_in.d ima\_out.d \\
Filters an image by the multiscale entropy method, assuming
Gaussian noise (its standard deviation is automatically estimated).
\item mw\_filter -G 10 -P image\_in.d ima\_out.d \\
Same as before, but the regularization will be stronger, and the solution
more smooth. Positivity constraint is imposed.
\item mw\_filter -G 10 -D image\_in.d ima\_out.d \\
The regularization is adaptive, depending on the wavelet SNR.
\item mw\_filter -T 2 image\_in.d ima\_out.d \\
The regularization parameter is automatically estimated in an iterative way.
\item mw\_filter -T 2 -G 2.5 image\_in.d ima\_out.d \\
Same as before, but the estimated parameter is multiplied by $2.5$ (the solution
becomes more smooth).
\item mw\_filter -T 3 image\_in.d ima\_out.d \\
On regularization parameter per scale is now used. All are automatically 
estimated in an iterative way.
\item mw\_filter -T 3 -D -s 5 image\_in.d ima\_out.d \\ 
Same as before, preserving also feature in the wavelet space greater than
$5\sigma$.
\end{itemize}


\subsection{2D combined filtering: mw\_comb\_filter}
\index{mw\_comb\_filter}
The program {\em mw\_comb\_filter} filters an image using the 
combined filtering method. 
{\bf 
\begin{center}
 USAGE: mw\_comb\_filter options image\_in image\_out
\end{center}}
where options are~:
\begin{itemize}
\baselineskip=0.4truecm
\itemsep=0.1truecm
\item {\bf [-t type\_of\_multiresolution\_transform]} 
\item {\bf [-O]} \\
 Filtering by an opening (erosion+dilation). \\
The structural element is a circle of size 3
\item {\bf [-M type\_of\_multiresolution\_transform]} \\
 Filtering using the MEM method (estimating automatically
 one  Alpha value per band). Default multiresolution transform is
 the \`a trous algorithm. 
\item {\bf [-D]} \\
Wavelet coefficient with a high signa to noise ratio are not
regularized. The Alpha parameter is modified using the SNR.
Default is no. Valid only when ``-d'' or ``-M'' option is set.
\item {\bf [-G RegulParam]} \\
Regularization parameter for MEM method. 
Valid only when ``-d'' or ``-M'' option is set. \\
Default is 1.
\item {\bf [-m type\_of\_noise]}
\begin{enumerate}
\baselineskip=0.4truecm
\item Gaussian noise 
\item Poisson noise 
\item Poisson noise + Gaussian noise 
\item Multiplicative noise 
\end{enumerate}
Default is Gaussian noise.
\item {\bf [-g sigma]} 
\item {\bf [-c gain,sigma,mean]}
\item {\bf [-s NSigma]} 
\item {\bf [-n number\_of\_scales]} 
\item {\bf [-d]} \\
Use default combined methods (methods are a thresholding by the 
\`a trous algorithm,
the multiscale median transform, the orthogonal wavelet transform, the haar
transform, and the multiscale entropy method. ``-d'' option is equivalent to
the set of following options together: ``-t 2 -t 4 -t 14 -t 18 -M 2''.
\item {\bf [-v]} \\
Verbose. Default is no.
\end{itemize}
\subsubsection*{Examples:}
\begin{itemize}
\item mw\_comb\_filter -d image\_in.d ima\_out.d \\
Filters an image by the combined filtered method (default option), assuming
Gaussian noise (its standard deviation is automatically estimated).
\item mw\_comb\_filter -t 2 -t 4 -t 14 -t 18 -M 2 image\_in.d ima\_out.d \\
Ditto. 
\item mw\_comb\_filter -d -O -t 1 image\_in.d ima\_out.d \\
Same as before, with in addition a filtering using the linear \`a trous
 algorithm, and the morphological operators.
\item mw\_comb\_filter -t 14 -M 14 image\_in.d ima\_out.d \\
Filtering using only two methods: thresholding the orthogonal WT, and the MEM 
method.
\item mw\_comb\_filter -t 14 -M 14 -G 2.5 image\_in.d ima\_out.d \\
Ditto, but the estimated MEM parameter is multiplied by $2.5$.
\end{itemize}

\section{Deconvolution: mw\_deconv}
 \index{ mw\_deconv}
The program {\em mw\_deconv} deconvolves  an image using the 
multiscale entropy method. 
{\bf 
\begin{center}
 USAGE: mw\_deconv options image\_in psf\_in image\_out
\end{center}}
where options are~:
\begin{itemize}
\baselineskip=0.4truecm
\itemsep=0.1truecm
\item {\bf [-t type\_of\_multiresolution\_transform]} \\
Default is 2.
\item {\bf [-H EntropyFunction]} 
\begin{enumerate}
\baselineskip=0.4truecm
\itemsep=0.1truecm
\item Entropy = $H$ = Wavelet coefficient energy.
\item Entropy = $H_n$ = Noise information (for Gaussian noise only), using 
N2-MSE approach.
\end{enumerate}
Default is 1. For Gaussian noise, default is 2.
\item {\bf [-g sigma]} 
\item {\bf [-c gain,sigma,mean]}
\item {\bf [-m type\_of\_noise]} \\
Noise models 1 to 9 are available.
\item {\bf [-n number\_of\_scales]} 
\item {\bf [-s NSigma]} 
\item {\bf [-i number\_of\_iterations]} \\
 Maximum number of iterations. Default is 500.
\item {\bf [-K]}  \\
Suppress the last scale. Default is no. 
\item {\bf [-R RMS\_Map\_File\_Name]} \\
\item {\bf [-P]} \\
Suppress the positivity constraint.

\item {\bf [-C]} \\
Convergence parameter.
Default is $1$.
\item {\bf  [-f ICF\_Fwhm]} \\
Intrinsic correlation function. \\
Fwhm = Full-width at half maximum.
\item {\bf [-I ICF\_FileName]} \\
Intrinsic correlation function file.
\item {\bf [-F First\_Guess]} \\
Input solution file.
\item {\bf [-W DataWeightingType]} 
\begin{enumerate}
\baselineskip=0.4truecm
\item no weighting 
\item soft weighting
\item hard weighting 
\end{enumerate}
Default is 3.
\item {\bf [-A RegulProtectType]}
\begin{itemize}
\baselineskip=0.4truecm
\itemsep=0.1truecm
\item{0: } no regularization (all protected) 
\item{1: } no protection from regularization 
\item{2: } soft protection 
\item{3: } hard protection 
\item{4: } soft + hard protection 
\end{itemize}
Default is 3.
\item {\bf [-G RegulParam]} \\
Regularization parameter. Default is 1.
\item {\bf [-M NSigmaObj]} \\
NSigma level for object multiresolution determination.
Default is 3. With hard weighting (-W 3), default is 1.

\item {\bf [-r residual\_file\_name]} \\
 Write the residual to disk. 
\item {\bf [-S]} \\
Do not shift automatically the maximum of the PSF at the center.
\item {\bf [-v]} \\
Verbose. Default is no.
\end{itemize}
\subsubsection*{Example:}
\begin{itemize}
\baselineskip=0.4truecm
\itemsep=0.1truecm
\item mw\_deconv image\_in.d psf ima\_out.d \\
 Deconvolve an image using default options
\item mw\_deconv -W 3 -A 0  image\_in.d psf ima\_out.d \\
Deconvolution from the multiresolution support.
\item mw\_deconv -A 3 image\_in.d psf ima\_out.d \\
Regularization with a protection of high SNR wavelet coefficients.
\item mw\_deconv -A 3 -G 2.5 image\_in.d psf ima\_out.d \\
Ditto, but increases the regularization (solution more smooth).
\item mw\_deconv -f 3 image\_in.d psf ima\_out.d \\
Impose that the solution is the result of the convolution product between
a Gaussian (full-width at half maximum equal to 3) and a hidden solution.
\end{itemize}
