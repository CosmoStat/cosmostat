
\chapter{Multiscale Entropy Theory}
\label{ch_entrop}
\index{entropy}
\section{Entropy and Image Restoration}

\subsection{Introduction}

The term ``entropy'' is due to Clausius (1865), and the concept of 
entropy was introduced by Boltzmann into statistical mechanics
in order to measure the number of microscopic ways that a given macroscopic
state can be realized. Shannon (1948) \cite{ima:shannon48} founded the mathematical theory of
communication when he suggested that the information gained in a 
measurement depends on the number of possible outcomes out of 
which one is realized. Shannon also suggested that 
the entropy can be used for maximization of the bit transfer rate under
a quality constraint. Jaynes (1957) \cite{entropy:jaynes57} 
proposed to use the entropy measure
for radio interferometric image deconvolution,  in
order to select between a set of possible solutions that which contains the
minimum of information or, following his entropy definition, that 
which has maximum entropy. In principle, the solution verifying such 
a condition should be the most reliable. A great deal of work has been 
carried out 
in the last 30 years on the use of the entropy for the general problem
of data filtering and deconvolution 
\cite{entropy:ables74,entropy:bontekoe94,entropy:burg67,entropy:frieden75,entropy:gull91,entropy:narrayan86,starck:pan96,entropy:skilling89,entropy:weir92,entropy:djafari94,entropy:djafari98}. 

Traditionally information and entropy are determined from events and the
probability of their occurrence.  Signal and noise are basic building-blocks 
of signal and data analysis in the physical sciences.  Instead of the 
probability of an event, we are led to consider the probabilities of our
data being either signal or noise. 

Observed data $Y$ in the physical sciences 
are generally corrupted by noise, which is often additive and which 
follows in many cases a Gaussian distribution, a Poisson distribution, or
a combination of both.  Other noise models may also be considered.
 Using Bayes' theorem to evaluate the probability distribution of the 
realization of the original signal $X$,
knowing the data $Y$, we have
 
\begin{eqnarray}
 \mathrm{p}(X|Y) = \frac{\mathrm{p}(Y|X).\mathrm{p}(X)}{\mathrm{p}(Y)}
\label{eqn_bayes}
\end{eqnarray}
$\mathrm{p}(Y|X)$ is the conditional probability distribution of getting the data 
$Y$ given an original signal $X$, i.e.\ it represents the distribution 
of the noise. It is given, in the case of uncorrelated Gaussian 
noise with variance $\sigma^2$, by:
\begin{eqnarray}
 \mathrm{p}(Y|X) = \mathrm{exp} \left\{
-\sum_{pixels} \frac{ (Y-X)^2}{2{\sigma}^2} \right\}
\label{eqn_proba}
\end{eqnarray}
The denominator in  equation \ref{eqn_bayes} is independent of $X$ and
 is considered as a constant (stationary noise). 
$\mathrm{p}(X)$ is the a priori distribution 
of the solution $X$. In the absence of any information on the solution 
$X$ except its positivity, a possible course of action 
is to derive the probability
of $X$ from its entropy, which is defined from information theory.

The main idea of information theory \cite{ima:shannon48} is to establish
a relation between the received information and the probability of
the observed event \cite{ima:bijaoui84}. If we note ${\cal I}(E)$ the information
related to the event $E$, and $p$ the probability of this
event happening, then we consider that
\begin{eqnarray}
%\cal{I}(E) =    f(p)       
{\cal I}(E) = f(p)
\end{eqnarray}

Then we assume the two following principles:
\begin{itemize}
\item The information is a decreasing function of the probability. This 
implies that the more information we have, the  less will be the probability 
associated with one event.
\item Additivity of the information. If we have two independent events 
$E_1$ and $E_2$, the information ${\cal I}(E)$ associated with the happening
of both is equal
to the addition of the information of each of them.
\begin{eqnarray}
{\cal I}(E) = {\cal I}(E_1) + {\cal I}(E_2)
\end{eqnarray}
\end{itemize}

Since $E_1$ (of probability $p_1$) and $E_2$ (of probability $p_2$) are 
independent, then the probability of both happening is equal to the
product of $p_1$ and $p_2$.  Hence
\begin{eqnarray}
f(p_1 p_2) = f(p_1) + f(p_2) 
\end{eqnarray}

Then we can say that the information measure is
\begin{eqnarray}
{\cal I}(E) = k \ln(p)
\end{eqnarray}
where k is a constant. Information must be positive, and $k$
is generally fixed at $-1$.

Another interesting measure is the mean information which is denoted
\begin{eqnarray}
H = - \sum_i p_i \ln(p_i)
\end{eqnarray}
This quantity is called the entropy of the system and was established by 
Shannon in 1948 \cite{ima:shannon48}.

This measure has several properties:
\begin{itemize}
\item It is maximal when all events have the same probability 
$p_i = 1/ N_e$ ($N_e$ being the number of events), and is equal to 
$\ln(N_e)$. It is in this 
configuration that the system is the most undefined.
\item It is minimal when one event is sure. In this case, the system is 
perfectly known, and no information can be added.
\item The entropy is a positive, continuous, and symmetric function.
%\item The mean information, obtained in two steps, can be added.
\end{itemize}

If we know the entropy $H$ of the solution (the next section 
describes different ways to calculate it), 
we derive its probability by
\begin{eqnarray}
\mathrm{p}(X) = \mathrm{exp}(- \alpha H(X))
\label{info_prop}
\end{eqnarray}

Given the data, the most probable image is obtained by maximizing
$\mathrm{p}(X|Y)$. Taking the logarithm of equation \ref{eqn_bayes}, we 
thus need to maximize
\begin{eqnarray}
 \ln (\mathrm{p}(X|Y))  = - \alpha  H(X) + \ln(\mathrm{p}(Y|X)) - 
\ln(\mathrm{p}(Y))
\end{eqnarray}
The last term is a constant and can be omitted.
Then, in the case of Gaussian noise, the solution is found by minimizing 
\begin{eqnarray}
J(X) = \sum_{pixels} \frac{{(Y-X)}^{2}}{2 {\sigma}^{2}} + {\alpha} H(X)
= \frac{{\chi}^2}{2} + {\alpha} H(X)
\label{eqn_j1}
\end{eqnarray}
which is a linear combination of two terms: the entropy of the signal,
and a quantity corresponding to ${\chi}^2$ in statistics measuring the
discrepancy between the data and the predictions of the model.
$\alpha$ is a parameter that can be viewed alternatively as 
a Lagrangian parameter or a value fixing the relative weight between 
the goodness-of-fit and the entropy H. 

For the deconvolution problem, the object-data relation is given by the
convolution
\begin{eqnarray}
Y = P * X
\end{eqnarray}
where $P$ is the point spread function, and the solution is found (in the case
of Gaussian noise) by minimizing
\begin{eqnarray}
J(X) = \sum_{pixels} \frac{{(Y-P*X)}^{2}}{2 {\sigma}^{2}} + {\alpha} H(X)
\end{eqnarray}

The way the entropy is defined is fundamental, because from its definition
will depend the solution. The next section discusses the different approaches 
which have been proposed in the past.

\clearpage
\newpage

\subsection{The concept of entropy}
\label{sect_entr}
We wish to estimate an unknown probability density $p(X)$ of the data.
 Shannon \cite{ima:shannon48}, in the framework of the information 
 theory, has defined the entropy of an image $X$ by 
\begin{eqnarray}
H_s(X) = - \sum_{k=1}^{N_b} p_k \log p_k
\end{eqnarray}
where  $X=\left\{X_1,.. X_N \right\}$ is an image 
containing integer values, $N_b$ is number of possible values which can 
take a given pixel $X_k$ 
(256 for a 8 bits image), and 
 $p_k$ values are derived the histogram of $X$:
\begin{eqnarray}
p_k = {  \mbox{\#} X_j = k \over  N} 
\end{eqnarray}
$\mbox{\#} X_j = k $ giving the number of pixels  $X_j = k$.

If the image contains floating values, it is possible to
to build up the histogram $L$ of values $L_i$, using
a suitable interval $\Delta$, counting up how many times $m_k$ each interval
$(L_k, L_k + \Delta)$ occurs among the N occurrences. Then the probability
that a data value belongs to an interval $k$ is $p_k = \frac{m_k}{N}$, and
each data value has a probability $p_k$.  

The entropy is minimum and equal to zero when the signal is flat, and
increases when we have some fluctuations. Using this entropy in 
equation~\ref{eqn_j1} leads to minimize:
\begin{eqnarray}
J(X) = \frac{{\chi}^2}{2} + {\alpha} H_s(X)
\label{eqn_j2}
\end{eqnarray}
It is a minimum entropy restoration method.
 
The trouble with this approach is that, because the number of occurrences is
finite, the estimate $p_k$ will be in error by an amount proportional
to $m_k^{-\frac{1}{2}}$~\cite{entropy:frieden91}. The error becomes significant when
$m_k$ is small. Furthermore this kind of entropy definition is not
easy to use for signal restoration, because the gradient of 
equation~\ref{eqn_j2} is not easy to compute. 
For these reasons, other
entropy functions are generally used. The main ones are:
\begin{itemize}
\item Burg \cite{entropy:burg67}:
\begin{eqnarray}
H_b(X) = -\sum_{k=1}^N \ln(X_k) 
\end{eqnarray}
\item Frieden \cite{entropy:frieden75}:
\begin{eqnarray}
H_f(X) = -\sum_{k=1}^N X_k \ln(X_k)
\end{eqnarray}
\item Gull and Skilling \cite{entropy:gull91}:
\begin{eqnarray}
H_g(X) = \sum_{k=1}^N  X_k - M_k - X_k \ln({X_k \over M_k})
\end{eqnarray}
where $M$ is a given model, usually taken as a flat image
\end{itemize}
where $N$ is the number of pixels, and $k$ represents an index pixel.

Each of these entropies can be used, and they correspond to different
probability distributions that one can associate with 
an image \cite{entropy:narrayan86}.
(See \cite{entropy:frieden75,entropy:skilling89} for descriptions).
The last definition of the entropy has the advantage of having a zero
 maximum when $X$ equals the model $M$. 
All of these entropy measures are negative (if $X_k > 1$), 
and maximum when the image is flat.
They are negative because an offset term is omitted which has no importance
for the minimization of the functional. The fact that we consider that
a signal has maximum information value when it is flat is evidently
a curious way to measure information. A consequence is that we must
now maximize the entropy if we want a smooth solution, and
the probability of $X$ 
must be redefined by:
\begin{eqnarray}
\mathrm{p}(X) = \mathrm{exp}(\alpha H(X))
\end{eqnarray}
The sign has been inverted
(see equation~\ref{info_prop}), which is natural if we want the best
solution to be the smoothest. These three entropies, above, lead to the 
Maximum Entropy Method method (MEM), for which the 
solution is found by minimizing 
(for Gaussian noise)
\begin{eqnarray}
J(X) = \sum_{k=1}^N \frac{{(Y_k-X_k)}^{2}}{2 {\sigma}^{2}} - {\alpha} H(X)
\label{eqn_j3}
\end{eqnarray}

These different entropy functions  which have
been proposed for image restoration have the property of being maximal when
the image is flat, and of decreasing when we introduce some information.
So minimizing the information is equivalent to maximizing the entropy, and
this has led to the well known Maximum Entropy Method (MEM). For the Shannon
entropy (which is obtained from the histogram of the data), 
this is the opposite. The entropy is null for a flat image, and increases
when the data contains some information. So, if the Shannon entropy were 
used for restoration, this would lead to a Minimum Entropy Method.

In 1986, Narayan and Nityanda \cite{entropy:narrayan86} 
compared several entropy functions,
 and finally 
concluded by saying that all were comparable if they have good
properties, i.e.\ they enforce positivity, and they have a negative 
second derivative which discourages ripple. They showed also that 
results varied strongly with the background level, and
that these entropy functions produced poor results
for negative structures, i.e.\ structures under the background level
(absorption area in an image, absorption band in a spectrum, etc.), and 
compact structures in the signal.
The Gull and Skilling entropy gives rise to  
the difficulty of estimating a model.
Furthermore it has been shown \cite{entropy:bontekoe94} 
that the solution is dependent on this choice.

The determination of the $\alpha$ parameter is also not an easy task and in 
fact it is a very serious problem facing the maximum entropy method.
In the historic MAXENT algorithm of Skilling and Gull, the choice of $\alpha$ 
is such that it must satisfy the ad hoc constraint $\chi^2=N$ when 
the deconvolution is achieved, $N$ being
 the number of degrees of freedom of the system i.e.\ the number of pixels 
in image deconvolution problems.
But this choice systematically leads to an under-fitting of the data 
 \cite{entropy:titterington85} which is clearly apparent for imaging problems 
with little blurring. In reality, the $\chi^2$ statistic is expected to 
vary in the range $N\pm\sqrt{2N}$ from one data realization to another.
 In the Quantified Maximum Entropy point of view \cite{entropy:skilling89}, the 
optimum value of $\alpha$ is determined by including its probability 
P($\alpha$) in Bayes' equation and then by maximizing the marginal 
probability of having $\alpha$, knowing the data and the model $m$.
 In practice, a value of $\alpha$ which is too large gives a resulting image 
which is too regularized
\index{regularization}
with a large loss of resolution.  A value which is too small
leads to a poorly regularized solution showing unacceptable artifacts. 
Taking a flat model of the prior image softens the discontinuities 
which may appear unacceptable for astronomical images often containing 
stars and other point-like objects. Therefore the basic maximum entropy 
method appears to be not very
appropriate for this kind of image which contains high and low spatial 
frequencies 
at the same time. Another point to be noted 
is a ringing effect of the maximum entropy method algorithm,
producing artifacts around bright sources.

 To solve these problems while still using the maximum entropy concept, some 
enhancements of the maximum entropy method have been proposed.
Noticing that neighboring pixels of reconstructed images with MAXENT 
could have values differing a lot in expected flat regions \cite{entropy:charter89}, 
Gull and Skilling introduced the concepts of hidden image $S$ and intrinsic 
 correlation function $C$ (Gaussian or cubic spline-like) 
in the Preblur MAXENT algorithm.
\index{maximum entropy method}
\index{MEM}
\index{intrinsic correlation function}

 The ICF describes a minimum scale length of correlation in the desired 
image $O$ which is achieved by assuming that
\begin{eqnarray}
 O=C*S
\end{eqnarray}
This corresponds to imposing a  
minimum resolution on the solution $O$. 
Since the hidden space image $S$ is not 
spatially correlated, this can be regularized by the entropy 
\begin{eqnarray}
H_g(h)=\sum_{k=1}^N  S_k - M_k - S_k \ln(\frac{S_k}{M_k})
\end{eqnarray}

Since in astronomical images many scale lengths are present, the 
{\it Multi-channel Maximum Entropy Method}, developed by Weir 
\index{maximum entropy method}
\index{MEM}
\cite{entropy:weir91,entropy:weir92}, uses a set of 
ICFs having different scale lengths, each defining a channel. The 
visible-space image is now formed by a weighted sum of 
 the visible-space image
channels $O_j$:
   
\begin{eqnarray}
  O= \sum_{j=1}^{N_c} p_j O_j
\end{eqnarray}
where $N_c$ is the number of channels.
Like in Preblur MAXENT, each solution $O_j$ is supposed to be the result of the
convolution between a hidden image $S_j$ with a low-pass filter (ICF) $C_j$:
\begin{eqnarray}
O_j = C_j * S_j
\end{eqnarray}

But such a method has several drawbacks:
\begin{enumerate}
\item The solution depends on the width of the ICFs \cite{entropy:bontekoe94}.
\item There is no rigorous way to fix the weights $p_j$ \cite{entropy:bontekoe94}.
\item The computation time increases linearly with the number of pixels.
\item The solution obtained depends on the choice of the models $M_j$ 
($j = 1 \dots N_c$) which were chosen independently of the channel.
\end{enumerate}
\index{intrinsic correlation function}

In 1993, Bontekoe et al.\ \cite{entropy:bontekoe94} used a special application 
of this method which 
they called Pyramid Maximum Entropy on infrared image data.
\index{pyramid}
The pyramidal approach allows the user to have constant ICF width, and the 
computation time is reduced. It is demonstrated \cite{entropy:bontekoe94} that
all weights can be fixed ($p_j = 1$ for each channel). 

This method eliminates the first three drawbacks, and
gives better reconstruction of the sharp 
and smooth structures. But in addition 
to the two last drawbacks, a new one is added:
as the images $O_j$ have different sizes (due to the pyramidal approach),
\index{pyramid}
the solution $O$ is built by duplicating the pixels of 
the subimages $O_j$ of each channel. This procedure is known to produce 
artifacts due to the  appearance of high frequencies which are
incompatible with the 
real spectrum of the true image $\hat{O}$.

However this problem can 
be easily overcome by duplicating the pixels before convolving with the ICF, 
or expanding the channels using linear 
interpolation. Thus the introduction of the ``pyramid of resolution'' has 
solved some problems and brought lots of improvements to the classic maximum
entropy method, but has also raised other questions. In order to derive the
model from a physical value, Pantin and Starck \cite{starck:pan96} 
introduced the wavelet 
transform, and defined entropy as follows:
\begin{eqnarray}
H(O) =  \frac{1}{\sigma_I^2}\sum_{j=1}^l \sum_{k=1}^{N_j} \sigma_j( w_{j,k}- M_{j,k}- |w_{j,k}|\ln{\frac{|w_{j,k}|}{M_{j,k}}})
\label{eqn_entr}
\end{eqnarray}
where $\sigma_I$ is the noise standard deviation in the data, $l$ is the number of scales, and
$N_j$ is the number of samples in the band $j$
($N_j = N$ for the \`a trous algorithm). 
The multiscale entropy is the sum of the entropy at each scale.

The coefficients $w_{j,k}$ are wavelet coefficients, and we 
take the absolute value of $w_{j,k}$ in this definition because the  
values of $w_{j,k}$ can be positive or negative, and a negative signal 
contains also some information in
the wavelet transform. 

The advantage of such a definition of entropy is
 the fact we can use previous work concerning the wavelet transform and
image restoration \cite{starck:mur95_2,starck:sta94_1,starck:sta94_4}. 
The noise behavior has already been studied in the wavelet transform 
and we can estimate the standard deviation of the noise $\sigma_j$ 
at scale $j$. These estimates can be naturally introduced in our 
models $m_j$
\begin{eqnarray}
M_{j,k} = k_{m} \sigma_j
\end{eqnarray}
 The model $M_j$ at scale $j$ represents 
the value taken by a wavelet coefficient in the absence of any relevant
 signal and, in  practice, it must be a small value compared to 
any significant signal value.
Following the Gull and Skilling procedure, we take $M_j$ as a fraction of the 
noise because the value of 
$\sigma_j$ can be considered as a sort of physical limit under which a signal 
cannot be distinguished from the noise ($k_m = \frac{1}{100}$).

\subsection{Conclusion}
\label{sect_5pt}

As described above, many studies   
have been carried out in order to improve the functional to be minimized.
But the question which should be raised is: what is a good entropy for 
signal restoration?

Trying to answer, this corresponds to asking what is the information
in the signal. We first assume that a signal $X$ can be decomposed in
several components:
\begin{eqnarray}
 X = S + B + N
\end{eqnarray}
where $S$ is the signal of interest, $B$ lis the background, and $N$  the noise.

The entropy should verify the following criteria \cite{starck:sta98_2}:
{\bf
\begin{enumerate}
\item The information in a flat signal is zero ($S=0$, $N=0$ et $B=\mathrm{Cst}$). 
\item The amount of information in a signal is independent of the background
($H(X)$ is independent of $B$).
\item The amount of information is dependent on the noise 
($H(X)$ is dependent of $N$). 
A given signal $X$ doesn't furnish the  same information if 
the noise $N$ is high or small.
\item The entropy must work in the same way for a pixel which
has a value $B + \epsilon$, and
for a pixel which has a value $B - \epsilon$.
$H(X)$ must be a function of the absolute value of $S$ instead of $S$.
\item The amount of information is dependent on the correlation in the signal.
If the signal $S$  presents large features above the noise, it contains
a lot of information. By generating a new set of  data from $S$, by 
randomly taking the pixel values in $S$, the large features will
evidently disappear, and this new signal will contain less information.
But the pixel values will be the same as in $S$.
\end{enumerate}
}
% \begin{figure}[htb]
% \centerline{
% \hbox{
% \psfig{figure=lenna256.ps,bbllx=1.8cm,bblly=12.9cm,bburx=14.5cm,bbury=25.5cm,width=8cm,height=8cm,clip=}
% \psfig{figure=scrambled_lenna.ps,bbllx=1.8cm,bblly=12.9cm,bburx=14.5cm,bbury=25.5cm,width=8cm,height=8cm,clip=}
% }}
% \caption{Lena image (left) and the same data distributed differently (right). 
% These
% two images have the same entropy, using any of the standard entropy methods.}
% \label{fig_lenna}
% \end{figure}
\begin{figure}[h]
\centerline{
\vbox{
\hbox{
\psfig{figure=fig_saturn.ps,bbllx=1.7cm,bblly=12.9cm,bburx=11.2cm,bbury=25.6cm,width=9.cm,height=12.3cm,clip=}
\psfig{figure=fig_saturn_scramble.ps,bbllx=1.7cm,bblly=12.9cm,bburx=11.2cm,bbury=25.6cm,width=9.cm,height=12.3cm,clip=}
% \vspace{21cm}
}
}}
\caption{Saturn image (left) and the same data distributed differently (right). 
These
two images have the same entropy, using any of the standard entropy 
definitions.}
\label{fig_saturn}
\end{figure}
 
Fig.~\ref{fig_saturn} illustrates the last point perfectly. 
The second image is obtained  by distributing randomly the Saturn image pixel 
values, and the standard entropy definitions  produce the same information
measurement for both images. The concept of information becomes really
subjective, or at least it depends on the application domain. Indeed, for 
someone who is
not involved in image processing, the second image contains 
{\em less} information
than the first one. For someone working on image transmission, it is clear
that the second image will require more bits for lossless transmission,
and from this point of view, he/she will consider that the second 
image contains
{\em more} information. Finally, for data restoration, all fluctuations
due to noise are not of interest, and do not contain relevant 
information. From this physical point of view, 
the standard  definition of entropy seems badly adapted to information 
measurement in signal restoration.

% \clearpage
% \newpage
