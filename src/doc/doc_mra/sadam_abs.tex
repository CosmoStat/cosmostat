\documentstyle[11pt]{article}

\textwidth 16.5truecm
\textheight 23.5truecm
\oddsidemargin=0pt
\evensidemargin=0pt

\hoffset -1truecm
\voffset -1.8truecm
\input psfig.tex

\pagestyle{empty}

\title{System for Astronomical Data Analysis using  Multiresolution}

\vspace{1cm}
\author{J.L. Starck \\ [12pt] 
 CEA, DSM/DAPNIA, CE-Saclay, F-91191 Gif-sur-Yvette Cedex, France}

\date{\today}

\begin{document}

\maketitle

SADAM (System for Astronomical Data Analysis using Multiresolution) is a 
software developed by CEA (Saclay-France) and Nice Observatory. The
goal of the project is to propose to the astronomical community 
a set of programs using multiresolution. Multiresolution technics have
been developed during the last years, and furnish a powerful representation
of the data. By this way, an image can be decomposed in a set 
of images (or scales), each scale containing only structures of a given size. 
This data representation, associated with a noise modeling, 
have been applied for different applications such data filtering, 
deconvolution, compression, objects detection, images registration ... 
The good quality of the results furnished by methods using the multiresolution 
is due the fact that multiresolution allows us  
to better understand how the information is distributed in an
astronomical image, and how the signal can be separated from the noise.

The SADAM software includes almost all applications presented in the book 
\cite{starck97}. SADAM goal is not to replace existing image
 processing  package, but to complement
them, offering the user a complete set of multiresolution tools. These tools
are executable programs, which work on different platforms, outside of any 
image processing system, and allows the users to perform some treatments
using multiresolution on their images like a wavelet transform, a filtering, 
a deconvolution, ...

The programs, written in C++, are builded upon three classes: the  
"image" class, the "multiresolution" class, and the "noise\_modeling" class.
The "image" class includes routines for reading/writing fits and MIDAS
images, and a lot of standard image processing routines. The "multiresolution"
class integrates eighteen different multiresolution transforms 
(orthogonal wavelet transform, \`a trous algorithms, pyramidal algorithms,
non linear multiresolution transform, ...). The "noise\_modeling" class
allows to determine whether at a given position of the image, 
and at a given scale, the multiresolution coefficients can be due to noise.
Several kind of noise can be considered (Gaussian noise, Poisson noise, 
Gaussian + Poisson noise, ...). From these three classes, the multiresolution
support of an image can be derived, which is then used by programs 
for the different applications. These classes can also
be used by a C++ programmer in order to built his own programs.

A graphical user interface has been developed for interactive 
wavelet coefficients analysis (the interface uses X11 and MOTIF), 
and a set of IDL (Interactive Data Language) routines
are including in the package which make the interface between IDL and 
the executables. 

\begin{thebibliography}{99}
\bibitem{starck97} J.L. Starck, F. Murtagh, and A. Bijaoui,  "Image Processing
 and Data Analysis in the Physical Sciences: The
        Multiscale Approach",  Cambridge University Press, forthcoming.
\end{thebibliography}


\end{document}
