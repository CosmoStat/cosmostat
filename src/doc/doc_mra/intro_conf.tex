\documentstyle[12pt]{article}
\setlength{\parskip}{\medskipamount}
\overfullrule 0pt
\tolerance 100000
\textwidth 15.2truecm
\textheight 23truecm
\hoffset -1truecm
\voffset -1.5truecm

\pagestyle{empty}

\title{Analyse en Ondelettes et Applications en Astronomie}

\author{Jean-Luc Starck \\ [12pt]
CEA/DSM/DAPNIA, CE-SACLAY, F-91191 Gif sur Yvette Cedex}
\date{\today}

\begin{document}
\maketitle

La transform\'ee en ondelettes permet de repr\'esenter des donn\'ees 
diff\'eremment des m\'ethodes traditionnelles. Elle fournit une vision
multi-\'echelles, c'est \`a dire que les structures composant le signal
sont s\'epar\'ees suivant leur taille. 
Nous pr\'esenterons dans une premi\`ere partie les principes de la 
transform\'ee en ondelettes, ainsi que les diff\'erents algorithmes
de transformation. Puis nous verrons comment le signal peut-\^etre
d\'etect\'e dans les diff\'erentes \'echelles par une mod\'elisation
du bruit. Enfin, nous passerons en revue un certain nombre d'applications
telles que la restauration d'images, la compression d'images, l'extraction
de sources dans les images astronomiques ...

\end{document}
