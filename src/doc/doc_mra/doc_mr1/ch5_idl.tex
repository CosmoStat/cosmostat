
\chapter{IDL Routines}
\label{ch_idl}
%\chapterhead{IDL Routines}
\markright{IDL Routines}

\section{Introduction}
A set of routines has been developed in IDL. Starting IDL using
the script program {\em mre} allows the user to get the multiresolution
environment, and all routines
described in the following can be called. An online help facility 
is also available by
invoking the {\em mrh} program under IDL.

\section{IDL Tools}
\subsection{del\_pattern}
\label{pat}
Suppress a pattern in an image. A pattern is considered as a peak in 
Fourier space. So it can be iteratively eliminated.
{\bf
\begin{center}
     USAGE: output = del\_pattern(Image, n\_iter=n\_iter, pattern=pattern, disp=disp, NSigma=NSigma, NbrScale=NbrScale)
\end{center}}
where
\begin{itemize}
\item {\em Image} is an IDL 2D array.
\item {\em n\_iter} is the number of iterations. Default is 3.
\item if {\em disp} set, then results are displayed during the iterations.
\item {\em output} is an IDL 2D array. It contains the input image, without
the pattern.
\item {\em pattern} is an IDL 2D array. It is equal to the difference between
{\em Image} and {\em output}.
\item {\em NSigma} is a real value (default is 3.0).
\item {\em NbrScale} is the number of scales used in the wavelet transform.
\end{itemize}


\subsection{block\_trans}
This routine separates the image into blocks of size {\em BlockSize} 
$\times$ {\em BlockSize}
and computes inside each block the mean and the variance. The 
variance versus mean plot gives information about the type of noise in
the data.
 
{\bf
\begin{center}
     USAGE: block\_trans, Image, TabVar=TabVar, TabMean=TabMean,  Plot=Plot, BlockSize=BlockSize
\end{center}}
where
\begin{itemize}
\item {\em Image} is an IDL 2D array.
\item {\em TabVar} is an output keyword and is the array of variances.
\item {\em TabMean} is an output keyword and is the array of means.
\item {\em Plot} is an input keyword. If set, the curve variance versus
mean is plotted.
\item {\em BlockSize} is the size of the blocks. Default is 8.
\end{itemize}

\subsection{delete}
Suppress a file in working directory.
{\bf
\begin{center}
     USAGE: delete, filename
\end{center}}

\subsection{im\_smooth}
Smooth an image, taking into account the border.
{\bf
\begin{center}
     USAGE:  im\_smooth, Im\_in, Im\_out, method=method, Step\_Trou=Step\_Trou, Border=Border, WinSize=WinSize
\end{center}}
where
\begin{itemize}
\item {\em Im\_in} is the input 2D IDL array.
\item {\em Im\_out} is the output 2D IDL smoothed array.
\item {\em method}: keyword specifying the method for smoothing. 
Available methods are ``linear'', ``bspline'', or ``median''. Default is 
``median''.
\item {\em Step\_Trou}: $2^{Step\_Trou}$ is the  distance between 
two adjacent 
pixels (default is 0).
\item {\em Border} defines the way  the borders of the image must be managed.
Available methods are: ``cont'' ($Im_{in}(-i,-j) = Im_{in}(0,0)$), ``mirror''  
($Im_{in}(-i,-j) = Im_{in}(i,j)$), ``zero'' ($Im_{in}(-i,-j) = 0$), ``nobord'' (the
borders of the image are not 
smoothed).
\item {\em WinSize} is the window size (only used if method = 
``median''). Default is 3.
\end{itemize}

\subsection{Tikhonov}
 Deconvolve  an image using Tikhonov regularization.
 This program does not shift the maximum of the PSF to the center. 
 The minimized functional is:
\begin{eqnarray}
  J(O) = \parallel I - P*O  \parallel + \alpha  \parallel H*O  \parallel
 \end{eqnarray}
where $O$ is the unknown object, $P$ the point spread function (PSF), $I$
the data and $H$ a high pass filter (the Laplacian).
{\bf
\begin{center}
  USAGE: Result = TIKHONOV(Imag, Psf, residu=residu, niter=niter, 
                         CvgParam=CvgParam, RegulParam=RegulParam, 
                         NoRegul=NoRegul)
 \end{center}}
 where
\begin{itemize}
\item {\em Imag} is an input 2D IDL  array, i.e.\ the image to deconvolve.
\item {\em Psf}  is an input 2D IDL  array, i.e.\ point spread function.
\item {\em niter} is the input number of iterations. Default is 10.
\item {\em CvgParam} is the convergence parameter. Default is 0.1.
\item {\em RegulParam} is the regularization parameter (i.e.\ $\alpha$).
\item {\em NoRegul} if set, no regularization is performed.
\item {\em Residu} is an output 2D IDL array containing the residual: 
residual = Imag -- Psf*Result.
\end{itemize}
 
If the keyword NoRegul is set, then no regularization is done, 
and the algorithm becomes a simple one step gradient method.
\subsubsection*{Examples:}
\begin{itemize}
\item Result = TIKHONOV(Imag, Psf) \\
Deconvolve an image with all default options.
\item  Result = TIKHONOV(Imag, Psf) \\
 Same example, but impose the number of iterations to be 30.
\item   Result = TIKHONOV (Imag, Psf, niter=30) \\
 Deconvolution by the one step gradient method, without 
 any regularization, with 30 iterations.
\item  Result = TIKHONOV (Imag, Psf, niter=30, /NoRegul)  
\end{itemize}
  
\section{Noise Related Routines}
\subsection{poisson\_image}
Create an image with Poisson-distributed values around the mean values
IMAGE, using Knuth's ``Algorithm Q"  (D.E. Knuth, The Art of
Computer Programming, Volume 2, ``Seminumerical Algorithms", 
Addison-Wesley, 1969, p.\ 117). This routine has been written by James Hamill
(Siemens Medical Systems, 2501 N. Barrington Rd., Hoffman Estates, IL  
60195-7372), and has been inserted in the \proj package.
{\bf
\begin{center}
     USAGE: output = poisson\_image ( image[, seed] )
\end{center}}
where {\em image} is a  numeric array (byte, integer, long, or float) of arbitrary dimensionality.  
This is the array of values around which values in the
result will be Poisson-distributed. {\em seed} is a  longword seed for the random number generator.  If this is not
supplied, the value --123456789L is used for generating the first random
value. 

\subsection{poisson\_to\_gauss}
Transforms the data with Poisson noise into a new set of data
 with Gaussian noise using the generalized Anscombe transform.
\begin{eqnarray}
T(x) = \frac{2}{gain} \sqrt{gain*x + \frac{3}{8}*gain^2 + \sigma^2 - gain * mean}
\end{eqnarray}
where
\begin{itemize}
\item {\em gain} = gain of the CCD
\item $\sigma$ = standard deviation of the readout noise
\item {\em mean} = mean of the read out noise
\end{itemize}
For pure Poisson noise, {\em gain}=1, {\em mean}=0, and $\sigma$=0.
If the {\em inv} keyword is set, then the inverse transform is applied:
\begin{eqnarray}
 T(x) = \frac{x^2}{4.} * \textrm{\em gain} - \frac{\frac{3}{8}*\textrm{\em gain}^2 + \sigma^2 - gain * mean}{gain}
\end{eqnarray}

{\bf
\begin{center}
     USAGE: output = poisson\_to\_gauss (Data, poisson=poisson, inv=inv)
\end{center}}
\begin{itemize}
\item {\em Data}: IDL array: data
\item {\em poisson}: float array (poisson(0) = gain, poisson(1)= $\sigma$, and
poisson(2) = mean). By default, poisson=[1.,0.,0.].
\item {\em inv}: If set, the inverse transform is applied.
\end{itemize}

\subsection{sigma\_clip}
Returns the noise standard deviation obtained by k-sigma.  
{\bf
\begin{center}
     USAGE: output = sigma\_clip(Data, sigma\_clip=sigma\_clip, mean=mean)
\end{center}}
where
\begin{itemize}
\item {\em output}: standard deviation estimated by k-sigma clipping.
\item {\em Data}: IDL array (input data).
\item {\em sigma\_clip}: number of iterations.  Default value is 3.
\item {\em mean}: If mean is set, the mean of the data (taking into 
account outliers) is returned.
\end{itemize}

\subsection{get\_noise}
Find the standard deviation of white Gaussian noise in the data.
{\bf
\begin{center}
     USAGE: SigmaNoise = get\_noise(Data, Niter=Niter)
\end{center}}
{\em Data} is an IDL array (1D, 2D, or 3D), and ({\em Niter}) is a keyword
which fixes the number of iterations used by the sigma clipping. Default
value is 3.

\section{Display}
\subsection{tvlut}
Display an image with a zoom factor, and the LUT is displayed
The zoom factor is calculated automatically in order to visualize
the image in a window of size 320 $\times$ 320 (for an image 32 $\times$ 
32, the
zoom factor is 320/32 = 10). An offset is automatically calculated
(or is set by keywords) in order to center the image in the IDL window. 
{\bf
\begin{center}
     USAGE: tvlut, Data, Depx=Depx, Depy=Depy
\end{center}}
where
\begin{itemize}
\item {\em Data}: 2D IDL array (input data) to visualize.
\item {\em Depx}: offset in x (default is 50).
\item {\em Depy}: offset in y (default is 80).
\end{itemize}

\subsection{xdisp}
XDISP is a widget program for image analysis. Several operations
can be carried out by using the mouse and pressing buttons:
\begin{itemize}
\baselineskip=0.4truecm
\item QUIT: quit the application
\item LOAD: load a FITS image. 
\item LUT: modify the LUT.
\item PROFILE: examine rows or columns.
\item CURSOR: examine pixel values. If the image format is FITS and if
the header contains astrometric position, then the pixel
position in the sky is given (right ascension and declination).
\item HISTO: plot the histogram.
\item CONTOURS: Contours of the image (isophotes).
\item 3D VISU: three-dimensional representation of the image.
\item INFO: print the min, max, mean, sigma of the image.
\item FFT: compute the Fourier transform and display either the power 
spectrum, the phase, the real part or the imaginary part.
\item PAN: make a zoom. Zoom factors are 2,4,8,1/2,1/4,1/8
\end{itemize}
{\bf
\begin{center}
     USAGE: xdisp, Data [, FitsHeader]
\end{center}}
{\em Data} is a two-dimensional IDL array, and {\em FitsHeader} (string array)
is an optional parameter for astrometric information (in FITS format).

\subsection{x3d}
X3D is a widget program for cube analysis. Several operations
can be carried out by using the mouse and buttons:
\begin{itemize}
\baselineskip=0.4truecm
\item  QUIT: quit the application
\item Temporal Cut: A temporal cut is displayed.
\item Horizontal Cut: A horizontal cut is displayed.
\item Vertical Cut: A vertical cut is displayed.
\item Frame Number: When the user enters a value V followed by carriage return
the image Cube(*,*,V) is displayed.
\item Window Size: Define the number of elements plotted.
\item Slider Frame Number: When the user moves the slider, 
the corresponding frame is displayed.
\end{itemize}

{\bf
\begin{center}
     USAGE: x3d, Data, from=from, to=to
\end{center}}
{\em Data} is an IDL 3D array. The keywords {\em from} and {\em to} allow
the user 
to define a subcube, which can analysed separately with x3d.
\subsubsection*{Example:}
\begin{verbatim}
IDL> HELP,MY_CUBE 
       MY_CUBE         INT       = Array(32, 32, 100)
IDL>  X3D, my_cube,from=[0,21,50],to=[20,49,99] 
\end{verbatim}

\subsection{xdump}
Save the selected window in a Postscript file. The file
name is ``idl.ps''. 
{\bf
\begin{center}
     USAGE: xdump, FILENAME=filename, WIN=win, LANDSCAPE=landscape, 
              SCALE=scale, TITLE=title, GIF=gif
\end{center}}
where keywords are
\begin{itemize}
\item {\em filename}: filename of output Postscript file (default is ``idl.ps")
\item {\em win}: identifier of IDL window (default is active window)
\item {\em landscape}: set landscape orientation
\item {\em title}: add the title at the upper left corner of window
\item {\em scale}: specify a scale to be applied to the entire graph
\item {\em gif}: store the data in GIF format instead of PS format.
 Default title in this case ``idl.gif".
\end{itemize}
 
 

\section{Multiresolution Routines (1D)}
\subsection{mr1d\_atrou}
Computes a multiresolution transform of a 1D Signal by the
\`a trous algorithm.
{\bf
\begin{center}
     USAGE: mr1d\_atrou, Signal, WaveTrans, Nscale, mirror=mirror, linear=linear
\end{center}}
where 
\begin{itemize}
\item {\em Signal}: input one-dimensional IDL array.
\item {\em WaveTrans}: output two-dimensional array (wavelet transform).
\item {\em Nscale}: number of scales (input parameter).
\item {\em mirror}: if set, then mirroring is used at the border. 
\item {\em linear}: if set, then a linear wavelet function is used.
If not set, then a B-spline wavelet function is used.
\end{itemize}
 
\subsection{mr1d\_pavemed}
Computes a multiresolution transform of a 1D Signal by the
multiresolution median algorithm.
{\bf
\begin{center}
     USAGE: mr1d\_pavemed, Signal, MedTrans, Nscale, cont=cont
\end{center}}
where 
\begin{itemize}
\item {\em Signal}: input one-dimensional IDL array.
\item {\em MedTrans}: output two-dimensional array (multiresolution transform).
\item {\em Nscale}: number of scales (input parameter).
\item {\em cont}: if set, consider the data as constant beyond
the borders. 
\end{itemize}
 
\subsection{mr1d\_minmax}
Computes a multiresolution transform of a 1D Signal by the
min-max algorithm.
{\bf
\begin{center}
     USAGE: mr1d\_minmax, Signal, MinMaxTrans, Nscale
\end{center}}
where 
\begin{itemize}
\item {\em Signal}: input one-dimensional IDL array.
\item {\em MinMaxTrans}: output two-dimensional array 
(multiresolution transform).
\item {\em Nscale}: number of scales (input parameter).
\end{itemize}

\subsection{mr1d\_pyrmed}
Computes a multiresolution transform of a 1D Signal by the
pyramidal median algorithm.
{\bf
\begin{center}
     USAGE: mr1d\_pyrmed, Signal, MedTrans, Nscale, TabNp=TabNp, interp=interp, cont=cont
\end{center}}
where 
\begin{itemize}
\item {\em Signal}: input one-dimensional IDL array.
\item {\em MedTrans}: output two-dimensional array (multiresolution transform).
\item {\em Nscale}: number of scales (input parameter).
\item {\em TabNp}: Number of pixels at each scale of the multiresolution
transform (MedTrans(0:TabNp(j)-1) are the multiresolution coefficients
of the scale j).
\item {\em interp}: if set, each scale is interpolated to the size of the 
input signal.
\item {\em cont}: if set, consider the data as constant outside
the borders. 
\end{itemize}
 
\subsection{mr1d\_paverec}
Reconstruct a signal from its multiresolution transform (\`a trous algorithm
or multiresolution median transform).
{\bf
\begin{center}
     USAGE: mr1d\_paverec, MR\_Trans, Rec\_Signal, Adjoint=Adjoint, 
mirror=mirror, nosmooth=nosmooth
\end{center}}
where 
\begin{itemize}
\item {\em MR\_Trans}: input two-dimensional 
IDL array (multiresolution transform).
\item {\em Rec\_Signal}: output one-dimensional 
IDL array. This is the reconstructed signal.
\item {\em Adjoint}: if set, the adjoint operator is used for the 
reconstruction.
\item {\em nosmooth}: if set, the last scale is not used.
\item {\em interp}:  if set, and if adjoint is set, 
then mirroring is used beyond the borders.
\end{itemize}

\subsection{mr1d\_pyrrec}
Reconstruct a signal from its pyramidal multiresolution transform.
{\bf
\begin{center}
     USAGE: mr1d\_pyrrec, MR\_Trans, Rec\_Signal
\end{center}}
where 
\begin{itemize}
\item {\em MR\_Trans}: input two-dimensional IDL array (multiresolution transform).
\item {\em Rec\_Signal}: output one-dimensional IDL array. This 
is the reconstructed signal.
\end{itemize}
 
\subsection{mr1d\_pyrinterp}
Interpolate each scale of a 1D pyramidal multiresolution transform to the size
of the original signal.
{\bf
\begin{center}
     USAGE: mr1d\_pyrinterp, MR\_Trans
\end{center}}
where {\em MR\_Trans} is a pyramidal multiresolution transform.


\subsection{mr1d\_tabcoef}
Return the pre-computed table for noise behavior in the multiresolution 
transform.
{\bf
\begin{center}
     USAGE: output =  mr1d\_tabcoef(TypeTransform)
\end{center}}
where {\em TypeTransform} is the type of multiresolution transform. Available
types are ``atrou", ``pyrmed" or ``pavemed". The output is a float array which
gives the standard deviation of the noise at each scale, when we apply
a multiresolution transform to a signal following a Gaussian distribution 
with a standard deviation equal to 1. 
 
\subsection{mr1d\_trans}
One-dimensional wavelet transform. 19 transforms are available
 (see \ref{sect_mr1dtrans}), which are grouped in 5 classes:
\bi
\item Class 1: no decimation (transform 1 to 7 and 11 to 14).
\item Class 2: pyramidal transform (transform 8 to 10).
\item Class 3: orthogonal transform (15 and 16).
\item Class 4: Wavelet packets (17 and 18).
\item Class 5: Wavelet packets from the \`a trous algorithm  (19).
\ei
Depending on the class, the transform does not contain the
same number of pixels, and the data representation differs.
{\bf
\begin{center}
     USAGE: MR1D\_Trans, Signal, Result, OPT=OPT, BAND=BAND, NODEL=NODEL 
\end{center}}
where 
\begin{itemize}
\item {\em Signal}: input one-dimensional IDL array.
\item {\em Result}: output IDL structure.
\item {\em Opt}: string which contains the different options 
(see the {\em mr1d\_trans} C++  program).
\item {\em Band}: if set, a tag per band is created in the output structure.
\item {\em Nodel}: if set, the two created file are not deleted: \\
xx\_result.fits: wavelet coefficients file  \\
xx\_info.fits: information about the transform
\end{itemize}
{\em Result} is IDL structure which contains the wavelet transform.
The structure contains the following tags:
{\small
\bi
\item N\_BAND: float; number of bands in the transfrom    
\item INFO: 2D float array (Array[2, NbrBand+3])
\begin{verbatim}
                       info[0,0] = transform number
                       info[1,0] = number of scales
                       info[0,1] = transform class number (5 classes)
                       info[1,1] = number of bands
                                 it is not equal to the number of scales
                                 for wavelet packets transform.
                       info[0,2] = number of pixels
                       info[1,2] = lifting scheme type
                       info[0,3] = type of filter
                       info[1,3] = type of normalization
                       for i=4 to Number_of_bands + 3  
                       info[0,i] = number of pixels in the band i
                       info[1,i] = position number of the pixel of the band
		       If a user filter file is given (i.e. -T 6,filename), 
                       with a filename of $L$ characters, $L$ lines are added 
                       to the array:
                       info[1,Number_of_bands + 4] = number of characters of 
                                                    the filter file name
                        for i=Number_of_bands+4 to  Number_of_bands+4+L-1
                             info[0,i] = ascii number of the ith character.
\end{verbatim}
If a user filter file is given (i.e. -T 6,filename), with a filename 
of $L$ characters, $L$ lines are added to the array:
\begin{verbatim}
        info[1,Number_of_bands + 4] = number of characters of the filter file name
        for i=Number_of_bands+4 to  Number_of_bands+4+L-1
	info[0,i] = ascii number of the ith character.
\end{verbatim}
\item FROM: 1D array (NbrBand); position of the first pixel 
\item TO: 1D array (NbrBand); position of the last pixel
\item COEF: 1D or 2D FLOAT array; wavelet coefficients
     for non-redundant transform (15,16,17), it is a 1D array \\
     for other transform, it is 2D array \\
     class 1 and 5: coeff[*,i] = band i (i in [0..NbrBand-1]) \\
     class 2: coeff[0:to[i],i] = band i \\
     class 3 and 4: coeff[from[i]:to[i]] = band i \\
\item Bandi: if BAND keyword is set, the array coef is also split 
into bands:
\begin{verbatim}
            BAND1   : band 1  
            BAND2   : band 2  
                      ...
            BANDi   : band i  
\end{verbatim}
\ei
}

\subsection{mr1d\_recons}
Reconstruct a one-dimensional signal from its wavelet transform.   
{\bf
\begin{center}
     USAGE:  MR1D\_RECONS, WT\_Struct, result
\end{center}}
where 
\begin{itemize}
\item {\em WT\_Struct}: input  IDL structure (obtained using IDL MR1D\_TRANS program)..
\item {\em Result}: output one-dimensional IDL array.
\end{itemize}

\subsection{mr1d\_filter}
Filter a 1D signal:
\begin{center}
     USAGE: mr1d\_filter, Signal, Result, Opt=Opt
\end{center}
where 
\begin{itemize}
\item {\em Signal}: input  one-dimensional IDL array.
\item {\em Result}: output one-dimensional IDL array (filtered signal).
\item {\em Opt}:  string which contains the different options 
(see the {\em mr1d\_filter} C++  program).
\end{itemize}

\section{Spectral Analysis}

\subsection{mr1d\_continuum}
Estimate the continuum of a 1D signal.
{\bf
\begin{center}
     USAGE: output = mr1d\_continuum(Signal, Nscale,Sigma=Sigma,median=median,mirror=mirror)
\end{center}}
where 
\begin{itemize}
\item {\em Signal}: 1D IDL array. Input spectrum.
\item {\em Output}: 1D IDL array. Estimated continuum.
\item {\em Nscale}: number of scales.
\item {\em median}: if set  use multiresolution median transform
\item {\em mirror}: if set, use mirroring at borders.
\item {\em niter}: number of iterations. Default is 5.
\item {\em Sigma}: noise estimation (output keyword).
\end{itemize}

\subsection{mr1d\_optical\_depth}
Estimate the optical depth of a spectrum.
{\bf
\begin{center}
     USAGE: output = mr1d\_optical\_depth(Signal, Nscale, Sigma=Sigma)
\end{center}}
where 
\begin{itemize}
\item {\em Signal}: 1D IDL array. Input spectrum.
\item {\em Output}: 1D IDL array. Estimated optical depth.
\item {\em Nscale}: number of scales.
\item {\em Sigma}: noise estimation (output keyword).
\end{itemize}

\subsection{mr1d\_detect}
{\bf
\begin{center}
     USAGE: mr1d\_detect, Signal, result, OPT=OPT, print=print, tabobj=tabobj, 
    NbrObj=NbrObj, nodel=nodel, tabw=tabw
\end{center}}
where
\begin{itemize}
\item {\em Signal}: input signal.
\item {\em result}: output signal (sum of all detected objects).
\item {\em OPT}: string which contains the different options 
(see the {\em mr1d\_detect} C++  program).
\item {\em print}: if set, information about each detected object is printed.
\item {\em tabobj}: IDL structure which contains the information about 
the objects. For each object, we have
\begin{itemize}
\item NumObj: Object number 
\item Pos: Position in the signal (in wavelength units). 
\item Sigma: standard deviation of the band (in wavelength units).
\item Fwhm:  Full-width at half-maximum (in wavelength units).
\item PosPix: Position in the signal (in pixel units).
\item SigmaPix: standard deviation of the band (in pixel units).
\item Flux:  integrated flux of the band.
\end{itemize}
\item {\em NbrObj}: number of detected objects
\item {\em nodel}: if set, the created files 
(by the {\em mr1d\_detect} C++ program)
are not deleted.
\item {\em tabw}:  wavelength array
\end{itemize}



\section{Multiresolution Routines (2D)}

\subsection{mr\_transform}
Computes a multiresolution transform of an image. If the
keyword {\em MR\_File\_Name} is set, a file is created
which contains the multiresolution transform. A multiresolution file
has a ``.mr" extension, and if the parameter file name does not 
specify this, then
the extension is added to the file name. Result is stored in the {\em 
DataTransf}. 
Depending on the options, DataTransf can be a cube, an
image, or an IDL structure. This routine calls 
the C++ executable {mr\_transform}. 
The keyword ``OPT" allows to pass to the executable all
options described in section~\ref{sect_trans}.

{\bf
\begin{center}
  USAGE: mr\_transform, Data, DataTransf, MR\_File\_Name=MR\_File\_Name, OPT=Opt
\end{center}}
\subsubsection*{Examples:}
\begin{itemize}
\item mr\_transform, I, Output, MR\_File\_Name='result.mr' \\
Compute the multiresolution of the image {\em I} with default options
(i.e.\ \`a trous algorithm with 4 scales). The result is stored in 
the file ``result.mr".
\item mr\_transform, I, Output, MR\_File\_Name='result\_pyr\_med', 
OPT='-t 10 -n 5' \\
Compute the multiresolution of {\em I}  by using the pyramidal median 
algorithm with 5 scales. The result is stored in the file
``result\_pyr\_med.mr".
\end{itemize}

\subsection{xmr\_transform}
Computes a multiresolution transform of an image. Parameters are
chosen through a widget interface. Creates a file which contains
the multiresolution transform of the image. The transform is made
by calling the routine mr\_transform. If there is an image parameter
the user doesn't need to load an image with the interface.  
{\bf
\begin{center}
   USAGE: xmr\_transform, TransfData, input=input
\end{center}}
{\em TransfData} is the output transform, and {\em input} is the input
image. 


\subsection{mr\_info}
Calculate statistical information about the wavelet transform
of an image. This routine calls 
the C++ executable {mr\_info}. 
The keyword ``OPT" allows to pass to the executable all
options described in section~\ref{sect_trans}.
The output is the 2D IDL array $T(*,0:4)$ with the following
syntax:
\begin{itemize}
\item  T(j,0) = standard deviation of the jth band
\item T(j,1) = skewness of the jth band
\item T(j,2) = kurtosis of the jth band
\item T(j,3) = minimum of the jth band
\item T(j,4) = maximum of the jth band
\end{itemize}

{\bf
\begin{center}
  USAGE: mr\_info, Data, TabStat, OPT=Opt, nodel=nodel, NameRes=NameRes
\end{center}}
\subsubsection*{Example:}
\begin{itemize}
\item mr\_info, I, Stat \\
Compute the multiresolution of the image {\em I} with default options
(i.e.\ \`a trous algorithm with 4 scales) and calculate statistical
information about each band.
\end{itemize}


\subsection{mr\_extract}
Extract a scale from a multiresolution transform. This routine 
calls the C++ executable {mr\_extract}. The keyword ``OPT" 
allows 
all options described in 
section~\ref{sect_extr}
to be passed to the executable.
{\bf
\begin{center}
   USAGE: mr\_extract, Multiresolution\_File\_Name, ScaleImage, OPT=Opt
\end{center}}
where {\em Multiresolution\_File\_Name} is a string which contain the 
name of multiresolution file (``.mr"), and {\em ScaleImage} is the output
image (IDL 2D array).

\subsection{mr\_read}
Read a multiresolution file (extension ``.mr"). If the multiresolution
transform is a cube, the output is a three-dimensional array. 
If it is an image the output is a two-dimensional IDL array.
If it is a pyramidal transform or a half-pyramidal transform, 
we have 3 different outputs:
\begin{enumerate}
\item an image containing several subimages (flag raw set)
\item a cube of interpolated or rebinned images (flag interpol set)
\item a structure containing several subimages  (default)
\end{enumerate}
{\bf
\begin{center}
     USAGE: output = mr\_read(filename, interpol=interpol, raw=raw, debug=debug)
\end{center}}
where 
\begin{itemize}
\item {\em filename}: string which contains the file name of 
the multiresolution transform (extension ``.mr").
\item {\em interpol}: integer (for pyramidal transform only) \\
0: the output will not be interpolated \\
1: the output will be rebinned \\
2: the output will be interpolated \\
\item {\em raw}: if set the output overwrites the 
input FITS file (for pyramidal transform only).
\item {\em debug}: if set, the routine is verbose
\end{itemize}

\subsection{mr\_compare}
Comparison between a reference image and an image or a sequence of images.
The comparison is carried out on the multiresolution scales.
If {\em Ima\_Or\_Cube} is a cube,  
the processing is repeated on each image of the
cube (*,*,i).
For each image {ima\_i} of the cube {\em Ima\_Or\_Cube} 
\begin{itemize}
\item  we compute the wavelet transform {\em WaveRef} of {\em ImaRef}
\item  we compute the wavelet transform {\em WaveIma} of {\em  ima\_i}
\item  we calculate the  correlation at each scale s: \\
if TabCorrel[s,i] = 1 then WaveRef[*,*,s] = WaveIma[*,*,s] are identical \\
else TabCorrel[s,i] is less than 1, and some differences exist.
\item  we calculate the  RMS at each scale:    \\
  TabRMS[s,i] = sigma( WaveRef[*,*,s] - WaveIma[*,*,s]])
\item  we calculate the normalized SNR at each scale (in dB): \\
  TabSNR[s,i] = 10 alog10 ( mean(WaveRef[*,*,s]\^2) \/ TabRMS[s,i]\^2)
\end{itemize}
This routine  calls the C++ 
executable {mr\_transform} described in section~\ref{sect_trans}.

{\bf
\begin{center}
     USAGE:  mr\_compare, ImaRef, Ima\_Or\_Cube, TabCorrel, TabRMS, TabSNR, Nscale=Nscale, plot=plot, title=title
\end{center}}
where 
\begin{itemize}
\item {\em ImaRef}: IDL 2D array. Reference image.
\item {\em Ima\_Or\_Cube}:  IDL 2D or 3D array.  Image or sequence of images 
to be compared.
\item {\em Nscale}: number of scales for the comparison
\item {\em plot}: if set, then plot the results
\item {\em title}:  is set, add the title to the plots
\item {\em TabCorrel}:  1D or 2D IDL array: output correlation table
\item {\em TabRMS}:  1D or 2D IDL array: output RMS table
\item {\em TabSNR}:  1D or 2D IDL array: output SNR table (in dB)
\end{itemize}
\subsubsection*{Example}
mr\_compare, imaref, ima1, tc, tr, ts, /plot, title='Comparison ImaRef-Ima1'\\
Comparison of the images {\em imaref} and {\em ima1}.


\subsection{mr\_background}
Estimate the background of an image. 
{\bf
\begin{center}
     USAGE:  output=mr\_background(Image, nscale=nscale, border=border)
\end{center}}
If {\rm border} is set, the background is estimated from the border of
the image. If not, the pyramidal median transform is used with {\em nscale}
scales (default is 3) by calling the C++ executable {mr\_background} described in section~\ref{sect_bgr}.


\subsection{mr\_filter}
Filter  an image by using a multiresolution transform. 
This routine is calling the C++ executable {mr\_filter}. The keyword 
``OPT" allows 
all options described in section~\ref{sect_filter}
to be passed to the executable.

{\bf
\begin{center}
     USAGE: mr\_filter, Data, FilterData, opt=opt
\end{center}}
{\em Data} is an image (2D IDL array), and {\em FilterData} is the result
of the filtering.

\subsection{im\_deconv}
Deconvolve an image by standard methods. This routine calls 
the C++ executable {im\_deconv}. The keyword ``OPT" 
allows 
all options described in section~\ref{sect_deconv} to be passed 
to the executable.

{\bf
\begin{center}
     USAGE: im\_deconv, Data, PSF, DeconvData, opt=opt
\end{center}}
\subsubsection*{Examples:} 
\begin{itemize}
\item im\_deconv, Imag, Psf, Result \\
deconvolve an image with all default options (gradient method).
\item  im\_deconv, Imag, Psf, Result, OPT='-i 30 -e 0' \\
same example, but impose the number of iterations to be 30.
\item  im\_deconv, Imag, Psf, Result, OPT='-d 4 -i 30' \\
deconvolution by the one step gradient method, without 
any regularization, with 30 iterations.
\end{itemize}


\subsection{mr\_deconv}
Deconvolve an image by using the multiresolution support. This routine calls 
the C++ executable {mr\_deconv}. The keyword ``OPT" 
allows 
all options described in section~\ref{sect_deconv} to be passed 
to the executable.

{\bf
\begin{center}
     USAGE: mr\_deconv, Data, PSF, DeconvData, opt=opt
\end{center}}
\subsubsection*{Examples:} 
\begin{itemize}
\item mr\_deconv, Imag, Psf, Result \\
deconvolve an image with all default options (Richardson-Lucy method + 
regularization in  wavelet space by using the \`a trous 
algorithm, etc.).
\item  mr\_deconv, Imag, Psf, Result, OPT='-i 30 -e 0'  \\
same example, but impose the number of iterations to be 30.
\item  mr\_deconv, Imag, Psf, Result, OPT='-d 2 -i 30' \\
deconvolution by the one step gradient method, without 
any regularization, with 30 iterations.
\end{itemize}

\subsection{mr\_detect}
Detect the sources in  an image by using a multiresolution transform. 
This routine calls the C++ executable {mr\_detect}. The keyword ``OPT" 
allows all options described in section~\ref{sect_detect}
to be passed to the executable. 

{\bf
\begin{center}
     USAGE: mr\_detect, imag, result, OPT=OPT, print=print, tabobj=tabobj, 
    NbrObj=NbrObj, nodel=nodel, RMS=RMS
\end{center}}
where
\begin{itemize}
\baselineskip=0.4truecm
\item {\em imag}: input image.
\item {\em result}: output image (sum of all detected objects).
\item {\em OPT}: option keyword.
\item {\em print}: if set, information about each detected object is printed.
\item {\em tabobj}: IDL structure which contains the information about 
the objects. For each object, we have
\begin{itemize}
\baselineskip=0.4truecm
\item ScaleObj: scale where the object is detected.
\item NumObj: object number 
\item PosX: X coordinate in the image.
\item PosY: Y coordinate in the image.
\item SigmaX: standard deviation on first main axis of the object.
\item SigmaY: standard deviation on second main axis of the object.
\item Angle: angle between the main axis and x-axis.
\item ValPixMax: value of the maximum of the object.
\item Flux: integrated flux of the object.
\item Magnitude: magnitude of the object.
\item ErrorFlux: flux error.
\item SNR\_ValMaxCoef: signal to noise ratio of the maximum of the wavelet
coefficient.
\item SNR\_Obj: signal to noise ratio of the object.
\item PosCoefMaxX: X coordinate  of the maximum wavelet coefficient
\item PosCoefMaxY: Y coordinate  of the maximum wavelet coefficient
\end{itemize}
\item {\em NbrObj}: number of detected objects
\item {\em nodel}: if set, the files created (by the program mr\_detect)
are not deleted.
\item {\em RMS}: RMS image related to the input image.
\end{itemize}
\subsubsection*{Examples:} 
mr\_detect, imag, result, tabobj=tabobj, NbrObj=NbrObj \\
detects all sources with default options in the image {\em imag}. \\
print, NbrObj \\
print the number of objects. \\
print, tabobj(0).PosX, tabobj(0).PosY \\
print the coordinates of the first object. \\

\subsection{xlive}
XLIVE is a widget program for large image analysis. The large image
has to be compressed  by mr\_comp or mr\_lcomp before being 
analyzed. Then the XLIVE data format is the MRC format.
When an MRC file is read, an image at very low resolution is displayed
and the user can improve the resolution of the image, or of a part
the image, using the RESOLUP button. When RESOLUP is called,
XLIVE reads from the MRC file the  wavelet coefficient needed for improving
the resolution. 
If the large image has been compressed by block (-C option), only  
the blocks needed are decompressed, and the image in memory has always
a size compatible with the window size.
The Dat parameter (if given) contains the last displayed image.
If it is not given, the user can however get it using the global
variable XIMA.

Several operations can be carried out by using the mouse and pressing buttons:
\begin{itemize}
\baselineskip=0.4truecm
\item QUIT: quit the application.
\item LOAD: load a FITS image. 
\item LUT: modify the LUT.
\item PROFILE: examine rows or columns.
\item CURSOR: examine pixel values. If the image format is FITS and if
the header contains astrometric position, then the pixel
position in the sky is given (right ascension and declination).
\item HISTO: plot the histogram.
\item CONTOURS: contours of the image (isophotes).
\item 3D VISU: three-dimensional representation of the image.
\item INFO: print the min, max, mean, sigma of the image.
\item FFT: compute the Fourier transform and display either the power 
spectrum, the phase, the real part or the imaginary part.
\item PAN: make a zoom. Zoom factors are 2,4,8,1/2,1/4,1/8
\item RESOLUP: Improve the image resolution. If the new image size is
greater than the window size, the user must click in the area
he/she wishes to see, and only this part of the image will be 
decompressed.
\item RESOLDOWN: decrease the resolution.
\item RESOL: goto a given resolution.\end{itemize}
{\bf
\begin{center}
     USAGE: xlive, [Dat,], FILEName=FileName, WindowSize=WindowSize 
\end{center}}
{\em Dat} is an output two-dimensional IDL array, 
and {\em FileName} (string array)
is an optional parameter containing the MRC filename to be read.  

