 

\chapter{ISOCAM data deconvolution using Markov random field}

\section{Use of the program im\_dec\_markov}

\begin{center}
 USAGE: ${\bf im\_dec\_markov}$ option image\_in psf\_in image\_out
\end{center}
where options are: 
\begin{itemize}
\item {\bf [-b]} \\
If this option is set, the support of the image is dilated by using I\_MIRROR to
obtain a solution with the same support. By default, we use I\_CONT.
\item {\bf [-d delta]} \\
Scaling parameter $\delta$. This parameter must be strictly positive. 
Default is 1 (See Parameter estimation section)
\item {\bf [-e epsilon]} \\
Convergence parameter. Default is 0.001
\item {\bf [-g sigma]} \\
Gaussian Noise standard deviation. Default is automatically estimated.
\item {\bf [-i number\_of\_iterations]} \\
Maximum number of iterations. Default is 500
\item {\bf [-l lambda]} \\
Regularizing parameter $\lambda$. This parameter must be strictly positive. 
Default is 1 (See Parameter estimation section)
\item {\bf [-o omega]} \\
Over-relaxation coefficient such as: $1\leq omega<2$. Omega is used to allow faster 
convergence. Default is 1 
\item {\bf [-r residual\_file\_name]} \\
If this option is set, the residual is written to 
the file of name {\em residual\_file\_name}. By default, the
residual is not written. The residual is equal to the difference between
the input image and the solution convolved by the PSF.

The maximum of the PSF must be at the center of its support.
So, dimensions of the PSF support must be odd.
\end{itemize}
\noindent

{\bf Example:}
\begin{itemize}
\item im\_dec\_markov -d 100 -i 100 -l 10 -r residu.fits 
image\_in.fits psf\_in.fits ima\_out.fits \\
deconvolves an image with $\delta=100$ and $\lambda=10$, assuming
a Gaussian noise (its standard deviation is automatically estimated).
\end{itemize}

For each iteration, the following informations will be printed:

...

It 95 D=90674.9 M=537432 V=628106 Ec=75.5482 Fx=9.38398e+07 min=1224.23 max=13732.2

It 96 D=90676.3 M=537430 V=628107 Ec=70.6314 Fx=9.38399e+07 min=1224.23 max=13731.7

It 97 D=90677.5 M=537429 V=628107 Ec=66.1543 Fx=9.38398e+07 min=1224.23 max=13731.3

It 98 D=90678.7 M=537429 V=628107 Ec=61.8843 Fx=9.38399e+07 min=1224.23 max=13730.8

It 99 D=90679.9 M=537428 V=628107 Ec=57.978 Fx=9.38399e+07 min=1224.23 max=13730.4
 
where D=$U({\bf y/x})$, M=$U({\bf x})$ and V=$U({\bf x/y})$ 
so V=D+M according to equation (\ref{mar_eqn_cr}). 

Ec is the correction energy compared
to the convergence parameter (epsilon).

Fx is the flux of the estimated
solution $\bf \hat x$.

Minimum and maximum of the restored image are also printed.

\newpage

\section{Data Simulation}

The original image (the object) is a simulated HST Wide field camera image of 
a distant cluster 
of galaxies (see Figure~\ref{fig_convol1}).
We used a Gaussian PSF (see Figure~\ref{fig_psf}) which has the following properties:
\begin{itemize}
\item{Support = 11 x 11 pixels}
\item{  Min = 2.00418e-08}   
\item{  Max = 0.0980603}
\item{  Mean = 0.0082643} 
\item{  sigma = 0.0183551}
\item{  Flux = 0.999981} 
\item{  Energy = 0.0490302}
\end{itemize}
\begin{figure}[htb]
\centerline{
\hbox{
\psfig{figure=mf_psf.ps,bbllx=2.5cm,bblly=13cm,bburx=19.5cm,bbury=25.5cm,width=16cm,height=16cm,clip=}
}}
\caption{Point Spread Function (PSF)}
\label{fig_psf}
\end{figure}

The image (see Figure~\ref{fig_convol1}) was blurred with the Gaussian PSF, and Gaussian noise ($\sigma=20$) was
added.
\begin{figure}[htb]
\centerline{
\hbox{
\psfig{figure=mf_simu256.ps,bbllx=1.8cm,bblly=12.9cm,bburx=14.5cm,bbury=25.5cm,width=8cm,height=8cm,clip=}
\psfig{figure=mf_data256.ps,bbllx=1.8cm,bblly=12.9cm,bburx=14.5cm,bbury=25.5cm,width=8cm,height=8cm,clip=}
}}
\caption{Left: the original image, Right: the blurred image.}
\label{fig_convol1}
\end{figure}
 
\section{Results}

\subsection{Visual comparisons} 

The proposed deconvolution method using a Markov model is compared to the 
basic Richardson-Lucy (RL) 
algorithm in 
Figure~\ref{fig_lucy} and to the regularized RL method in Figure~\ref{fig_mr}.
\begin{figure}[htb]
\centerline{
\hbox{
\psfig{figure=mf_Lresultd4.ps,bbllx=1.8cm,bblly=12.9cm,bburx=14.5cm,bbury=25.5cm,width=8cm,height=8cm,clip=}
\psfig{figure=mf_D100L10.ps,bbllx=1.8cm,bblly=12.9cm,bburx=14.5cm,bbury=25.5cm,width=8cm,height=8cm,clip=}
}}
\caption{Left: Deconvolution with RL method, Right: Deconvolution with
Markov model ($\delta=100$ and $\lambda=10$)}
\label{fig_lucy}
\end{figure}
\begin{figure}[htb]
\centerline{
\hbox{
\psfig{figure=mf_Lresultd8.ps,bbllx=1.8cm,bblly=12.9cm,bburx=14.5cm,bbury=25.5cm,width=8cm,height=8cm,clip=}
\psfig{figure=mf_D100L10.ps,bbllx=1.8cm,bblly=12.9cm,bburx=14.5cm,bbury=25.5cm,width=8cm,height=8cm,clip=}
}}
\caption{Left: Deconvolution with regularized RL method, Right: Deconvolution with
Markov model ($\delta=100$ and $\lambda=10$)}
\label{fig_mr}
\end{figure}

\subsection{Flux linearity}

The simulated HST image in Figure~\ref{fig_convol1} provides enough sources 
to test
the flux linearity. The procedure followed was to detect all sources 
respectively 
in the observed image and also in the image restored by the proposed 
regularization method.
Then, the flux of each source before and after deconvolution is 
displayed in Figure~\ref{fig_flux}.

\begin{figure}[htb]
\centerline{
\hbox{
\psfig{figure=fx256.ps,bbllx=1.5cm,bblly=13cm,bburx=20.5cm,bbury=25.5cm,width=15cm,height=15cm,clip=}
}}
\caption{FLUX LINEARITY TEST}
\label{fig_flux}
\end{figure}

\section{Conclusion}

The problem of image deconvolution has been considered in the context of the
regularization theory. An example of regularized method is proposed. 
The approach we have taken involves constraints in the estimated solution 
that are locally due to the Markov 
neighbor system. Nevertheless, the choice of parameters $\delta$ and $\lambda$
is global. Then, it could be interesting to introduce 
local parameters to open up new experimental directions.




