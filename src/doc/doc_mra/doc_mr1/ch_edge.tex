\chapter{\proj Edge Detection}
\label{ch_edge}
\index{edge detection}
% \chapterhead{Programs}
\markright{Edge detection}

\section{Introduction}
% An edge in an image  corresponds to a discontinuity in the intensity
% surface of the underlying scene.
An edge is defined as a local variation of image intensity. Edges can
be detected by the computation of a local derivative operator. 

\begin{figure}[htb]
\centerline{
\vbox{
\psfig{figure=fig_inflex.ps,bbllx=3.5cm,bblly=4cm,bburx=19.5cm,bbury=12cm,width=12cm,height=6cm,clip=} 
}}
\caption{First and second derivative of  $G_\sigma * f$. (a) Original signal,
(b) signal convolved by a Gaussian, (c) first derivative of (b), (d) second
derivative of (b).}
\label{fig_inflex}
\end{figure}

Figure~\ref{fig_inflex} shows how the inflection point of a signal can be
found from its first and second derivative.
Two methods
can be used for generating first order derivative edge gradients.

\section{First Order Derivative Edge Detection}

\subsection{Gradient}
The gradient of an image $f$ at location $(x,y)$, along the line
normal to the edge slope, is the vector \cite{ima:pratt91,ima:gonzalez93,ima:jain89}:
\begin{eqnarray}
\bigtriangledown f = & 
\left[ \begin{array}{c}
f_x  \\
f_y
\end{array} \right] & = 
\left[ \begin{array}{c}
 \partial f \over \partial x  \\
 \partial f \over \partial y
\end{array} \right]
\end{eqnarray}

% The gradient of an image $f$ along $r$ in a direction $\alpha$ is
%\begin{eqnarray}
%{\partial f \over \partial r} & = &{\partial f \over \partial x} \cos{\alpha} +
%{\partial f \over \partial y} \sin{\alpha} \\
%  & = & f_x \cos{\alpha} + f_y \sin{\alpha}
%\end{eqnarray}
%The maximum value of ${\partial f \over \partial r}$ is obtained when $

The spatial gradient amplitude is given by:
\begin{eqnarray}
G(x,y) = \sqrt{f_x^2 + f_y^2}
\end{eqnarray}
and the gradient direction with respect to the row axis is
\begin{eqnarray}
\Theta(x,y) = \arctan{\frac{f_y}{f_x}}
\end{eqnarray}

The first oder derivative edge detection can be carried out either by using
 two orthogonal directions in an image or by using a set of
directional derivatives. 
 
\subsection{Gradient mask operators}
% details dans le Jain, p348 et dans le pratt p504

Gradient estimates can be obtained by using gradient operators of the form:
\begin{eqnarray}
f_x & = f & \otimes  \ \ H_x \nonumber \\
f_y & = f & \otimes \ \ H_y 
\end{eqnarray}
where $\otimes$ denotes the convolution product, and $H_x$ and $H_y$ are $3
\times 3$
row and column operators, called gradient masks. Table \ref{tab_grad1} shows 
the main gradient masks proposed in the literature. Pixel difference is
the simplest one, which consists just of making the difference of   pixels
along rows and columns of the image:
\begin{eqnarray}
f_x(x_m, y_n) & = & f(x_m , y_n) - f(x_m-1, y_n) \nonumber \\
f_y(x_m, y_n) & = & f(x_m , y_n )- f(x_m, y_n-1)
\end{eqnarray}

The Roberts gradient masks \cite{edge:roberts65} are more sensitive 
to diagonal edges. Using these masks,
the orientation must be calculated by
\begin{eqnarray}
\Theta(x_m, y_n) = {\pi \over 4} + \arctan \left[ 
{f_y(x_m, y_n) \over f(x_m , y_n)} \right]
\end{eqnarray}

Prewitt \cite{edge:prewitt70}, Sobel, and Frei-Chen \cite{edge:frei77} produce
better results than the pixel difference, separated pixel difference and 
Roberts
operator, because the mask is larger, and provides averaging of small luminance
fluctuations. The Prewitt operator is more sensitive to 
horizontal and vertical edges  than diagonal edges, and the 
reverse is true for the Sobel operator. The Frei-Chen  edge detector has the 
same sensitivity for diagonal, vertical, and horizontal edges.

% table
\begin{table}[htb]
\begin{center}
\begin{tabular}{lllc}
Operator &    $H_x $     & $H_y$   & Scale factor\\
         &              &        \\
Pixel difference&
\( \left[ \begin{array}{ccc}
0 & 0 & 0\\
0 & 1 & -1\\
0 & 0 & 0
\end{array}\right]  \) &
\( \left[ \begin{array}{ccc}
0 & -1 & 0\\
0 & 1 & 0\\
0 & 0 & 0
\end{array}\right]  \) & 1\\

\begin{tabular}{l}
Separated \\
pixel difference \\
\end{tabular} &

\( \left[ \begin{array}{ccc}
0 & 0 & 0\\
1 & 0 & -1\\
0 & 0 & 0
\end{array}\right]  \)&
\( \left[ \begin{array}{ccc}
0 & -1 & 0\\
0 & 0 & 0\\
0 & 1 & 0
\end{array}\right]  \) & 1\\ 
Roberts&
\( \left[ \begin{array}{ccc}
0 & 0 & -1\\
0 & 1 & 0\\
0 & 0 & 0
\end{array}\right]  \)&
\( \left[ \begin{array}{ccc}
-1 & 0 & 0\\
0 & 1 & 0\\
0 & 0 & 0
\end{array}\right]  \) & 1 \\
Prewitt&
\( \left[ \begin{array}{ccc}
1 & 0 & -1\\
1 & 0 & -1\\
1 & 0 & -1
\end{array}\right]  \)&
\( \left[ \begin{array}{ccc}
-1 & -1 & -1\\
0 & 0 & 0\\
1 & 1 & 1
\end{array}\right]  \) & 1\\
Sobel&
 \( \left[ \begin{array}{ccc}
1 & 0 & -1\\
2 & 0 & -2\\
1 & 0 & -1
\end{array}\right]  \)&
 \( \left[ \begin{array}{ccc}
-1 & -2 & -1\\
0 & 0 & 0\\
1 & 2 & 1
\end{array}\right]  \) & $ \frac{1}{4} $ \\ 
Fei-Chen&
 \( \left[ \begin{array}{ccc}
1 & 0 & -1\\
\sqrt{2} & 0 & \sqrt{2}\\
1 & 0 & -1
\end{array}\right]  \)&
 \( \left[ \begin{array}{ccc}
-1 & -\sqrt{2} & -1\\
0 & 0 & 0\\
1 & \sqrt{2} & 1
\end{array}\right]  \) & $\frac{1}{2+\sqrt{2}}$ \\
 \end{tabular}\par
\caption{Gradient edge detector masks.}
\label{tab_grad1}
  \end{center}
\end{table}
\vspace{0.3cm}


\subsection{Compass operators}
Compass operators measure gradients in a selected number of directions.
The directions are $\Theta_k = k {\pi \over 4}$,  $k = 0,...,7$. The edge
template gradient is defined as:
\begin{eqnarray}
G(x_m, y_n) = \max_{k=0}^7 { \mid f(x_m, y_n) \otimes H_k(x_m, y_n) \mid }
\end{eqnarray}

Table~\ref{tab_grad2} shows the principal template gradient operators.

\begin{table}[htb]
\begin{center} 

\begin{tabular}{lllll}
\begin{tabular}{c}
 Gradient \\
 direction \\
 \end{tabular} &
\begin{tabular}{c}
 Prewitt \\
 compass   \\
 gradient \\
\end{tabular} &
Kirsch&
\begin{tabular}{c}
Robinson \\
3-level \\
\end{tabular}&
\begin{tabular}{c}
Robinson \\
5-level\\
\end{tabular} \\
East&
\( \left[ \begin{array}{ccc}
1 & 1 & -1\\
1 & -2 & -1\\
1 & 1 & -1
\end{array}\right]  \)&
\( \left[ \begin{array}{ccc}
5 & -3 & -3\\
5 & 0 & -3\\
5 & -3 & -3
\end{array}\right]  \)&
\( \left[ \begin{array}{ccc}
1 & 0 & -1\\
1 & 0 & -1\\
1 & 0 & -1
\end{array}\right]  \)&
\( \left[ \begin{array}{ccc}
1 & 0 & -1\\
2 & 0 & -2\\
1 & 0 & -1
\end{array}\right]  \)\\
 Northeast&
\( \left[ \begin{array}{ccc}
1 & -1 & -1\\
1 & -2 & -1\\
1 & 1 & 1
\end{array}\right]  \)&
\( \left[ \begin{array}{ccc}
-3 & -3 & -3\\
5 & 0 & -3\\
5 & 5 & -3
\end{array}\right]  \)&
\( \left[ \begin{array}{ccc}
0 & -1 & -1\\
1 & 0 & -1\\
1 & 1 & 0
\end{array}\right]  \)&
\( \left[ \begin{array}{ccc}
0 & -1 & -2\\
1 & 0 & -1\\
2 & 1 & 0
\end{array}\right]  \)\\
 North&
\( \left[ \begin{array}{ccc}
-1 & -1 & -1\\
1 & -2 & 1\\
1 & 1 & 1
\end{array}\right]  \)&
\( \left[ \begin{array}{ccc}
-3 & -3 & -3\\
-3 & 0 & -3\\
5 & 5 & 5
\end{array}\right]  \)&
\( \left[ \begin{array}{ccc}
-1 & -1 & -1\\
0 & 0 & 0\\
1 & 1 & 1
\end{array}\right]  \)&
\( \left[ \begin{array}{ccc}
-1 & -2 & -1\\
0 & 0 & 0\\
1 & 2 & 1
\end{array}\right]  \)\\
 Northwest&
\( \left[ \begin{array}{ccc}
-1 & -1 & 1\\
-1 & -2 & 1\\
1 & 1 & 1
\end{array}\right]  \)&
\( \left[ \begin{array}{ccc}
-3 & -3 & -3\\
-3 & 0 & 5\\
-3 & 5 & 5
\end{array}\right]  \)&
\( \left[ \begin{array}{ccc}
-1 & -1 & 0\\
-1 & 0 & 1\\
0 & 1 & 1
\end{array}\right]  \)&
\( \left[ \begin{array}{ccc}
-2 & -1 & 0\\
-1 & 0 & 1\\
0 & 1 & 2
\end{array}\right]  \)\\
 West&
\( \left[ \begin{array}{ccc}
-1 & 1 & 1\\
-1 & -2 & 1\\
-1 & 1 & 1
\end{array}\right]  \)&
\( \left[ \begin{array}{ccc}
-3 & -3 & 5\\
-3 & 0 & 5\\
-3 & -3 & 5
\end{array}\right]  \)&
\( \left[ \begin{array}{ccc}
-1 & 0 & 1\\
-1 & 0 & 1\\
-1 & 0 & 1
\end{array}\right]  \)&
\( \left[ \begin{array}{ccc}
-1 & 0 & 1\\
-2 & 0 & 2\\
-1 & 0 & 1
\end{array}\right]  \)\\
 Southwest&
\( \left[ \begin{array}{ccc}
1 & 1 & 1\\
-1 & -2 & 1\\
-1 & -1 & 1
\end{array}\right]  \)&
\( \left[ \begin{array}{ccc}
-3 & 5 & 5\\
-3 & 0 & 5\\
-3 & -3 & -3
\end{array}\right]  \)&
\( \left[ \begin{array}{ccc}
0 & 1 & 1\\
-1 & 0 & 1\\
-1 & -1 & 0
\end{array}\right]  \)&
\( \left[ \begin{array}{ccc}
0 & 1 & 2\\
-1 & 0 & 1\\
-2 & -1 & 0
\end{array}\right]  \)\\
 South&
\( \left[ \begin{array}{ccc}
1 & 1 & 1\\
1 & -2 & 1\\
-1 & -1 & -1
\end{array}\right]  \)&
\( \left[ \begin{array}{ccc}
5 & 5 & 5\\
-3 & 0 & -3\\
-3 & -3 & -3
\end{array}\right]  \)&
\( \left[ \begin{array}{ccc}
1 & 1 & 1\\
0 & 0 & 0\\
-1 & -1 & -1
\end{array}\right]  \)&
\( \left[ \begin{array}{ccc}
1 & 2 & 1\\
0 & 0 & 0\\
-1 & -2 & -1
\end{array}\right]  \)\\
 Southeast&
\( \left[ \begin{array}{ccc}
1 & 1 & 1\\
1 & -2 & -1\\
1 & -1 & -1
\end{array}\right]  \)&
\( \left[ \begin{array}{ccc}
5 & 5 & -3\\
5 & 0 & -3\\
-3 & -3 & -3
\end{array}\right]  \)&
\( \left[ \begin{array}{ccc}
1 & 1 & 0\\
1 & 0 & -1\\
0 & -1 & -1
\end{array}\right]  \)&
\( \left[ \begin{array}{ccc}
2 & 1 & 0\\
1 & 0 & -1\\
0 & -1 & -2
\end{array}\right]  \)\\
 \begin{tabular}{c}
 Scale  \\
 factor    \\
 \end{tabular} &  
 \begin{tabular}{c}
\( \frac{1}{5} \) \\
 \end{tabular}&
 \begin{tabular}{c}
\( \frac{1}{15} \) \\
 \end{tabular}&
 \begin{tabular}{c}
 \( \frac{1}{3} \) \\
  \end{tabular} &
  \begin{tabular}{c}
\( \frac{1}{4} \) \\
\end{tabular} \\
 \end{tabular} 
\caption{Template gradients.}
\label{tab_grad2}
\end{center}
\end{table}


\clearpage

\subsection{Derivative of Gaussian}
The previous methods are relatively sensitive to the noise.  A solution 
could be to extend the window size of the gradient mask operators.
Another approach is to use  the derivative of the 
convolution of the image by a Gaussian.
The derivative of a Gaussian (DroG) operator is
\begin{eqnarray}
\bigtriangledown (g \otimes f) & =  & {\partial  (g \otimes f) \over \partial x} +
 {\partial  (g \otimes f) \over \partial y} \nonumber \\
  & =  & f_x + f_y
\end{eqnarray}
with $g = \exp{- {x^2+y^2 \over {2\sigma^2}}}$. Partial derivatives of the
Gaussian function are
\begin{eqnarray}
g_x(x,y) & = {\partial  g  \over \partial x} = & - {x \over \sigma^2} \exp{- {x^2+y^2 \over {2\sigma^2}}} \nonumber \\
g_y(x,y) & = {\partial  g  \over \partial y} = & - {y \over \sigma^2} \exp{- {x^2+y^2 \over {2\sigma^2}}}
\end{eqnarray}
The filters are separable so we have
\begin{eqnarray}
g_x(x,y) & = & g_x(x) * g(y) \nonumber \\
g_y(x,y) & = & g_y(y) * g(x)  
\end{eqnarray}
Then
\begin{eqnarray}
f_x & = & g_x (x) \otimes g(y) \otimes f \nonumber \\
f_y & = & g_y (y) \otimes g(x) \otimes f
\end{eqnarray}
 
\subsection{Thinning the contour}
From the gradient map, we may want to consider only pixels which belong to the
contour. This can be done by looking for each pixel in the direction of 
gradient. For each point P0 in the gradient map, we determine the two 
adjacent pixels P1,P2
in the direction orthogonal to the gradient. If P0 is not a maximum in 
this direction (i.e.\ P0 $<$ P1, or P0 $<$ P2), then we threshold P0 to 
zero.
 
\section{Second Order Derivative Edge Detection}
Second derivative operators allow us to accentuate the 
edges. The most frequently
used operator is the Laplacian one, defined by 
\begin{eqnarray}
\bigtriangledown^2 f = {\partial^2  f  \over \partial x^2} 
                     + {\partial^2  f  \over \partial y^2}
\end{eqnarray}

Table~\ref{tab_laplacian} gives three discrete approximation of this operators.
 
\begin{table}[htb]
\begin{center}
\begin{tabular}{ccc}
 Laplacian 1 &   Laplacian 2 &   Laplacian 3\\
             &           &    \\
\( {1 \over 4} \left[ \begin{array}{ccc}
0 & -1 & 0\\
-1 & 4 & -1\\
0 & -1 & 0
\end{array}\right]  \) &
\( {1 \over 8} \left[ \begin{array}{ccc}
-1 & -1 & -1\\
-1 & 8 & -1\\
-1 & -1 & -1
\end{array}\right]  \) &  
\( {1 \over 8} \left[ \begin{array}{ccc}
-1 & -2 & -1\\
-2 & 4 & -2\\
-1 & -2 & -1
\end{array}\right]  \) \\
\end{tabular}
\caption{Laplacian operators.}
\label{tab_laplacian}
\end{center}
\end{table}

Marr and Hildreth \cite{edge:marr80} have proposed the Laplacian of Gaussian (LoG) edge detector operator.
It is defined as
\begin{eqnarray}
L(x,y) = {1 \over{\pi s^4}} 
	  \left[ 1 - {{x^2+y^2} \over {2s^2}} \right] 
\exp \left( - {{x^2+y^2}\over {2s^2}} \right)
\end{eqnarray}
where $\sigma$ controls the width of the Gaussian kernel. 

Zero-crossings of a given image $f$ convolved with $L$  
give its edge locations.

A simple algorithm for zero-crossings is:
\begin{enumerate}
\item For all pixels i,j do
\item ZeroCross(i,j) = 0
\item P0 = G(i,j); P1 =  G(i,j-1); P2 =  G(i-1,j); P3 =  G(i-1,j-1)
\item If (P0*P1 $<$  0) or  (P0*P2 $<$  0) or (P0*P3 $<$ 0) then ZeroCross(i,j) = 1
\end{enumerate} 


\section{Edge Detection Program: im\_edge}
\index{im\_edge}
The program {\em im\_edge} detects the edges in an image by the
methods previously described.
{\bf
\begin{center}
     USAGE: im\_edge option file\_name\_in file\_name\_out
\end{center}}
where options are: 
\begin{itemize} 
\baselineskip=0.4truecm
\item {\bf [-M edge\_detection\_method}
{\small
\begin{enumerate}
\baselineskip=0.4truecm
\itemsep=0.1truecm
\item Derivative of a Gaussian (DroG) 
\item Pixel difference 
\item Separated pixel difference 
\item Sobel 
\item Prewitt 
\item Roberts 
\item Frei Chen 
\item Laplacian: filter 1 
\item Laplacian: filter 2 
\item Laplacian: filter 3 
\item Marr and Hildreth: Laplacian of Gaussian (LoG) 
\item Prewitt compass gradient 
\item Kirsch 
\item Robinson 3-level 
\item Robinson 5-level 
\end{enumerate}}
Default is Sobel method
\item {\bf [-k]} \\
Select local maxima in gradient direction for first derivative methods,
and zero-crossings for second derivative methods. \\
Default is not to do these operations.
\item {\bf [-t ThresholdValue]} \\
Threshold all values in the edge map lower than {\em ThresholdValue}.
\item {\bf [-S Sigma]} \\
Scale parameter. Only used for DroG and LoG methods. \\
Default value is ${1 \over \sqrt{3}}$.
\end{itemize}

\clearpage
\newpage

\section{Wavelets and Edge Detection}
The LoG operator, also called the Mexican hat, is a well-known 
wavelet function
(even if wavelets did not exist when the LoG operator was proposed!).
Furthermore, a scale parameter ($\sigma$) in this method leads to 
a multiscale approach when $\sigma$ is varying. The advantage of using
the wavelet framework is the existence of very fast WT, such as the \`a trous
algorithm, which furnishes us with a  way to get the edges directly at all 
dyadic scales, even for large scales.

\subsection{Multiscale first derivative}

Generalizing the concept of multiscale edge detection, 
Mallat \cite{edge:mallat92a,edge:mallat92b,ima:mallat98} showed that 
the DroG operators can be easily associated with a wavelet
transform.
If $\phi(x)$ is a smoothing function (i.e. $\int \phi(x) dx = 1$), 
converges to zero at infinity, and is differentiable, we denote
\begin{eqnarray}
\psi^{1}(x,y) = {{d \phi(x,y)} \over {dx}} \quad \mbox{ and } \quad
\psi^{2}(x,y) = {{d \phi(x,y)} \over {dy}} \quad \mbox{     } \quad
 \end{eqnarray}
By definition, $\psi^x$ and $\psi^y$ are wavelets
(their integral is equal to zero). The local extrema of the wavelet coefficients
using $\psi^{x_1},\psi^{y_1}$ correspond to the inflection points of $f*\phi_s$ 
(with $\phi_s = {1 \over s} \phi({x\over s})$).

Using the directional \`a trous algorithm, sometimes also called dyadic wavelet
transform, we have at each scale $j$ and at pixel location $(x,y)$ two wavelet
coefficients $w_{j,x}, w_{j,y}$. The modulus of the gradient is then defined by
\begin{eqnarray}
G_j(x,y) = \sqrt{ w_{j,x}^2 + w_{j,y}^2}
\end{eqnarray}
and the directional angle $\theta_j$ is
\begin{eqnarray}
\theta_j(x,y) = \left\{ \begin{array}{cc}
 \arctan({w_{j,y} \over w_{j,x}})  & \mbox{ if } w_{j,x} \ge 0 \\
 \pi - \arctan({w_{j,y} \over w_{j,x}})  & \mbox{ if } w_{j,x} < 0
\end{array}\right.
\end{eqnarray}
  
Multiscale edge points, also called modulus maxima, are points where 
the modulus is locally maximum with respect to its neighbors along the
direction $\theta_j$.

A similar approach can be developed for the
 LoG operator \cite{edge:mallat92a}.


\subsection{Image reconstruction from its multiscale edges}

\begin{figure}[htb]
\centerline{
\vbox{
\psfig{figure=fig_project.ps,bbllx=4cm,bblly=16cm,bburx=20.5cm,bbury=26cm,width=12cm,height=8cm,clip=} 
\caption{Approximation of the wavelet transform of $f$.}
}}
\label{fig_project}
\end{figure}

An image can be reconstructed (approximately) from its multiscale edges
\cite{ima:mallat98} using an iterative algorithm. It does not
converge exactly towards the image, but in practice the error is very
small. The algorithm consists of searching for an image $h$
such that its wavelet transform has the same modulus maxima (i.e.\ 
same number of modulus maxima, same positions, and same amplitudes) as 
those of the original image $f$.
Denoting as $w_j(X)$ the wavelet coefficients of an image $X$ at a scale $j$,
and $w^m_j(X)$ its modulus maxima, 
we require that $w^m_j(h) = w^m_j(f)$.
The algorithm is the following:
\begin{enumerate}
\itemsep=0.1truecm
\item Set $w^m_j(h)$  to zero.
\item Loop:
\item Calculate the difference $w^m_j(f) - w^m_j(h)$, and
interpolate between the maxima: we get $w_j^d$.
\item Update $w_j(h)$: $w_j(h) = w_j(h) + w_j^d$.
\item Assign in $w_j(h)$ the correct values at maxima positions.
\item Reconstruct $h$  from  $w_j(h)$
\item Do a wavelet transform of $h$: $w_j(h)$ is then updated.  
\item Goto 2
\item Assign in $w_j(h)$ the correct values at maxima positions.
\item Reconstruct $h$  from  $w_j(h)$.
\end{enumerate}
The algorithm converges quickly to the solution after a few iterations.
The interpolation \cite{edge:mallat92a} can be viewed 
as a projection on an affine space
$\Gamma$, and the inverse and forward wavelet transform as a projection
on the space V of all possible wavelet transform signals. The two 
projections are visualized in Figure~\ref{fig_project} ($Wh$ and $Wf$ 
are respectively the wavelet transform of $h$ and $f$).


\section{Multiscale Edge Detection Program}
\subsection{First derivative: mr\_edge}
\index{mr\_edge}
The program {\em mr\_edge} detects the edges in an image by the
methods previously described. The wavelet transform used is 
the Mallat dyadic wavelet transform. The output file is a multiresolution
file (``.mr''). For each scale, two bands are stored, one 
the maxima map and the second the gradient angle map. 

{\bf
\begin{center}
     USAGE: mr\_edge option file\_name\_in file\_name\_out
\end{center}}
where options are: 
\begin{itemize} 
\item {\bf [-n number\_of\_scales]} \\
Number of scales used in the multiresolution transform.
Default is 4.
\end{itemize} 
\subsubsection*{Examples:}
\begin{itemize} 
\baselineskip=0.4truecm
\itemsep=0.1truecm
\item mr\_edge image.d med.mr \\
Build the multiscale edge detection file.
\item mr\_extract -B -s 1 med.mr mod1 \\
Create an image which contains the detected edge map (maxima) 
of the first scale.
\item mr\_extract -B -s 2 med.mr ang1 \\
Create an image which contains the  gradient angle map
of the previous edge map.
\item mr\_extract -B -s 5 med.mr mod3 \\
\item mr\_extract -B -s 6 med.mr ang3 \\
Ditto for scale 3.
\end{itemize} 

\subsection{Second derivative: mr\_at\_edge}
\index{mr\_at\_edge}
The program {\em mr\_at\_edge} detects the edges in an image by the
methods previously described. The wavelet transform used is 
the \`a trous algorithm. Only zero crossings of each scale are
kept (i.e.\ not thresholded).
The output file is a multiresolution
file (``.mr''). 
{\bf
\begin{center}
     USAGE: mr\_at\_edge option file\_name\_in file\_name\_out
\end{center}}
where options are: 
\begin{itemize} 
\item {\bf [-n number\_of\_scales]} \\
Number of scales used in the multiresolution transform.
Default is 4.
\end{itemize} 
\subsubsection*{Examples:}
\begin{itemize} 
\baselineskip=0.4truecm
\itemsep=0.1truecm
\item mr\_at\_edge image.d med.mr \\
Build the multiscale edge detection file.
\item mr\_extract -s 1 med.mr zero1 \\
Create an image which contains the detected edge map of the first scale.
\item mr\_extract -s 2 med.mr zero2 \\
Ditto for scale 2.
\end{itemize} 

\subsection{Image reconstruction: mr\_rec\_edge}
\index{mr\_rec\_edge}
The program {\em mr\_rec\_edge} reconstructs an image from its multiscale
edges. 
The multiscale edge file must have been obtained from the {\em mr\_edge}
program. 
{\bf
\begin{center}
     USAGE: mr\_rec\_edge option file\_name\_in file\_name\_out
\end{center}}
where options are: 
\begin{itemize} 
\item {\bf [-i number\_of\_iterations]} \\
Number of iterations.
Default is 10.
\end{itemize} 

\newpage

\section{Contrast Enhancement}
\label{sec_contrast}

\subsection{Introduction}
Because some features are hardly detectable by eyes in a image, we
often transform it before visualization. Histogram equalization is
certainly one the most well known method for contrast enhancement.
Such an approach is general useful for images
with a poor  intensity distribution. As edges play a fundamental 
role in image understanding, a way to enhance the contrast is to 
enhance the edges. For example, we can add to the original image its Laplacian
($I^{'}= I + \gamma \Delta I$, where $\gamma$ is a parameter). Only
features at the finest scale are enhanced (linearly). For a high 
$\gamma$ value, only the high frequencies are visible.
Multiscale edge enhancement \cite{col:velde99} can be seen 
as a generalization of this approach to all resolution levels.  
Images with a high dynamic range are also 
difficult to analyze. For example, astronomers generally visualize their
images using a logarithmic transformation. We see in the next
section that wavelet can also be used to compress the dynamic range
at all scales, and therefore allows us to clearly see some very faint
features.

\subsection{Multiscale Edge Enhancement}
\subsubsection{Gray Images}
Velde has proposed the following algorithm \cite{col:velde99}:
\begin{enumerate}
\baselineskip=0.4truecm
\itemsep=0.1truecm
\item The image  $L$ is mapped nonlinearly according to:
\begin{eqnarray}
   L(i) \rightarrow L(i)^{1-q} 100^q
\end{eqnarray}
\item The $L$ image is decomposed into a multiscale gradient
pyramid by using the dyadic wavelet transform (two directions per scale).
The gradient at the scale $j$, the pixel position $i$  is
calculated by: $G_j(i) = \sqrt{ (w_j^{(h)}(i))^2 + (w_j^{(v)}(i))^2}$ where 
$w_j^{(h)}$ and $w_j^{(v)}$ are the wavelet coefficients in both 
vertical and diagonal directions
at pixel position $i$. 
\item The two wavelet coefficients at scale $j$ and at position $i$ 
are multiplied by  $y(G_j(i))$, 
where $y$ is defined by:
\begin{eqnarray}
  y(x) & = & ({m \over c})^p \mbox{ if } \mid x \mid < c \nonumber \\
  y(x) & = & ({m \over \mid x \mid })^p  \mbox{ if } c \le \mid x \mid < m \nonumber \\
  y(x) & = & 1  \mbox{ if } \mid x \mid \ge m
\end{eqnarray}
\item The $\tilde L$image is reconstructed 
from the modified wavelet coefficients.
\item the $\tilde L$ image is mapped nonlinearly according to:
\begin{eqnarray}
  \tilde L(i) \rightarrow \tilde L(i)^{1 \over 1-q} 100^{- {q \over {1-q}}}
\end{eqnarray}
\end{enumerate}

\begin{figure}[htb]
\vbox{
\centerline{  
\hbox{
\psfig{figure=fig_velde.ps,bbllx=3cm,bblly=13cm,bburx=20cm,bbury=25.cm,width=11cm,height=8cm,clip=}
}}
}
\caption{Enhanced coefficients versus  original coefficients.  
Parameters are m=30,c=3,p=0.5, and q=0.
}
\label{fig_velde1}
\end{figure}

Four parameters are needed $p$,$q$,$m$,$c$. 
$p$ determines the degree of non-linearity in the nonlinear rescaling
of the luminance, and must be in $]0,1[$. $q$ must be in $[-0.5,0.5]$.
When $q > 0$, then darker parts are less enhanced than the lighter parts.
When $ q < 0$, then the dark parts are more enhanced than lighter parts.
Coefficients larger than $m$ are not modified by the algorithm.
The $c$ parameter corresponds to the noise level.  

Figure~\ref{fig_velde1} shows the modified wavelet coefficients versus
the original wavelet coefficients for a given set parameters 
(m=30,c=3,p=0.5, and q=0).


\subsection{The LOG-Wavelet Representation of Gray Images}

By using the \`a trous wavelet transform algorithm, an image $I$ 
\index{a trous wavelet transform}
\index{wavelet transform}
can be defined as the sum of its $J$ wavelet scales and the last smooth 
array:
\begin{eqnarray}
I(x,y) = c_{J}(x,y) + \sum_{j=1}^{J} w_j(x,y) 
\label{resid}
\end{eqnarray}
where the first term on the right is the last smoothed array, 
and $w$ denotes a wavelet scale. See \cite{starck:book98} more
details about this algorithm.
The wavelet-log representations consists in replacing $w_j(x,y)$
by $\log(\mid w_j(x,y)\mid)$:
\begin{eqnarray}
I_w(x,y) = \log(c_{J}(x,y)) +  \sum_{j=1}^{J} \mathrm{sgn}(w_j(x,y)) \log(\mid w_j(x,y)\mid) 
\end{eqnarray}

\begin{figure}[htb]
\vbox{
\centerline{  
\hbox{
\psfig{figure=fig_comet.ps,bbllx=1.8cm,bblly=12.7cm,bburx=14.5cm,bbury=25.4cm,width=9cm,height=9cm,clip=}
}}
\centerline{ 
\hbox{
\psfig{figure=comet_wlog_m8.ps,bbllx=1.8cm,bblly=12.7cm,bburx=14.5cm,bbury=25.4cm,width=9cm,height=9cm,clip=}
}}}
\caption{Hale-Bopp Comet image and its wavelet log representation.}
\label{fig_halebopp_wavelet}
\end{figure}
Figure~\ref{fig_halebopp_wavelet} left shows  the  logarithm of 
Hale-Bopp comet image and right its wavelet log representation. The jets
clearly appears in the last representation. 


\subsection{Contrast Enhancement Program: mr\_contrast}
\index{mr\_contrast}
The program {\em mr\_contrast} enhances the contrast of a gray image. 
Several methods are available. 
{\bf
\begin{center}
 USAGE: mr\_contrast option in\_image out\_image
\end{center}}
where options are:
\begin{itemize}
\baselineskip=0.4truecm
\itemsep=0.1truecm
\item {\bf [-m contrast\_enhancement\_method]} 
\begin{enumerate}
\baselineskip=0.4truecm
\itemsep=0.1truecm
\item Histogram Equalization 
\item Wavelet Coefficients Enhancement 
\item Wavelet-Log function: $f(w) = log(\mid w \mid +L) $
\item Wavelet-Log function: $f(w) = sgn(w).log(\mid w \mid+L)$ 
\item K-Sigma clipping.
\item Add to the image its Laplacian. 
\end{enumerate}
Default is Wavelet Coefficients Enhancement.
\item {\bf [-n number\_of\_scales]} \\
Number of scales used in the wavelet transform.
Default is 4. 
\item {\bf [-M M\_parameter]} \\
M Parameter. Only used if ``-e'' option is set. 
Coefficients larger than M are not enhanced. 
Default is 100.
\item {\bf [-P P\_parameter]} \\
P Parameter. Only used if ``-e'' option is set. 
P must be in the interval $]0,1[$.
Default is $0.5$.
\item {\bf [-Q Q\_parameter]} \\ 
Q Parameter. Only used if ``-e'' option is set. 
Q must be in the interval $[-0.5,0.5]$. When $Q > 0$,
darker part are less enhanced than lighter part.
When $Q < 0$, darker part are more enhanced than lighter part.
Default is $0$.
\item {\bf [-C C\_parameter]} \\  
C Parameter. Only used if ``-e'' option is set. 
Default is $0$.
\item {\bf [-K ClippingValue]} \\
Clipping value. Default is 3.
\item {\bf [-L Param]} \\
Parameter for for the Laplacian or the log method. Default is 0.1.
\end{itemize}
\subsubsection*{Examples:}
\begin{itemize}
\baselineskip=0.4truecm
\itemsep=0.1truecm
\item mr\_contrast image.fits image\_out.fits\\
Enhance the contrast by multiscale edge method.
\item mr\_contrast -n6 -m4 image.fits image\_out.fits\\
Enhance the contrast by wavelet log representation using six
resolution levels.
\end{itemize}
