\chapter*{Appendix C: Nonlinear Pyramidal Wavelet Transform (WT-PMT)}
\addcontentsline{toc}{chapter}{Appendix C: Pyramidal Wavelet Transform (PMWT)}

One of the advantages of the pyramidal wavelet transform over the pyramidal
median transform (PMT) is the ability to have robust noise estimation in the 
different scales, while the advantage of the PMT is a better separation
of the structures in the scales. Using the PMT, 
a strong structure (like a bright star, cosmic rays, bad pixels, etc.) will 
not be spread over all scales as when using a wavelet transform. In fact, when
there is no signal in a given region, a wavelet transform would be better, and
if a strong signal appears, it is the PMT that we would like to use. So the 
idea arises 
to try to merge both transforms, and to adapt the analysis at each position
and at each scale, depending on the amplitude of the coefficient we measure.

A possible algorithm to perform this on an image $I$ is the following:
\begin{enumerate}
\item Let $c_j = I$, and $j=0$.
\item Filter the the image $c_j$ by the median: we get $m_j$
\item Set $c_j^* = c_j$ and $d_j= c_j - m_j$
\item For all pixels $k$ do \\
\ \ \ \ \ \ \ \ \ \ \ \ \ 
 if $\mid d_j(k) \mid > k \sigma_j$ then $c_j^*(k) = m_j(k)$
\item Smooth $c_j^*$  by a B$_3$-spline: we get $b_j$
\item Set $w_j = c_j - b_j$
\item  $c_{j+1} = dec(b_j)$ (where the decimation operation, dec, entails 1 pixel replacing
each 2 $\times$2 subimage).
Let $j=j+1$ and return to step 2 if $j < N_s$ ($N_s$ being the number of scales).
\end{enumerate}
In this algorithm, the linear filtering relative to the wavelet transform is
not applied to the strong features contained in the image. Indeed, significant
pixel values are detected at step 4, and are replaced by the median. Regions
containing no bright object are treated as if the pyramidal wavelet 
transform  is used. The parameter $k$ used at step 4 must be large enough
in order to be sure that noise is not filtered by the median ($k=5$ seems
high enough). As this transform merges the wavelet transform and the PMT, 
we call it the PMWT transform. 
PMWT takes more time than the PMT, but this algorithm has the
advantages of the PMT without the drawbacks. The reconstruction is the
same as for the PMT.


\chapter*{Appendix D: Half Pyramidal Wavelet Transform}
\addcontentsline{toc}{chapter}{Appendix D: Half Pyramidal Wavelet Transform}

At each iteration of the pyramidal transform, there is a smoothing and 
a decimation of the image. But the smoothing is not strong enough to
reduce the cut-off frequency by a factor two. The decimation then violates
Shannon's theorem. To avoid this problem, which may create artifacts, 
Bijaoui \cite{wave:bijaoui97}
proposed not to decimate the image at the first iteration of the algorithm.
This means that the two first scales have the same size as the original image
(instead of just one in the standard pyramidal transform). Then the two first
iterations are identical as in the \`a trous algorithm. For the
following iterations, there is a decimation and the filtering is done
as for the second iteration (i.e.\ by taking pixels interspersed by 
one pixel for the filtering).

\newpage
$ $

