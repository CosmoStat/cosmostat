\chapter{\proj Geometric Registration}
\label{ch_register}
\index{registration}
% \chapterhead{Programs}
\markright{Geometric registration}

\section{Introduction}
Image registration is a procedure which determines 
the best spatial fit between two or more images that overlap the same 
scene, and were acquired at the same or at a different time, by identical or 
different sensors. 
Thus, registration is required for  processing a new set of data 
in such a way that its 
image under an appropriate transform is in a proper geometrical 
relationship with the previous set of data. 
\bigskip

Several digital techniques have been used for automatic registration of 
images such as cross-correlation, normal cross-correlation and minimum 
distance criteria.  
The advantage of the wavelet transform is that it produces both 
spatial and frequency domain information which allow the study of 
the image by frequency bands \cite{reg:djamdji1,reg:djamdji2,reg:djamdji3}. \\
 
An automatic image registration procedure can be helpful for several
applications:
\begin{enumerate}
\item Comparison between two images obtained at the same wavelength.
\item Comparison between two images obtained at different wavelengths.
\item Pixel field of view distortion estimation.
\end{enumerate}
 \bigskip
 
The geometrical correction is usually performed by three operations:
\begin{itemize}
\item The measure of a set of well-defined ground control points (GCPs), which are 
      features well located both in the input image and in the reference image.
\item The determination of the warping or deformation model, by specifying a 
      mathematical deformation model defining the relation between the 
      coordinates $(x,y)$ and $(X,Y)$ in the reference and input image respectively.
\item The construction of the corrected image by output-to-input mapping.
\end{itemize}

\bigskip
The main difficulty lies in the automated localization of the corresponding 
GCPs, since the accuracy of their determination will affect the overall 
quality of the registration.  In fact, there are always ambiguities in 
matching two sets of points, as a given point corresponds to a small region 
{\em D}, which takes into account the prior geometric uncertainty between 
the two images and many objects could be contained in this region.
\bigskip

 One property of the wavelet transform is to have a sampling step proportional 
to the scale.  When we compare the images in the wavelet transform space, 
we can choose a scale corresponding to the size of the region {\em D}, so 
that no more than one object can be detected in this area, and the matching 
is done automatically.

\section{Deformation model}%Polynomial Transformation
Geometric correction requires a spatial transformation to invert an unknown 
distortion function.  A general model for characterizing misregistration 
between two sets of remotely sensed data is a pair of bivariate polynomials 
of the form:
\begin{eqnarray*}
x_i = \displaystyle{ \sum^{N}_{p=0} \sum^{N-p}_{q=0} a_{pq} 
X^{p}_{i} Y^{q}_{i} = Q(X_{i}, Y_{j}) } \\
y_i = \displaystyle{ \sum^{N}_{p=0} \sum^{N-p}_{q=0} b_{pq} 
X^{p}_{i} Y^{q}_{i} = R(X_{i}, Y_{j}) }
\end{eqnarray*}
where $(X_{i},Y_{i})$ are the coordinates of the $i^{th}$ GCP in the 
reference image, $(x_{i},y_{i})$ the corresponding GCP in the input 
image and $N$ is the degree of the polynomial.

Usually, for images taken 
under the same {\em imaging direction}, polynomials of degree one or two 
are sufficient as they can model most of the usual deformations like shift, 
scale, skew, perspective and rotation (see Table 3.1). 

We then compute the unknown parameters 
($(N+1)(N+2)/2$ for each polynomial) using the least mean square 
estimator. \\ 

\begin{table}[h]
\begin{center}
\begin{tabular}{l|c} \hline 
Shift       & $x  =  a_0 + X$                        \\ 
            & $y  =  b_0 + Y$                        \\ \hline
Scale       & $x  =  a_1 X$                          \\ 
            & $y  =  b_2 Y$                          \\ \hline
Skew        & $x  =  X + a_2 Y$                      \\
            & $y  =  Y$                              \\ \hline
Perspective & $x  =  a_3 X Y$                        \\
            & $y  =  Y$                              \\ \hline
Rotation    & $x  =   \cos \theta X + \sin \theta Y$ \\
     	    & $x  =  -\sin \theta X + \cos \theta Y$ \\ \hline 
\end{tabular}
\caption{Some common deformations.}
\end{center}
\label{table:deformations}
\end{table}


\section{Image registration: mr\_fusion}
\index{mr\_fusion}
Program {\em mr\_fusion} performs the geometrical registration 
of two images having the same size, and same resolution. Four deformation
models are available (``-d" option). The program may fail to register the image
when not enough control points are detected. In this case, the user can try
another deformation model which may be more adapted to his data, or modify
the noise model parameters. Here the noise model is only used for structure
detection. If the image contains strong features, the threshold parameter 
``-s"
can be set at a greater value. Hence, the registration will be performed only
from the strongest feature of the image. The number of scales is also very
important. Indeed, the maximum distance between two pixels of the same point
in the two images  must be less than or equal to the size of the wavelet at the
last scale. The number of scales can be fixed either using 
directly the ``-n" option,
or using the ``-D" option. 
By choosing the latter, the user gives the maximum distance
between two pixels of the same point, and the program calculates automatically
the correct number of scales. 

{\bf
\begin{center}
 USAGE: mr\_fusion option image\_ref image\_in image\_out
\end{center}}
Options are:
\begin{itemize}
\baselineskip=0.4truecm
\itemsep=0.1truecm
\item {\bf [-p]} \\
Poisson Noise. Default is no Poisson component (just Gaussian).
\item {\bf [-g SigmaNoise]} \\
SigmaNoise = Gaussian noise standard deviation. Default is automatically estimated.
\item {\bf [-c gain,sigma,mean]} \\
See section~\ref{sect_support}.
\item {\bf [-n number\_of\_scales]} \\
Number of scales used in the multiresolution transform. Default is 4.
\item {\bf [-s NSigma]} \\
Thresholding at NSigma * SigmaNoise. Default is 5.
\item{\bf [-r res\_min]} \\
Minimum resolution for the reconstruction. 
The registration procedure is stopped at scale res\_min and 
the resulting deformation model is used to register the input image. 
Default value is 1.
\item{\bf [-D dist\_max]} \\
Maximum estimated distance between two identical points in both images. 
This value is used to estimate the number of scales for the wavelet transform.
% \item{\bf [-l]} \\
%  Sub-scene and scene registration:
%  the sub-scene is considered to be part of a larger scene. 
% image\_in is registered, and the resulting deformation model is used to 
% register the larger scene.
\item{\bf [-i Interpolation type]} \\
Type of interpolation:
\begin{itemize}
\baselineskip=0.4truecm
\item  0: Zero order interpolation -- nearest neighbor.
\item  1: First order interpolation -- bilinear.
\item  2: Second order interpolation -- bicubic.
\end{itemize}
Default is 2.
\item{\bf [-d DeforModel] } \\
Type of registration deformation model: \\
The type of polynomial model used for the geometrical registration. Three 
types
are available:
      \begin{itemize}
      \baselineskip=0.4truecm
      \itemsep=0.1truecm
      \item 0: Polynomial of the first order of type I:
            \begin{eqnarray}
            x^{'} & = & aX - bY + c_x \\
            y^{'} & = & bX + aY + c_y 
            \end{eqnarray}
      \item 1: Polynomial of the first order of type II:
            \begin{eqnarray}
             x^{'} & = & aX + bY + c \\
             y^{'} & = & dX + eY + f
            \end{eqnarray}
      \item 2: Polynomial of the second order:
            \begin{eqnarray}
            x^{'} & = & aX^{2} + bY^{2} + cXY + dX + eY + f \\
            y^{'} & = & gX^{2} + hY^{2} + iXY + jX + kY + l 
            \end{eqnarray}
      \item 3: Polynomial of the third order.
      \end{itemize}
Default is 1.
\item{\bf [-o]} \\
Manual Options specifications:\\
A few options are provided in order to have more control on the procedure. 
Using manual options, the following parameters can be fixed by the user 
for each scale:
\begin{itemize}
\baselineskip=0.4truecm
\item Matching distance: \\
      The distance used for the matching procedure. The procedure looks for each control
      point candidate in the reference image which is the corresponding control point
      candidate in the input image within a radius of {\em Matching distance}.
\item Threshold level: \\
      The threshold level used for thresholding the wavelet transform. 
\item Type of registration deformation model (same as option -d).
\item Type of interpolation (same as option -i).
\end{itemize}
The available manual options are the following:
\begin{itemize}
\baselineskip=0.4truecm
\itemsep=0.1truecm
\item 0:  Everything is taken care of by the program.
\item 1:  The matching distance is specified manually for each resolution.
\item 2:  The threshold level is specified manually for each resolution  and for
      both the reference image and the input image.
\item 3:  The type of deformation model is specified manually for each resolution.
\item 4:  The matching distance, the Threshold level and the Type of deformation model 
      are specified manually for each resolution.
\item 5:  The matching distance, the threshold level, the type of deformation model 
      and the type of interpolation are specified manually for each resolution.
\end{itemize}
The default is none (0).
\item{\bf [-w]} \\
The following files are written to  disk:
\begin{itemize}
\baselineskip=0.4truecm
\item deform\_model.txt: contains the calculated coefficients of the 
deformation model, allowing us to calculate the coordinates in the 
second image of a given point from its coordinates in the reference image.
\item scale\_j\_control\_points.dat: contains the set of control points for
 each scale. The first line contains the number of control points, and all
 other lines the values Xr,Yr,Xi,Yi,
 where  Xr and Yr are the coordinates of the control point in the reference 
 image (origin is (0,0)), and  Xi and Yi are the coordinates of the 
 corresponding control point in the input image (second image).
 \item xx\_grill\_in: an artificial image which contains a ``grilled'' image.
 \item xx\_grill\_out: the resulting image after applying the deformation 
 model to the artificial one.
\end{itemize}
The default is none.
\end{itemize}

\begin{figure}[htb]
\centerline{
\vbox{
\hbox{
\psfig{figure=ch5_dec_ngc.ps,bbllx=1.8cm,bblly=7cm,bburx=19.2cm,bbury=24.3cm,width=8cm,height=8cm}
\psfig{figure=ch5_diff_ngc_dec.ps,bbllx=1.8cm,bblly=7cm,bburx=19.2cm,bbury=24.3cm,width=8cm,height=8cm}
}
\hbox{
\psfig{figure=ch5_rec_ngc.ps,bbllx=1.8cm,bblly=7cm,bburx=19.2cm,bbury=24.3cm,width=8cm,height=8cm}
\psfig{figure=ch5_diff_ngc_rec.ps,bbllx=1.8cm,bblly=7cm,bburx=19.2cm,bbury=24.3cm,width=8cm,height=8cm}
}
}}
\caption{Synthetic image (upper left) and difference between the original 
and the synthetic image (upper right). Registered image (bottom left) 
and difference between NGC2997 and the registered image (bottom right). }
\label{fig_ngc_register}
\end{figure}

A strong distortion was applied to the galaxy NGC2997 (see 
Figure \ref{fig_ngc}). 
A synthetic image was made by shifting it by 5 and 10 pixels 
in each axis direction. Then this image
was rotated by 10 degrees and Gaussian noise was added. 
Figure \ref{fig_ngc_register} shows the synthetic image 
(upper left panel), and also the difference
between the original image and the synthetic one (upper right panel).
Figure \ref{fig_ngc_register} (bottom left and right) shows the 
corrected image and the residual between the original image 
and the corrected image.
The two images have been correctly -- and automatically -- registered.
\subsubsection*{Example:}
\begin{itemize}
\item mr\_fusion -n 6 -s 10 -d 1 ngc2997.fits  dec\_ngc  register\_ima\\ 
Register the image dec\_ngc on ngc2997, with a 10 sigma detection, 6 scales,
and using the first order deformation model of type 2. The 
result is presented
in Figure \ref{fig_ngc_register}.
\end{itemize}

\clearpage
\newpage

If the deformation model is not sophisticated enough to resolve a specific
problem, the result will certainly not be correct, but the user can still
use the CGP points which do not depend on any model.
