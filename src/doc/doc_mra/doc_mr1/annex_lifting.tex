\chapter*{Appendix A: The Lifting Scheme}
\addcontentsline{toc}{chapter}{Appendix A: The Lifting Scheme}

\subsection*{Introduction}

\begin{figure}[htb]
\centerline{
\hbox{
\psfig{figure=fig_lift.ps,bbllx=4.5cm,bblly=13cm,bburx=18.5cm,bbury=19.5cm,width=14cm,height=6.5cm,clip=}}
}
\caption{The lifting scheme -- forward direction.}
\label{fig_lift}
\end{figure}
The lifting scheme \cite{wave:sweldens96} is a flexible technique that has 
been used in several different settings, for easy construction and implementation
of traditional wavelets \cite{wave:sweldens96}, and of second generation
wavelets \cite{wave:sweldens95b} such as spherical wavelets \cite{wave:sweldens95a}.

Its principle is to compute the difference between a true coefficient and its
prediction:
\begin{eqnarray}
w_{j+1,l} = c_{j,2l+1} - {\cal P}(c_{j,2l-2L}, ..., c_{j,2l-2},c_{j,2l},c_{j,2l+2}, ..., c_{j,2l+2L})
\end{eqnarray}
A pixel at an odd location 2l+1 is then predicted using pixels at even
locations.

The transformation is done in three steps:
\begin{enumerate}
\item Split the signal into even and odd number samples:
\begin{eqnarray*}
c_{j+1,l} & = & c_{j,2l} \\
w_{j+1,l} & = & c_{j,2l+1}
\end{eqnarray*}
\item Set \[w_{j+1,l} = w_{j+1,l} - {\cal P}(c_{j+1,l})\]
\item Set \[c_{j+1,l} = c_{j+1,l} + {\cal U}(w_{j+1,l})\]
where $\cal U$ is the update operator.
\end{enumerate}
The reconstruction is obtained by:
\begin{eqnarray*}
c_{j,2l} & = & c_{j+1,l} - {\cal U}(w_{j+1,l}) \\
c_{j,2l+1} & = & w_{j+1,l} + {\cal P}(c_{j+1,l}) \\
\end{eqnarray*}
 
\subsection*{Example of transforms}

\subsubsection*{Haar wavelet via lifting}

The Haar transform can be performed via the lifting scheme by taking
the predict operator equal to the identity, and an update operator which
halves the difference. The transform becomes:
\begin{eqnarray*}
w_{j+1,l} =  w_{j+1,l} - c_{j+1,l} \\
c_{j+1,l} = c_{j+1,l} + {w_{j+1,l} \over 2}
\end{eqnarray*}
All computation can be done in place. Every wavelet transform can be written
via lifting.


\subsubsection*{Linear wavelets via lifting}
The identity predictor used before is correct when the signal is constant.
In the same way, we can use a linear predictor which is correct when
the signal is linear. The predictor and update operators are now:
\begin{eqnarray*}
{\cal P}(c_{j-1,l} & = & {1 \over 2} (c_{j-1,l} + c_{j-1,l+1}) \\
 {\cal U}(w_{j-1,l}) & = & {1 \over 4} (w_{j-1,l-1} + w_{j-1,l})
\end{eqnarray*}
It is easy to verify that:
\begin{eqnarray*}
c_{j-1,l} = -{1 \over 8} c_{j,2l-2} + {1 \over 4} c_{j,2l-1} + {3 \over 4} c_{j,2l}
+ {1 \over 4} c_{j,2l+1} -{1 \over 8} c_{j,2l+2}
\end{eqnarray*}
which is the bi-orthogonal Cohen-Daubechies-Feauveau \cite{wave:cohen92}
 wavelet transform.


\subsection*{Integer wavelet transform}

When the input data are integer values, the wavelet transform no longer consists
of integers. For lossless coding, it is useful to have a wavelet 
transform which produces integer values.
We can build an integer version of every wavelet
transform \cite{wave:calderbank96}. For instance, denoting $\lfloor x \rfloor$ 
as the largest integer
not exceeding x, the integer Haar transform  (also called ``S'' transform)
can be calculated by:
\begin{eqnarray}
w_{j+1,l} & = & c_{j,2l+1} - c_{j,2l} \\ \nonumber
c_{j+1,l} & = & c_{j,2l} + \lfloor {w_{j+1,l} \over 2 } \rfloor = c_{j+1,l} + \lfloor {w_{j+1,l} \over 2 } \rfloor
\end{eqnarray}
while the reconstruction is
\begin{eqnarray}
c_{j,2l} & = &  c_{j+1,l} - \lfloor {w_{j+1,l} \over 2} \rfloor \\ \nonumber
c_{j,2l+1} & = &  w_{j+1,l} +  c_{j,2l}
\end{eqnarray}

More generally, the lifting operators for an integer version of the wavelet transform are:
\begin{eqnarray*}
{\cal P}(c_{j+1,l}) & = & \lfloor \sum_k p_k c_{j+1,l-k} + {1 \over 2 } \rfloor\\
 {\cal U}(w_{j+1,l}) & = &  \lfloor \sum_k u_k w_{j+1,l-k} + {1 \over 2 } \rfloor
\end{eqnarray*}

The linear integer wavelet transform is 
\begin{eqnarray*}
w_{j+1,l} & = & w_{j+1,l} - \lfloor {1 \over 2 }(c_{j+1,l}+c_{j+1,l+1})+ {1 \over 2 } \rfloor\\
c_{j+1,l} & = & c_{j+1,l} + \lfloor {1 \over 4 }(w_{j+1,l-1}+w_{j+1,l})+ {1 \over 2 } \rfloor
\end{eqnarray*}
  
Even if there is no filter that consistently performs better than all
the other filters on all images, the following one performs generally better
than others \cite{wave:calderbank96}:
\begin{eqnarray*}
w_{j+1,l} & = & w_{j+1,l} - \lfloor {1 \over 2 }(c_{j+1,l}+c_{j+1,l+1})+ {1 \over 2 } \rfloor\\
c_{j+1,l} & = & c_{j+1,l} + \lfloor {1 \over 4 }(w_{j+1,l-1}+w_{j+1,l})+ {1 \over 2 } \rfloor
\end{eqnarray*}
More filters can be found in \cite{wave:calderbank96}. 
% (the notation (x,y) means that the numbers of wanishing moments
% of the analysing and synthesis high pass filter are respectively x and y):
% \item{integer linear transform: (2,2)}
% More details can be found in \cite{wave:calderbank96}. \\

