\chapter{\proj 1D Data}
\label{ch_1d}
\markright{1D Data}


\section{1D Tools}
\subsection{Conversion: im1d\_convert}
\index{im1d\_convert}
Program {\em im1d\_convert} converts a 1D signal from one data format 
to another one. 
Currently supported signal formats are the ascii format, the FITS format, 
  and Excel format. Suffixes for these formats are 
respectively ``.dat'', ``.fits'',
  and ``.csv"''. The ``-s'' option allows the user to 
suppress a given number
  of lines at the beginning of the file. This option has an effect only  
  for ascii and Excel input formats. The ascii format consists of a series of
  numbers separated by a space, a tab, or a new line.
{\bf
\begin{center}
     USAGE: im1d\_convert [-s NbrLines] file\_name\_in file\_name\_out
\end{center}}
\subsubsection*{Examples:}
\begin{itemize}
\item im1d\_convert sig.dat image.fits \\
Converts an ascii file to FITS format.
\item im1d\_convert -s 2 sig.dat image.fits  \\
Ditto, but the first lines are not taken into account.
\end{itemize}


\subsection{Statistical information: im1d\_info}
\index{im1d\_info}

{\em im1d\_info} gives information about a signal:
\begin{itemize}
\baselineskip=0.4truecm
\itemsep=0.1truecm
\item the number of pixels
\item the minimum, the maximum 
\item the arithmetic mean: $  {\bar x} = \frac{1}{N}\sum_k x_k$
\item the standard deviation: $\sigma = \frac{1}{N} \sum_k (x_k - {\bar x})^2 = \sqrt{{\bar {x^2}} - {\bar x}^2}$
\item the flux: $F  = \sum_k x_k$
\item the energy: $E = \sum_k x_k^2$
\item the skewness: 
$
S  =   \frac{1}{N \sigma^3}\sum_k (x_k - {\bar x})^3  
        =   \frac{1}{\sigma^3} ({\Bar {x^3}} - 3 {\bar x}{\Bar {x^2}} + 2{\bar x}^3)
$ \\
\item the kurtosis:
$
C  =   \frac{1}{N \sigma^4}\sum_k (x_k - {\bar x})^4 - 3 
   =  \frac{1}{\sigma^4} ({\Bar {x^4}} - 4 {\bar x}{\Bar {x^3}} + 6 {\Bar {x^2}}{\bar x}^2 - 3 {\bar x}^4) - 3
$ \\
\item Measure of dependence: calculate the autoregressive model  
which fits the data, for all orders between $1$ and $M$, and select the 
AR model which minimizes the following equation: 
\begin{eqnarray*}
J(p) = \log \sigma_{A(p)}^2 + {\cal P}(A(p))
\end{eqnarray*}
where $\sigma_{A}$ is the prediction error, and
${\cal P}$ is a penalty function, which increases with the AR order.
Examples of penalty functions are:
\begin{itemize}
\item AIC: $AIC = \log \sigma_{A_j}^2 + \frac{2 A_j}{N}$ 
\item AICC: $AICC = \log \sigma_{A_j}^2 + \frac{N +  A_j}{N - A_j - 2}$
\item SIC: $SIC = \log \sigma_{A_j}^2 + \frac{ A_j\log N }{N}$
\end{itemize}
\end{itemize}
When the ``-a'' option is set, then the  autocorrelation function $\rho(h)$
is also calculated:
\begin{eqnarray}
\rho(h) = \frac{\gamma(h)}{\gamma(0)} 	
\end{eqnarray}
 where $\gamma(h)$ is the autocovariance function defined by:
 \begin{eqnarray}
 \gamma(h)= \frac{1}{N} \sum_k (x_{k+h} - {\bar x})(x_k - {\bar x})
 \end{eqnarray}

The command line is:
{\bf
\begin{center}
     USAGE: im1d\_info file\_name\_in 
\end{center}
}
where options are:
\begin{itemize}
\baselineskip=0.4truecm
\itemsep=0.1truecm
\item {\bf[-a Nbr\_of\_Lags]} \\
 Calculate the autocorrelation function (AF) with a given number of lags.
The output AF is saved in a file of name ``autocor''.
Default number of lags is 10.
\item {\bf[-O Estimation\_AR\_Order\_Method]} 
\begin{enumerate}
\baselineskip=0.4truecm
\itemsep=0.1truecm
\item AIC 
\item AICC
\item BIC
\end{enumerate}
Default is BIC method.
\item {\bf[-M MaxAROrder]} \\
Maximum AR model order. Default is 10.
\end{itemize}
\subsubsection*{Examples:}
\begin{itemize}
\itemsep=0.1truecm
\item im1d\_info data.dat  \\
Gives information about the data.
\item im1d\_info -a 10 data.dat \\
Ditto, but calculates also autocorrelation function with 10 lags.
\end{itemize}

\subsection{Tendency estimation: im1d\_tend}
\index{im1d\_tend}
If the signal exhibits a tendency, it may be convenient to remove it
before starting the analysis. A well-known fitting
method is the polynomial one.
A polynomial of order $p$ is fitted around each pixel 
in a window of size $T$ ($p$ equals 2 in this program). 
In order to have smooth tendency estimation,
it is recommended to weight the pixels with weights from 1 to zero for pixels
in the middle to the border of the window.
{\bf
\begin{center}
 USAGE: im1d\_tend option in\_data out\_tend out\_signal\_no\_tend
\end{center}}
where options are:
\begin{itemize}
\baselineskip=0.4truecm
\itemsep=0.1truecm
\item {\bf [-T WindowSize\_for\_tendency\_estimation]} \\
Default is 100.
\item {\bf [-f FirstPixels]} 
Default is the input signal size. If this option is set, the tendency is
estimated only on the first {\em FirstPixels} pixels.
\end{itemize}
\subsubsection*{Examples:}
\begin{itemize}
\item im1d\_tend sig.dat tend.dat sig\_out \\
Remove the tendency in the input data with all default options.
\item im1d\_tend -T 200 sig.dat tend.dat sig\_out   \\
Ditto, but increase the window size.
\end{itemize}
\begin{figure}[htb]
\centerline{
\vbox{ 
\psfig{figure=fig_ar_tend.ps,bbllx=2cm,bblly=12.5cm,bburx=20cm,bbury=25.5cm,width=13.5cm,height=10cm,clip=}
}}
\caption{Top, input signal containing an AR(4) signal and a tendency. Middle,
estimated tendency. Bottom, difference between the two previous signals.}
\label{fig_artend}
\end{figure}  
Figure~\ref{fig_artend} shows the results when applying the {\em tend\_est}
program (with a window size equal to 400) 
to a time series containing a tendency and an AR(4). 


\section{Wavelet Transform}
\subsection{Multiresolution transform: mr1d\_trans}
\label{sect_mr1dtrans}
\index{mr1d\_trans}

Program {\em mr1d\_trans} applies a one-dimensional multiresolution
transform to a signal. The ouput file contains an image, in which
each row is a band of the multiresolution transform. Morlet, Mexican hat,
and French hat wavelet transforms are non-dyadic (the resolution is
not decreased by a factor two between two scales), 
and 12 voices (fixed value) are calculated (instead of one for the
dyadic case) when the resolution is divided by two. Then the number of 
rows in the output image will be equal to 12*{\em number\_of\_scales}.
The Morlet wavelet is complex, so the wavelet transform is complex too.
Using this transform, the first 12*{\em number\_of\_scales}  lines represent
the real part of the transform, and the 12*{\em number\_of\_scales} last lines 
the imaginary part. Four transforms are non-redundant: the bi-orthogonal,
the lifting scheme, and the wavelet packet methods (transforms 16 and 17). 
In this case, the output is not an 
image but a signal (i.e.\ 1D rather than 2D), 
which has the same size as the original one.
Position and length of a given band in the output signal can be found by
reading the file created using the ``-w'' option. For the lifting scheme
based method, the type of lifting can be changed using the ``-l'' option,
and for the (bi-) orthogonal and packet one, the filter can be changed
by the ``-f'' option.

19 transforms are available, which are grouped into 5 classes 
\begin{itemize}
\item{Class 1:} no decimation (transforms 1 to 7 and 11 to 14).
\item{Class 2:} pyramidal transform (transforms 8 to 10).
\item{Class 3:} orthogonal transform (15 and 16).
\item{Class 4:} Wavelet packets (17 and 18).
\item{Class 5:} Wavelet packets via the \`a trous algorithm  (19).
\end{itemize}
Depending on the class, the transform does not contain the
same number of pixels, and the data representation differs.
By default, the number of scales is calculated from
the length of the signal.
{\bf
\begin{center}
 USAGE: mr1d\_trans option signal\_in image\_out
\end{center}}
where options are:
\begin{itemize}
\baselineskip=0.4truecm
\itemsep=0.1truecm
\item {\bf [-t type\_of\_multiresolution\_transform]}
\begin{enumerate}
\baselineskip=0.4truecm
\itemsep=0.1truecm
\item Linear wavelet transform: \`a trous algorithm 
\item B$_1$-spline wavelet transform: \`a trous algorithm 
\item B$_3$-spline wavelet transform: \`a trous algorithm 
\item Derivative of a B$_3$-spline: \`a trous algorithm 
\item Undecimated Haar wavelet transform: \`a trous algorithm
\item Morphological median transform 
\item Undecimated (bi-) orthogonal wavelet transform 
\item Pyramidal linear wavelet transform 
\item Pyramidal B$_3$-spline wavelet transform 
\item Pyramidal median transform 
\item Morlet's wavelet transform \\
Continuous wavelet transform with a complex wavelet 
which can be decomposed into two parts, one for
the real part, and the other for the imaginary part:
\begin{eqnarray*}
\psi_r(x) & = & \frac{1}{\sqrt{2 \pi}\sigma}  e^{-\frac{x^2}{2\sigma^2}} \cos(2\pi\nu_0 \frac{x}{\sigma}) \nonumber \\
\psi_i(x) & = & \frac{1}{\sqrt{2 \pi}\sigma}  e^{-\frac{x^2}{2\sigma^2}} \sin(2\pi\nu_0 \frac{x}{\sigma}) 
\end{eqnarray*}
First scale $\sigma_0$ is chosen equal to $2 \nu_0$ and $\nu_0 = 0.8$, 
using 12 voices per octave.
\item Mexican hat wavelet transform \\
Continuous wavelet transform with the Mexican hat wavelet.
This is the second derivative of a Gaussian
\begin{eqnarray}
\psi(x) = (1 - \frac{x^2}{\sigma^2}) e^{-\frac{ x^2}{2\sigma^2}} 
\end{eqnarray}
First scale $\sigma_0$ is chosen equal to $\frac{1}{\sqrt{3}}$, using 12 voices per octave.
\item French hat wavelet transform \\
Continuous wavelet transform with the French  hat wavelet
\begin{eqnarray}
\psi(x) =  \left \{
\begin{array}{ll}
     0  & \mbox{ if }  \mid \frac{x}{\sigma} \mid > 1 \\
     -1 & \mbox{ if } \mid  \frac{x}{\sigma} \mid \in [\frac{1}{3},1]  \\
     2  & \mbox{ if } \mid  \frac{x}{\sigma} \mid < \frac{1}{3}
\end{array}
\right.
\end{eqnarray}  
First scale $\sigma_0$ equal to $0.66$, and 12 voices per octave. 
\item Gaussian derivative wavelet transform \\
 Continuous wavelet transform. The wavelet is the first 
derivative of a Gaussian
 \begin{eqnarray}
 \psi(x) =  - x e^{-\frac{1}{2}x^2} 
 \end{eqnarray}
 First scale $\sigma_0$ equal to $\frac{1}{\sqrt{3}}$, and 12 voices per octave. 
\item (bi-) orthogonal transform. 
\item (bi-) orthogonal transform via lifting scheme. 
\item Wavelet packets.
\item Wavelet packets via lifting scheme. 
\item Wavelet packets using the \`a trous algorithm.
\end{enumerate}
\item {\bf [-n number\_of\_scales]} \\
Number of scales used in the multiresolution transform.
\item {\bf [-r]} \\
Rebin all scales to the input signal size 
(for pyramidal transforms only).
\item {\bf [-k]} \\
Set to 0 pixels contaminated by the border problem.
\item {\bf [-T type\_of\_filters]}  
{\small
\begin{enumerate}
\baselineskip=0.4truecm
\itemsep=0.1truecm
\item Antonini 7/9 filters. 
\item Daubechies filter 4. 
\item Biorthogonal 2/6 Haar filters.
\item Biorthogonal 2/10 Haar filters.
\item Odegard 7/9 filters.
\item User's filters.
\end{enumerate}}
Default is Antonini 7/9 filters. \\
 This option is only available if the chosen transform method is
 the (bi-) orthogonal transform (-t option in [7,15,17]).
\item {\bf [-L]} \\
Use an $L_2$ normalization. Default is $L_1$.
\item {\bf [-l type\_of\_lifting\_transform]}  
{\small 
\begin{enumerate}
\baselineskip=0.4truecm
\itemsep=0.1truecm
\item Lifting scheme: CDF WT. 
\item Lifting scheme: median prediction.
\item Lifting scheme: integer Haar WT. 
\item Lifting scheme: integer CDF WT. 
\item Lifting scheme: integer (4,2) interpolating transform. 
\item Lifting scheme:  Antonini 7/9 filters.
\item Lifting scheme: integer Antonini 7/9 filters. 
\end{enumerate}}
 Default is Lifting scheme: integer Haar WT. \\
 This option is only available if the chosen transform method is
 the lifing scheme (-t 24).
\item {\bf [-w InfoFileName]} \\
Write in a file the size and the starting index of each band.
This file contains a 2D float array (Array[2, NbrBand+3]).
\begin{verbatim}
        info[0,0] = transform number
        info[1,0] = number of scales
        info[0,1] = transform class number (5 classes)
        info[1,1] = number of bands
                  it is not equal to the number of scales
                  for wavelet packets transform.
        info[0,2] = number of pixels
        info[1,2] = lifting scheme type
        info[0,3] = type of filter
        info[1,3] = type of normalization
        for i=4 to Number_of_bands + 3
        info[0,i] = number of pixels in the band i
        info[1,i] = position number of the pixel of the band
\end{verbatim}
If a user filter file is given (i.e. -T 6,filename), with a filename 
of $L$ characters, $L$ lines are added to the array:
\begin{verbatim}
        info[1,Number_of_bands + 4] = number of characters of the filter file name
        for i=Number_of_bands+4 to  Number_of_bands+4+L-1
	info[0,i] = ascii number of the ith character.
\end{verbatim}

\end{itemize}
\subsubsection*{Examples:}
\baselineskip=0.4truecm
\begin{itemize}
\baselineskip=0.4truecm
\itemsep=0.1truecm
\item mr1d\_trans -n 7 -t 3 sig.fits image.fits \\
\`A trous algorithm with 7 scales.
\item mr1d\_trans -n 7 -T 5 -t 15 sig.fits image.fits  \\
Bi-orthogonal wavelet transform (Odegard 7/9 filters).
\item mr1d\_trans -n 7 -T 5 -t 15 -w info.fits sig.fits image.fits  \\
Ditto, but an information file is written. This information 
is necessary for reconstruction.
\item mr1d\_trans -n 7 -T 5 -t 15 -w info.fits sig.fits image.fits  \\
Ditto, but an information file is written. This information 
is necessary for reconstruction.
\item mr1d\_trans -n 7 -T 6,dau4 -t 15 -w info.fits sig.fits image.fits  \\
Ditto, but the filters defined in the file ``dau4.wvf'' are used instead
of the Odegard filters.
\end{itemize}


\subsection{Reconstruction: mr1d\_recons}
\index{mr1d\_recons}
Program {\em mr1d\_recons} reconstructs a signal from its transform.
To be able to reconstruct it, the program needs the information file
which has been created by the ``-w'' option during the transformation.
{\bf
\begin{center}
 USAGE: mr1d\_recons TransformFileName\_in InfoFileName\_in  signal\_out
\end{center}}
\subsubsection*{Examples:}
\begin{itemize}
\item mr1d\_trans -n 7 -t 3 -w info sig.fits image.fits \\
      mr1d\_recons image.fits info sig\_rec.fits \\
\`A trous algorithm with 7 scales, and reconstruction.
\item mr1d\_trans -n 7 -T 4 -t 15 -w info sig.fits image.fits  \\
mr1d\_recons image.fits info sig\_rec.fits \\
Bi-orthogonal wavelet transform (Odegard 7/9 filters), and reconstruction.
\end{itemize}

Reconstruction is not implemented for transforms 10, 11, and 13. 
For pyramidal transforms,
the reconstruction cannot be performed by {\em mr1d\_recons} if 
the rebin option ``-r'' was set.


\section{Filtering: mr1d\_filter}
\index{mr1d\_filter}
Program {\em mr1d\_filter} filters a signal using different methods. 
{\bf
\begin{center}
 USAGE: mr1d\_filter option signal\_in signal\_out
\end{center}}
where options are:
\begin{itemize}
\baselineskip=0.4truecm
\itemsep=0.1truecm
\item {\bf [-f type\_of\_filtering]} 
\begin{enumerate}
\baselineskip=0.4truecm
\itemsep=0.1truecm
\item  Multiresolution Hard K-Sigma Thresholding 
\item Multiresolution Soft K-Sigma Thresholding 
\item Iterative Multiresolution Thresholding 
\item Universal Hard Thesholding 
\item Universal Soft Thesholding 
\item SURE Hard Thesholding 
\item SURE Soft Thesholding 
\item MULTI-SURE Hard Thesholding 
\item MULTI-SURE Soft Thesholding 
\item Median Absolute Deviation (MAD) Hard Thesholding 
\item Median Absolute Deviation (MAD) Soft Thesholding 
\item Total Variation + Wavelet Constraint 
\end{enumerate}
Default is Multiresolution Hard K-Sigma Thresholding.
\item {\bf [-t type\_of\_multiresolution\_transform]} 
\begin{enumerate}
\baselineskip=0.4truecm
\itemsep=0.1truecm
\item Linear wavelet transform: \`a trous algorithm 
\item B$_1$-spline wavelet transform: \`a trous algorithm 
\item B$_3$-spline wavelet transform: \`a trous algorithm 
\item Derivative of a B$_3$-spline: \`a trous algorithm 
\item Undecimated Haar wavelet transform: \`a trous algorithm 
\item morphological median transform 
\item Undecimated (bi-) orthogonal wavelet transform 
\item pyramidal linear wavelet transform 
\item pyramidal B$_3$-spline wavelet transform 
\item pyramidal median transform 
\end{enumerate}
Default is 3. 
\item {\bf [-T type\_of\_filters]}  
{\small
\begin{enumerate}
\baselineskip=0.4truecm
\itemsep=0.1truecm
\item Antonini 7/9 filters. 
\item Daubechies filter 4. 
\item Biorthogonal 2/6 Haar filters.
\item Biorthogonal 2/10 Haar filters.
\item Odegard 7/9 filters.
\item User's filters.
\end{enumerate}}
Default is Antonini 7/9 filters. \\
This option is only available if the chosen transform method is
the (bi-) orthogonal transform (-t 7).
\item {\bf [-m type\_of\_noise]}
{\small
\begin{enumerate}
\itemsep=0.1truecm
\baselineskip=0.4truecm
\item Gaussian Noise 
\item Poisson Noise 
\item Poisson Noise + Gaussian Noise 
\item Multiplicative Noise 
\item Non-stationary additive noise 
\item Non-stationary multiplicative noise 
\item Undefined stationary noise 
\item Undefined noise 
\item Stationary correlated noise 
\item Poisson noise with few events 
\end{enumerate}
}
Default is Gaussian noise.
\item {\bf [-g sigma]} 
\item {\bf [-c gain,sigma,mean]} 
\item {\bf [-E Epsilon]} \\
Epsilon = precision for computing thresholds
(only used in case of poisson noise with few events).
Default is $1e-03$.
\item {\bf [-n number\_of\_scales]} 
\item {\bf [-s NSigma]} 
\item {\bf [-i number\_of\_iterations]} 
\item {\bf [-e epsilon]}  \\
Convergence parameter. Default is $1e-4$.
\item {\bf [-K]}  \\
Suppress the last scale. Default is no.
\item {\bf [-p]}  \\
Detect only positive structure. Default is no.
\item {\bf [-k]}  \\
Suppress isolated pixels in the support. Default is no.
\item {\bf [-S SizeBlock]} \\
Size of the  blocks used for local variance estimation. Default is 7.
\item {\bf [-N NiterSigmaClip]} \\
Iteration number used for local variance estimation. Default is 1.
\item {\bf [-F first\_detection\_scale]} \\
If this option is set, all wavelet coefficients detected at scales lower 
than {\em first\_detection\_scale} are considered as significant.
\item {\bf [-G RegulParam]} \\
Regularization parameter for filtering methods 11 and 12. default is 0.01.
\item {\bf [-v]}  \\
Verbose. Default is no.
\end{itemize}

\subsubsection*{Examples:}
\begin{itemize}
\item mr1d\_filter sig.fits filter\_sig.fits \\
Filtering using the \`a trous algorithm, and a Gaussian noise model.
\item mr1d\_filter -m 2 sig.fits filter\_sig.fits   \\
Ditto, but assuming Poisson noise.
\end{itemize}


\section{Band Detection: mr1d\_detect}
\index{mr1d\_detect}

Program {\em mr1d\_detect} detects emission or absorption bands in 
a spectrum \cite{starck:sta97}. The output is an image, in which each row
contains a detected band. By default, {\em mr1d\_detect} detects
both emission and absorption bands. The program gives better results
if we detect only emission (``-e" option) bands, or only absorption bands
(``-a" option). This is due to the fact that  positivity and negativity 
constraints can be introduced in the iterative reconstruction for, 
respectively, 
emission and absorption band detection. If the user wants to 
detect both kinds of band at the same time, he/she should 
use the multiresolution
median transform (``-M" option) which gives in this case better results
than the wavelet transform.
{\bf
\begin{center}
 USAGE: mr1d\_detect option signal\_in tab\_band\_out
\end{center}}
where options are:
\begin{itemize}
\baselineskip=0.4truecm
\itemsep=0.1truecm
\item {\bf [-n number\_of\_scales]} 
\item {\bf [-s NSigma]} 
\item {\bf [-m type\_of\_noise]}
{\small
\begin{enumerate}
\baselineskip=0.4truecm
\itemsep=0.1truecm
\item Gaussian Noise 
\item Poisson Noise 
\item Poisson Noise + Gaussian Noise 
\item Multiplicative Noise 
\item Non-stationary additive noise 
\item Non-stationary multiplicative noise 
\item Undefined stationary noise 
\item Undefined noise 
\end{enumerate}
}
Description in section~\ref{sect_filter}. Default is Gaussian noise.
\item {\bf [-g sigma]} 
\item {\bf [-c gain,sigma,mean]} 
\item {\bf [-a]} \\
Detection of Absorption lines only. Default is not to do this. 
\item {\bf [-e]} \\
Detection of Emission lines only. Default is not to do this. 
\item {\bf [-f FirstScale]} \\
First scale. Default is 1. All bands detected at the scale smaller than 
{\em FirstScale} are not reconstructed.
\item {\bf [-l LastScale]} \\
Last scale. Default is number\_of\_scales -- 2.
All bands detected at the scale higher than {\em LastScale} are not reconstructed.
\item {\bf [-i IterNumber]} \\
Number of iterations for the reconstruction. Default is 20.
 \item {\bf [-M]} \\
Use the multiresolution median transform 
instead of the \`a trous algorithm. If this option is set, the reconstruction
is not iterative, and the previous option (``-i") is not active.
\item {\bf [-A ]} \\
Detect only negative multiresolution coefficients. Default is no. 
\item {\bf [-E ]} \\
Detect only positive multiresolution coefficients. Default is no. 
\item {\bf [-w ]} \\
Write two other files: 
\begin{itemize}
\item tabadd.fits: sum of the reconstructed objects. All detected bands are co-added. 
\item tabseg.fits: segmented wavelet transform.
\end{itemize}
\item {\bf [-v ]} \\
Verbose. Default is no.
\end{itemize}

\subsubsection*{Examples:}
\begin{itemize}
\item mr1d\_detect -a -A sig.fits band\_sig.fits \\
Detection using the \`a trous algorithm, and a Gaussian noise model.
The detection is performed from the positive wavelet coefficients, and
only emission bands are reconstructed.
\item mr1d\_detect -m 2 -a -A -s 5 sig.fits band\_sig.fits   \\
Ditto, but assuming Poisson noise and a five sigma detection.
\end{itemize}

\begin{figure}[htb] 
\centerline{
\vbox{
\hbox{
\psfig{figure=ch4_sig10_wid02.ps,bbllx=1.5cm,bblly=19cm,bburx=20cm,bbury=25.5cm,height=5cm,width=16cm,clip=}}
\hbox{
\psfig{figure=ch4_extr_simu3.ps,bbllx=1.5cm,bblly=13cm,bburx=20cm,bbury=25.5cm,height=10cm,width=16cm,clip=}}
}}
\caption{Simulation: top, simulated spectrum, 
middle, reconstructed simulated band (full line) and original band (dashed line). Bottom,
simulated spectrum minus the reconstructed band.}
\label{fig_detect_spect}
\end{figure}

A simulated spectrum was created containing a smooth continuum  
with superimposed, variable Gaussian noise and an emission band at $3.50\mu$m 
with a maximum of five times the local noise standard 
deviation and a width of 
FWHM = $0.01 \mu$m (see Figure~\ref{fig_detect_spect} top). 
Figure~\ref{fig_detect_spect}
(middle) shows the comparison of the 
reconstructed emission band with the original Gaussian.
Figure~\ref{fig_detect_spect} (bottom) shows the  difference between 
the simulated spectrum and the reconstructed band.


\section{Modulus Maxima Representation}
\subsection{Modulus maxima detection: mr1d\_max}
\index{mr1d\_max}
Program {\em mr1d\_max} detects the modulus maxima using the dyadic
  \`a trous algorithm  with the wavelet function equal to the derivative
of a B-spline (transform number 4 in {\em mr1d\_trans}).
{\bf
\begin{center}
 USAGE: mr1d\_max option signal\_in  output
\end{center}}
where options are:
\begin{itemize}
\item {\bf [-n number\_of\_scales]} 
\item {\bf [-v ]} \\
Verbose. Default is no.
\end{itemize}
\subsubsection*{Example:}
\begin{itemize}
\item mr1d\_max sig.fits band\_sig.fits   \\
Modulus maxima detection.
\end{itemize}


\subsection{Reconstruction: mr1d\_maxrecons}
\index{mr1d\_maxrecons}
Program {\em mr1d\_maxrecons} reconstructs a signal from its modulus maxima.
{\bf
\begin{center}
 USAGE: mr1d\_maxrecons option tab\_band\_in signal\_out 
\end{center}}
where options are:
\begin{itemize} 
 \item {\bf [-i number\_of\_iterations]}  \\
Maximum number of iterations for the reconstruction. Default is 10.
\item {\bf [-v ]} \\
Verbose. Default is no.
\end{itemize}
\subsubsection*{Example:}
\begin{itemize}
\item mr1d\_max sig.fits band\_sig.fits   \\
      mr1d\_maxrecons band\_sig.fits sig\_rec.fits \\
Modulus maxima detection, and reconstruction from the maxima.
\end{itemize}

