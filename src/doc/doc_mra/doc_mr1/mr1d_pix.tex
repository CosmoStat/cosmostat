\subsection{Transition detection: mr1d\_pix}
Program {\em mr1d\_pix} detects the transitions in all scales at a 
given position. The used wavelet transform is the Haar transform,
so a given wavelet coefficient at position $x$ and at scale $j$ ($j=1..P$,
$P$ being the number of scales)
is calculated from pixel values between  positions $x-2^{j}+1$ and $x$.
Only pixels in the signal which are on the left of a given position $x$
(or before a given time for temporal signal) are
used for the calculation of the wavelet coefficients at position $x$.
It allows us to detect new event in a temporal serie whatever the time 
scale of the event.
By default, the analysed position is the last one of the signal, but other
position can as well being analysed using the "-x" option. The program
prints for each scale $j$ the following information corresponding to 
the position $x$:
\begin{itemize}
\item {\bf No detection} \\
 if the wavelet coefficient $\mid w_j(x) \mid  < k \sigma_j$
\item {\bf New upward detection}\\
 if $  w_j(x) > k \sigma_j$  and $\mid w_j(x-1) \mid  < k \sigma_j$
\item {\bf New downward detection}\\
 if $  w_j(x) < - k \sigma_j$ and $\mid w_j(x-1) \mid  < k \sigma_j$
\item {\bf Positive significant structure}\\
 if $  w_j(x) > k \sigma_j$ and $\mid w_j(x-1) \mid  > k \sigma_j$ \\
The first detected coefficient of the structure is also given.
\item {\bf Negative significant structure}\\
 if $  w_j(x) < -k \sigma_j$ and $\mid w_j(x-1) \mid  > k \sigma_j$ \\
The first detected coefficient of the structure is also given.
\item {\bf End of significant structure}\\
 if $\mid w_j(x) \mid  < k \sigma_j$ and $\mid w_j(x-1) \mid  > k \sigma_j$
\end{itemize}
Furthermore the signal to noise ratio of the wavelet coefficient is given.
{\bf
\begin{center}
 USAGE: mr1d\_pix option signal\_in  
\end{center}}
where options are:
\begin{itemize}
\item {\bf [-m type\_of\_noise]}
{\small
\begin{enumerate}
\baselineskip=0.4truecm
\item Gaussian Noise 
\item Poisson Noise 
\item Poisson Noise + Gaussian Noise 
\item Multiplicative Noise 
\item Non-stationary additive noise 
\item Non-stationary multiplicative noise 
\item Undefined stationary noise 
\item Undefined noise 
\end{enumerate}
}
Description in section~\ref{sect_filter}. Default is Gaussian noise.
\item {\bf [-g sigma]} 
\item {\bf [-c gain,sigma,mean]} 
\item {\bf [-n number\_of\_scales]} 
\item {\bf [-s NSigma]} 
\item {\bf [-n number\_of\_scales]} 
\item {\bf [-x Position]}  \\
Position to analyse. Default is the last point.
\end{itemize}
\subsubsection*{Examples:}
\begin{itemize}
\item mr1d\_pix sig.dat  \\
Analyse the last point of the signal with all default option.
\item mr1d\_pix -x 55 -s 10 sig.dat  \\
Analyse the point at position 55, and detect the transition with a 
signal to noise ratio equal to 10.
\end{itemize}


