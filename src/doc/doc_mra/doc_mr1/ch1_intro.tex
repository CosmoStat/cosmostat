 
\chapter{Introduction}
\label{ch_intro}
% \chapterhead{Introduction}
\markright{Introduction}

% \proj (\defproj) 
\proj is a
set of software components 
developed by CEA (Saclay, France) and Nice Observatory. This
project originated in astronomy, and involved the development of a range of 
innovative methods built around multiscale analysis.
Multiresolution techniques have
been developed  in recent years, and furnish a powerful and 
insightful representation
of the data. By means of multiresolution or multiscale analysis, an 
image can be decomposed into a set of images (or scales), each scale 
containing only structures of a given size. 
This data representation, associated with noise modeling, 
has been applied to very different applications such as data filtering, 
deconvolution, compression, object detection, and so on. Results are 
enhanced in all such processing because the multiresolution approach
allows a better understanding of how the data values 
are distributed in an image, 
and how the signal can be separated from the noise.

The \proj software 
components include almost all applications 
presented in the book {\em Image and Data Analysis: the Multiscale Approach}
 \cite{starck:book98}. The goal of 
\proj is not to replace 
existing image processing  packages, but to complement
them, offering the user a complete set of multiresolution tools. These tools
are executable programs, which work on a wide range of platforms, 
independently of current 
image processing systems.  They allow the user to perform various tasks 
using multiresolution, such as wavelet transforms, filtering, 
deconvolution, and so on.

The programs, written in C++, are built on three classes: the  
``image" class, the ``multiresolution" class, and the 
``noise\_modeling class".
Fig. \ref{fig_sadam1} illustrates this architecture. A multiresolution
transform is applied to the input data, and noise modeling is performed.
Hence the multiple scales  can be derived, and the programs can use this
in order to know at which scales, and at which positions, significant
signal has been detected. 
A wide range of 
multiresolution transforms are available (see Fig.~\ref{fig_sadam2}),
allowing significant flexibility.
Fig.\ \ref{fig_modelnoise} summarizes how the multiresolution support data
structure 
is derived from the data and the noise-modeling. 


A set of  IDL\footnote{Research Systems 
Inc., 2995 Wilderness Place, Boulder, Colorado 80301.} (Interactive 
Data Language) 
and PV$\sim$Wave\footnote{Visual Numerics Inc.,  6230 Lookout Road, 
Boulder, Colorado 80301, USA.}
routines
are included in the package which interface the executables to these
image processing packages.  

\proj is an important package, 
introducing front-line methods to scientists 
in the physical, space and medical domains among other fields; to engineers in 
such disciplines as geology and electrical engineering; and to financial 
engineers and those in  other fields requiring control and analysis of 
large quantities of noisy data.  

% The first release of the package is at the 
% beginning of 1997.  
% Further information on the pre-release version, codenamed 
% \proj, can be found at  \\ 
% http://www.dapnia.cea.fr/Sadam

\begin{figure}[t]
\centerline{
\hbox{
\psfig{figure=ch_annex2_sadam.ps,bbllx=2.5cm,bblly=3.cm,bburx=17.5cm,bbury=25cm,height=12cm,width=12cm,clip=}
}}
\caption{\proj diagram.}
\label{fig_sadam1}
\end{figure}

\begin{figure}[htb]
\centerline{
\hbox{
% \psfig{figure=ch_annex2_mr_trans.ps,bbllx=0.5cm,bblly=3.5cm,bburx=20.5cm,bbury=24.5cm,height=16cm,width=15cm,clip=}
\psfig{figure=fig_mr1_transf.ps,bbllx=0.2cm,bblly=3.5cm,bburx=20.5cm,bbury=24.5cm,height=16cm,width=15cm,clip=}
}}
\caption{Multiresolution transforms available in \proj (a selection).}
\label{fig_sadam2}
\end{figure}

\begin{figure}[htb]
\centerline{
\hbox{
\psfig{figure=ch2_noise.ps,bbllx=0.5cm,bblly=1.5cm,bburx=22cm,bbury=26.5cm,width=13cm,height=16.66cm,clip=}}
}
\caption{Determination of multiresolution support from noise modeling.}
\label{fig_modelnoise}
\index{multiresolution support}
\index{support, multiresolution}
\end{figure}
