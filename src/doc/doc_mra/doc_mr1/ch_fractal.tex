\chapter{\proj Wavelet and Multifractal Analysis}
\label{ch_fractal}
\index{fractal}
% \chapterhead{Programs}
\markright{Wavelet and Multifractal Analysis}

\section{Fractal}
\subsection{Introduction}
The word ``fractal'' was introduced by Mandelbrot 
(1977) \cite{frac:mandel83}, and
comes from the Latin word {\em fractus} which means ``to break''.
According to Mandelbrot, a fractal is an object which has 
a greater dimension than its topological dimension. A typical
fractal is the Cantor set.
 
\subsubsection*{Example: the triadic Cantor set} 
\label{triadic Cantor set}
The Cantor set is built in the following way: considering a
segment of dimension $L$, we separate it into three equal parts,
and suppress the middle part. There remain two segments of size
${L \over 3}$. Repeating the process on both segments, we get 
four segments of size ${3^{-2}L}$. After $n$ iterations, we have
$2^n$ segments of size ${3^{-n}L}$. The Cantor set is obtained 
when $n$ tends to infinity. The set has the following properties:
\begin{enumerate}
\item It is self-similar.
\item It has a fine structure, i.e.\ detail on arbitrary small
scales.
\item It is too irregular to be described in traditional 
geometrical language, both locally and globally.
\item It is obtained by successive iterations.
\item It has an infinite number of points but its length is null.
\end{enumerate}
These properties define in fact a fractal object. 

A real fractal does not exist in the nature, and we always need
to indicate at which scales we are talking about fractality.

\subsection{The Hausdorff and Minkowski measure} 
\label{Hausdorff}
\subsubsection*{Measure}
An object dimension describes how an object $F$ fills the space.
A simple manner of measuring the length of
curves, the area of surfaces or the volume of objects is to divide the 
space into small boxes (segment in one dimension, surface in 2D, and cubes
in 3D)  of diameter $\delta$ as shown in Figure \ref{fig_mes} 
\cite{frac:falconer90}. These boxes
are chosen so that their diameter is not greater than a given size $\delta$,
which corresponds to the measure resolution.
\begin{figure}[htb]
\centerline{
\hbox{
\psfig{figure=fig_frac1.ps,bbllx=5cm,bblly=12cm,bburx=16cm,bbury=18cm,height=6cm,width=12cm,clip=}
}}
\caption{Measuring the ``size'' of curves (Feder 1988).}
\index{Mallat's multiresolution}
\label{fig_mes}
\end{figure}

We consider now the quantity:
\begin{eqnarray}
L_{\delta}^d(F) = \sum diam(B_i)^d
\end{eqnarray}
where $d$ is a real number, $diam(B_i)$ is the diameter of the  box  $i$.
$L_{\delta}^s(F)$ represents an estimation of the size of $F$ at the resolution
$\delta$. Depending on the choice of the boxes, the measure is more or less 
correct. Generally, it is easier to manipulate the Minkowski-Bouligand 
measure which fixes all the boxes to the same size $\delta$. Using this
measure, the size of $F$ is given by:
\begin{eqnarray}
M^s(F) = \lim_{\delta \rightarrow 0} \sum diam(B_i)^s = \delta^s N_B(\delta)
\end{eqnarray}
where $N_B$ is the number of boxes needed to cover the object $F$.
Then the curves of length $L_\ast$ in Figure~\ref{fig_mes}   
can be measured by finding the number $N_B(\delta)$ of line segments
 (respectively squares and cubes for the second and third object)
of length $\delta$ needed to cover the object. The three sizes are:
\begin{eqnarray}
L  &  = M^1(F) = & N_B(\delta) \delta^1 \mathop\rightarrow_{\delta \to 0} 
      L_\ast \delta^0  \nonumber  \\
A  & = M^2(F) = & N_B(\delta) \delta^2 \mathop\rightarrow_{\delta \to 0}
      L_\ast \delta^1 \nonumber \\
V  & = M^3(F) = & N_B(\delta) \delta^3  \mathop\rightarrow_{\delta \to 0}
      L_\ast \delta^2         
\end{eqnarray}

\subsection{The Hausdorff and Minkowski dimension}

The Hausdorff dimension $d_H$ of the set $F$ is the {\it
critical dimension} for which the measure $H^d(F)$ jumps from infinity
to zero:
\begin{eqnarray}
  H^d(F) =  \left\{ \begin{array}{ll}
                      0,      &  d > d_H, \\
                      \infty, &  d < d_H.
                   \end{array}
           \right.
 \label{eq-ch2-1}
\end{eqnarray}
But $H^{d_H}(F)$ can be finite or infinite. For a simple set 
(segment, square, cube),
Hausdorff dimension is equal to the topological dimension (i.e.\ 1, 2, 
or 3). This is not true for a more complex set, such as the Cantor set.
Minkowski dimension $d_M$ (and $d_H (F) \le d_M(F)$) is defined 
in a similar way using Minkowski measure.

By definition, we have:
\begin{eqnarray}
M^{d_M} = \lim_{\delta \rightarrow 0} \delta^{d_M} N_B(\delta)
\end{eqnarray}
When $\delta \rightarrow 0$, we have $d_M \ln M= d_M\ln{\delta} + \ln{N_N(\delta)}$. If
$M$ is finite, the Minkowski dimension, also called box-counting,
 can be defined by
\begin{eqnarray}
 d_M = \lim_{\delta \rightarrow 0} {\ln{N_B(\delta)} \over - \ln{\delta}}
\end{eqnarray}

In the case of the Cantor set, at iteration $n$, we have $2^n$ segments
of size $3^{-n}$ ($\delta = 3^{-n}$).  When $n  \rightarrow  \infty$, we have
\begin{eqnarray}
d_M (Cantor) = {\ln{2^n} \over {- \ln{3^{-n}}}}  = {\ln{2} \over \ln{3}}
\end{eqnarray}

\section{Multifractality}
The multifractal picture is a refinement and
generalization of the fractal properties
that arise naturally in the case of self-similar distributions.
The singularity spectrum $f(\alpha)$ can be introduced as a quantity
which characterizes the degree of regularity and homogeneity of a 
fractal measure.

\subsection{Definition}
\subsubsection*{H\"older exponent}
A multifractal measure describes a non-homogeneous set $A$. 
Such a measure is called multifractal if it is everywhere self-similar, i.e.\
if the measure varies locally as a power law, at any point of A. 
Denoting $\mu$ a measure, we call the H\"older exponent or singularity exponent
at $x_0$  the limit
\be
\alpha(x_0) =  \lim_{\delta \rightarrow 0} { \ln{\mu(B_{x_0}(\delta))} 
            \over \ln{\delta} }
\ee
where $B_{r_0}$ is a box centered at $r_0$ of size $\delta$.
We have:
\be
\mu(B_{x_0}(\delta)) \propto \delta^{\alpha(x_0)}
\ee
The smaller the value $\alpha(x_0)$, the less the measure is regular around
$x_0$. For example, if $\mu$ corresponds to a Dirac distribution centered 
at $0$, 
then $\alpha(0) = 0$, and if $\mu$ corresponds to a Gaussian distribution, then
$\alpha(0) = -1$.

\subsubsection*{Singularity spectrum}

The singularity spectrum, associated with a measure $\mu$, is the function
which associates with $\alpha$ the fractal dimension of 
any point $x_0$ such that
$\alpha(x_0) = \alpha$:
\be
f(\alpha) = d_F(\{ x_0 \in A \mid \alpha(x_0) = \alpha \})
\ee
 The function $f(\alpha)$ is usually (\cite{frac:paladin86})
a single-humped
function with the maximum at $max_{\alpha} f(\alpha) = D$, 
where $D$ is the dimension of the support. In the case
of a single fractal, the function $f(\alpha)$ is reduced to
a single point: $f(\alpha) = \alpha = D$.

The singularity spectrum describes statistically the $\alpha$ exponent
distribution on the measure support. For example, if we split the
support into boxes of size $\delta$, then the number of boxes with a
measure varying as $\delta^\alpha$ for a given $\alpha$ is
\be
N_\alpha(\delta) \propto \delta^{-f(\alpha)}
\ee
$f(\alpha)$ describes the histogram of $N_\alpha(\delta)$ when
$\delta$ is small. A measure is homogeneous if its singularity spectrum
is concentrated in a single point. If $f(\alpha)$ is large, the
measure is multifractal.

\subsubsection*{Multifractal quantities}
From a practical point of view one does not determine directly the
spectrum of exponents $[f(\alpha), \alpha]$; it is more convenient to
compute its Legendre transformation $[\tau(q),q]$ given by
\be
\left\{
\begin{array}{l}
\label{mf12}
f(\alpha) = q \cdot \alpha - \tau(q)\\
\alpha    = \frac{d\tau(q)}{dq}
\end{array}
\right.
\ee
In the case of a simple fractal one has $\alpha=f(\alpha)=D$.  In
terms of the Legendre transformation this corresponds to
\be
\tau(q) = D(q-1)
\ee
i.e.\ the behavior of $\tau(q)$ versus $q$ is a straight line with
coefficient given by the fractal dimension.


\subsection{Generalized fractal dimension}

\subsubsection*{Definition}
The generalized fractal dimension, also called Reyni dimension of
order $q$, is given by:
\be
D_q = { \tau(q) \over q -1 }
\ee
$D_0$ is also called capacity dimension, and coincides with the
Hausdorff dimension. Dimensions $D_1$, and $D_2$ are respectively 
called information and correlation dimension.
 
\subsubsection*{Partition function}
The partition function $Z$ is defined by:
\be
Z(q,\delta) = \sum_{i=1}^{N(\delta)} \mu_i^q(\delta)
\ee
where we denote $\mu_i(\delta) = \mu(B_i(\delta))$.  If the measure
$\mu$ is multifractal, $Z$ follows a power law in the limit $\delta
\rightarrow 0$.
\be
Z(q,\delta) \propto \delta^{\tau(q)} 
% \label{partition function}
\ee

The box-counting method consists of calculating the partition
function, to derive from $Z$ $\tau(q)$, and to obtain, the multifractal
spectrum by a Legendre transform.

\section{Wavelets and Multifractality}

\subsection{Singularity analysis}

Let $f(x)$ be the input signal, $x_0$ the singularity location,
$\alpha(x_0)$ the H\"older exponent at the singularity point $x_0$ and
$n$ the degree of Taylor development such that
$n\leq\alpha(x_0)<n+1$. We have
\be
f(x) & = & f(x_0) + (x-x_0)\,f^{(1)}(x_0) + ... +  \\ \nonumber
     &   & \frac{(x-x_0)^n}{n!}\,f^{(n)}(x_0) + C\,|x-x_0|^{\alpha(x_0)}
\ee

Letting $\psi$ be the wavelet with $n_\psi>n$ vanishing moments, then we have
for the wavelet transform of $f(x)$ at $x_0$ when the scale goes to 0
($\psi$ is orthogonal to polynomials up to order $n$):

\be
\lim_{scale s \rightarrow 0^+} T_\psi[f](x_0,s) \sim {a^{\alpha(x_0)}}
% \label{Mallat Hwang}
\ee

One can prove that if $f$ is $C^\infty$, then we have

\be
\lim_{scale s \rightarrow 0^+} T_\psi[f](x_0,s) \sim {a^{n_\psi}}
\ee

Thus, we have 
\begin{eqnarray}
\left\{  \begin{array}{ll}
 	T_\psi[f] \sim s^{n_\psi} & \mbox{ where the signal f is regular} \\
	T_\psi[f] \sim s^{\alpha} (>>s^{n_\psi}) & \mbox{ around a singular zone}
	\end{array}
	\right.
\end{eqnarray} 

For a fixed scale s, $T_\psi[f](.,s)$ will be greater when the signal is
singular. This local maximum is organized in maxima lines (function of s) 
which
converges, when s goes to 0, to a singularity of the signal.
Mallat and Hwang (1992) demonstrate that to recover the H\"older exponent 
$\alpha(x_0)$ at $x_0$, one need only study the  wavelet transform along these
lines of maxima which converge (when the scale goes to 0) towards the singularity
point $x_0$.

Along this maxima line l we have
\be
T_\psi[f](b,s) \sim a^{\alpha(x_0)}, \,\,(b,a)\in l, s\rightarrow 0^+
\ee
Figure~\ref{fig_2} displays the function $f(x)=K(x-x_0)^{0.4}$ with a singular
point at $x_0$. The H\"older exponent at $x_0$ is equal to 0.4.
In Figure~\ref{fig_3}, we display the wavelet transform of $f(x)$ with a wavelet
$\psi$ which is the first derivative of a Gaussian. In Figure~\ref{fig_4}, we
display $log_2 {|T_\psi[f](x,s)|}$ as a  function of $\log_2(s)$. 

\begin{figure}[htb]
\centerline{
\vbox{
\psfig{figure=singul.ps,height=5cm,width=8cm}
}}
\caption{Function $f(x)=K(x-x_0)^{0.4}$ \cite{frac:arneodo90}.}
\label{fig_2}
\end{figure}

\begin{figure}[htb]
\centerline{
\vbox{
\psfig{figure=WT.ps,height=5cm,width=8cm}
}}
\caption{
Wavelet transform of $f(x)$ with a wavelet $\psi$ which is the first 
derivative of
a Gaussian. The small scales are at the top. The maxima line converges to the
singularity point at $x_0$.}
\label{fig_3}
\end{figure}

\begin{figure}[htb]
\centerline{
\vbox{
\psfig{figure=holder_expo.ps,height=5cm,width=8cm}
}}
\caption{
Estimation of the H\"older exponent $\alpha(x_0)$ at 
$x_0$ by computing the slope of $\log_2 {|T_\psi[f](x,s)|}$ versus
$\log_2(s)$ along a maxima line which converges to $x_0$.}
\index{Mallat's multiresolution}
\label{fig_4}
\end{figure}

When we compute the slope of the curve $log_2 {|T_\psi[f](x,s)|}$
versus $\log_2(s)$ along a maxima line which converges at $x_0$, 
we obtain an estimation of the H\"older exponent (in this case
$\alpha(x_0) \approx 0.4$
which corresponds to the theoretical value).


\subsection{Wavelet transform of multifractal signals }

The estimation of the H\"older exponent by this method 
 becomes inaccurate
in the case of multifractal signals \cite{frac:arneodo90}. 
We need to use a more global
method. One can define the wavelet-based partition function by

\be
Z(q,s) = \sum_{b_i} |T_\psi[\mu](b_i,s)|^q
\ee
where ${(b_i,s)}_i$ are all local maxima at scale $s$.

Let $\tau(q)$ be the scaling exponent. We can prove that we have
\be
Z(q,s) \sim a^{\tau(q)}
\ee

We can then calculate the singularity spectrum $D(\alpha)$ by its Legendre
transformation 
\be
D(\alpha) = \underset{q}{\min} (q\alpha-\tau(q))
\ee
 
This method is called the {\em Wavelet Transform Modulus Maxima} (WTMM).
 

\subsection{Numerical applications of WWTM method}

The calculation of the singularity spectrum of signal f proceeds as follows:
\begin{list}{--}{\itemindent=-5mm \itemsep=-1mm}
\item compute the wavelet transform and the modulus maxima $T_\psi[f]$ for
all $(s,q)$. We chain all maxima across scale
lines of maxima
\item compute the partition 
function $Z(q,s) = \sum_{b_i} |T_\psi[f](b_i,s)|^q$
\item compute $\tau(q)$ with 
$\log_2{Z(q,s)}\,\approx\,\tau(q)\log_2(s)+C(q)$
\item compute $D(\alpha) = \underset{q}{\min}  (q\alpha-\tau(q))$
\end{list}

\subsubsection*{The triadic Cantor set}

Definition : The measure associated with the triadic Cantor set is defined by 
$f(x) = \int_0^x d\mu$ where $\mu$ is the uniform measure lying on the triadic 
Cantor set described in section \ref{triadic Cantor set}. To compute $f(x)$, we used the next recursive
function called the Devil's Staircase function 
(which looks like a staircase whose steps are
uncountable and infinitely small, see Figure~\ref{fig_5}). 

\begin{eqnarray}
f(x)=	\left\{   \begin{array}{ll}
	p_1f(3x)       & \mbox{ if $x \in [0,\frac{1}{3}]$}\\
	p_1            & \mbox{ if $x \in [\frac{1}{3}, \frac{2}{3}]$ }\\
	p_1+p_2f(3x-2) & \mbox{ if $x \in [\frac{2}{3}, 1]$}
	\end{array}
	\right.
% \label{recursive Cantor func}
\end{eqnarray}

This is a continuous function that increases  from 0 to 1 
on [0,1]. The recursive
construction of $f(x)$ implies that $f(x)$ is self-similar.

\begin{figure}[htb]
\centerline{
\hbox{
	\psfig{figure=DSC3.ps,height=5cm,width=8cm}
    \psfig{figure=DSC3ParFun.ps,height=5cm,width=8cm}
}}
\caption{Classical Devil's Staircase function (associated with
triadic Cantor set) with $p_1=0.5$ and $p_2=0.5$
(left) and partition function $Z(q,s)$ for several 
values of $q$ (right). The
wavelet transform is calculated with $\psi$ equal to the 
first derivative of a Gaussian.}
\label{fig_5}
\end{figure}


\begin{figure}[htb]
\centerline{
\hbox{
	\psfig{figure=DSC3ScaExp.ps,height=5cm,width=8cm}
    \psfig{figure=DSC3SinSpe.ps,height=5cm,width=8cm}
}}
\caption{Scaling Exponent estimation $\tau(q)$ (left) and Singularity
Spectrum $D(\alpha)$ (right). The Scaling Exponent curve  corresponds
 to the theoretical curve
$\tau(q)=(q-1)\log_2(2)/\log_2(3)$. Points of the Singularity Spectrum
where $q \neq \infty(\max\,q)$ and  $q \neq-\infty(\min\,q)$ are reduced 
to a single point
($\alpha=\log_2(2)/\log(3), D(\alpha)=\log _2(2)/\log _2(3))$. This point
corresponds to
the Hausdorff dimension of the triadic Cantor Set.}
\label{fig_6}
\end{figure}



\subsubsection*{The generalized Devil's Staircase with $p_1=0.4$ and $p_2=0.6$}

One can prove \cite{frac:arneodo90} that the theoretical singularity spectrum $D(\alpha)$
of the generalized Devil's Staircase function $f(x)=\int_0^x{d\mu}$ verifies
the following: 
\begin{itemize}
\item The singular spectrum is a convex curve with a maximum value
$\alpha_{max}$ which corresponds to the fractal dimension of the
support of the measure ($\mu$). 
\item The theoretical support of $D(\alpha)$ is reduced at the interval
$[\alpha_{min},\alpha_{max}]$:

\begin{eqnarray}
\left\{ \begin{array}{l}
 \alpha_{min}=\min({\frac{\ln p_1}{\ln (1/3)}, \frac{\ln p_2}{\ln (1/3)}})\\
 \alpha_{min}=\min({\frac{\ln p_1}{\ln (1/3)}, \frac{\ln p_2}{\ln (1/3)}})
	\end{array}
	\right.
\end{eqnarray}
\end{itemize}

Figure~\ref{fig_7} displays the generalized Devil's Staircase and its 
partition 
function $Z(q,s)$. In Figure~\ref{fig_8}, we can see the Scaling exponent and
the Singularity spectrum. This one is in perfect ``accord'' with 
the theoretical
values: bell curve, $D(\alpha)_{max}=\log_2(2)/\log_2(3)$,
$\alpha_{min}\approx 0.47$ and $\alpha_{max}\approx 0.83$

\begin{figure}[htb]
\centerline{
\hbox{
    \psfig{figure=DSC3b.ps,height=5cm,width=8cm}
    \psfig{figure=DSC3bParFun.ps,height=5cm,width=8cm}
}}
\caption{Devil's Staircase function with $p_1=0.4$ and $p_2=0.6$
(left) and partition function $Z(q,s)$ for several values of $q$ (right). 
The wavelet transform is calculated
with $\psi$ equal to the first derivative of a Gaussian.}
\label{fig_7} 
\end{figure}



\begin{figure}[htb]
\centerline{
\hbox{
	\psfig{figure=DSC3bScaExp.ps,height=5cm,width=8cm}
    \psfig{figure=DSC3bSinSpe.ps,height=5cm,width=8cm}
}}
\caption{Scaling Exponent estimation $\tau(q)$ (left) and Singularity
Spectrum $D(\alpha)$ (right). The theoretical maximum value of $D(\alpha)$ is
obtained for $\alpha_{max D}=\log_2(2)/\log_2(3)$ and
$D(\alpha_{max D})=\log_2(2)/\log_2(3)$.}
\label{fig_8}
\end{figure}


\newpage
\section{Program}
\subsection{Devil's Staircase: mf1d\_create\_ds}
\index{mf1d\_create\_ds}
Program {\em mf1d\_create\_ds} creates a Devil's Staircase function lying on the
Cantor sets measure. If the ``-b'' option is set, a border is added, which
is equal to 0 on the left side, and to 1 on the right side. The size of the
border is ${N\over 3}$ where $N$ is the number of points. Hence the output
signal size becomes $N + {2N\over 3} = {5N\over 3}$. It is recommended to
choose a number of points which can be divided by 3.

{\bf
\begin{center}
USAGE:  mf1d\_create\_ds options image out
\end{center}}
where options are 
\begin{itemize}
\item {\bf [-n number\_of\_points]} \\
Number of points of Devil's Staircase. Default is 243.
\item {\bf [-p Prob1]} \\
Coefficient $p_1$ of the first interval of the Devil's Staircase. 
$p_2$ is equal to $1 - p_1$. \\
Default value is $0.5$. 
\item {\bf [-b]} \\
Add a border. Default is no.
\item {\bf [-v]} \\
Verbose. Default is False.
\end{itemize}
\subsubsection*{Examples:}
\begin{itemize}
\item mf1d\_create\_ds NameOut \\
Create a Devil's Staircase  with all default values (243 points,
probabilities equal to $0.5$, ...).
\item mf1d\_create\_ds -p 0.4 NameOut \\
Ditto, but probabilities $p_1$ and $p_2$ are respectively equal to $0.4$ and
$0.6$.
\end{itemize}



\subsection{Chain maxima wavelet coefficients: mf1d\_chain}
\index{mf1d\_chain}

Program {\em mf1d\_chain} calculates the wavelet transform of a signal,
finds its maxima wavelet coefficients, and chains them. Finally, chains which
do not respect some criteria are deleted. Two files are created: the first
contains the maxima, and the second the skeleton.
{\bf
\begin{center}
 USAGE:  mf1d\_chain options  signal\_in
\end{center}}
where options are 
\begin{itemize}
\item {\bf [-t type\_of\_transform]} \\
0: Gaussian derivative wavelet transform \\
1: Mexican hat wavelet transform  \\
Default is 0.
\item {\bf [-r min\_length\_chain]} \\
Remove chains with length $<$ min\_length. Default is 0.
\item {\bf [-s  dynamique(\%)]} \\
Remove points of the skeleton map 
if level $<$ dynamique $*$ max(current scale). \\
Default is 0.
% \item {\bf [-c]} \\
% Draw chains of max. \\
% Default is false.
\item {\bf [-v]} \\
Verbose
\end{itemize}
Wavelet transform file name is {\bf WTMM\_signal\_in.fits}. \\
Skeleton file name is {\bf WTMMskel\_signal\_in.fits}.
\subsubsection*{Example:}
\begin{itemize}
\item mf1d\_chain Data01.fits \\
 Create the files{\bf WTMM\_Data01.fits} and {\bf WTMMskel\_Data01.fits}.
\end{itemize}

\subsection{Partition function: mf1d\_repart}
\index{mf1d\_repart}
Program {\em mf1d\_repart} computes the partition function. It needs the two
output files of the {\em mf1d\_chain} program (i.e.\ WTMMxxx.fits 
and WTMMskelxxx.fits), and creates three files:
\begin{itemize}
\item {\bf q\_FileNameOut.fits}: which contains the one-dimensional array $Q$ 
of the $q$ values for which the
partition function is calculated.
\item {\bf s\_FileNameOut.fits}: which contains the one-dimensional array $S$
of the $\log_2 s$ values for which the
partition function is calculated.
\item {\bf Z\_FileNameOut.fits}: which contains the partition functions. It
is a two dimensional array. $Z(i,j)$ is the partition function value 
 for $q = Q(i)$ and $s = S(j)$ ($i$ indexing the x-axis, and $j$ the y-axis).
\end{itemize}
{\bf
\begin{center}
USAGE: mf1d\_repart options MaxFileName MaxSupFileName FileNameOut
\end{center}}
where options are 
\begin{itemize}
\item {\bf [-a q\_minq]} \\
Value min of vector $q$. Default is -10.00.
\item {\bf [-b q\_max]} \\
Value max of vector $q$. Default is +10.00.
\item {\bf [-c number\_of\_q]} \\
Number of values of vector $q$. Default is 20.
\end{itemize}
$Z(q,s)$ file name is {\bf Z\_FileNameOut.fits} \\
$q$ file name is {\bf q\_FileNameOut.fits} \\
$s$ file name is {\bf s\_FileNameOut.fits}. \\
\subsubsection*{Example:}
\begin{itemize}
\item mf1d\_chain Data01.fits \\
      mf1d\_repart  WTMM\_Data01.fits WTMMskel\_Data01.fits Repart.fits  \\
 Create the files {\bf Z\_Data01.fits}, {\bf q\_Data01.fits}, and
 {\bf s\_Data0.fits}.
\end{itemize}


\subsection{Multifractal analysis: mf\_analyse}
\index{mf\_analyse}
Program {\em mf\_analyse} realizes the multifractal analysis. It
 computes the scaling
exponent estimation $\tau(q)$ and the singularity spectrum $\alpha(q)$.
It needs the three files created by {\em mf1d\_repart}. Four files are 
created:
\begin{itemize}
\item {\bf X\_Tau.fits}: which contains the x-coordinates of the 
$\tau(q)$ curve.
\item {\bf Y\_Tau.fits}: which contains the y-coordinates of the 
$\tau(q)$ curve.
\item {\bf X\_FracSpectrum.fits}: which contains the x-coordinates of the 
singularity spectrum curve (i.e. $\alpha$ values).
\item {\bf Y\_FracSpectrum.fits}: which contains y-coordinates of the 
singularity spectrum curve (i.e. $D(\alpha)$ values).
\end{itemize}

{\bf \begin{center}
 USAGE:  mf\_analyse options FileNameIn
\end{center}}
where options are 
\begin{itemize}
\item {\bf [-d $\alpha_{min}$]} \\
Value min of vector $\alpha$. Default is 0.00.
\item {\bf [-e $\alpha_{max}$]} \\
Value max of vector $\alpha$. Default is 1.00.
\item {\bf [-f number\_of\_$\alpha$]} \\
Number of values of vector $\alpha$. Default is 20.
\item {\bf [-m ind\_scale\_min]} \\
ind scale min for computing $\tau$. Default is 30. 
\item {\bf [-M ind\_scale\_max]} \\
ind scale max forn computing $\tau$. Default is scale\_max-2.
\end{itemize}
$Z(q,s)$ are read in {\bf Z\_FileNameIn.fits} file \\
$q$ are read in {\bf q\_FileNameIn.fits} file\\
$s$ are read in {\bf s\_FileNameIn.fits} file .
\subsubsection*{Example:}
\begin{itemize}
\item mf1d\_chain Data01.fits \\
      mf1d\_repart WTMM\_Data01.fits WTMMskel\_Data01.fits Repart  \\
      mf\_analyse  Repart \\
  Create the files  {\bf X\_Tau.fits},{\bf Y\_Tau.fits},{\bf X\_FracSpectrum.fits},{\bf Y\_FracSpectrum.fits},
  and {\bf Y\_TauTheo.fits}.
\end{itemize}


\subsection{Examples}

\subsubsection*{Fractal analysis of Devil's Staircase function}
Create Devil's Staircase function with $p_1=.6$ and $p_2=.4$.
\begin{verbatim}
mf1d_create_ds  -p .4 -b DSC3
\end{verbatim}
Compute the wavelet transform WTMM\_DSC3.fits, the skeleton WTMMskel\_DSC3.fits
(default transform: $\psi$ is the first derivative of a Gaussian).
\begin{verbatim}
mf1d_chain DSC3
\end{verbatim}
Compute the partition function $Z(q,s)$ for $q\in[-10.,10.]$ with 60 values,
write the $Z(q,s)$ to file Z\_DSC3.fits, the $s$(scale) file to s\_DSC3.fits and the
$q$ file to q\_DSC3.fits.
\begin{verbatim}
mf1d_repart -c 60 WTMM_DSC3 WTMMskel_DSC3 DSC3
\end{verbatim}
Compute the scaling exponent $\tau(q)$, write it in X\_Tau.fits and Y\_Tau.fits,
compute the singularity spectrum $\alpha(q)$, write it in X\_FracSpectrum.fits
and Y\_FracSpectrum.fits (100 values are used for $\alpha$).
\begin{verbatim}
mf_analyse -f 100 DSC3
\end{verbatim}

\subsubsection*{Fractal analysis}

Compute the wavelet transform WTMM\_DSC3.fits, the skeleton WTMMskel\_DSC3.fits
(use the Mexican hat for $\psi$)
\begin{verbatim}
mf1d_chain -t 1 signal
\end{verbatim}
Compute the partition function $Z(q,s)$ for $q\in[-5.,5.]$ with 100 values,
write the $Z(q,s)$ in file Z\_signal.fits, the $s$(scale) file in s\_signal and the
$q$ file in q\_signal.
\begin{verbatim}
mf1d_repart -a -5. -b 5. -c 100 WTMM_signal WTMMskel_signal sigout
\end{verbatim}
Compute the scaling exponent $\tau(q)$, write it in X\_Tau.fits and Y\_Tau.fits,
compute the singularity spectrum $\alpha(q)$, write it in X\_FracSpectrum.fits
and Y\_FracSpectrum.fits ($\alpha \in [0,2]$, 100 values are used).
\begin{verbatim}
mf_analyse -d 0. -e 2. -f 100 sigout
\end{verbatim}
 





