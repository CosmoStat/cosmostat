 
\chapter{Installation}
\label{ch_install}
% \chapterhead{Installation}
 
\textbf{IMPORTANT!} Before going any further, please read and accept the terms
in the file COPYRIGHT.


\section{Download}

To install MR/1 on your system you need the MR-install script and 
a distribution file.


\subsection{Distribution files}

First you have to choose the distribution of MR/1 binaries appropriate to your
host's system.

\vspace{0.3cm}
{\centering \begin{tabular}{|c|c|}
\hline 
Identification&
Host System\\
\hline 
\hline 
alpha\_osf&
Alpha (DEC) running OSF/1\\
\hline 
hp\_ux\_09&
HP 9000 series 700 or 800 or HP UX 10.x\\
\hline 
i486\_linuxlibc2&
Intel x86 Linux libc2 '99 distributions\\
\hline 
rs6000\_ibm\_aix&
IBM RS/6000 \\
\hline 
sun4x\_5.6&
Sun 4 Sparc or Ultra running Solaris 2.6\\
\hline 
i586\_cygwin32 & Windows 95,98, and NT\\
\hline
\end{tabular}\par}
\vspace{0.3cm}

For each system (xx\_xx is the identification of your host system), 4 gzip'ed
tar archives are provided:

\begin{itemize}
\item MR\-3.0\-xx\_xx.tgz contains all the MR/1,MR/2, and MR/3 executables
\item cea\_lm-3.0-xx\_xx.tgz contains the binary of the floating licence server
\item mr\_decomp-3.0-xx\_xx.tgz and mr\_upresol-3.0-xx\_xx.tgz contain 
the decompression utilities
\end{itemize}

A set of IDL routines (for IDL Version 5) is also available in the
file idlv5.tar.gz, allowing the executables to be called directly from IDL.

\subsection{Example}

If you wish to run MR/1 on IBM, you will need the following files:

\begin{itemize}
\item MR-3.0-rs6000\_ibm\_aix.tgz,
\item cea\_lm-3.0-rs6000\_ibm\_aix.tgz.
\end{itemize}

\subsection{Installation}

To proceed, you need to have: the MR-install script and the distribution
archive. During the installation procedure, you will be asked for an 
installation
directory (with writable authorization and enough available space) and licence
information dedicated to your host (hostname and hostid).

If you plan to install only the decompression utility, you do not need any 
licence.

This table gives the maximum space needed to fully install 
MR/1 on various systems:

\vspace{0.3cm}
{\centering \begin{tabular}{|c|c|}
\hline 
Alpha &
200 MB\\
\hline 
HP &
128 MB\\
\hline 
Linux &
140 MB\\
\hline 
IBM RS/6000&
330 MB\\
\hline 
Sun Solaris 2.5&
120 MB\\
\hline 
WINDOWS 95, 98, and NT &
225 MB\\
\hline 
\end{tabular}\par}
\vspace{0.3cm}

MR-install is a shell script which will help to easily extract the binaries
from the distribution archive and to configure the licence server.

If you get the distribution using Internet (i.e., ftp), you will have to 
check the file permission
of the shell script MR-install. ``\textsl{chmod +x MR-install}''
may be needed to add execute permission.

Ater running this script, \textsl{./MR-install}, you
will need to modify the \textsl{PATH} system variable to include
the installation directory. You may ask your system administrator to modify
your host so that it will automatically start the licence manager 
whenever you reboot your computer.

Once everything is fine, you can remove the MR-install script and 
the distribution archive from your host.


\subsection{How to start?}

The directory with executables corresponding to the selected
type of computers must be in the user path.  
The licence manager can be run (if not already done so by your system manager)
by typing 
\begin{center}
 {\bf cea\_lm -file fileName}
 \end{center}
 or directly {\bf cea\_lm} if the default file name has been chosen. 

For non-floating licences, the programs
can only be executed on the computer where the  licence manager is installed.
For floating licences, the programs can be run from any computer (with the 
correct type of platform) on the network 
if the environment variable {\bf CEA\_LICENSE\_HOST} is set 
to the correct address (for example ariane.saclay.cea.fr), or if the licence file is 
in the directory {\bf /usr/local/cea} (and {\bf  C:\verb+\+Program Files\verb+\+MR} for WINDOWS).

When the licence manager is running, all executables described in 
this document can be called.

\subsection{JAVA Interface}
 
\subsubsection*{Required packages}
\begin{itemize}
\item JDK1.2. (http://java.sun.com/products/index.html)
\item Swing1.1. (http://java.sun.com/products/index.html)
\item MR/1 interface Java class (mr.class, ....)
\end{itemize} 

\subsubsection*{Running MR Java application}

On unix computers, the {\bf CLASSPATH} variable must contain the Java class 
directory and the Swing class directory. \\ 
Run the program by typing:
\begin{verbatim}
> java mr
\end{verbatim}

On WINDOWS, run the program by typing:
\begin{verbatim}
> jmr1
\end{verbatim}
if the default installation directory has been choosen. Otherwise you
will have to edit the jmr1.bat file, and modify the variable accordingly
to your installation.

If the license manager is not running, it has to be started from the 
menu ``System''$>$``cea\_lm...'', by selecting the ``Status'' option.


Exit the program by selecting ``Quit...'' in the ``System'' menu.
\begin{verbatim}
"System">"Quit..."
\end{verbatim}
The license manager can be stopped before from the 
menu ``System''$>$11cea\_lm...'', by selecting the ``Kill'' option.

\subsubsection*{Description}

The MR/1 graphical user inteface allows the user to execute all the MR/1  
programs.
The application window contains a menu bar with 5 tear-off menus: 
\begin{itemize}
\item System: quit, and licence manipulation.
\item MR/1 IMAGE: image manipulation programs (multiresolution not used).
\item MR/1 Multiresol: manipulation of multiresolution objects.
\item MR/1 Application: filtering, deconvolution, compression, detection, etc.
\item MR/1 1D: 1D programs.
\end{itemize}
To execute an MR program select the associated item (program name): \\
``MR/1 Image'' $>$ ``Manipulation tools'' $>$ ``im\_info'', 
for the im\_info program.  

The application displays a popup window. All the options of the 
program can be
selected. If a popup window is validated (by pressing the  
OK button) with an obligatory 
option not initialized, a warning dialog box prevents the user continuing.

When the dialog box is validated the programm is executed. 
Information on the execution is displayed in the current window: 
\begin{verbatim}
  Program is running ...
  ... information relative to the program execution ...
  Program is finished!
The execution is now terminated.
\end{verbatim}

The help button of the dialog box displays a window with the help of the 
program.


\subsection{IDL routines}

To use IDL routines, the {\bf CEA\_MR\_DIR} variable 
must be initialized to the directory
where the MR/1-IDL software has been installed. Then 
the command
\begin{center}
{\bf idl \$CEA\_MR\_DIR/idl/pro/mre} 
\end{center}
runs the IDL session with the 
multiresolution environment. An IDL help facility on the available routines
is obtained by typing {\bf mrh} in the IDL session. 





