\chapter{ISOCAM Programs}

 
\section*{Deglitching}

\begin{center}
 USAGE: mr1d\_deglitch option cube\_in cube\_out
\end{center}
where options are:
\begin{itemize}
\item {\bf [-n number\_of\_scales]} \\
number of scales used in the multiresolution transform. Default is 3.
\item {\bf [-s Nsigma]} \\
\item {\bf [-f]} \\
Suppress the noise, but not the glitches.
\end{itemize}

\section*{Transient Correction}

\begin{center}
 USAGE: transient\_fit option cube\_in cube\_out
\end{center}
where options are:
\begin{itemize}
\item {\bf [-d downward\_model]} \\
for fitting downward transients.
\item {\bf [-e ending\_scd]} \\
e = position of the ending scd to be considered  for fitting transients.
\item {\bf [-g]} \\
Global correction. upward \& downward transients are fitted over all the scds.
 \item {\bf [-m method]} \\
m = name of the method used for the minimization of the merit-function:
'cauchy','exponential','gaussian'.
\item {\bf [-n number\_of\_iterations]} \\
n = max number of iterations when fitting the model with the data.
\item {\bf [-o]} \\
overshoot upward model.
\item {\bf [-s starting\_scd]} \\
s = position of the starting scd to be considered  for fitting transients.
\item {\bf [-t down\_threshold]} \\
t = the greatest negative value of the slope of a downward transient above 
which   no fitting is undertaken.
\item {\bf [-T up\_Threshhold]} \\
T = the lowest positive value of the slope of a upward transient beneath
 which no fitting is undertaken.
\item {\bf [-u upward\_model]} \\
for fitting upward transients.
\end{itemize}

\section*{Field of view Distortion Correction}
\begin{center}
 USAGE: projection  options input1 input2 ... output  
\end{center}
where options are:
\begin{itemize}
\item {\bf [-R CROTA2]} \\
Creates a map with given {\em CROTA2} (decimal degrees).
Default is 0 .
\item {\bf [-d CDELT1]} \\
Creates a map with a given resolution (in arcsec). \\
if {\em CDELT1} is set and not {\em CDELT2}, then  \\
                 CDELT1 = -{\em CDELT1} and CDELT2 = {\em CDELT1} \\
to ensure North top and West left astronomical convention  with CROTA2=0 images.
\item {\bf [-D CDELT2]} \\
 Creates a map with a given CDELT2 (in arcsec). CDELT1 must have been provided.
\item {\bf [-M map]} \\
 Will use the orientation of the map input number map for the result map.
\item {\bf [-C]} \\
Will produce as many outputs as inputs, with the same astrometry, 
but without coadding the maps. Fits files will be numbered from 1 to n 
after the given output filename.
\item {\bf [-o output file name]} \\
Output file name. Default is "result.fits". 
\item {\bf [-x line of pixel to be examined]} \\
Default is -1000.
\item {\bf [-y column of pixel to be examined]} \\
 Default is -1000.
\end{itemize}


\section*{The PRETI Program}
The {\em mr\_preti} applies the calibration from pattern recognition.
The first input (raw data + observation parameters) file should be created 
using the {\em mosaic2fits} routine from the CAM Interactive Analysis package.
\begin{center}
 USAGE: mr\_preti option cube\_in cube\_in\_or\_out
\end{center}
where options are:
\begin{itemize}
\item {\bf [-n number\_of\_scales]} \\
Number of scales used in the multiresolution transform for the short glitches
removal. 
Default is 3.
\item {\bf [-N number\_of\_scales]} \\
Nnumber of scales used for the suppresion of long glitches. 
Default is 8.
\item {\bf [-g sigma]} \\
sigma = noise standard deviation. By default, it is automatically
estimated for each temporal vector.
\item {\bf [-s Nsigma]} \\
Detection level at nsigma * SigmaNoise.
\item {\bf [-d]} \\
Apply a short  glitches removal.  Default is no.
\item {\bf [-D]} \\
Apply a long glitches removal. Default is no.
\item {\bf [-b]} \\
Suppress the base line. Default is no.
\item {\bf [-B number\_of\_scales]} \\
 Number of scales used for baseline suppression.
\item {\bf [-f]} \\
Flat field correction. Default is no.
\item {\bf [-t]} \\ 
 Transient correction. Default is no.
\item {\bf [-K BorderNumber]} \\
Do not take into account the border.
\item {\bf [-a]} \\
Average the frames belonging to the same sky position. Default is no.
\item {\bf [-r]} \\
Create a rasmap.
\item {\bf [-S NSigma\_Long\_Glitch]} \\
NSigma for long glitch detection. Default is 5. 
\item {\bf [-F]} \\ 
Kill detected objects containing more than 50 masked  pixels.
\item {\bf [-G]} \\
Kill detected objects if the first 3 pixels are masked .
\item {\bf [-M]} \\
Mask pixels where long glitches have been detected.
\item {\bf [-A]} \\
(for all) Apply a flat field correction, a short and long glitches removal,
a baseline suppression, average the frame, create the raster map. It is equivalent
to options "-d -D -f -b -a -r".
\item {\bf [-L]} \\
Use the slope of an object for source/glitch separation.
\item {\bf [-m]} \\
Creates a micro-scan map instead of a raster map.
\end{itemize}

There is two modes for using the {\em mr\_isocalib} program: \\
If option "-d" and/or "-D" is set then the second file of the command
line is an output file, which contains the deglitched data cube, without
the baseline (if "-b" is set), and hence contains only the 
noise and the sources. In this case  the program will also create
a set of other files:
\begin{itemize}
\item xx\_baddata.fits: data cube which contains the detected long glitches.
\item xx\_mask.fits: data cube which contains the mask. mask(x,y,z) equal
to 1 if a short glitch is found at position (x,y,z). If "-M" is set, 
positions were long glitches have been detected are set also to 1.
\item xx\_bgr.fits: data cube which contains the baseline.
\item xx\_sigmanoise.fits: image which contains the estimated noise for each
detector pixel.
\item xx\_source.fits: data cube which contains the detected sources.
\end{itemize}

If neither "-d" or "-D" is set (second mode), all the files described above are read 
by {\em mr\_isocalib}. This allows us to treat in several steps the data.

other created files are created:
\begin{itemize}
\item Results concerning the flat field (only if "-f" is set):
\begin{itemize}
\item "xx\_auto\_flat.fits": estimated flat field.
\item "xx\_rms\_flat.fits": RMS of the estimated flat field.
\end{itemize}
\item Results concerning the reduced data (frame at the same sky position
have been averaged). Only if "-r" or/and "-f", or/and "-m" is set.
\begin{itemize}
\item "xx\_reduce.fits": reduced and flat-fielded (if "-f") cube.
\item "xx\_rms.fits": RMS cube of the averaged cube.
\item "xx\_npix.fits": number of pixels used for pixel of the averaged cube.
\item "xx\_rms\_noise.fits": Estimated noise on the averaged cube.
\item "xx\_raster.fits": raster scan or micro scan final map.
\item "xx\_rmsraster.fits": RMS of the final map
\item "xx\_npixraster.fits": number of pixels used
 for the final map calculation.
\end{itemize}
\item Results concerning the baseline:
\begin{itemize}
\item "xx\_bgrreduce.fits"
\item "xx\_bgrrms.fits"
\item "xx\_bgrnpix.fits"
\item "xx\_bgrraster.fits"
\item "xx\_bgrrmsraster.fits"
\item "xx\_bgrnpixraster.fits"
\end{itemize}
\item Results concerning the bad data (long glitches):
\begin{itemize}
\item "xx\_badreduce.fits"
\item "xx\_badraster.fits"
\end{itemize}
\end{itemize}

\section*{Markov Random Field Deconvolution}

\begin{center}
 USAGE: ${\bf im\_dec\_markov}$ option image\_in psf\_in image\_out
\end{center}
where options are: 
\begin{itemize}
\item {\bf [-b]} \\
If this option is set, the support of the image is dilated by using I\_MIRROR to
obtain a solution with the same support. By default, we use I\_CONT.
\item {\bf [-d delta]} \\
Scaling parameter $\delta$. This parameter must be strictly positive. 
Default is 1 (See Parameter estimation section)
\item {\bf [-e epsilon]} \\
Convergence parameter. Default is 0.001
\item {\bf [-g sigma]} \\
Gaussian Noise standard deviation. Default is automatically estimated.
\item {\bf [-i number\_of\_iterations]} \\
Maximum number of iterations. Default is 500
\item {\bf [-l lambda]} \\
Regularizing parameter $\lambda$. This parameter must be strictly positive. 
Default is 1 (See Parameter estimation section)
\item {\bf [-o omega]} \\
Over-relaxation coefficient such as: $1\leq omega<2$. Omega is used to allow faster 
convergence. Default is 1 
\item {\bf [-r residual\_file\_name]} \\
If this option is set, the residual is written to 
the file of name {\em residual\_file\_name}. By default, the
residual is not written. The residual is equal to the difference between
the input image and the solution convolved by the PSF.

The maximum of the PSF must be at the center of its support.
So, dimensions of the PSF support must be odd.
\end{itemize}
\noindent

{\bf Example:}
\begin{itemize}
\item im\_dec\_markov -d 100 -i 100 -l 10 -r residu.fits 
image\_in.fits psf\_in.fits ima\_out.fits \\
deconvolves an image with $\delta=100$ and $\lambda=10$, assuming
a Gaussian noise (its standard deviation is automatically estimated).
\end{itemize}

For each iteration, the following informations will be printed:

...

It 95 D=90674.9 M=537432 V=628106 Ec=75.5482 Fx=9.38398e+07 min=1224.23 max=13732.2

It 96 D=90676.3 M=537430 V=628107 Ec=70.6314 Fx=9.38399e+07 min=1224.23 max=13731.7

It 97 D=90677.5 M=537429 V=628107 Ec=66.1543 Fx=9.38398e+07 min=1224.23 max=13731.3

It 98 D=90678.7 M=537429 V=628107 Ec=61.8843 Fx=9.38399e+07 min=1224.23 max=13730.8

It 99 D=90679.9 M=537428 V=628107 Ec=57.978 Fx=9.38399e+07 min=1224.23 max=13730.4
 
where D=$U({\bf y/x})$, M=$U({\bf x})$ and V=$U({\bf x/y})$ (V=D+M). 

Ec is the correction energy compared
to the convergence parameter (epsilon).

Fx is the flux of the estimated
solution $\bf \hat x$.

Minimum and maximum of the restored image are also printed.


\section*{IDL routine}
{\em mr1d\_deglitch} IDL routine eliminates glitches in 
a cube $D$ calling the previous program. Each vector $D(s,y,*)$
at position $(x,y)$ is treated individually. 
\begin{center}
     USAGE: mr1d\_deglitch, Cube, filter=filter, Nscale=Nscale, Nsigma=Nsigma, mask=mask
\end{center}
where 
\begin{itemize}
\item {\em Cube}: 3D IDL array (input-ouput parameter). Cube to deglitch.
\item {\em Nscale}: number of scale for the multiresolution analysis.
\item {\em Nsigma}: glitch detection at Nsigma.
\item {\em filter}: if set, then an adaptive temporal filter is applied.
\item {\em mask}: output 3D IDL array: Mask where glitches have been detected.
\end{itemize}
