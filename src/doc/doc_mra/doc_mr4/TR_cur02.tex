
% \documentclass[11pt,a4paper]{article}
\documentclass[11pt,a4paper]{article}

% \renewcommand{\baselinestretch}{2}
\usepackage{amssymb,alltt}
\usepackage{latexsym}
\usepackage{graphicx}
\usepackage{psfig}

\newcommand{\psdir}{/export/home/starck/tex/report/FIG}
\psfigurepath{\psdir/filter_xima:\psdir/contrast:\psdir/curvelet:\psdir/book98:\psdir/compress:\psdir/entropy:\psdir/deconv:\psdir/cours:\psdir/demo:\psdir/detect}
% \psfigurepath{./PS}

% \usepackage[latin1]{inputenc}
% \usepackage[widemargins]{a4}
% \usepackage{graphics}

\textwidth 15.6truecm
\textheight 22.5truecm
\hoffset -1.5truecm
\voffset -1.8truecm
\newtheorem{theorem}{Theorem}
\newtheorem{lemma}[theorem]{Lemma}
\newtheorem{corollary}[theorem]{Corollary}
\newtheorem{proposition}[theorem]{Proposition}
\newtheorem{definition}[theorem]{Definition}
\newcommand{\goto}{\rightarrow}
\newcommand{\bitem}{\begin{itemize}}
\newcommand{\eitem}{\end{itemize}}
\newcommand{\bR}{{\bf R}}
\newcommand{\eps}{{\epsilon}}
\newcommand{\cR}{{\cal R}}
\newcommand{\cQ}{{\cal Q}}
\newcommand{\cW}{{\mathcal W}}
\newcommand{\cA}{{\mathcal A}}
\newcommand{\cP}{{\mathcal P}}
\newcommand{\cS}{{\mathcal S}}
\newcommand{\bZ}{{\bf Z}}
\newcommand{\cM}{{\cal M}}
\newcommand{\<}{\langle}
\renewcommand{\>}{\rangle}
\newcommand{\qed}{\quad\hbox{\vrule width 4pt height 6pt depth 1.5pt}}
\newcommand{\cH}{{\cal H}}
\newcommand{\cL}{{\cal L}}



\title{  {\Huge \bf  Image Processing by the Curvelet Transform \\
              \underline{ \hspace{15cm}} }}

\author{ \huge {Technical Report} \\ 
       \vspace{2cm} \\
        J-L. Starck     \\  [12pt]
  DSM/DAPNIA/SEDI-SAP, CEA-Saclay, 91191 Gif-sur-Yvette, France   \\
       \\ 
      \vspace{1cm}}

\date{September 2002 - Version 1.0\\
     \vspace{1cm}}

\newcommand{\be}{\begin{eqnarray}}
\newcommand{\ee}{\end{eqnarray}}

 

% \includeonly{ch_cover,ch_curvelet}
% \includeonly{ch_curfilter,ch_combfilter,ch_combdeconv,ch_curcontrast,ch_curprog,ch_mga,ch_curbiblio,Annex_FFT,Annex_Atrous}

\begin{document}
% \input{epsf.tex}


\include{ch_cover}

\chapter{The Curvelet Transform}
\section{Introduction}
 Wavelets and related multiscale representations pervade  all
areas of signal processing. The recent inclusion of wavelet algorithms
in JPEG 2000, the new still-picture compression standard, testifies
to this lasting and significant impact.  
The most used
wavelet transform algorithm is the decimated bi-orthogonal
wavelet transform (OWT). Using the OWT, a signal $s$ can be decomposed
by:
\begin{eqnarray}
  s(l) = \sum_{k} c_{J,k} \phi_{J,l}(k) 
        +  \sum_{k} \sum_{j=1}^J \psi_{j,l}(k) w_{j,k}
 \end{eqnarray}
 with $\phi_{j,l}(x) = 2^{-j} \phi(2^{-j}x-l)$ and $\psi_{j,l}(x) =
 2^{-j} \psi(2^{-j}x-l)$, where $\phi$ and $\psi$ are respectively the
 scaling function and the wavelet function.  $J$ is the number of
 resolutions used in the decomposition, $w_{j}$ the wavelet (or
 detail) coefficients at scale $j$, and $c_{J}$ is a coarse or smooth
 version of the original signal $s$.  Thus, the algorithm outputs $J+1$
 subband arrays.  The indexing is such that, here, $j = 1$ corresponds
 to the finest scale (high frequencies).

The application of the OWT to image compression, using the 7-9 filters
 \cite{wave:antonini92} and the zerotree coding
\cite{compress:shapiro93,compress:said96} has lead to impressive
results, compared to previous methods like JPEG.   

A series of recent papers
\cite{Roysoc,Curvelets-StMalo}, however, argued that wavelets and related
classical multiresolution ideas are playing with a limited dictionary
made up of roughly isotropic elements occurring at all scales and
locations. We view as a limitation the facts that those dictionaries
do not exhibit highly anisotropic elements and that there is only a
fixed number of directional elements, independent of scale.  
Despite the success of the classical wavelet viewpoint, there are objects, 
e.g.\  images, that do
not exhibit isotropic scaling and thus call for other kinds of
multiscale representation. In short, the theme of this line of
research is to show that classical multiresolution ideas only address
a portion of the whole range of interesting multiscale phenomena and
that there is an opportunity to develop a whole new range of
multiscale transforms.

Following on this theme, Cand\`es and Donoho in\-tro\-du\-ced new mul\-ti\-scale
systems like curvelets \cite{Curvelets-StMalo} and ridgelets \cite{Harmnet}
which are very different from wavelet-like systems. Curvelets and
ridgelets take the form of basis elements which exhibit very high
directional sensitivity and are highly anisotropic.  In
two-dimensions, for instance, curvelets are localized along curves, in
three dimensions along sheets, etc.  Continuing at this informal level
of discussion we will rely on an example to illustrate the fundamental
difference between the wavelet and ridgelet approaches, postponing
the mathematical description of these new systems.

We investigate in this document the best way to implement 
the ridgelet and the curvelet transform for the purpose of image restoration.
The second and third sections describe respectively
the ridgelet transform and the curvelet transform.
Section four shows how the ridgelet and the wavelet coefficients can be
thresholded in order to filter an image. Comparisons with other methods
are presented. Instructions for using the programs are given in the last 
section.

\begin{figure}[htb]
\centerline{
\vbox{
\hbox{
\psfig{figure=fig_bar_noise.ps,bbllx=1.7cm,bblly=12.5cm,bburx=14.7cm,bbury=25.5cm,width=6.5cm,height=6.5cm,clip=}
\psfig{figure=fig_bar_coupe.ps,bbllx=2.5cm,bblly=13.5cm,bburx=19.5cm,bbury=25.5cm,width=6.5cm,height=6.5cm,clip=}
}
\hbox{
\psfig{figure=fig_bar_wave.ps,bbllx=1.7cm,bblly=12.5cm,bburx=14.7cm,bbury=25.5cm,width=6.5cm,height=6.5cm,clip=}
\psfig{figure=fig_bar_rid.ps,bbllx=1.7cm,bblly=12.5cm,bburx=14.7cm,bbury=25.5cm,width=6.5cm,height=6.5cm,clip=}
}}}
\caption{Top left, original image containing a vertical band embedded in 
white noise with relatively large amplitude.  Top right, row sums 
illustrating the hardly perceptable band.  Bottom left,
reconstructed image using undecimated wavelet coefficients. Bottom right,
reconstructed image using ridgelet coefficients.}
\label{fig_bar}
\end{figure}

Consider an image which contains a vertical band embedded in white
noise with relatively large amplitude.  Figure~\ref{fig_bar} (top
left) represents such an image.  The parameters are as follows: the
pixel width of the band is 20 and the SNR (signal-to-noise ratio)
is set to be $0.1$. Note
that it is not possible to distinguish the band by eye. The wavelet
transform (undecimated wavelet transform) is also incapable of
detecting the presence of this object; roughly speaking, wavelet
coefficients correspond to averages over approximately isotropic
neighborhoods (at different scales) and those wavelets clearly do not
correlate very well with the very elongated structure (pattern) of
the object to be detected.

We now turn our attention towards procedures of a very different
nature which are based on line measurements. To be more specific,
consider an {\em ideal} procedure which consists of integrating the
image intensity over columns; that is, along the orientation of our
object.  We use the adjective ``ideal'' to emphasize the important
fact that this method of integration requires a priori knowledge about
the structure of our object. This method of analysis gives of course
an improved signal-to-noise ratio for our linear functional better
which is better correlated with the object in question: 
see the top right panel of
Figure~\ref{fig_bar}.

This example will make our point. Unlike wa\-velet trans\-forms, the
rid\-gelet trans\-form pro\-cesses data by first computing integrals over
lines with all kinds of orientations and locations. We will explain in
the next section how the ridgelet transform further processes those
line integrals. For now, we apply naive thresholding of the ridgelet
coefficients and ``invert'' the ridgelet transform; the bottom right
panel of Figure~\ref{fig_bar} shows the reconstructed image. The
qualitative difference with the wavelet approach is striking. We
observe that this method allows the detection of our object even in
situations where the noise level (standard deviation of the white
noise) is five times superior to the object intensity. 

\section{Continuous Ridgelet Transform}

The two-dimensional continuous ridgelet transform in $\bR^2$ can be
defined as follows \cite{Harmnet}. We pick a smooth univariate
function $\psi:\bR \goto \bR$ with sufficient decay and satisfying the
admissibility condition
\begin{equation}
\label{eq:admissibility}
\int |\hat{\psi}(\xi)|^2/|\xi|^2 \, d\xi < \infty,
\end{equation}
which holds if, say, $\psi$ has a vanishing mean $\int
\psi(t) dt = 0$.  We will suppose that $\psi$ is normalized so that
$\int |\hat{\psi}(\xi)|^2 \xi^{-2} d\xi = 1$.

For each $a > 0$, each $b \in \bR$ and each $\theta \in [0,2\pi)$, we
define the bivariate {\em ridgelet} $\psi_{a,b,\theta}: \bR^2 \goto
\bR^2$
by
\begin{equation}
\label{eq:ridgelet}
       \psi_{a,b,\theta} (x) = a^{-1/2} \cdot
     \psi( (x_1 \cos\theta + x_2 \sin\theta  - b)/a);
\end{equation}
this function is constant along lines $x_1 \cos\theta + x_2 \sin\theta
= const$. Transverse to these ridges it is a wavelet.  

\begin{figure}[htb]
\centerline{
\vbox{
\hbox{
\psfig{figure=ridgelet.eps,bbllx=0.5cm,bblly=6cm,bburx=20.5cm,bbury=21cm,width=5.5cm,height=5.5cm,clip=}
\psfig{figure=ridgerotate.eps,bbllx=0.5cm,bblly=6cm,bburx=20.5cm,bbury=21cm,width=5.5cm,height=5.5cm,clip=}
}
\hbox{
\psfig{figure=ridgescale.eps,bbllx=0.5cm,bblly=6cm,bburx=20.5cm,bbury=21cm,width=5.5cm,height=5.5cm,clip=}
\psfig{figure=ridgeshift.eps,bbllx=0.5cm,bblly=6cm,bburx=20.5cm,bbury=21cm,width=5.5cm,height=5.5cm,clip=}
}
}}
\caption{A few ridgelets.}
\label{fig_rid_function}
\end{figure}

Figure~\ref{fig_rid_function} graphs a few ridgelets with different
parameter values. The top right, bottom left and right panels
are obtained after simple geometric manipulations of the upper left
ridgelet, namely rotation, rescaling, and shifting.  

Given an
integrable bivariate function $f(x)$, we define its ridgelet
coefficients by
\[
\cR_f(a,b,\theta) = \int \psi_{a,b,\theta}(x) f(x) dx.
\]
We have the exact reconstruction formula
\begin{equation}
\label{eq:CRT}
f(x) = \int_0^{2\pi} \int_{-\infty}^\infty \int_0^\infty
\cR_f(a,b,\theta) {\psi}_{a,b,\theta}(x) \frac{da}{a^3} db
\frac{d\theta}{4\pi}
\end{equation}
valid a.e.\ (almost everywhere) 
for functions which are both integrable and square
integrable. Furthermore, this formula is stable since we have a Parseval
relation
\begin{equation}
\label{eq:Parseval}
       \int |f(x)|^2 dx = \int_0^{2\pi} \int_{-\infty}^\infty
\int_0^\infty
|\cR_f(a,b,\theta)|^2 \frac{da}{a^3} db \frac{d\theta}{4\pi}.
\end{equation}
Hence, much like the wavelet or Fourier transforms, the identity
(\ref{eq:CRT}) expresses the fact that one can represent any arbitrary
function as a continuous superposition of ridgelets.  Discrete
analogs of (\ref{eq:CRT})-(\ref{eq:Parseval}) exist, see
\cite{Harmnet}, or \cite{Ortholinear} for a slightly different
approach.

\subsection{The Radon Transform}

A basic tool for calculating ridgelet coefficients is to view ridgelet
analysis as a form of wavelet analysis in the Radon domain. We recall
that the Radon transform of an object $f$ is the collection of line
integrals indexed by $(\theta,t) \in [0,2\pi) \times \bR$ given by
\begin{equation}
      \label{eq:Radon}
      Rf(\theta,t) = \int f(x_1,x_2)
\delta(x_1 \cos\theta + x_2 \sin\theta - t)\, dx_1 dx_2,
\end{equation}
where $\delta$ is the Dirac distribution. The ridgelet coefficients
$\cR_f(a,b,\theta)$ of an object $f$ are given by analysis of the Radon
transform via
\[
\cR_f(a,b,\theta) = \int Rf(\theta,t) a^{-1/2} \psi((t - b)/a) \, dt.
\]
Hence the ridgelet transform is precisely the application of a
1-dimensional wavelet transform to the slices of the Radon transform
where the angular variable $\theta$ is constant and $t$ is varying.

\subsection{Ridgelet Pyramids}
\label{sec:pyramid}

Let $Q$ denote a dyadic square $Q = [k_1/2^s,(k_1+1)/2^s) \times
[k_2/2^s, (k_2+1)/2^s)$ and let ${\cal Q}$ be the collection of all
such dyadic squares. We write ${\cal Q}_s$ for the collection of all
dyadic squares of scale $s$. Associated with the squares
$Q \in \cQ_s$ we construct a partition of energy
as follows. With $w$ a nice smooth window obeying
$ \sum_{k_1,k_2} w^2(x_1-k_1 , x_2 -k_2 ) = 1$,
we dilate and transport $w$ to all squares $Q$ at scale $s$,
producing a collection of windows $(w_Q)$ such that the
$w_Q^2$'s, $Q \in {\cal Q}_s$, make up a partition of unity.
We also let $T_Q$
denote the transport operator acting on
functions $g$ via
\[
(T_Q g)(x_1,x_2) = 2^s g(2^s x_1 - k_1,2^s x_2 - k_2).
\]
With this notation, it is not hard to see that
\[
f w_Q = \int \<f, w_Q T_Q {\psi}_{a,b,\theta} \>
     T_Q {\psi}_{a,b,\theta} \frac{da}{a^3} db
\frac{d\theta}{4\pi}
\]
and, therefore,
summing the above equality across squares at a given scale gives
\begin{equation}
      \label{eq:MRT}
f = \sum_{Q \in {\cal Q}_s} f w_Q^2 = \sum_Q \int \<f, w_Q T_Q
{\psi}_{a,b,\theta} \> w_Q T_Q {\psi}_{a,b,\theta} \frac{da}{a^3} db
\frac{d\theta}{4\pi}.
\end{equation}
The identity (\ref{eq:MRT}) expresses the fact that one can represent
any function as a superposition of elements of the form $w_Q T_Q
{\psi}_{a,b,\theta}$; that is, of ridgelet elements localized near the
squares $Q$. Associated with the 
function $T_Q {\psi}_{a,b,\theta}$ is the ridgelet
$\psi_{a_Q,\theta_Q,b_Q}$ (\ref{eq:ridgelet}) with parameters obeying
\[
a_Q = 2^{-s} a, \quad \quad \theta_Q = \theta, \quad \quad b_Q = b +
k_1 2^{-s} \cos\theta + k_2 2^{-s}\sin\theta
\]
and thus $w_Q T_Q {\psi}_{a,b,\theta}$ is a windowed ridgelet,
supported near the square $Q$, hence the name {\em local ridgelet
      transform}.

The previous paragraph discussed the construction of local ridgelets
of fixed length, roughly $2^{-s}$ ($s$ fixed). Letting the scale $s$
vary defines the multiscale ridgelet dictionary
$\{\psi^Q_{a,b,\theta}: s \ge s_0, Q \in {\cal Q}_s, a > 0, b \in \bR,
\theta \in [0,2\pi)\}$ by
\[
\psi^Q_{a,b,\theta} = w_Q \, T_Q \psi_{a,b,\theta};
\]
that is, a whole pyramid of local ridgelets at various lengths and
locations. This is, of course, a massively overcomplete representation
system and no formula like (\ref{eq:MRT}) is available for this
multiscale ridgelet pyramid, because it is highly overcomplete.
 
\section{Digital Ridgelet Transform}
So the basic strategy for calculating the
continuous ridgelet transform is first to compute the Radon transform
$Rf(t,\theta)$ and second, to apply a one-dimensional wavelet
transform to the slices $Rf(\cdot,\theta)$.  

Several digital ridgelet transforms have been proposed, and we will described
three of them in this section, based on different implementations of the
Radon transform.

\subsection{The RectoPolar Ridgelet transform}
In this section 
we develop a digital procedure which is inspired by this viewpoint,
and is realizable on $n$ by $n$ numerical arrays.

 A fundamental fact about the Radon transform is the
projection-slice formula \cite{Deans}:
\[
\hat{f}(\lambda \cos\theta, \lambda \sin\theta) = \int Rf(t,\theta)
e^{-i\lambda t} dt.
\]
This says that the Radon transform can be obtained by applying the
one-dimensional inverse Fourier transform to the two-dimensional
Fourier transform restricted to radial lines going through the origin.

This of course suggests that approximate Radon transforms
for digital data can be based on discrete fast Fourier
transforms. This is a widely used approach,
in the literature of medical imaging and
synthetic aperture radar imaging, for which the
key approximation errors and artifacts have been widely discussed.
In outline, one
simply does the following, for
gridded data $(f(i_1,i_2))$, $0 \le i_1, i_2 < n-1$.
\begin{enumerate}
\item {\em 2D-FFT}. Compute the two-dimensional FFT of $f$ giving the
      array $(\hat{f}(k_1,k_2))$, $ -n/2 \le k_1, k_2 \le n/2 - 1$.
\item {\em Cartesian to Polar Conversion}. Using an interpolation
      scheme, substitute the sampled values of the Fourier transform
      obtained on the square lattice with sampled values of $\hat{f}$ on a
polar
      lattice: that is, on a lattice where the points fall on lines going
      through the origin.
\item {\em 1D-IFFT}. Compute the one-dimensional IFFT (inverse FFT)
     on each line,
      i.e.\ for each value of the angular parameter.
\end{enumerate}

The use of this strategy in connection with ridgelet transforms has
been discussed in the articles \cite{DRT,FRT,starck:sta01_3}.

\subsubsection{A Polar Sampling Scheme for Digital Data}

For our implementation of the Cartesian-to-polar
conversion, we have used a pseudo-polar
grid, in which the pseudo-radial variable has level sets which are squares
rather than circles. Starting with Oppenheim and Mersereau
\cite{MERSEREAU} this grid has often been called the {\it concentric
squares} grid in the signal processing literature; in the medical
tomography literature it is associated with the {\it linogram}
\cite{cur:edholm87,cur:edholm88}, while
in \cite{cur:averbuch01} it is called the rectopolar grid; see this  last
reference for a complete bibliographic treatment.  The geometry of the
rectopolar  grid is illustrated in Figure \ref{fig_radon}. We select $2
n$ radial lines in the frequency plane obtained by connecting the
origin to the vertices
$(k_1,k_2)$ lying on the boundary of the array $(k_1, k_2)$ , i.e. such
that $k_1$ or $k_2 \in \{-n/2, n/2\}$. The polar grid
$\xi_{\ell,m}$ ($\ell$ serves to index a given radial line while the
position of the point on that line is indexed by $m$) that we shall use
is the intersection between the set of radial lines and that of
Cartesian lines parallel to the axes.  To be more specific, the sample
points along a radial line ${\cal L}$ whose angle with the vertical
axis is less than or equal to $\pi/4$ are obtained by intersecting ${\cal
      L}$ with the set of horizontal lines $\{x_2 = k_2, \, k_2 = -n/2,
-n/2 + 1, \ldots, n/2\}$.  Similarly, the intersection with the
vertical lines $\{x_1 = k_1, \, k_1 = -n/2, -n/2 + 1, \ldots, n/2\}$
defines our sample points whenever the angle between ${\cal L}$ and the
horizontal axis is less than or equal to $\pi/4$.  The cardinality of the
rectopolar grid is equal to $2 n^2$ as there are $2 n$ radial lines and
$n$ sampled values on each of these lines. As a result,  data
structures associated with this grid will have a rectangular format. We
observe that this choice corresponds to irregularly spaced values of
the angular variable $\theta$.

\begin{figure}[htb]
\centerline{
\hbox{\psfig{figure=polar.eps,height=8cm,width=10cm,clip=}}
}
\caption{Illustration of the digital polar grid in the frequency domain for
          an $n$ by $n$ image $(n=8)$. The figure displays the set of
          radial lines joining pairs of symmetric points from the boundary
          of the square. The rectopolar grid is the set of points -- marked with
          circles -- at the intersection between those radial lines and
          those which are parallel to the axes.}
\label{fig_radon}
\end{figure}


\subsubsection{Interpolation to Rectopolar Grid}

To obtain samples on the rectopolar grid,
we should, in general, interpolate from
nearby samples at the Cartesian grid.
In principle, cf.\
\cite{cur:averbuch01,cur:donoho_02}, 
the interpolation of Fourier transforms is a
very delicate matter because of the well-known fact that
the Fourier transform of an image is highly oscillatory,
and the phase contains crucial information about the image.
In our approach, however, we use a crude interpolation method:
we simply impute for
$\hat{f}(\xi_{\ell,m})$ the value of the Fourier transform taken at
the point on the Cartesian grid nearest to
$\xi_{\ell,m}$.

There are, of course, more sophisticated ways to realize the
Cartesian-to-polar conversion; even simple bilinear interpolation
would offer better theoretical accuracy. A very high accuracy
approach used in \cite{FRT} consists of viewing the data
$(\hat{f}(k_1,k_2))$ as
samples of the trigonometric polynomial $F$ defined by

\begin{equation}
\label{eq:2d-poly}
F(\omega_1,\omega_2) = \sum_{i_1 = 0}^{n-1}  \sum_{i_2 =0}^{n-1}
    f(i_1,i_2) \exp\{-i(\omega_1 i_1 + \omega_2 i_2)\}
\end{equation}
on a square lattice; that is, with $\hat{f}(k_1,k_2) = F(\frac{2\pi
      k_1}{n},\frac{2\pi k_2}{n})$ with $ -n/2 \le k_1, k_2 < n/2$.
There turns out \cite{FRT,cur:averbuch01} to be an exact algorithm for
rapidly
finding the values of $F$ on the polar grid. The high-accuracy
approach can be used in reverse, allowing for exact reconstruction
of the original trigonometric polynomial from its rectopolar samples.

Our nearest-neighbor interpolation, although admittedly simple-minded,
happens to give  good results in our applications.
In fact numerical experiments show that in overall system
performance, it rivals the exact interpolation scheme.
This is explainable as follows.  Roughly speaking, the high-frequency
terms in the
trigonometric polynomial $F$ are associated with pixels at the boundary of
the underlying $n$ by $n$ grid.  Our crude interpolation evidently will
fail at reconstructing high-frequency terms. However, in the curvelet
application -- see below -- we use a window function to downweight
the contributions of our reconstruction near the boundary of the image
array.
So, inaccuracies in
reconstruction caused by our crude interpolation can be expected to
be located mostly in regions which make little visual
impact on the reconstruction.

A final point about our implementation. Since we are
interested in noise removal, avoiding 
artifacts is very important.  At the  signal-to-noise ratios we
consider, high-order-accuracy interpolation formulas which  generate
substantial artifacts  (as many high-order formulas do) can be less
useful than  low-order-accuracy schemes which are relatively
artifact-free. A known artifact of exact interpolation of trigonometric
polynomials is as follows: substantial long-range disturbances can be
generated by local perturbations such as discontinuities. In this
sense, our crude interpolation may actually turn out to be preferable
for some purposes.

\subsubsection{Exact Reconstruction and Stability}

The Cartesian-to-rectopolar conversion we have suggested here
is reversible. That is to say, given the rectopolar values output from
this method, one can recover the original Cartesian values exactly.
To see this, take as given the following: {\it {\bf Claim:} the assignment
of Cartesian points as nearest neighbors of rectopolar points happens
in such a way that   each Cartesian point is assigned as the nearest
neighbor of at least one rectopolar point.}  It follows from this claim
that each value in the original Cartesian input array is copied into at
least one place in the output rectopolar array. Hence, perfect
reconstruction is obviously possible in principle -- just by keeping track
of where the entries are, have been copied to, and undoing the process.

Our reconstruction rule obtains, for each point on the Cartesian grid,
the {\it arithmetic mean of all the values in the rectopolar grid
  which have that Cartesian point as their nearest point}. This
provides a numerically stable left inverse.  Indeed, if applied to a
perturbed set of rectopolar values, this rule gives an approximate
reconstruction of the original unperturbed Cartesian values in which
the approximation errors are smaller than the size of the
perturbations suffered by the rectopolar values. (This final comment
is reassuring in the present de-noising context, where our
reconstructions will always be made by perturbing the empirical
rectopolar FT of the noisy data.) Phrased in mathematical terms
this gives
\[
x_C = 1/\# R(C) \sum_{R \in R(C)} y_R, 
\]
where $C$ is a given point on the Cartesian grid and $R(C)$ is the set
of rectopolar points that are closest to $C$. Cardinality is denoted
$\#$. Stability in $\ell_2$,
for instance, follows from the observation
\[
|x_C|^2 \le 1/\# R(C) \sum_{R \in R(C)} |y_R|^2 
\le \sum_{R \in R(C)} |y_R|^2; 
\]
Since we have a partition of the set of rectopolar points,
summing this last inequality across the Cartesian grid gives $\|x\|^2
\le \|y\|^2$.

It remains to explain the italicized claim, because, as we have seen,
from it flows the exact reconstruction property and stability of the
inverse. Consider the rectopolar points in the hourglass region made of
`basically vertical lines', i.e. lines which make an angle
less than $\pi/4$ with vertical, and more specifically those
points on a single horizontal scan line. Assuming the scan line
is not at the extreme top or bottom of the array, these points are spaced {\it
strictly less than one unit apart}, where our unit is the spacing of 
the Cartesian
grid. Therefore, when we consider a Cartesian grid point $C$ belonging to this
scan line and ask about the rectopolar points $R_L$ and $R_R$ which 
are closest to
it on the left and right respectively, these two points cannot be as much as 1
unit apart:
$|| R_L - R_R || < 1$. Therefore at least one of the two points must be
strictly less than 1/2 unit away from the Cartesian point: i.e. either
$|| R_L - C || < 1/2$ or $|| R_R - C || < 1/2$. Without loss of
generality suppose that $|| R_L - C || < 1/2$. Then clearly $R_L$ has
$C$ as its closest Cartesian point. In short, every Cartesian point in
the strict interior of the `hourglass' associated with the `basically
vertical' lines arises as the strict closest Cartesian point of at
least one rectopolar point. Similar statements can be made about points
on the boundary of the hourglass, although the arguments supporting
those statements are much simpler, essentially mere inspection. Similar
statements can be made about the points in the transposed hourglass.
The italicized claim is established.


\subsubsection{One-dimensional Wavelet Transform}

To complete the ridgelet transform, we must take
a one-dimensional wavelet transform along the
radial variable in Radon space. We now discuss the choice of
digital one-dimensional wavelet transform.

Experience has shown that compactly-supported wavelets can lead to many
visual artifacts when  used in conjunction with nonlinear processing --
such as hard-thresholding of individual wavelet coefficients --
particularly for decimated wavelet schemes used at critical sampling.
Also, because of the lack of localization of such compactly-supported
wavelets in the frequency domain, fluctuations in coarse-scale wavelet
coefficients can introduce fine-scale fluctuations; this is
undesirable in our setting. Here we take a frequency-domain approach,
where the discrete Fourier transform is reconstructed from the inverse
Radon transform. These considerations lead us to use a band-limited
wavelet, whose support is compact in the Fourier domain rather than
the time domain. Other implementations have made a choice of compact
support in the frequency domain as well \cite{DRT,FRT}. However, we
have chosen a specific overcomplete system, based on work of Starck et
al. \cite{starck:sta94_3,starck:book98}, who constructed such  a
wavelet transform and applied it to interferometric image
reconstruction. The wavelet transform algorithm is based on a scaling
function $\phi$ such that $\hat{\phi}$ vanishes outside of the interval
$[-\nu_c, \nu_c]$.  We defined the scaling function
$\hat{\phi}$ as a renormalized $B_3$-spline
\[
\hat{\phi}(\nu)= {3\over 2} B_3(4\nu),
\]
and $\hat{\psi}$ as the difference between two consecutive resolutions
\[
\hat \psi(2\nu) = \hat \phi(\nu) - \hat \phi(2\nu).
\]
Because $\hat{\psi}$ is compactly supported, the sampling theorem
shows than one can easily build a pyramid of $n + n/2 + \ldots + 1 =
2n$ elements: see \cite{starck:book98} for details.

This transform enjoys the following features:
\begin{itemize}
\item The wavelet coefficients are directly calculated in the Fourier
      space. In the context of the ridgelet transform, this allows
      avoiding the computation of the one-dimensional inverse Fourier
      transform along each radial line.
\item Each subband is sampled above the Nyquist rate, hence
      avoiding aliasing -- a phenomenon typically encountered by critically
      sampled orthogonal wavelet transforms \cite{Steerable}.
\item The reconstruction is trivial. The wavelet coefficients simply
      need to be co-added to reconstruct the input signal at any given
      point. In our application, this implies that the ridgelet
coefficients
      simply need to be co-added to reconstruct Fourier coefficients.
\end{itemize}

This wavelet transform introduces an extra redundancy factor, which
might be viewed as an objection by advocates of orthogonality
and critical sampling. However,
we note that our goal in this implementation is not data
compression/efficient
coding -- for which critical sampling might be relevant -- but
instead
noise removal, for which it is well-known that overcompleteness
can provide substantial advantages
\cite{rest:donoho95}.

\begin{figure}[htb]
\centerline{
\hbox{
\psfig{figure=fig_rid.ps,bbllx=4.5cm,bblly=5cm,bburx=21cm,bbury=22cm,width=10cm,height=10cm,clip=}
}}
\caption{Ridgelet transform flowgraph. Each of the $2n$  radial lines
      in the Fourier domain is processed separately. The 1D inverse FFT
      is calculated along each radial line followed by a 1D nonorthogonal
      wavelet transform.  In practice, the one-dimensional wavelet
      coefficients are directly calculated in the Fourier space. }
\label{fig_ridgelet}
\end{figure}

\subsubsection{Combining the Pieces}

Figure~\ref{fig_ridgelet} shows the flowgraph of the ridgelet
transform.  The ridgelet transform of an image of size $n \times n$ is
an image of size $2n \times 2n$, introducing a redundancy factor
equal to 4.

We note that, because our transform is made up of a chain
of steps, each one of which is invertible, the whole
transform is invertible, and so has the
exact reconstruction property. For the same
reason, the reconstruction
is stable under perturbations of the coefficients.

Last but not least, our discrete transform is computationally
attractive. Indeed, the algorithm we presented here has low complexity
since it runs in $O(n^2\log(n))$ flops for an $n \times n$ image.

\subsection{The Orthonormal Finite Ridgelet Transform}
The orthonormal finite ridgelet transform (OFRT)  has been 
recently proposed  
\cite{cur:do00,cur:do00b} for image compression and filtering. 
The transform, based on the finite Radon transform \cite{cur:matus93} and
a 1D orthogonal wavelet transform, is not redundant, and reversible.
It would be a great alternative to the previously described ridgelet transform
if the OFRT were not based on a strange definition 
of a line. In fact, a line in the OFRT is defined as 
a set of periodic equidistant points \cite{cur:matus93}. 
Figure~\ref{fig_cmp_backproj} shows the backprojection of a ridgelet 
coefficient by the FFT-based ridgelet transform (left) and by the 
 OFRT (right). It is clear that the backprojection
of the  OFRT is nothing like a ridge function.

\begin{figure}[htb]
\centerline{
\vbox{
\hbox{
\psfig{figure=fig_backproj_ridfft.ps,bbllx=2cm,bblly=13.5cm,bburx=13cm,bbury=24.5cm,width=7.cm,height=7.cm,clip=}
\psfig{figure=fig_backproj_fird.ps,bbllx=2cm,bblly=13.5cm,bburx=13cm,bbury=24.5cm,width=7.cm,height=7.cm,clip=}
}}}
\caption{Left, backprojection of a ridgelet coefficient by the 
FFT-based ridgelet transform,
and right, backprojection of a finite ridgelet coefficient.}
\label{fig_cmp_backproj}
\end{figure}

\begin{figure}[htb]
\centerline{
\vbox{
\hbox{
\psfig{figure=fig_l257.ps,bbllx=2cm,bblly=13.5cm,bburx=13cm,bbury=24.5cm,width=7.cm,height=7.cm,clip=}
\psfig{figure=fig_ridl257.ps,bbllx=2cm,bblly=13.5cm,bburx=13cm,bbury=24.5cm,width=7.cm,height=7.cm,clip=}
}}}
\caption{Left, part of Lena image,
and right, reconstruction after finite ridgelet coefficient thresholding.}
\label{fig_cmp_recl257}
\end{figure}

Because of this specific definition of a line, the thresholding of the OFRT
coefficients produces strong artifacts. 
Figure~\ref{fig_cmp_recl257} left shows a part 
of the original standard Lena image, and 
Figure~\ref{fig_cmp_recl257} right shows the reconstruction after the 
hard thresholding of the OFRT. A kind of noise has been added to the 
noise-free image!
Finally, the OFRT presents another limitation: the image size  
must be a prime number. This last point is however not too restrictive, 
because we generally use a partitioning when denoising the data, and a
prime number block size can be used. 
The OFTR is interesting from the conceptual point of view, but will
certainly be of no help for real applications.  

\subsection{The Slant Stack Ridgelet Transform}

\begin{figure}[htb]

\centerline{
\vbox{
\hbox{
\psfig{figure=fig_back_ssrad.ps,bbllx=8.8cm,bblly=13.7cm,bburx=12.1cm,bbury=16.cm,width=3cm,height=3cm,clip=}
\psfig{figure=fig_back_ssrad1.ps,bbllx=8.8cm,bblly=13.7cm,bburx=12.1cm,bbury=16.cm,width=3cm,height=3cm,clip=}
}
\hbox{
\psfig{figure=fig_back_ssrad2.ps,bbllx=8.8cm,bblly=13.7cm,bburx=12.1cm,bbury=16.cm,width=3cm,height=3cm,clip=}
\psfig{figure=fig_back_ssrad3.ps,bbllx=8.8cm,bblly=13.7cm,bburx=12.1cm,bbury=16.cm,width=3cm,height=3cm,clip=}
}}}
\caption{Backproject of a point at four different locations in the 
Radon space.}
\label{fig_back_stack_radon}
\end{figure}

The Fast Slant Stack \cite{cur:averbuch01}
 is geometrically more accurate than the 
previously decribed methods. The backprojection of a point 
in Radon space is exactly a ridge function in the spatial domain
(see Figure~\ref{fig_back_stack_radon}).
The transformation of an $n \times n$ image is a $2n \times 2n$ image.
$n$ line integrals with angle between $[-\frac{\pi}{4}, \frac{\pi}{4}]$
are calculated from the zero padded image on the y-axis, 
and $n$ line integrals with angle between $[\frac{\pi}{4}, \frac{3\pi}{4}]$
are computed by zero padding the image on the x-axis. 
For a given angle inside $[-\frac{\pi}{4}, \frac{\pi}{4}]$,
$2n$ line integrals are calculated by first shearing the zero-padded image, 
and then integrating the pixel values 
along all horizontal lines (resp.\ vertical lines
for angles in $[\frac{\pi}{4}, \frac{3\pi}{4}]$). The shearing is 
performed column per column (resp.\ line per line) by using the 1D FFT.
Figure~\ref{fig_stack_radon} shows an example of the image shearing step
with two different angles ($5\frac{\pi}{4}$  and $-\frac{\pi}{4}$).
A ridgelet transform based on the Fast Slant Stack transform has been
proposed in \cite{cur:donoho_02}. The connection between the Fast Slant Stack
and the linogram has been investigated in \cite{cur:averbuch01}, and
a Fast Slant Stack is proposed, based on the 2D Fourier transform.

\begin{figure}[htb]
\centerline{
\hbox{
\psfig{figure=fig_slant_radon.ps,bbllx=3cm,bblly=9cm,bburx=18cm,bbury=21.cm,width=15cm,height=12cm,clip=}
}}
\caption{Slant Stack Transform of an image.}
\label{fig_stack_radon}
\end{figure}

\section{Local Ridgelet Transforms}

A digital version of the ideas presented in section \ref{sec:pyramid}
decomposes the original $n$ by $n$ image into smoothly overlapping
blocks of sidelength $b$ pixels in such a way that the overlap between
two vertically adjacent blocks is a rectangular array of size $b$ by
$b/2$; we use overlap to avoid blocking artifacts. For an $n$ by $n$
image, we count $2n/b$ such blocks in each direction.

The partitioning introduces redundancy, as a pixel belongs to 4
neighboring blocks. We present two competing strategies to perform the
analysis and synthesis:
\begin{enumerate}
\item The block values are weighted (analysis) in such a way that the
      co-addition of all blocks reproduce exactly the original pixel value
      (synthesis).
\item The block values are those of the image pixel values (analysis)
      but are weighted when the image is reconstructed (synthesis).
\end{enumerate}
Of course, there are intermediate strategies and one could apply
smooth windowing at both the analysis and synthesis stages as discussed
in Section \ref{sec:pyramid}, for example. In the first approach, the
data are smoothly windowed and this presents the advantage to limit
the analysis artifacts traditionally associated with boundaries.  The
drawback, however, is a loss of sensitivity. Indeed, suppose for the
sake of simplicity that a vertical line with intensity level $L$
intersects a given block of size $b$. Without loss of generality
assume that the noise standard deviation is equal to 1. When the
angular parameter of the Radon transform coincides with that of the
line,
we obtain a measurement with a signal intensity equal to $b L$ while
the noise standard deviation is equal to $\sqrt{b}$ (in this case, the
value of the signal-to-noise ratio (SNR) is $\sqrt{b} L$). If weights
are applied at the analysis stage, the SNR is roughly equal to $L \,
\sum_{i=1}^b w_i/\sqrt{\sum_{i=1}^b w_i^2} < \sqrt{b} L$. Experiments
have shown that this sensitivity loss may have substantial effects
in filtering applications and, therefore, the second approach
seems more appropriate since our goal is image restoration.

We calculate a pixel value $f(i,j)$ from its four corresponding block
values of half-size $\ell = b/2$, namely, $B_1(i_1,j_1)$,
$B_2(i_2,j_1)$, $B_3(i_1,j_2)$ and $B_4(i_2,j_2)$ with $i_1, j_1 >
b/2$ and $i_2 = i_1 - \ell, j_2 = j_1- \ell$, in the following way:
\begin{eqnarray}
      f_1 &  = &  w(i_2/\ell) B_1(i_1,j_1) + w(1-i_2/\ell)
B_2(i_2,j_1)\nonumber \\
      f_2 &  = &  w(i_2/\ell) B_3(i_1,j_2) + w(1-i_2/\ell)
B_4(i_2,j_2)\nonumber \\
      f(i,j) &  = &  w(j_2/\ell) f_1 +  w(1-j_2/\ell) f_2
\end{eqnarray}
with $w(x) = \cos^2(\pi x/2)$.  Of course, one might select any other
smooth, non-increasing function satisfying $w(0) = 1$, $w(1) = 0$,
$w'(0) = 0$ and obeying the symmetry property $w(x) + w(1-x) = 1$.

It is worth mentioning that the spatial partitioning introduces a
redundancy factor equal to 4.

Finally, we note that in order to be in better agreement with the
theory one should of course introduce a normalizing factor depending
on the block-size. However, since we are concerned about de-noising
and the thresholding of individual coefficients, the normalization is
a non-issue.  Renormalizing coefficients automatically renormalizes
corresponding thresholds in the exact same way: see section 
\ref{sec:filtering}.


\section{Digital Curvelet Transform}
\label{sec:curvelet}

\subsection{Discrete Curvelet Transform of Continuum Functions}

We now briefly return to the continuum viewpoint of section
\ref{sec:pyramid}.
Suppose we set an initial goal to produce a decomposition using the
multiscale ridgelet pyramid. The hope is that this
would allow us to use thin `brushstrokes' to reconstruct
the image, with all lengths and widths available to us.
In particular, this would seem allow us to trace sharp edges precisely
using a few elongated elements with very narrow widths.

As mentioned in section \ref{sec:pyramid},
the full multiscale ridgelet pyramid is highly overcomplete.
As a consequence,  convenient algorithms like
simple thresholding will not find sparse
decompositions when such good decompositions exist.
An important ingredient of the curvelet
transform is to restore sparsity by reducing redundancy across scales.
In
detail, one introduces interscale orthogonality by means of
subband filtering. Roughly speaking, different levels of the
multiscale ridgelet pyramid are used to represent different
subbands of a filter bank output.  At the same time, this subband
decomposition imposes a relationship between the width and length
of the important frame elements so that they are anisotropic and obey
$width = length^2$.

The discrete curvelet transform of a continuum
function $f(x_1,x_2)$ makes use of a dyadic sequence of scales,
and a bank of filters $(P_0 f, \Delta_1 f ,
\Delta_2 f , \dots )$ with the property that the passband filter
$\Delta_s$ is concentrated near the frequencies $[2^{2s}, 2^{2s+2}]$,
e.g.
\[
\Delta_s = \Psi_{2s} * f, \quad \widehat{\Psi_{2s}}(\xi) =
\widehat{\Psi}(2^{-2s}\xi).
\]
In wavelet theory, one uses a decomposition into dyadic subbands
$[2^s, 2^{s+1}]$. In contrast, the subbands used
in the discrete curvelet transform of continuum functions have
the non-standard form
$[2^{2s}, 2^{2s+2}]$. This is a non-standard feature of the
discrete curvelet transform well worth remembering.

With the notation of section \ref{sec:pyramid}, the curvelet
decomposition is the sequence of the following steps:
\begin{itemize}
     \item {\it Subband Decomposition.}  The object $f$ is
     decomposed into subbands:
     \[
       f \mapsto (P_0 f, \Delta_1 f , \Delta_2 f , \dots ) .
     \]
\item {\it Smooth Partitioning.} Each subband is smoo\-thly win\-do\-wed
      into ``squares'' of an appropriate scale (of sidelength $\sim
      2^{-s}$):
     \[
        \Delta_s f \mapsto ( w_Q    \Delta_s f )_{Q \in {\cal Q}_s} .
     \]
     \item {\it Renormalization.} Each resulting square is renormalized to
unit scale
     \begin{equation}
      \label{eq:gQ}
        g_Q =  (T_Q)^{-1}(w_Q \Delta_s f ),  \qquad {Q \in {\cal Q}_s} .
     \end{equation}
\item {\it Ridgelet Analysis.} Each square is analyzed via the
      discrete ridgelet transform.
\end{itemize}

In this definition, the two dyadic subbands $[2^{2s}, 2^{2s+1}]$ and
$[2^{2s+1}, 2^{2s+2}]$ are merged before applying the ridgelet
transform.
\subsection{Digital Realization}

In developing a transform for digital $n$ by $n$ data
which is analogous to the discrete curvelet transform of a
continuous function $f(x_1,x_2)$,
we replace each of the continuum concepts
with the appropriate digital concept mentioned in the sections above.
In general, the translation is rather obvious and direct. However,
experience shows that one modification is essential;
we found that, rather than merging the two
dyadic subbands $[2^{2s}, 2^{2s+1}]$ and
$[2^{2s+1}, 2^{2s+2}]$ as in the theoretical work,
in the digital application, leaving these subbands separate,
applying spatial partitioning to each subband
and applying the ridgelet transform on
each subband separately led to improved visual and numerical results.

We believe that the ``\`a trous'' subband filtering algorithm is
especially
well-adapted to the needs of the digital curvelet transform.
The algorithm decomposes an $n$ by $n$ image $I$ as a superposition of
the form
\[
I(x,y) = c_{J}(x,y) + \sum_{j=1}^{J} w_j(x,y),
\]
where $c_{J}$ is a coarse or smooth version of the original image $I$
and $w_j$ represents ``the details of $I$'' at scale $2^{-j}$: see
\cite{starck:book98} for more information. Thus, the algorithm outputs
$J+1$ subband arrays of size $n \times n$. (The indexing is such that,
here, $j = 1$ corresponds to the finest scale, or high frequencies.)

\begin{figure}[htb]
\centerline{
\hbox{\psfig{figure=fig_cur.ps,width=8cm,height=10cm,clip=}}
}
\caption{Curvelet transform flowgraph. The figure illustrates
the decomposition of the original image  into subbands followed by
the spatial partitioning of each subband. The ridgelet transform
is then applied to each block.}
\label{fig_curvelet}
\end{figure}

\subsection{Algorithm}
We now present a sketch of the discrete curvelet transform algorithm:
\begin{enumerate}
\item Apply the \`a trous algorithm with $J$ scales,
\item set $B_1 = B_{min}$,
\item for $j = 1, \ldots, J$ do,
\begin{enumerate}
\item partition the subband $w_j$ with a block size $B_j$ and apply the
digital ridgelet transform to each block,
\item if $j \mbox{ modulo } 2 = 1$ then $B_{j+1} = 2B_{j}$,
\item else $B_{j+1} = B_{j}$.
\end{enumerate}
\end{enumerate}
The sidelength of the localizing windows is doubled {\em at every
      other} dyadic subband, hence maintaining the fundamental property of
the curvelet transform which says that elements of length about
$2^{-j/2}$ serve for the analysis and synthesis of the $j$-th subband
$[2^j, 2^{j+1}]$.  Note also that the coarse description of the image
$c_J$ is not processed. Figure~\ref{fig_curvelet}
gives an overview of the organization of the algorithm.


This implementation of the curvelet transform is also redundant. The
redundancy factor is equal to $16J+1$ whenever $J$ scales are employed.
Finally, the method enjoys exact reconstruction and stability,
because this invertibility holds for each element of the
processing chain.

\begin{figure}[htb]
\centerline{
\hbox{
\psfig{figure=fig_sparse_s1.ps,bbllx=4cm,bblly=7cm,bburx=17cm,bbury=21cm,width=6.5cm,height=7cm,clip=}
\psfig{figure=fig_sparse_s2.ps,bbllx=4cm,bblly=7cm,bburx=17cm,bbury=21cm,width=6.5cm,height=7cm,clip=}
}}
\caption{A few curvelets.}
\label{fig_ex_curve}
\end{figure}
Figure~\ref{fig_ex_curve} shows a few curvelets at different scales,
orientations and locations.  
 




 

\chapter{Filtering}
\label{sec:filtering}
\section{Curvelet Coefficient Thresholding}
We now apply our digital transforms for
removing noise from image data. The methodology is standard and is
outlined mainly for the sake of clarity and self-containedness.

Suppose that one is given noisy data of the form
\[
x_{i,j} = f(i,j) + \sigma z_{i,j},
\]
where $f$ is the image to be recovered and $z$ is white noise, i.e.
$z_{i,j} \stackrel{i.i.d.}{\sim} N(0,1)$. Unlike FFTs or FWTs, our
discrete ridgelet (resp. curvelet) transform is not norm-preserving
and, therefore, the variance of the noisy ridgelet (resp. curvelet)
coefficients will depend on the ridgelet (resp. curvelet) index
$\lambda$.  For instance, letting $F$ denote the discrete curvelet
transform matrix, we have $F z \stackrel{i.i.d.}{\sim} N(0,F F^T)$
where $T$ denotes transpose.
Because the computation of $F F^T$ is prohibitively expensive, we
calculated an approximate value $\tilde{\sigma}^2_\lambda$ of the
individual variances using Monte Carlo simulations where the diagonal
elements of $F F^T$ are simply estimated by evaluating the curvelet
transforms of a few standard white noise images.

Let $y_\lambda$ be the noisy curvelet coefficients
($y = F x$). We use the following hard-thresholding rule
for estimating the unknown curvelet coefficients:
\begin{eqnarray}
\hat y_\lambda = y_\lambda  & \mbox{ if }  &
|y_\lambda|/\sigma \geq k \tilde{\sigma}_\lambda\\
\hat y_\lambda = 0 &  \mbox{ if } & |y_\lambda|/\sigma
<  k \tilde{\sigma}_\lambda.
\end{eqnarray}
In our experiments, we actually chose a scale-dependent value for $k$;
we have $k = 4$ for the first scale $(j = 1)$ while $k = 3$
for the others $(j > 1)$.

\subsection*{Poisson Observations}
Assume now that we have Poisson data $x_{i,j}$ with unknown mean
$f(i,j)$.  The Anscombe transformation \cite{rest:anscombe48}
\begin{equation}
\tilde{x} = 2\sqrt{x + \frac{3}{8}}
\end{equation}
stabilizes the variance and we have $\tilde{x} = 2 \sqrt{f} +
\epsilon$ where $\epsilon$ is a vector with independent and
approximately standard normal components. In practice, this is a good
approximation whenever the number of counts is large enough, greater
than 30 per pixel, say.
 
For a small number of counts, a possibility is to compute the Radon
transform of the image, and then to apply the Anscombe transformation
to the Radon data. The rationale is that, roughly speaking, the
Radon transform corresponds to a summation of pixel values over lines
and  the sum of independent Poisson random variables is a Poisson
random variable with intensity equal to the sum of the individual
intensities. Hence, the intensity of the sum may be quite large (hence
validating the Gaussian approximation) even though the individual
intensities may be small. This might be viewed as an interesting
feature  since,  unlike wavelet transforms, the ridgelet and curvelet
transforms tend to average data over elongated and rather large
neighborhoods.

\section{Filtering Experiments}

\subsubsection{Recovery of Linear Features}

\begin{figure}[htb]
\centerline{
\vbox{
\hbox{
\psfig{figure=fig_line.ps,bbllx=1cm,bblly=12cm,bburx=15cm,bbury=26cm,width=7.5cm,height=7.5cm,clip=}
\psfig{figure=fig_line_g0p5.ps,bbllx=1cm,bblly=12cm,bburx=15cm,bbury=26cm,width=7.5cm,height=7.5cm,clip=}
}
\hbox{
\psfig{figure=fig_f24_line_g0p5.ps,bbllx=1cm,bblly=12cm,bburx=15cm,bbury=26cm,width=7.5cm,height=7.5cm,clip=}
\psfig{figure=fig_cur_line_g0p5.ps,bbllx=1cm,bblly=12cm,bburx=15cm,bbury=26cm,width=7.5cm,height=7.5cm,clip=}
}}}
\caption{The top panels display a geometric image and that same image
      contaminated with Gaussian white noise. The bottom left and right
      panels display the restored images using the undecimated wavelet
      transform and the curvelet transform, respectively.}
\label{fig_cur_line}
\end{figure}
The next experiment (Figure~\ref{fig_cur_line}) consists of an
artificial image containing a few bars, lines and a square.  The
intensity is constant along each individual bar; from left to right,
the intensities of the ten vertical bars (these are in fact thin
rectangles which are 4 pixels wide and 170 pixels long) are equal to
${32 \over 2^i}$, $i = 0, \ldots 9$.  The intensity along all the
other lines is equal to 1, and the noise standard deviation is $1/2$.
Displayed images have been log-transformed in order to better see
the results at low signal to noise ratio.

The curvelet reconstruction of the nonvertical lines is obviously
sharper than that obtained using wavelets.  The curvelet transform
also seems to go one step further as far as the reconstruction of the
vertical lines is concerned. Roughly speaking, for those templates,
the wavelet transform stops detecting signal at a SNR equal to 1
(we defined here the SNR as the intensity level of the pixels
on the line, divided by the noise standard deviation of the noise)
while the cut-off value equals 0.5 for the curvelet approach.  It is
important to note that the horizontal and vertical lines correspond to
privileged directions for the wavelet transform, because the underlying
basis functions are direct products of functions varying solely in
the horizontal
and vertical directions.  Wavelet methods will
give even poorer results on lines
of the same intensity but tilting substantially
away from the Cartesian axes.
Cf.\ the reconstructions of the faint diagonal
lines in the image.

\subsubsection{Recovery of Curves}
In this experiment (Figure~\ref{fig_cur_picasso}), we have added 
Gaussian noise to ``War and Peace,'' a drawing from Picasso which
contains many curved features.  Figure \ref{fig_cur_picasso} bottom left
and right show respectively the restored images by the undecimated
wavelet transform and the curvelet transform. Curves are more sharply
recovered with the curvelet transform.

The authors are working on new methods (some of which will be based on
the curvelet transform) to extract and recover curves from noisy data
with greater accuracy and, therefore, this example is merely to be
taken for illustrative purposes.

\begin{figure}[htb]
\centerline{
\vbox{
\hbox{
\psfig{figure=fig_picasso.ps,bbllx=2cm,bblly=13cm,bburx=18cm,bbury=25cm,width=8cm,height=6cm,clip=}
\psfig{figure=fig_picasso_g50.ps,bbllx=2cm,bblly=13cm,bburx=18cm,bbury=25cm,width=8cm,height=6cm,clip=}
}
\hbox{
\psfig{figure=fig_picasso_wt.ps,bbllx=2cm,bblly=13cm,bburx=18cm,bbury=25cm,width=8cm,height=6cm,clip=}
\psfig{figure=fig_picasso_cur.ps,bbllx=2cm,bblly=13cm,bburx=18cm,bbury=25cm,width=8cm,height=6cm,clip=}
}}}
\caption{The top panels display a Picasso picture (War and Peace) 
      and that same image
      contaminated with Gaussian white noise. The bottom left and right
      panels display the restored images using the undecimated wavelet
      transform and the curvelet transform respectively.}
\label{fig_cur_picasso}
\end{figure}


\subsubsection{Denoising of a Color Image}

\begin{figure}[htb]
\centerline{
\vbox{
\hbox{
\psfig{figure=fig_cmp_lena_rgb.ps,bbllx=2cm,bblly=13cm,bburx=20cm,bbury=25cm,width=6cm,height=4.cm,clip=}
\psfig{figure=fig_cmp_pepper_rgb.ps,bbllx=2cm,bblly=13cm,bburx=20cm,bbury=25cm,width=6cm,height=4.cm,clip=}
}
\hbox{
\psfig{figure=fig_cmp_mandrill_rgb.ps,bbllx=2cm,bblly=13cm,bburx=20cm,bbury=25cm,width=6cm,height=4.cm,clip=}
\psfig{figure=fig_cmp_barbara_rgb.ps,bbllx=2cm,bblly=13cm,bburx=20cm,bbury=25cm,width=6cm,height=4.cm,clip=}
}
}}
\caption{PSNR versus noise standard deviation using different filtering
      methods. YUV and curvelet, YUV and undecimated wavelet, and YUV and
      decimated wavelet transforms are represented respectively with a
      continuous, dashed, and dotted line.  The upper left panel
      corresponds to {\tt Lena} (RGB), the upper right to {\tt pepper}
      (RGB), the bottom left to {\tt Baboon} (RGB), and the bottom right
      to {\tt Barbara} (RGB).}
\label{fig_exp_rgb_curv}
\end{figure}

In a wavelet based denoising scenario, color RGB images are generally
mapped into the YUV space, and each YUV band is then filtered
independently from the others. The goal here is to see whether the
curvelet transform would give improved results. We used four of the
classical color images, namely {\tt Lena}, {\tt Peppers}, {\tt
      Baboon}, and {\tt Barbara} (all images except perhaps {\tt Barbara}
are available from the USC-SIPI Image Database \cite{SIPI}).
We performed
a series of experiments and
summarized our findings in Figure~\ref{fig_exp_rgb_curv} which again
displays the PSNR (peak signal-to-noise ratio) 
versus the noise standard deviation for the four
images.

In all cases, the curvelet transform outperforms the wavelet
transforms in terms of PSNR -- at least for moderate and large values
of the noise level.  In addition, the curvelet transform outputs
images that are visually more pleasant.
 
\subsubsection{Saturn Rings}

\begin{figure}[htb]
\centerline{
\vbox{
\hbox{
\psfig{figure=SATURN/fig_sat_sub_g20.ps,bbllx=1.8cm,bblly=12.7cm,bburx=14.5cm,bbury=25.4cm,width=7cm,height=7cm,clip=}
\psfig{figure=SATURN/fig_sat_sub_fil_owt.ps,bbllx=1.8cm,bblly=12.7cm,bburx=14.5cm,bbury=25.4cm,width=7cm,height=7cm,clip=}
}
\hbox{
\psfig{figure=SATURN/fig_sat_sub_fil_atrou.ps,bbllx=1.8cm,bblly=12.7cm,bburx=14.5cm,bbury=25.4cm,width=7cm,height=7cm,clip=}
\psfig{figure=SATURN/fig_sat_sub_fil_cur.ps,bbllx=1.8cm,bblly=12.7cm,bburx=14.5cm,bbury=25.4cm,width=7cm,height=7cm,clip=}
}}
}
\caption{Top left, part of Saturn image with Gaussian noise. Top right,
filtered image using the undecimated bi-orthogonal wavelet transform.
Bottom left and right, filtered image by the \`a trous wavelet 
transform algorithm and the curvelet transform.}
\label{fig_sub_saturn_cur_filter}
\end{figure}

Gaussian white noise with a standard deviation
fixed to 20 was added to the {\tt Saturn} image.
We employed several methods to filter the noisy image:
\begin{enumerate}
\item Thresholding of the Curvelet transform.
\item Bi-orthogonal undecimated wavelet de-noising methods using  the
      Dau\-che\-chies-An\-to\-ni\-ni 7/9 fil\-ters (FWT-7/9) and hard thresholding.
\item A trous wavelet transform algorithm and hard thresholding.
\end{enumerate}
Our experiments are reported in Figure~\ref{fig_sub_saturn_cur_filter}.  The
curvelet reconstruction does not contain the quantity of disturbing artifacts
along edges that one sees in wavelet reconstructions. An examination
of the details of the restored images % (Figure~\ref{fig_sub_saturn_cur_filter})
is instructive. One notices that the decimated wavelet transform
exhibits distortions of the boundaries and suffers substantial loss of
important detail.  The \`a trous  wavelet transform gives better
boundaries, but completely omits to reconstruct certain ridges. 
In addition, it exhibits numerous small-scale embedded
blemishes; setting higher thresholds to avoid these blemishes would
cause even more of the intrinsic structure to be missed.

% Further results are visible at the following URL: {\tt
%  http://www-stat.stanford.edu/$\sim$jstarck}.

\subsubsection{Supernova with Poisson noise}

\begin{figure}[htb]
\centerline{  
\hbox{
\psfig{figure=fig_kepler1604.ps,bbllx=1.8cm,bblly=12.7cm,bburx=14.5cm,bbury=25.4cm,width=7cm,height=7cm,clip=}
\psfig{figure=fig_kepler1604_fil_rid.ps,bbllx=1.8cm,bblly=12.7cm,bburx=14.5cm,bbury=25.4cm,width=7cm,height=7cm,clip=}
}}
\caption{Left, XMM/Newton image of the Kepler SN1604 supernova.
Right, ridgelet filtered image.}
\label{fig_sn1604}
\end{figure}
Figure~\ref{fig_sn1604} shows an example of an X-ray image filtering by the
ridgelet transform using such an approach. Figure~\ref{fig_sn1604} left
and right 
show respectively the  XMM/Newton image of the Kepler SN1604 supernova and
the ridgelet filtered image (using a five sigma hard thresholding). 


% \clearpage
% \newpage


\chapter{The Combined Filtering Method}
% \section{The Combined Filtering Method}
\section{Introduction}
\label{sect_comb}

\begin{figure}[htb]
\centerline{
\hbox{
\psfig{figure=fig_resi_lena_f24.ps,bbllx=2cm,bblly=13.5cm,bburx=13cm,bbury=24.5cm,width=7.5cm,height=7.5cm,clip=}
\psfig{figure=fig_resi_lena_cur.ps,bbllx=2cm,bblly=13.5cm,bburx=13cm,bbury=24.5cm,width=7.5cm,height=7.5cm,clip=}
}}
\caption{Residual for thresholding of the undecimated
wavelet transform and thresholding of 
the curvelet transform.}
\label{fig_lenna_resi}
\end{figure}

Although the results obtained by simply thresholding the curvelet expansion
are encouraging, there is of course ample room for further
improvement. A quick inspection of the residual images for both the
wavelet and curvelet transforms shown in Figure~\ref{fig_lenna_resi}
reveals the existence of very different features.  For instance,
wavelets do not restore long edges with high fidelity while curvelets
are seriously challenged by small features such as {\tt Lena}'s eyes. Loosely
speaking, each transform has its own area of expertise and this
complementarity may be of great potential. This section will develop a
denoising strategy based on the idea of combining both transforms.

In general, suppose that we are given $K$ linear transforms $T_1,
\ldots, T_K$ and let $\alpha_k$ be the coefficient sequence of an
object $x$ after applying the transform $T_k$, i.e. $\alpha_k = T_k
x$. We will suppose that for each transform $T_k$ we have available a
reconstruction rule that we will denote by $T^{-1}_k$ although this is
clearly an abuse of notation.  Finally, $T$ will denote the block
diagonal matrix with the $T_k$'s as building blocks and $\alpha$ the
amalgamation of the $\alpha_k$'s.

A hard thresholding rule associated with the transform $T_k$ synthesizes 
an estimate $\tilde{s}_k$ via the formula 
\begin{equation}
\label{eq:ht}
\tilde{s}_k = T_k^{-1} \delta(\alpha_k)
\end{equation}
where $\delta$ is a rule that sets to zero all the coordinates of
$\alpha_k$ whose absolute value falls below a given sequence of
thresholds (such coordinates are said to be non-significant).

In practice, a widely used approach is to compute the average of the
$\tilde{s}_k$'s giving a reconstruction of the form
 \begin{equation}
\tilde{s} = \sum_k \tilde{s}_k/K. 
\end{equation}
For instance, in the literature of image processing it is common to
average reconstructions obtained after thresholding the wavelet
coefficients of translated versions of the original dataset
(cycle-spinning), i.e. the $T_k$'s are obtained by composing
translations and the wavelet transform.  In our setup, we do not find
this solution very appealing since this creates the opportunity to
average high-quality and low-quality reconstructions.

\section{The Combined Filtering Principle}

Given data $y$ of the form $y = s + \sigma z$, where $s$ is the image
we wish to recover and $z$ is standard white noise, we propose solving
the following optimization problem \cite{starck:spie01a}:
\begin{equation}
  \label{eq:l1-min}
  \min \|T\tilde{s}\|_{\ell_1}, \quad \mbox{subject to} \quad s \in C,  
\end{equation}
where $C$ is the set of vectors $\tilde{s}$ 
which obey the linear constraints
\begin{equation}
\label{eq:constraints}
\left\{  \begin{array}{ll}
  \tilde{s} \ge 0, \\
  |T\tilde{s} - Ty| \le e; 
  \end{array}
  \right. 
\end{equation}
here, the second inequality constraint 
only concerns the set of significant coefficients, 
i.e. those indices $\mu$ such that $\alpha_\mu =
(Ty)_\mu$ exceeds (in absolute value) a threshold $t_\mu$. Given a
vector of tolerance $(e_\mu)$, we seek a solution whose coefficients
  $(T\tilde{s})_\mu$ are within $e_\mu$ of the noisy
empirical $\alpha_\mu$'s.  Think of $\alpha_\mu$ as being given by
\[
y = \langle y, \varphi_\mu \rangle, 
\]
so that $\alpha_\mu$ is normally distributed with mean $\langle f,
\varphi_\mu \rangle$ and variance $\sigma^2_\mu = \sigma^2
\|\varphi_\mu\|^2_2$. In practice, the threshold values range
typically between three and four times the noise level $\sigma_\mu$
and in our experiments we will put $e_\mu = \sigma_\mu/2$. In short,
our constraints guarantee that the reconstruction will take into
account any pattern which is detected as significant by  any of the
$K$ transforms.
   
We use an $\ell_1$ penalty on the coefficient sequence because we are
interested in {\em low complexity} reconstructions. There are other
possible choices of complexity penalties; for instance, an alternative
to (\ref{eq:l1-min}) would be
\[
\label{eq:tv-min}
  \min \|\tilde{s}\|_{TV}, \quad \mbox{subject to} \quad s \in C. 
\]
where $\|\cdot\|_{TV}$ is the Total Variation norm, i.e.\ the discrete
equivalent of the integral of the Euclidean norm of the gradient.

\section{The Minimization Method}

We propose solving (\ref{eq:l1-min}) using the method of hybrid
steepest descent (HSD) \cite{wave:yamada01}. HSD consists of building
the sequence
\begin{eqnarray}
 s^{n+1} = P(s^{n}) - \lambda_{n+1} \nabla_J(P(s^{n})); 
\end{eqnarray}
Here, $P$ is the $\ell_2$ projection operator onto the feasible set
$C$, $\nabla_J$ is the gradient of equation~\ref{eq:l1-min}, and
$(\lambda_{n})_{n \ge 1}$ is a sequence obeying $(\lambda_{n})_{n\ge
  1} \in [0,1] $ and $\lim_{ n \rightarrow + \infty } \lambda_{n} = 0$.

Unfortunately, the projection operator $P$ is not easily determined
and in practice we will use the following proxy; compute $T \tilde{s}$
and replace those coefficients which do not obey the constraints $|T
\tilde{s} - Ty| \le e$ (those which fall outside of the prescribed
interval) by those of $y$; apply the inverse transform.  

% {\tt Needs work.} Then apply thresholding. This approximation gives
The combined filtering algorithm is:
\begin{enumerate}
\baselineskip=0.4truecm
\itemsep=0.1truecm
\item Initialize $L_{\max} = 1$, the number of iterations $N_i$, and
  $\delta_{\lambda} = \frac{L_{\max}}{N_i}$.
\item Estimate the noise standard deviation $\sigma$, and set $e_k =
  {\sigma \over 2}$.
\item For k = 1, .., $K$ calculate the transform: $\alpha^{(s)}_k
  = T_k s$.
\item Set $\lambda = L_{\max}$, $n = 0$, and $\tilde s^{n}$ to 0.
\item While $\lambda >= 0$ do
\begin{itemize}
\item $u = \tilde s^{n}$.
\item For k = 1, .., $K$ do
  \begin{itemize}
  \item Calculate the transform $\alpha_{k} = T_k u$.
  \item For all coefficients $\alpha_{k,l}$ do
     \begin{itemize}
     \item Calculate the residual $r_{k,l} = \alpha^{(s)}_{k,l} -
       \alpha_{k,l}$
       
     \item if $\alpha^{(s)}_{k,l}$ is significant and $ \mid r_{k,l}
       \mid > e_{k,l}$ then $\alpha_{k,l} = \alpha^{(s)}_{k,l}$
     \item $\alpha_{k,l} = sgn(\alpha_{k,l}) ( \mid \alpha_{k,l} \mid - \lambda)_{+}$.
     \end{itemize}
   \item $u = T_k^{-1} \alpha_{k}$
  \end{itemize}
\item Threshold negative values in $u$ and $\tilde s^{n+1} = u$.
\item $n = n + 1$, $\lambda = \lambda - \delta_{\lambda} $, and goto 5.
\end{itemize}
\end{enumerate}

\section{Experiments}

\voffset -1truecm
\begin{table}[htb]
\begin{center}
\begin{tabular}{lccccc} \hline \hline
Method                          & PSNR   &  Comments   \\ \hline \hline
Noisy image                     & 22.13  &     \\
OWT7-9 + k-sigma  Hard thresh.   & 28.35  &   many artifacts \\
UWT7-9 + k-sigma  Hard thresh.   & 31.94  &   very few artifacts \\
Curvelet (B=16)                 & 31.95  &   no artifact  \\ 
Combined filtering              & 32.72  &   no artifact  \\ \hline \hline
\end{tabular}
\caption{PSNR after filtering the simulated image (Lena + Gaussian 
noise, sigma=20).
In the combined filtering, a curvelet and an undecimated wavelet
transform were used.}
\vspace{0.5cm}
\label{comptab2}
\end{center}
\end{table}

The noisy {\tt Lena} image (the noise standard deviation being equal 20)
was filtered by the undecimated
wavelet transform, the curvelet transform, and by our combined
transform approach (curvelet and undecimated wavelet transforms). The
results are reported in Table~\ref{comptab2}.
Figure~\ref{fig_cb2_lenna} displays the noisy image (top left), and
the restored image after denoising by the combined transforms (bottom right). 
Details are
displayed in Figure~\ref{fig_cb2_lenna} bottom left.  
Figure~\ref{fig_cb2_lenna}  bottom  right shows the full residual image, and 
can be compared  to the residual images shown in Figure~\ref{fig_lenna_resi}.
The residual is much better when the combined filtering is applied, and no
feature can be detected any more by eye. This was not the case for either the
wavelet and the curvelet filtering.

\begin{figure}[htb]
\centerline{ 
\vbox{ 
\hbox{
 \psfig{figure=fig_lena_g20.ps,bbllx=2cm,bblly=13.5cm,bburx=13cm,bbury=24.5cm,width=7.5cm,height=7.5cm,clip=}
 \psfig{figure=fig_lenna_cfil_wt_cur.ps,bbllx=2cm,bblly=13.5cm,bburx=13cm,bbury=24.5cm,width=7.5cm,height=7.5cm,clip=}
 }
\hbox{
\psfig{figure=fig_lena_cfil_rid_cur_x185_y342_sub.ps,bbllx=2cm,bblly=13.5cm,bburx=13cm,bbury=24.5cm,width=7.5cm,height=7.5cm,clip=}
\psfig{figure=fig_resi_cfil_wt_cur.ps,bbllx=2cm,bblly=13.5cm,bburx=13cm,bbury=24.5cm,width=7.5cm,height=7.5cm,clip=}
}	
}}
\caption{Noisy image (top left), and filtered image based on the
  combined transform (top right). Bottom left panel shows a detail 
  of the filtered image. The full residual image is displayed on the bottom right.}
\label{fig_cb2_lenna}
\end{figure}

\begin{figure}[htb]
\centerline{
\hbox{
\psfig{figure=fig_ctm_iter.ps,bbllx=2.5cm,bblly=13.cm,bburx=19.5cm,bbury=25cm,width=10cm,height=7cm,clip=}
}}
\caption{PSNR versus the number of iterations.}
\label{fig_iter_cvg}
\end{figure}

Figure~\ref{fig_iter_cvg} displays the PSNR of the solution versus the
number of iterations. In this example, the algorithm is shown to
converge rapidly. From a practical viewpoint only four or five
iterations are truly needed.  Note that the result obtained after a
single iteration is already superior to those available using
methods based on the thresholding of wavelet or curvelet coefficients
alone.

\begin{table*}[htb]
\begin{center}
\begin{tabular}{lcc} \hline \hline
Method                         & PSNR   & PSNR  \\ 
                               & (coeff. $l_1$ norm minim.) & (TV minim.) \\ \hline \hline
 Undecimated wavelet only      & 32.00   &  32.43 \\
 Curvelet only                 & 32.03   &  32.40 \\ 
 Wavelet + curvelet            & 32.72   &  32.77 \\ 
 (Combined filtering)          &         &   \\ \hline \hline
\end{tabular}
\caption{PSNR after filtering the simulated image 
(Lena + Gaussian noise, sigma=20).
In the combined filtering, a curvelet and an undecimated wavelet
transform have been used.}
\vspace{0.5cm}
\label{cur_comptab3}
\end{center}
\end{table*}

Several papers have been recently published, based on the concept
of minimizing the total variation under constraints in the 
wavelet domain \cite{rest:froment01,rest:froment02a,rest:malgouyres02} or in 
the curvelet domain \cite{rest:candes02}. Our combined approach
can be seen as a generalization of these methods.
We carried out a set of experiments in order to estimate (i) if the
total variation is better than the $l_1$ norm of the multiscale
coefficients, and (ii) if the combined approach improves the 
results compared to a single transform based method.

In our example, Gaussian noise with a standard deviation equal
to 20 was added to the classical {\tt Lena} image (512 by 512).
Several methods were used to filter the noisy image:
\begin{enumerate}
\item TV + constraint in the wavelet domain.
\item TV + constraint in the curvelet domain.
\item Wavelet $l_1$ norm minimization + wavelet constraints.
\item Curvelet $l_1$ norm minimization + curvelet constraints.
\item Combined filtering method using multiscale coefficient $l_1$ 
norm minimization.
\item Combined filtering method using TV minimization.
\end{enumerate}
We use the PSNR as an ``objective'' measure of performance.  
The noisy image PSNR is $22.13$. 
PSNR results from the different tested methods 
are reported in Table~\ref{cur_comptab3}.

We observe that combined filtering leads to a significant improvement
when compared to a single transform based method. 
The TV penalization gives better results when a single transform is used,
while it 
seems not to have too much importance for the combined filtering approach.
We will see in the following that the latter
is not true for the deconvolution problem.

% \clearpage
\section{Discussion}
We believe that the denoising experiments presented in this paper are
of very high quality: 
\begin{enumerate}
\item Combined filtering leads to a real
improvement both in terms of PSNR and visual appearance. 
\item The combined approach arguably challenges the eye to distinguish
  structure/features from residual images of real image data (at least
  for the range of noise levels that was considered here). Single
  transforms cannot manage such a feat.  
\end{enumerate}
We also note that the combined reconstruction may tend to be free of
major artifacts which is very much unlike typical thresholding rules.
Although the ease of implementation is clear we did not address the
computational issues associated with our method. In a nutshell, the
algorithms we described require calculating each transform and its
inverse only a limited  number of times. 

In our examples, we constructed a combined transform from linear
transforms (wavelets, ridgelets and curvelets) but our paradigm
extends to any kind of nonlinear transform such as the Pyramidal
Median Transform \cite{starck:book98} or morphological multiscale
transforms \cite{wave:goutsias99b}.


\clearpage
\newpage




\chapter{Deconvolution}
\section{Wavelet and Deconvolution}
\label{wdec}
Consider an image characterized by its intensity
distribution $I$, corresponding to the observation of a
``real image'' $O$ through an optical system. If the
imaging system is linear and shift-invariant, the relation between
the data and the image in the same coordinate frame is a
convolution:
$I(x,y) = (P * O)(x,y) + N(x,y)$,
where
$P$ is the point spread function (PSF) of the imaging system, and $N$
is additive noise. We want to determine $O(x,y)$ knowing $I$ and $P$. This
inverse problem has led to a large amount of work, the main difficulties 
being the existence of: (i) a cut-off frequency of the 
PSF, 
and (ii) the additive noise (see for example \cite{ima:bertero98}).

The wavelet based non-iterative algorithm, 
the wavelet-vaguelette decomposition \cite{rest:donoho95b},
consists of first applying an inverse filtering
($F = P^{-1} * I  +  P^{-1} * N = O + Z$
where $\hat{P}^{-1}(\nu) = \frac{1}{\hat{P}(\nu)}$). 
The noise $Z =  P^{-1} * N$ is not white but remains 
Gaussian. It is amplified when the deconvolution
problem is unstable. 
Then, a wavelet transform is applied on $F$, the wavelet coefficients
are soft or hard thresholded \cite{rest:donoho93_2}, 
and the inverse wavelet transform 
furnishes the solution. 

The method has been refined by 
adapting the wavelet basis to the frequency response of the inverse of $P$
\cite{rest:kalifa99}. This leads to a special basis,
the {\em Mirror Wavelet Basis}.  This basis has a 
time-frequency tiling structure different from the conventional wavelets one.
It isolates the frequency $\nu_s$ where $\hat{P}$ is close to zero, because 
a singularity in $\hat{P}^{-1}(\nu_s)$ influences the noise variance in
the wavelet scale corresponding to the frequency band which includes $\nu_s$.
Because it may not be possible to isolate all singularities, Neelamani
\cite{rest:neelamani99} has advocated a hybrid approach,
and proposes to still use the Fourier domain to restrict excessive noise
amplification. These approaches are fast and competitive compared to linear methods, and
the wavelet thresholding removes the Gibbs oscillations. 
This  presents however several drawbacks:
(i) the first step (division in the Fourier space by the PSF) 
cannot always be done properly,
(ii) the positivity a priori is not used, and
(iii) it is not trivial to consider non-Gaussian noise.

As an alternative, 
several wavelet-based iterative algorithms have been proposed 
 \cite{starck:book98}, especially in the astronomical
domain where the positivity a priori is known to improve 
significantly the result. The simplest method consists of first 
estimating the multiresolution support $M$ (i.e. $M(j,x,y)= 1$ if   
the wavelet transform of the data presents a significant coefficient 
at band $j$ and at pixel position $(x,y)$, and $0$ otherwise), and to
apply the following iterative scheme:
\begin{equation}
O^{n+1} = O^{n} + P^* * \cW^{-1}[M.\cW (I - P * O^n)]
\end{equation}
where $\cW$ is the wavelet transform operator.
At each iteration, information is extracted from the 
residual only at scales and positions defined by the multiresolution support.
$M$ is estimated from the input data and the correct noise modeling
can easily be considered.

 
\section{The Combined Deconvolution Method}
\label{cbdec}
 Similar to the filtering, we expect that the combination 
of different transforms can improve the quality of the result.
The combined approach for the deconvolution leads to two different 
methods. 

If the noise is Gaussian and if the division by the PSF in the Fourier space 
can be carried out properly, then the deconvolution problem becomes a 
filtering problem where the noise is still Gaussian, but not white.
The Combined Filtering Algorithm can then be applied using the curvelet 
transform and the wavelet transform, but by estimating first the 
correct thresholds in the different bands of both transforms.  
Since the mirror wavelet basis is known to produce better results than
the wavelet basis, it is recommended to use it instead of the standard
undecimated wavelet transform.

An iterative deconvolution method 
 is more general and can always be applied. 
Furthermore,
the correct noise modeling can much more easily be taken into account.
This approach consists of detecting, first, all the significant coefficients
with all multiscale transforms used. If we use $K$ transforms
$T_1, \dots, T_K$, we derive $K$ multiresolution supports $M_1, \dots, M_K$
from the input image $I$ using noise modeling.

For instance, in the case of Poisson noise, we apply the Anscombe transform
to the data (i.e. $\cA(I) = 2 \sqrt{I + \frac{3}{8}}$). Then
we detect the significant coefficients with the kth transform $T_k$, 
assuming Gaussian noise with standard deviation equal to 1,
in $T_k \cA(I)$ instead of $T_k I$.  $M_k(j,x,y) = 1$ if a coefficient
in band $j$ at pixel position $(x,y)$ is detected , and  $M_k(j,x,y) = 0$
otherwise. For the band $J$ which corresponds to the smooth array
in transforms such as the wavelet or the curvelet transform, 
we force $M_k(J,x,y) = 1$ for all $(x,y)$.

Following determination of a set of multiresolution supports, 
we propose to solve
the following optimization problem:
\begin{equation}
  \label{eq:dec-min}
  \min \cS(\tilde O), \quad \mbox{subject to} \quad \tilde O \in C,  
\end{equation}
where $\cS$ is an edge preservation penalization term defined by:
\[ \cS(\tilde O) = \int \parallel \nabla \tilde O \parallel_p, \] 
with $p=1.1$.
 $C$ is the set of images $\tilde{O}$ 
which obey the two constraints:
\begin{enumerate}
\item $\tilde{O} \ge 0$ (positivity).
\item $ M_k T_k I = M_k T_k [P *\tilde{O}]$, for all $k$. 
\end{enumerate}
The second constraint
imposes fidelity to the data, or more exactly, 
to the significant coefficients of the data, 
obtained by the different transforms.
Non-significant (i.e. noisy) coefficients are not taken into account, 
preventing any noise amplification in the final algorithm.

The solution is computed by using the 
projected Landweber method \cite{ima:bertero98}:
\begin{eqnarray}
\tilde O^{n+1} = {\cP}_c \left[ 
\tilde O^n + \alpha( P^* * {\bar R}^n - \lambda \frac{\partial \cS(\tilde O)}{\partial O}) \right]
\end{eqnarray}
where ${\cP}_c$ is the projection operator which enforces the positivity
(i.e. set to 0 all negative values).
${\bar R}^n$ is the significant residual which
 is obtained using the following algorithm:
\begin{itemize}
\item Set $I^n_{0} = I^n = P * \tilde O^n$.
\item For $k=1,\dots,K$ do 
$
I^n_k = I^n_{k-1} + T_k^{-1} \left[ M_k (T_k I - T_k I^n_{k-1}) \right] \nonumber
$
\item The significant residual ${\bar R}^n$ is obtained by:
  $ {\bar R}^n = I^n_K - I^n $.
\end{itemize}

$\alpha$ is a convergence parameter and $\lambda$ is the 
regularization hyperparameter. Since the noise is controlled by the
multiscale transforms, the regularization parameter does not   
have the same importance as in standard deconvolution methods.
A much lower value is enough to remove the artifacts relative to the
use of the wavelets and the curvelets. The positivity constraint can
be applied at each iteration.

\begin{figure}[htb]
\vbox{
\centerline{
\hbox{
\psfig{figure=phantom.ps,bbllx=8.2cm,bblly=12.6cm,bburx=12.7cm,bbury=17.1cm,width=7cm,height=7cm,clip=}
\psfig{figure=phantom_p.ps,bbllx=8.2cm,bblly=12.6cm,bburx=12.7cm,bbury=17.1cm,width=7cm,height=7cm,clip=}
}}
\centerline{
\hbox{
\psfig{figure=dec_mr_t24_p_n5_i400_s5.ps,bbllx=8.2cm,bblly=12.6cm,bburx=12.7cm,bbury=17.1cm,width=7cm,height=7cm,clip=}
\psfig{figure=dec_cb_s5_i400.ps,bbllx=8.2cm,bblly=12.6cm,bburx=12.7cm,bbury=17.1cm,width=7cm,height=7cm,clip=}
}}
}
\caption{Top, original image (phantom) and simulated data
(i.e. convolved image plus Poisson noise). Bottom, deconvolved image
by the wavelet based method and the combined approach.}
\label{fig_dec_phantom}
\end{figure}
Figure~\ref{fig_dec_phantom}, top, shows the Logan-Shepp Phantom 
and the simulated data, i.e. original image convolved by a Gaussian 
PSF (full width at half maximum,
FWHM=3.2) and Poisson noise. Figure~\ref{fig_dec_phantom}, 
bottom, shows the deconvolution with  (left) a pure wavelet deconvolution
method (no penalization term) and (right) the combined deconvolution method
(parameter $\lambda = 0.4$).


\clearpage
\newpage


\chapter{Contrast Enhancement}
\section{Introduction}
Because some features are hardly detectable by eye in an image, we
often transform it before display. Histogram equalization is
one the most well-known methods for contrast enhancement.
Such an approach is generally useful for images
with a poor  intensity distribution. Since edges play a fundamental 
role in image understanding, a way to enhance the contrast is to 
enhance the edges. For example, we can add to the original image its Laplacian
($I^{'}= I + \gamma \Delta I$, where $\gamma$ is a parameter). Only
features at the finest scale are enhanced (linearly). For a high 
$\gamma$ value, only the high frequencies are visible.
Multiscale edge enhancement \cite{col:velde99} can be seen 
as a generalization of this approach to all resolution levels.  
 
In color images, objects can exhibit variations in color saturation
with little or no correspondence in luminance variation. 
Several methods have been proposed in the past for color image
enhancement \cite{col:toet92}.
  The retinex concept was introduced by Land \cite{col:land86} as a model
for human color constitancy. 
The single scale retinex (SSR) method \cite{col:jobson97a} consists of
applying the following transform to each band $i$ of the color image:
\begin{eqnarray}
R_i(x,y) = \log( I_i(x,y)) - \log(F(x,y) * I_i(x,y)) 
\end{eqnarray}
where $R_i(x,y)$ is the retinex output, $I_i(x,y)$ is the image 
distribution in the $i$th spectral band, and $F$ is a Gaussian function.
A gain/offset is applied to the retinex output which clips the highest and
lowest signal excursions. This can be done by a k-sigma clipping.
The retinex method is efficient for dynamic range compression, but does not provide good
tonal rendition \cite{col:rahman96}. 
The Multiscale Retinex (MSR) combines several SSR outputs to produce 
a single output image which has both good dynamic range compression and
color constancy, and good tonal rendition \cite{col:jobson97b}.
The MSR can be defined by:
\begin{eqnarray}
R_{MSR_i} = \sum_{j=1}^N w_j R_{i,j}
\end{eqnarray}
with 
\begin{eqnarray}
R_{i,j}(x,y) = \log( I_i(x,y)) - \log(F_j(x,y) * I_i(x,y)) 
\end{eqnarray}
$N$ is the number of scales, $R_{i,j}$ is the $i$th spectral
component of the MSR output, and $w_j$ is the weight associated with
the scale $j$. The Gaussian $F_j$ is given by:
\begin{eqnarray}
F_j(x,y) = K \exp{- {r^2 \over c_j^2}}
\end{eqnarray}
$c_j$ defines the width of the Gaussian.
In \cite{col:jobson97b}, three scales were recommended with $c_j$ values
equal respectively to 15,80,250, and all weights $w_j$ fixed to ${1 \over N}$.
The Multiscale Retinex introduces the concept of multiresolution for
contrast enhancement. Velde \cite{col:velde99} has explicitly introduced
the wavelet transform and has proposed an algorithm which modifies the 
wavelet coefficients in order to amplify faint features.
% \section{Contrast Enhancement using the Wavelet Transform}
% Velde \cite{col:velde99} proposed  to use the wavelet transform
% for edge enhancement.
 The idea is to first transform the image
using the dyadic wavelet transform (two directions per scale).
The gradient $G_{j,k}$ at scale $j$ and at pixel location $k$
is calculated at each scale $j$ from
the wavelet coefficients $w_{j,k}^{(h)}$ and  $w_{j,k}^{(v)}$ relative to
the horizontal and vertical wavelet bands: 
$G_{j,k} = \sqrt{ (w_{j,k}^{(h)})^2 + (w_{j,k}^{(v)})^2}$. Then the two 
wavelet coefficients at scale $j$ and at position $k$   
are multiplied by  $y(G_{j,k})$, where $y$ is defined by:
\begin{eqnarray}
  y(x) & = & ({m \over c})^p \mbox{ if } \mid x \mid < c \nonumber \\
  y(x) & = & ({m \over \mid x \mid })^p  \mbox{ if } c \le \mid x \mid < m \nonumber \\
  y(x) & = & 1  \mbox{ if } \mid x \mid \ge m
\label{eqn_velde}
\end{eqnarray}
\begin{figure}[htb]
\vbox{
\centerline{  
\hbox{
\psfig{figure=fig_velde.ps,bbllx=3cm,bblly=13cm,bburx=20cm,bbury=25.cm,width=6.5cm,height=4cm,clip=}
}}
}
\caption{Enhanced coefficients versus  original coefficients.  
Parameters are m=30, c=3 and p=0.5.
}
\label{fig_velde}
\end{figure}
Three parameters are needed: $p$, $m$ and $c$. 
$p$ determines the degree of non-linearity in the nonlinear rescaling
of the luminance, and must be in $]0,1[$.  
Coefficients larger than $m$ are not modified by the algorithm.
The $c$ parameter corresponds to the noise level.  
Figure~\ref{fig_velde} shows the modified wavelet coefficients versus
the original wavelet coefficients for a given set of parameters 
($m=30$, $c=3$ and $p=0.5$). 
Finally, the enhanced image is obtained by the inverse wavelet transform
from the modified wavelet coefficients. 
For color images, a similar method can be used, but by calculating 
the multiscale gradient $\Gamma_{j,k}$ from the multiscale gradient of 
the three $L$, $u$, $v$ components: $\Gamma_j(i) = \sqrt{ \parallel G_{j,k}^L \parallel^2 + 
                     \parallel G_{j,k}^u \parallel^2 +
		     \parallel G_{j,k}^v \parallel^2 }$.
All wavelet coefficients at scale $j$ and at position $k$ 
are multiplied by $y(\Gamma_{j,k})$,  the enhanced $\tilde L$, $\tilde u$, $\tilde v$ components are reconstructed 
from the modified wavelet coefficients, and 
the ($\tilde L$,$\tilde u$,$\tilde v$) image is transformed into
an RGB image. More details can be found in \cite{col:velde99}.

Wavelet bases present some limitations,
because they are not adapted to the detection of highly anisotropic elements,
such as alignments in an image, or sheets in a cube. 
Recently, other multiscale
systems like ridgelets \cite{Harmnet} and 
curvelets \cite{Curvelets-StMalo,starck:sta01_3}    
which are very different from wavelet-like systems have been developed. 
Curvelets and ridgelets take  the form of basis elements which 
exhibit very high directional sensitivity and are highly anisotropic. 
The curvelet transform uses the ridgelet transform in its digital 
implementation. We first describe the ridgelet and the curvelet 
transform, then we show how contrast enhancement can be obtained 
from the curvelet coefficients.

\section{Contrast Enhancement by the Cur\-ve\-let Trans\-form}

Since the curvelet transform is well-adapted to represent images containing edges,
it is a good candidate for edge enhancement \cite{starck:capri02,starck:sta02_4}. 
Curvelet coefficients
can be modified in order to enhance edges in an image. A function $y_c$
must be defined which modifies the values of the curvelet 
coefficients. It could be a function similar to the one defined for the 
wavelet coefficients \cite{col:velde99} (see equation~\ref{eqn_velde}).
This function presents however the drawback of amplifying the noise (linearly)
as well as the signal of interest. We introduce explicitly the noise standard 
deviation $\sigma$ in the equation:
\begin{eqnarray}
  y_c(x, \sigma) & = & 1 \mbox{ if }   x < c \sigma \nonumber \\
  y_c(x, \sigma) & = & \frac{x-c\sigma}{c \sigma}(\frac{m}{c \sigma})^p + \frac{2c\sigma-x}{c \sigma}  \mbox{ if } x < 2c \sigma \nonumber \\
  y_c(x, \sigma) & = & (\frac{m}{x})^p  \mbox{ if } 2c\sigma \le x < m \nonumber \\
  y_c(x, \sigma) & = & (\frac{m}{x})^s \mbox{ if }x \ge m
\label{eqn_velde_curve}
\end{eqnarray}

\begin{figure}[htb]
\centerline{  
\hbox{
\psfig{figure=fig_velde_mod.ps,bbllx=3cm,bblly=13cm,bburx=20cm,bbury=25.cm,width=6.5cm,height=4cm,clip=}
\psfig{figure=fig_velde_mod_sat.ps,bbllx=3cm,bblly=13cm,bburx=20cm,bbury=25cm,width=6.5cm,height=4cm,clip=}
}}
\caption{Enhanced coefficients versus  original coefficients. Left, 
parameters are m=30,c=0.5,s=0, and p=0.5. Right, 
parameters are m=30,c=0.5,s=0.7,p=0.9.
}
\label{fig_velde_cur_enhance}
\end{figure}

We have fixed $m=c=p=0.5$ and $s=0$ in all our experiments. $p$ determines 
the
degree of non-linearity and $s$ introduces a saturation.
$c$ becomes a normalized parameter, and a $c$ value larger than $3$ 
guaranties that the noise 
will not be amplified. The $m$ parameter can be defined either from
the noise standard deviation ($m = K_m \sigma$) or from the maximum curvelet
coefficient $M_c$ of the relative band ($m = l M_c$, with $l < 1$). The first
choice allows the user to define the coefficients to amplify as a function
of their signal-to-noise ratio, while the second one gives an easy    
and general way to fix the $m$ parameter independently of the range of the
pixel values. Figure~\ref{fig_velde_cur_enhance} shows the curve representing
the enhanced coefficients versus the original coefficients for two
sets of parameters. In the second case, a saturation is added.

The curvelet enhancement method for grayscale images consists of 
the following steps:
\begin{enumerate}
\item Estimate the noise standard deviation $\sigma$ in the input
image $I$.
\item Calculate the curvelet transform of the input image. We get a set 
of bands $w_{j}$, each band $w_j$ contains $N_j$ coefficients 
and corresponds to a given resolution level. 
\item Calculate the noise  standard deviation $\sigma_j$ for each
band $j$ of the curvelet transform (see \cite{starck:sta01_3} more
details on this step).
\item For each band $j$ do
\begin{itemize}
\item Calculate the maximum $M_j$ of the band.
\item Multiply each curvelet coefficient $w_{j,k}$ by $y_c(\mid w_{j,k} \mid ,\sigma_j)$.
\end{itemize}
\item Reconstruct the enhanced image from the modified curvelet coefficients.
\end{enumerate}

For color images, we apply first the curvelet transform on the
three components $L,u,v$. For each cur\-velet coef\-fi\-cient, we  
cal\-cu\-la\-te $e = \sqrt{ c_L^2 + c_u^2 + c_v^2}$, where $(c_L, c_u, c_v)$
are respectively the curvelet coefficients of the three components,
and the mo\-di\-fied coef\-fi\-cients are obtained by:
$(\tilde c_L, \tilde  c_u, \tilde c_v) = 
(y_c(e, \sigma)c_L , y_c(e, \sigma)c_u, y_c(e, \sigma)c_v)$. 

Values in the enhanced components can be larger than the 
authorized upper limit (in general $255$),
and we found it necessary to add a final step to our method, which is
a sigma-clipping saturation.

\section{Examples}
\subsubsection*{Saturn Image}
\begin{figure}[htb]
\centerline{  
\vbox{
\hbox{
\psfig{figure=fig_sat512.ps,bbllx=1.8cm,bblly=12.7cm,bburx=14.5cm,bbury=25.4cm,width=8cm,height=8cm,clip=}
\psfig{figure=fig_sat_contrast_histo.ps,bbllx=1.8cm,bblly=12.7cm,bburx=14.5cm,bbury=25.4cm,width=8cm,height=8cm,clip=}
}
\hbox{
\psfig{figure=fig_sat_contrast_wedge.ps,bbllx=1.8cm,bblly=12.7cm,bburx=14.5cm,bbury=25.4cm,width=8cm,height=8cm,clip=}
\psfig{figure=fig_sat_contrast_cur.ps,bbllx=1.8cm,bblly=12.7cm,bburx=14.5cm,bbury=25.4cm,width=8cm,height=8cm,clip=}
}}
}
\caption{Top, Saturn image and its histogram equalization. Bottom,
enhancement image by the wavelet transform and the curvelet transform.}
\label{fig_saturn_cur_enhance}
\end{figure}

Figure~\ref{fig_saturn_cur_enhance} shows respectively from left to right
and from top to bottom 
the Saturn image, the histogram equalized image, the wavelet multiscale
edge enhanced image and the curvelet multiscale
edge enhanced image (parameters were $s=0$, $p=0.5$, $c=3$, and $l=0.5$). 
The curvelet multiscale edge enhanced image shows clearly better the 
rings and edges of Saturn.

\subsubsection*{Satellite Image}
\begin{figure}[htb]
\centerline{  
\vbox{
\hbox{
\psfig{figure=fig_marseille.ps,bbllx=1.9cm,bblly=12.8cm,bburx=14.6cm,bbury=25.5cm,width=10.cm,height=10cm,clip=}
}
\hbox{
\psfig{figure=fig_cur_marseille.ps,bbllx=1.9cm,bblly=12.8cm,bburx=14.6cm,bbury=25.5cm,width=10.cm,height=10cm,clip=}
}}
}
\caption{Top, grayscale image, and bottom,
curvelet enhanced image.}
\label{fig_marseille_bw_cur_enhance}
\end{figure}
 
\begin{figure}[htb]
\centerline{  
\vbox{
\hbox{
\psfig{figure=kodak140501.ps,bbllx=5.9cm,bblly=8.1cm,bburx=15cm,bbury=21.7cm,width=5.5cm,height=8cm,clip=}
\psfig{figure=kodak140501_ret.ps,bbllx=5.9cm,bblly=8.1cm,bburx=15cm,bbury=21.7cm,width=5.5cm,height=8cm,clip=}
}
\hbox{
\psfig{figure=kodak140501_mret.ps,bbllx=5.9cm,bblly=8.1cm,bburx=15cm,bbury=21.7cm,width=5.5cm,height=8cm,clip=}
\psfig{figure=kodak140501_cur.ps,bbllx=5.9cm,bblly=8.1cm,bburx=15cm,bbury=21.7cm,width=5.5cm,height=8cm,clip=}
}}
}
\caption{Top, color image (Kodak picture of the day 14/05/02) and retinex
method. Bottom, multiscale retinex method and multiscale edge enhancement.}
\label{fig_kodak_col_wt_enhance}
\end{figure}

\begin{figure}[htb]
\centerline{  
\vbox{
\hbox{
\psfig{figure=K111201.ps,bbllx=4.3cm,bblly=10.8cm,bburx=16.7cm,bbury=19.1cm,width=12.5cm,height=8.2cm,clip=}
}
\hbox{
\psfig{figure=K111201_cur.ps,bbllx=4.3cm,bblly=10.8cm,bburx=16.7cm,bbury=19.1cm,width=12.5cm,height=8.2cm,clip=}
}}
}
\caption{Left, color image (Kodak picture of the day 11/12/01), and right,
curvelet enhanced image.}
\label{fig_kodak2_col_cur_enhance}
\end{figure}

Figure~\ref{fig_marseille_bw_cur_enhance}  
shows the results for the enhancement of a grayscale satellite image, and
Figure~\ref{fig_kodak_col_wt_enhance}  
shows the results for the enhancement of a color image (Kodak image of
the day 14/05/01) by the retinex,
the multiscale retinex and the curvelet multiscale edge enhancement methods.
 Figure~\ref{fig_kodak2_col_cur_enhance} 
shows the results for the enhancement of a color image (Kodak image of
the day 11/12/01).

\section{Discussion}
A number of properties, respected by the curvelet filtering 
described here, are important for contrast stretching:
\begin{enumerate}
\item Noise must not be amplified in enhancing edges.
\item Colors should not be unduly modified.  In multiscale retinex,
for example, a tendancy towards increased grayness is seen.  This is 
not the case using curvelets.
\item It is very advantageous if block effects do not occur. 
Block overlapping is usually not necessary in curvelet-based contrast
enhancement, unlike in the case of noise filtering.  
\end{enumerate}
% A range of further examples can be seen at \\
% http://www-stat.stanford.edu/$\sim$jstarck/contrast.html.

% \clearpage
% \newpage





\chapter{Morphological Component Analysis}
\label{sect_ctm}
\section{Introduction}
The content of an image is often complex, and there is not a single
transform which is optimal to represent all the contained features. For
example, the Fourier transform better represents some textures, while
the wavelet transform better represents singularities. Even if we limit
our class of transforms to the wavelet one, decision have to
be taken between an isotropic wavelet transform which produce good 
results for isotropic objects (such stars and galaxies in astronomical 
images, cells in biological images, etc), or an orthogonal wavelet 
transform, which is better for images with edges. 
This has motivated
the development of different methods \cite{wave:donoho98,wave:meyer98,cur:huo99}, 
and the two most frequently discussed approaches are the Matching Pursuit (MP)
\cite{wave:mallat93} and the Basis pursuit (BP) \cite{wave:donoho98}.
A dictionary ${\cal D}$ being defined as a collection of waveforms 
$(\varphi_{\gamma})_{\gamma \in \Gamma}$,
the general principe consists in representing a signal $s$ as a ``sparse''
linear combination  of a small number of basis such that:
\begin{eqnarray}
 s = \sum_{\gamma} a_{\gamma} \varphi_{\gamma}
\end{eqnarray}
or an approximate decomposition
\begin{eqnarray}
 s = \sum_{i=1}^m a_{\gamma_i} \varphi_{\gamma_i} + R^{(m)} .
\end{eqnarray}
 
Matching pursuit \cite{wave:mallat93,ima:mallat98} method (MP) uses a greedy
algorithm which adaptively refines the signal approximation with an
iterative procedure:
\begin{itemize}
\item Set $s^0 = 0$ and $R^0 = 0$.
\item Find the element $\alpha_k \varphi_{\gamma_k}$ which best correlates with the 
residual.
\item Update $s$ and $R$:
\begin{eqnarray}
s^{k+1} & = & s^k + \alpha_k \varphi_{\gamma_k} \nonumber \\
R^{k+1} & = & s -  s^k .
\end{eqnarray}
\end{itemize}
In case of non orthogonal dictionaries, it has been shown \cite{wave:donoho98}
 that MP may
spend most of the time correcting mistakes made in the first few terms,
and therefore is suboptimal in term of sparsity.

\bigskip
Basis pursuit method  \cite{wave:donoho98} (BP) is a global procedure which
synthesizes an approximation $\tilde{s}$ to $s$ by minimizing a
functional of the type
\begin{equation}
\|s - \tilde{s}\|_{\ell_2} ^2 + \lambda \cdot \|\alpha\|_{\ell_1}, 
\quad \tilde{s} = \Phi \alpha.  
\label{eqn_mp}
\end{equation}
Between all possible solutions, the chosen one has the minimum $l^1$ norm.
This choice of $l_1$ norm is very important. A $l_2$ norm, as 
used in the method of frames \cite{wave:daube88b},
 does not preserve the sparsity \cite{wave:donoho98}. 
 
In many cases, BP or MP synthesis algorithms are computationally very
expensive. We present in the following an alternative approach, that we call 
{\em Morphological Component Analysis} (MCA), which combines  
 the different available transforms 
in order to benefit of the advantages of each of them. 

\section{The Combined Transformation}

Depending on the content of the data, several transforms 
can be combined in order to get an optimal representation of all 
features contained in our data set. In addition to the ridgelet
and the curvelet transform, we may want to use the \`a trous algorithm
which is very well suited to astronomical data, or the undecimated
wavelet transform which is commonly used in the signal processing domain.

Other transform such wavelet packets, the Fourier transform, 
the Pyramidal median transform \cite{starck:book98}, 
or other multiscale morphological transforms,
could also be considered. However, we found that in practice,
these four transforms (i.e. curvelet, ridgelet, \`a trous algorithm,
and undecimated wavelet transform)
furnishes a very large panel of waveforms
which is generally large enough to well 
represents all features contained in the data.

In general, suppose that we are given $K$ linear transforms $T_1,
\ldots, T_K$ and let $\alpha_k$ be the coefficient sequence of an
object $x$ after applying the transform $T_k$, i.e. $\alpha_k = T_k
x$. We will suppose that for each transform $T_k$ we have available a
reconstruction rule that we will denote by $T^{-1}_k$ although this is
clearly an abuse of notations.

Therefore, we search a vector  
${\bf \alpha} = {\alpha_1, \dots, \alpha_{K}}$ such that
\begin{equation}
s =  \Phi {\bf \alpha}
\end{equation}
where $\Phi {\bf \alpha} = \sum_{k=1}^K T_k^{-1} \alpha_k$.
As our dictionary is overcomplete, there is an infinity of vectors
verifing this condition, and we need to solve the following optimization 
problem:
\begin{equation}
min \parallel s - \phi {\bf \alpha} \parallel^2 + {\cal C}(\bf \alpha)  
\end{equation}
where ${\cal C}$ is a penalty term. We easily see that chosing
${\cal C}({\bf \alpha}) =  \parallel {\bf \alpha} \parallel_{l_1}$ leads to the
BP method, where the dictionary ${\cal D}$ is only composed of the
basis elements of the chosen transforms.
 
Two iterative methods, {\em soft-MCA} and {\em hard-MCA}, allowing us 
to realize such a combined transform, are described in this section.

\section{Soft-MCA}
\label{sect_l1}
% $l^1$ optimization by soft thresholding

Noting $T_1, ..., T_{K}$ the $K$ transform operators, 
a solution ${\bf \alpha}$  is obtained by minimizing a functional of the form:
\begin{eqnarray}
J({\bf \alpha}) = \parallel s - \sum_{k=1}^{K} T_k^{-1} \alpha_k  \parallel_2^2 + \lambda \sum_k \parallel \alpha_k \parallel_1
\label{eqn_min_l1}
\end{eqnarray}
where $s$ is the original signal, and $\alpha_k$ are the coefficients 
obtained with the transform $T_k$.

An simple algorithm to achieve such an solution is \cite{starck:spie01b,starck:sta02_3}:
\begin{enumerate}
\item Initialize $L_{\max}$, the number of iterations $N_i$, 
$\lambda = L_{\max}$, and $\delta_{\lambda} = \frac{L_{\max}}{N_i}$.
\item While $\lambda >= 0$ do
\item For k = 1, .., $K$ do
\begin{itemize}
\item Calculate the residual $R = s - \sum_k T_k^{-1} \alpha_k$.
\item Calculate the  transform $T_k$ of the residual:
$r_k = T_k R$.
\item Add the residual to  $\alpha_k$:  $\alpha_k = \alpha_k + r_k$.
\item Soft threshold the coefficient $\alpha_k$ with the $\lambda$ threshold.
\end{itemize}
\item $\lambda = \lambda - \delta$, and goto 2.
\end{enumerate}

Figure~\ref{fig_cb1_synt} illustrates the result in the case where the
input image contains only lines and Gaussians. In this experiment, we have 
initialized  $L_{\max}$ to $20$, and $\delta$ to $2$ (10 iterations).
Two transform operators
were used, the \`a trous wavelet transform and the ridgelet transform. The 
first is well adapted to the detection of Gaussian due to the isotropy of
the wavelet function~\cite{starck:book98}, while the second is optimal
to represent lines \cite{cur:candes99_1}. Figure~\ref{fig_cb1_synt} top,
bottom left, and bottom right represents respectively the 
original image, the reconstructed image from the \`a trous wavelet
coefficient, and the reconstructed image from the ridgelet
coefficient. The addition of both reconstructed images reproduces the
original one. 

\begin{figure}[htb]
\vbox{
\centerline{
\hbox{
\psfig{figure=fig_cb2_line_g.ps,bbllx=2cm,bblly=13.5cm,bburx=13cm,bbury=24.5cm,width=5.5cm,height=5.5cm,clip=}}
}
\centerline{
\hbox{
\psfig{figure=fig_cb2_line_g_atrou.ps,bbllx=2cm,bblly=13.5cm,bburx=13cm,bbury=24.5cm,width=5.5cm,height=5.5cm,clip=}
\psfig{figure=fig_cb2_line_g_rid.ps,bbllx=2cm,bblly=13.5cm,bburx=13cm,bbury=24.5cm,width=5.5cm,height=5.5cm,clip=}
}}}
\caption{Top, original image containing lines and gaussians. Botton left,
reconstructed image for the \`a trous wavelet coefficient, bottom right,
reconstructed image from the ridgelet coefficients.}
\label{fig_cb1_synt}
\end{figure}

In some specific cases where the data are sparse in all bases, 
it has been shown \cite{cur:huo99,cur:donoho01}  
that the solution is identical to the solution when using a 
$\parallel . \parallel_0$ penalty term. This is however generally not the 
case. 
The problem we met in image restoration applications,
when minimizing equation~\ref{eqn_min_l1}, is 
that both the signal and noise are split into the bases. The way the noise
is distributed in the coefficients $\alpha_k$ is not known, and leads to the problem
that we do not know at which level we should threshold the coefficients. Using
the threshold we would have used with a single transform 
makes a strong over-filtering of the data. Using the $l^1$ optimization 
for data restoration implies to first study how the noise is 
distributed in the coefficients. The hard-MCA method does not present
this drawback.

\section{Hard-MCA}
\label{sect_l0}
%  $l^0$ optimization by hard thresholding
The following algorithm consists in hard thresholding the residual 
successively on the different bases \cite{starck:spie01b,starck:sta02_3}. 
\begin{enumerate}
\item For noise filtering, estimate the noise standard deviation $\sigma$,
and set $L_{\min} = k_{\sigma}$. 
Otherwise, set $\sigma=1$ and $L_{\min} = 0$.
\item Initialize $L_{\max}$, the number of iterations $N_i$, 
$\lambda = L_{\max}$ and $\delta_{\lambda} = \frac{L_{\max} - L_{\min}}{N_i}$.
\item Set all coefficients $\alpha_k$ to 0.
\item While $\lambda >= L_{\min}$ do
\item for k = 1, .., $K$ do
\begin{itemize}
\item Calculate the residual $R = s - \sum_k T_k^{-1} \alpha_k$.
\item Calculate the transform $T_k$ of the residual:
$r_k = T_k R$.
\item For all coefficients $\alpha_{k,i}$ do
\begin{itemize}
\item  Update the coefficients:  
 if $\alpha_{k,i} \ne 0$ or $\mid r_{k,i}  \mid > \lambda \sigma$ 
then $\alpha_{k,i}  =  \alpha_{k,i} + r_{k,i}$.
\end{itemize}
\end{itemize}
\item $\lambda = \lambda - \delta_{\lambda} $, and goto 5.
\end{enumerate}
For an exact representation of the data, $k_{\sigma}$ must be set to 0.
Choosing $k_{\sigma} > 0$ introduces a filtering. If a single transform is used,
it corresponds to the standard  $k$-sigma hard thresholding.

It seems that starting with a high enough $L_{\max}$ and a high
number of iterations would lead to the $l^0$ optimization solution,
but this remains to be proved.


\section{Experiments}

\subsubsection{Experiment 1: Infrared Gemini Data}
\begin{figure}[htb]
\centerline{
\vbox{
\hbox{
\psfig{figure=fig_gemini.ps,bbllx=1.8cm,bblly=12.5cm,bburx=14.8cm,bbury=25.5cm,width=5cm,height=5cm,clip=}
\psfig{figure=fig_gem_cb_rid.ps,bbllx=1.8cm,bblly=12.5cm,bburx=14.8cm,bbury=25.5cm,width=5cm,height=5cm,clip=}
\psfig{figure=fig_gem_cb_cur.ps,bbllx=1.8cm,bblly=12.5cm,bburx=14.8cm,bbury=25.5cm,width=5cm,height=5cm,clip=}
}
\hbox{
\psfig{figure=fig_gem_cb_atrou.ps,bbllx=1.8cm,bblly=12.5cm,bburx=14.8cm,bbury=25.5cm,width=5cm,height=5cm,clip=}
\psfig{figure=fig_gem_cb_resi_cur_rid_atrou.ps,bbllx=1.8cm,bblly=12.5cm,bburx=14.8cm,bbury=25.5cm,width=5cm,height=5cm,clip=}
\psfig{figure=fig_gem_cb_resi_cur_rid.ps,bbllx=1.8cm,bblly=12.5cm,bburx=14.8cm,bbury=25.5cm,width=5cm,height=5cm,clip=}
}
}}
\caption{Upper left, galaxy SBS 0335-052 (10 $\mu$m), upper middle, upper middle,
and bottom left, reconstruction respectively from the ridgelet, the curvelet
and wavelet coefficients. Bottom middle, residual image. Bottom right, 
artifact free image.}
\label{fig_ctm_gemini1}
\end{figure}
$ $ 
\begin{figure}[htb]
\centerline{
\vbox{
\hbox{
\psfig{figure=fig_gemini2_bw.ps,bbllx=1.8cm,bblly=12.5cm,bburx=14.8cm,bbury=25.5cm,width=7cm,height=7cm,clip=}
\psfig{figure=fig_gem2_cb_cur_rid_bw.ps,bbllx=1.8cm,bblly=12.5cm,bburx=14.8cm,bbury=25.5cm,width=7cm,height=7cm,clip=}
}
\hbox{
\psfig{figure=fig_gem2_cb_atrou_bw.ps,bbllx=1.8cm,bblly=12.5cm,bburx=14.8cm,bbury=25.5cm,width=7cm,height=7cm,clip=}
\psfig{figure=fig_gem2_cb_resi_cur_rid_atrou_bw.ps,bbllx=1.8cm,bblly=12.5cm,bburx=14.8cm,bbury=25.5cm,width=7cm,height=7cm,clip=}
}
}}
\caption{Upper left, galaxy SBS 0335-052 (20 $\mu$m), upper right, addition
of the reconstructed images from both the ridgelet and the curvelet coefficients,
bottom left, reconstruction from the wavelet coefficients, and bottom right,
residual image.}
\label{fig_ctm_gemini2}
\end{figure}
\clearpage
Fig.~\ref{fig_ctm_gemini1} upper left shows
a compact blue galaxy located at 53 Mpc. The data have been
obtained on ground with the GEMINI-OSCIR instrument at $10$ $\mu$m. 
The pixel field of
view is $0.089^{\prime\prime}$/pix, and the  source was observed during 1500s.
The data are contaminated by a noise and a stripping artifact due to the 
instrument electronic. The same kind of artifact pattern were observed with
the ISOCAM instrument \cite{starck:sta99_1}.

This image, noted $D_{10}$, has been decomposed using wavelets, ridgelets, and curvelets.
Fig.~\ref{fig_ctm_gemini1} upper middle,  upper  right, and bottom left
show the three images $R_{10}$, $C_{10}$, $W_{10}$
reconstructed respectively from the ridgelets, the curvelets, and the wavelets.
Image in Fig.~\ref{fig_ctm_gemini1} bottom middle shows the residual, i.e.
$e_{10} = D_{10} - (R_{10} + C_{10} + W_{10})$. Another interesting
image is the artifact free one, obtained by subtracting $R_{10}$ and
$C_{10}$ from the input data (see Fig.~\ref{fig_ctm_gemini1} bottom right).
The galaxy has well been detected in the wavelet space, while all stripping
artifact have been capted by the ridgelets and curvelets.

Fig.~\ref{fig_ctm_gemini2} upper left shows the same galaxy, but at
$20$ $\mu$m. We have applied the same decomposition on $D_{20}$. 
Fig.~\ref{fig_ctm_gemini2} upper right shows the coadded 
image $R_{20} + C_{20}$, and we can see bottom left and right the 
wavelet reconstruction $W_{20}$ and the residudal
$e_{20} = D_{20} - (R_{20} + C_{20} + W_{20})$.

\subsubsection{Experiment 2: A370}
 
\begin{figure}[htb]
\centerline{
\vbox{
\hbox{
\psfig{figure=a370.ps,bbllx=1.5cm,bblly=6cm,bburx=19.5cm,bbury=24cm,width=7cm,height=7cm,clip=}
\psfig{figure=a370_ridcur.ps,bbllx=1.5cm,bblly=6cm,bburx=19.5cm,bbury=24cm,width=7cm,height=7cm,clip=}
}
\hbox{
\psfig{figure=a370_atrou.ps,bbllx=1.5cm,bblly=6cm,bburx=19.5cm,bbury=24cm,width=7cm,height=7cm,clip=}
\psfig{figure=a370_comb.ps,bbllx=1.5cm,bblly=6cm,bburx=19.5cm,bbury=24cm,width=7cm,height=7cm,clip=}
}
}}
\caption{Top left, HST image of A370, top right coadded image from the
reconstructions from the ridgelet and the curvelet coefficients, bottom
left  reconstruction from  the \`a trous wavelet coefficients, 
and bottom right addition of the three reconstructed images. }
\label{fig_a370}
\end{figure}

Figure~\ref{fig_a370} upper left shows the HST A370 image. It contains many 
anisotropic features such the gravitationnal arc, and the arclets. The
image has been decomposed using three transforms: the ridgelet transform, 
the curvelet transform, and the \`a trous wavelet transform. Three images
have then been reconstructed from the coefficients of the three basis.
Figure~\ref{fig_a370} upper right shows the coaddition of the ridgelet 
and curvelet reconstructed images. The \`a trous reconstructed image
is displayed in Figure~\ref{fig_a370} lower left, and the coaddition 
of the three images can be seen in Figure~\ref{fig_a370} lower right.
The gravitational arc and the arclets are all represented in the 
ridgelet and the curvelet basis, while all isotropic features are 
better represented in the wavelet basis.

We can see that this Morphological Component Analysis (MCA) allows
us to separate automatically features in an image which have different
morphological aspects. It is very different from other techniques such as
Principal Component Analysis or 
Independent Component Analysis \cite{mc:cardoso98} where the separation
is performed via statistical properties.


\clearpage
\newpage


\chapter{\projcur Programs}
% \section{Program}
\section{Ridgelet Transform}
\subsection{im\_radon}
\index{im\_radon}
Program {\em im\_radon} makes an (inverse-) Radon transform of a square
$n \times n$ image.
The output file which contains the transformation has a 
suffix, .rad. If the output file name
given by the user does not contain this suffix, it is automatically
added. The ``.rad'' file is a FITS format file, and can be manipulated by
any package implementing the FITS format, or can be converted to another
format using the {\em im\_convert} program.
For the two first Radon transform methods, the user can change the
number of directions and the resolution. For other methods, the 
number of directions and the resolution are fixed, and the
x and y options are not valid. Options f, w and s allow the user to perform
a filtering-backprojection. They are valid only when the selected Radon
method is the second one. ``w'' fixes the width of the filter and the
``s'' is the sigma parameter of the Gaussian filter.
{\bf
\begin{center}
 USAGE:  im\_radon  options image\_in trans\_out
\end{center}}
where options are:
\begin{itemize}
\baselineskip=0.4truecm
\itemsep=0.1truecm
\item {\bf [-m type\_of\_radon\_method]}  
{\small 
\begin{enumerate}
\baselineskip=0.4truecm
\itemsep=0.1truecm
\item  Radon transform (resp.\ backprojection) in spatial domain. \\
       By default, the output image is a $2n \times n$ image.
\item  Radon projection in spatial domain and reconstruction in Fourier domain.
       By default, the output image is a $2n \times n$ image.
       The reconstruction is available only for image with
       a size $n$ being a power of 2.
\item  Radon transformation and reconstruction in Fourier space 
       (i.e.\ Linogram). \\
       The output is a 
       $2n \times n$ image. The number of rows is multiplied by two.
\item  Finite Radon Transform. \\
       The output image is a $(n+1) \times n$ image. The input image size $n$
       must be a prime number.
\item  Slant Stack Radon transform. \\
       The output image has twice the number of rows and twice the number
       of columns of the input image.
       The output is a $2n \times 2n$ image.
       The reconstruction is not available with this transform.
\end{enumerate}}
Default is Radon transformation and reconstruction in Fourier space.

\item {\bf [-y OutputLineNumber]} \\
For the RADON transform, OutputLineNumber = number of projection,
 and default is twice the number of input image rows. Only valid for Radon
methods 1 and 2.\\
% For the inverse RADON transform, OutputLineNumber = number of lines,
% and default is the input image column number.

\item {\bf [-x OutputColumnNumber]} \\
For the RADON transform, OutputLineNumber = number of pixels per projection,
and default is the input image column number. Only valid for Radon
methods 1 and 2.\\
% For the inv. RADON transform, OutputLineNumber = number of column,
% and default is the input image column number.

\item {\bf [-r]} \\
Inverse Radon transform.
\item {\bf [-f]} \\
Filter each scan of the Radon transform. Only valid for Radon
method 2.
\item {\bf [-w FilterWidth]} \\
Filter width. Only valid for Radon
method 2. Default is 100. 
\item {\bf [-s SigmaParameter} \\
 Sigma parameter for the filtering. Only valid for Radon
method 2. Default is 10.
\item {\bf [-v]} \\
Verbose. Default is no
\end{itemize}

\subsubsection*{Examples:}
\begin{itemize}
\item im\_radon image.fits trans\\
Apply the Radon transform to an image.
\item im\_info -r trans.rad rec\\
 Reconstruct an image from its Radon transform.
\end{itemize}

\subsection{rid\_trans}
\index{rid\_trans}
Program {\em rid\_trans} makes the  ridgelet transform 
(and the inverse when -r option is set).  
The output file which contains the transformation has a 
suffix, .rid. If the output file name
given by the user does not contain this suffix, it is automatically
added. The ``.rid'' file is a FITS format file, and can be manipulated by
any package implementing the FITS format, or can be converted to another
format using the {\em im\_convert} program.
The default transform is the second one 
(RectoPolar Ridgelets using a  FFT based WT). 
The two first transform are based on the RectoPolar
(i.e. linogram) radon transform, but the first applies a standard 
bi-orthogonal wavelet transform (WT) on the Radon image rows, while the
second uses the Fourier-based WT which introduces a redundancy of 2.
For an $n \times n$ image, the output has $2n$ lines and $n$ column 
with the first transform, and is a $2n \times 2n$ image for the second.
When the overlapping is set, the size is doubled in each direction.

{\bf
\begin{center}
 USAGE: rid\_trans options image\_in trans\_out
\end{center}}
where options are:

\begin{itemize}
\baselineskip=0.4truecm
\itemsep=0.1truecm
\item {\bf [-t type\_of\_ridgelet]}  
\begin{enumerate}
\baselineskip=0.4truecm
\itemsep=0.1truecm
\item RectoPolar Ridgelet Transform using a standard bi-orthogonal WT.
\item RectoPolar Ridgelet Transform using a FFT based Pyramidal WT.
\item Finite ridgelet transform.
\end{enumerate}
Default is 2.
\item {\bf [-n number\_of\_scale]} \\
 Number of scales used in the wavelet transform.
 Default is automatically calculated.
\item {\bf [-b BlockSize]} \\
Block Size. Default is image size.
\item {\bf [-i]} \\
Print statistical information about each band. Default is no. 
\item {\bf [-O]} \\
 No block overlapping. Default is no. When this option is set, the 
 number of rows and columns is multiplied by two. 
\item {\bf [-r]} \\
Inverse Ridgelet transform.
\item {\bf [-x]} \\
 Write all bands separately as images in the FITS format with prefix 'band\_j' 
(j being the band number).
\item {\bf [-v]} \\
Verbose. Default is no
\end{itemize}

\subsubsection*{Examples:}
\begin{itemize}
\item rid\_transform image.fits trans\\
Apply the Ridgelet transform to an image.
\item rid\_transform -r trans.rid rec\\
 Reconstruct an image from its Ridgelet transform.
\end{itemize}

\subsection{rid\_stat}
\index{rid\_stat}
Program {\em rid\_stat} makes the  ridgelet transform, and 
gives statistical information on the ridgelet coefficients.
At each scale, it caculates the standard deviation, the skewness,
the kurtosis, the minimum, and the maximum. The output file is a 
fits file containing a two-dimensional array $T[J-1,5]$ ($J$ being the
number of scales), with the following syntax:
\begin{itemize}
\baselineskip=0.4truecm
\itemsep=0.1truecm
\item $T[j,0] = $ standard deviation of the jth ridgelet band.
\item $T[j,1] = $ skewness of the jth ridgelet band.
\item $T[j,2] = $ kurtosis of the jth ridgelet band.
\item $T[j,3] = $ minimum of the jth ridgelet band.
\item $T[j,4] = $ maximum of the jth ridgelet band.
\end{itemize}
The last ridgelet scale is not used.
If the ``-A'' option is set, these statistics are calculated only for 
the ridgelet coefficients relative the specified angle.
{\bf
\begin{center}
 USAGE: rid\_stat options image\_in trans\_out
\end{center}}
where options are:

\begin{itemize}
\baselineskip=0.4truecm
\itemsep=0.1truecm
\item {\bf [-t type\_of\_ridgelet]}  
\begin{enumerate}
\baselineskip=0.4truecm
\itemsep=0.1truecm
\item RectoPolar Ridgelet Transform using a standard bi-orthogonal WT.
\item RectoPolar Ridgelet Transform using a FFT based Pyramidal WT.
\item Finite ridgelet transform.
\end{enumerate}
Default is 2.
\item {\bf [-n number\_of\_scale]} \\
 Number of scales used in the wavelet transform.
 Default is automatically calculated.
\item {\bf [-b BlockSize]} \\
Block Size. Default is 16.
\item {\bf [-O]} \\
   Use overlapping block. Default is no.
\item {\bf [-A Angle]} \\
 Statistics for a given angle. The value must be given  in degrees.
 Default is no, statistics are calculated from all coefficients.
\item {\bf [-v]} \\
Verbose. Default is no
\end{itemize}

\subsubsection*{Example:}
\begin{itemize}
\item rid\_stat -v  image.fits tabstat\\
\end{itemize}

\subsection{rid\_filter}

Program {\em rid\_filter} filters an image using the ridgelet transform.
\begin{center}
 USAGE:  rid\_filter options image\_in imag\_out
\end{center}
where options are 
\begin{itemize}
\baselineskip=0.4truecm
\itemsep=0.1truecm
\item {\bf [-t type\_of\_ridgelet]}  
\begin{enumerate}
\baselineskip=0.4truecm
\itemsep=0.1truecm
\item RectoPolar Ridgelet Transform using a standard bi-orthogonal WT.
\item RectoPolar Ridgelet Transform using a FFT based pyramidal WT.
\item Finite ridgelet transform.
\end{enumerate}
Default is 2.
\item {\bf [-n number\_of\_scale]} \\
 Number of scales used in the wavelet transform.
 Default is automatically calculated.
\item {\bf [-h]} \\
Apply the ridgelet transform only on the high frequencies.
Default is no.

\item {\bf [-b BlockSize]} \\
Block Size. Default is image size.

\item {\bf [-F FirstDetectionScale]} \\
 First detection scale. Default is 1. 

% \item {\bf [-i NbrIter]}  \\
%  Number of iteration for the constraint reconstruction.
%   Default is no. 
% \item {\bf [-G RegulParam]}  \\
%   Regularization parameter for the constraint reconstruction.
%  Default is 0.2.
% \item {\bf [-C ConvergParam]}  \\
%  Convergence parameter. Default is 1.

\item {\bf [-s Nsigma]} \\
False detection rate. The false detection rate for a detection is given
\begin{eqnarray}
\epsilon =  \mbox{erfc}( NSigma / \sqrt{2})
\end{eqnarray}
{\em Nsigma} parameter allows us to express the false detection rate
even if it is not Gaussian noise. \\
Default is 3.

\item {\bf [-g sigma]} \\
Gaussian noise: sigma = noise standard deviation.  \\
 Default is automatically estimated.

\item {\bf [-p]} \\
Poisson noise.

\item {\bf [-O]}  \\
Do not apply block overlapping. By default, block overlapping is used.

\item {\bf [-v]} \\
Verbose.
\end{itemize}
\subsubsection*{Examples:}
\begin{itemize}
\item rid\_filter  image.fits fima\\
Filter an image using all default options.
\item rid\_filter -h -s5 image.fits fima\\
Five sigma filtering, filtering only the high frequencies.
\end{itemize}


\section{Curvelet Transform}
\subsection{cur\_trans}
\index{cur\_trans}
Program {\em cur\_trans} determines the  curvelet transform 
(and the inverse when -r option is set).  
The output file which contains the transformation has a 
suffix, .cur. If the output file name
given by the user does not contain this suffix, it is automatically
added. The ``.cur'' file is a 3D FITS format file, and can be manipulated by
any package implementing the FITS format.
The curvelet transform uses the ridgelet transform, and the default 
ridgelet transform is the RectoPolar one with a FFT based pyramidal WT.

{\bf
\begin{center}
 USAGE: cur\_trans options image\_in trans\_out
\end{center}}
where options are:
\begin{itemize}
\baselineskip=0.4truecm
\itemsep=0.1truecm
\item {\bf [-t type\_of\_ridgelet]}  
\begin{enumerate}
\baselineskip=0.4truecm
\item RectoPolar Ridgelet Transform using a standard bi-orthogonal WT.
\item RectoPolar Ridgelet Transform using a FFT based pyramidal WT.
\item Finite ridgelet transform.
\end{enumerate}
Default is 2.
\item {\bf [-n number\_of\_scale]} \\
 Number of scales used in the 2D wavelet transform.
 Default is 4. 
\item {\bf [-N number\_of\_scale]} \\
 Number of scales used in the ridgelet transform.
 Default is automatically calculated.
\item {\bf [-b BlockSize]}  \\
Block Size. Default is 16.
\item {\bf [-r]}  \\
Inverse Curvelet transform.
\item {\bf [-i]}  \\
Print statistical information about each band. Default is no. 
\item {\bf [-O]}  \\
 Block overlapping. Default is no. 
\item {\bf [-x]} \\
 Write all bands separately as images in the FITS format with prefix 'band\_j' 
(j being the band number).
\item {\bf [-v]} \\
Verbose. Default is no.
\end{itemize}

\subsubsection*{Examples:}
\begin{itemize}
\item cur\_trans -i image.fits trans\\
Curvelet transform of an image.
\item cur\_trans -r  trans.cur rec\\
Image reconstruction from its curvelet transform.
\end{itemize}


\subsection{cur\_stat}
\index{cur\_stat}
Program {\em cur\_stat} determines the  curvelet transform, and 
gives statistical information on the curvelet coefficients.
At each scale, it caculates the standard deviation, the skewness,
the kurtosis, the minimum, and the maximum. The output file is a 
FITS file containing a two-dimensional array $T[J-1,5]$ ($J$ being the
number of bands), with the following syntax:
\begin{itemize}
\baselineskip=0.4truecm
\itemsep=0.1truecm
\item $T[j,0] = $ standard deviation of the jth ridgelet band.
\item $T[j,1] = $ skewness of the jth ridgelet band.
\item $T[j,2] = $ kurtosis of the jth ridgelet band.
\item $T[j,3] = $ minimum of the jth ridgelet band.
\item $T[j,4] = $ maximum of the jth ridgelet band.
\end{itemize}
{\bf
\begin{center}
 USAGE: cur\_stat options image\_in trans\_out
\end{center}}
where options are:
\begin{itemize}
\baselineskip=0.4truecm
\item {\bf [-n number\_of\_scale]} \\
 Number of scales used in the wavelet transform.
 Default is automatically calculated.
\item {\bf [-b BlockSize]} \\
Block Size. Default is 16.
\item {\bf [-O]} \\
   Use overlapping block. Default is no.
\item {\bf [-v]} \\
Verbose. Default is no
\end{itemize}

\subsubsection*{Example:}
\begin{itemize}
\item cur\_stat -v  image.fits tabstat\\
\end{itemize}


\subsection{cur\_filter}
\index{cur\_filter}

Program {\em cur\_filter} filters an image using the curvelet transform.
\begin{center}
 USAGE:  cur\_filter options image\_in imag\_out
\end{center}
where options are 
\begin{itemize}
\baselineskip=0.4truecm
\itemsep=0.1truecm
\item {\bf [-t type\_of\_ridgelet]} 
\begin{enumerate}
\baselineskip=0.4truecm
\itemsep=0.1truecm
\item RectoPolar Ridgelet Transform using a standard bi-orthogonal WT.
\item RectoPolar Ridgelet Transform using a FFT based Pyramidal WT.
\item Finite ridgelet transform.
\end{enumerate}
Default is 2.
\item {\bf [-n number\_of\_scale]} \\
 Number of scales used in the 2D wavelet transform.
 Default is 4. 

\item {\bf [-N number\_of\_scale]} \\
 Number of scales used in the ridgelet transform.
 Default is automatically calculated.

\item {\bf [-b BlockSize]}  \\
Block Size. Default is 16.

\item {\bf [-g sigma]} \\
Gaussian noise: sigma = noise standard deviation.  \\
 Default is automatically estimated.

\item {\bf [-s Nsigma]} \\
False detection rate. \\
Default is 3.

\item {\bf [-O]}  \\
Do not apply block overlapping. By default, block overlapping is used.

\item {\bf [-P]}  \\
 Supress the positivity constraint. Default is no. 

\item {\bf [-I NoiseFileName]}  \\
If the noise is stationary, the program can estimate the correct 
thresholds from a realization of the noise.

\item {\bf [-v]} \\
Verbose
\end{itemize}

\subsubsection*{Examples:}
\begin{itemize}
\item cur\_filter image.fits sol\\
Curvelet filtering of an image.
\item cur\_filter -n 5 -s4  image.fits sol\\
Curvelet filtering of an image, using five resolution levels, and
a 4-sigma detection.
\end{itemize}


\subsection{cur\_colfilter}
\index{cur\_colfilter}

Program {\em cur\_colfilter} filters a
color image using the curvelet transform.
\begin{center}
 USAGE:  cur\_colfilter options image\_in imag\_out
\end{center}
where options are 
\begin{itemize}
\baselineskip=0.4truecm
\itemsep=0.1truecm
\item {\bf [-n number\_of\_scale]} \\
 Number of scales used in the 2D wavelet transform.
 Default is 4. 

\item {\bf [-N number\_of\_scale]} \\
 Number of scales used in the ridgelet transform.
 Default is automatically calculated.

\item {\bf [-b BlockSize]}  \\
Block Size. Default is 16.

\item {\bf [-g sigma]} \\
Gaussian noise: sigma = noise standard deviation.  \\
 Default is automatically estimated.

\item {\bf [-s Nsigma]} \\
False detection rate. \\
Default is 3.

\item {\bf [-O]}  \\
Do not apply block overlapping. By default, block overlapping is used.

\item {\bf [-v]} \\
Verbose
\end{itemize}

\subsubsection*{Example:}
\begin{itemize}
\item cur\_colfilter image.fits sol\\
Curvelet transform of a color image.
\end{itemize}


\subsection{cur\_contrast}
\index{cur\_contrast}

Program {\em cur\_contrast} filters a color image using the curvelet transform.
\begin{center}
 USAGE:  cur\_contrast options image\_in imag\_out
\end{center}
where options are 
\begin{itemize}
\baselineskip=0.4truecm
\itemsep=0.1truecm
\item {\bf [-n number\_of\_scales]} \\
Number of scales used in the wavelet transform.
Default is 4. 
\item {\bf [-N number\_of\_scales]} \\
Number of scales used in the ridgelet transform.
Default is automatically calculated.
\item {\bf [-b BlockSize]} \\
Block size used by the curvelet transform. Default is 16.
\item {\bf [-O]} \\
Use overlapping block. Default is no.
\item {\bf [-g sigma]} \\
Noise standard deviation. Only used when filtering is performed.
Default is automatically estimated.
\item {\bf [-s NSigmalLow]} \\
 Coefficient $<$ NSigmalLow*SigmaNoise is not modified.
 Default is   5.
\item {\bf [-S NSigmalUp]} \\
 Coefficient $>$ NSigmalUp*SigmaNoise is not modified.
 Default is  20.
\item {\bf [-M MaxCoeff]} \\
If MaxBandCoef is the maximum coefficient in a given curvelet band,
 Coefficient $>$ MaxBandCoef*MaxCoeff is not modified.
 Default is 0.5.
\item {\bf  [-P P\_parameter]} \\
Determine the degree on non-linearity. P must be in ]0,1[.  
Default is 0.5.
\item {\bf [-T P\_parameter]} \\
 Curvelet coefficent saturation parameter. T must be in [0,1].  
Default is 0.
\item {\bf [-c]} \\
By default a sigma clipping is performed. When this option is set, no
sigma clipping is performed.
\item {\bf [-K ClippingValue]} \\
Clipping value. Default is 3.
\item {\bf [-L Saturation]} \\
Saturate the reconstructed image.
A coefficient larger than Saturation*MaxData is set to Saturation*MaxData.
Default is  1. If L is set to 0, then no saturation is applyied.
\end{itemize}

\subsubsection*{Examples:}
\begin{itemize}
\baselineskip=0.4truecm
\itemsep=0.1truecm
\item cur\_contrast image.fits image\_out.fits\\
Enhance the contrast using the curvelet transform.
\item cur\_contrast -O image.fits image\_out.fits\\
Enhance the contrast using the curvelet transform and block overlapping.
\item cur\_contrast -M 0.8 image.fits image\_out.fits\\
Enhance more the contrast.
\end{itemize}

\subsection{cur\_colcontrast}
\index{col\_colcontrast}
The program {\em col\_colcontrast} enhances the contrast of a color image
using the curvelet transform.
The command line is:
{\bf
\begin{center}
 USAGE: cur\_colcontrast option in\_image out\_image
\end{center}}
where options are:
\begin{itemize}
\baselineskip=0.4truecm
\itemsep=0.1truecm
\item {\bf [-n number\_of\_scales]} \\
Number of scales used in the wavelet transform.
Default is 4. 
\item {\bf [-N number\_of\_scales]} \\
Number of scales used in the ridgelet transform.
Default is automatically calculated.
\item {\bf [-b BlockSize]} \\
Block size used by the curvelet transform. Default is 16.
\item {\bf [-O]} \\
Use overlapping block. Default is no.
\item {\bf [-g sigma]} \\
Noise standard deviation.  
Default is automatically estimated.
\item {\bf [-s NSigmalLow]} \\
 Coefficient $<$ NSigmalLow*SigmaNoise is not modified.
 Default is   5.
\item {\bf [-S NSigmalUp]} \\
 Coefficient $>$ NSigmalUp*SigmaNoise is not modified.
 Default is  20.
\item {\bf [-M MaxCoeff]} \\
If MaxBandCoef is the maximum coefficient in a given curvelet band,
 Coefficient $>$ MaxBandCoef*MaxCoeff is not modified.
 Default is 0.5.
 \item {\bf  [-P P\_parameter]} \\
Determine the degree of non-linearity. P must be in ]0,1[.  
Default is 0.5.
\item {\bf [-T P\_parameter]} \\
 Curvelet coefficent saturation parameter. T must be in [0,1].  
Default is 0.
\item {\bf [-c]} \\
By default a sigma clipping is performed. When this option is set, no
sigma clipping is performed.
\item {\bf [-K ClippingValue]} \\
Clipping value. Default is 3.
\item {\bf [-L Luminance\_Saturation]} \\
Values in the luminance map which are 
larger than Saturation*MaxData are set to Saturation*MaxData.
Default is  1. 
\end{itemize}

\subsubsection*{Examples:}
\begin{itemize}
\baselineskip=0.4truecm
\itemsep=0.1truecm
\item cur\_colcontrast image.tiff image\_out.tiff\\
Enhance the contrast using the curvelet transform.
\item cur\_colcontrast -n 5 image.tiff image\_out.tiff\\
Ditto, but use five scales instead of four.
\item cur\_colcontrast -M 0.9 image.tiff image\_out.tiff\\
Enhance more the contrast.
\end{itemize}


\section{Combined Filtering}
\subsection{cb\_filter}
\index{cb\_filter}
Program {\em cb\_filter} filters an image corrupted by Gaussian noise by
 the combined filtering method. By default, the undecimated  bi-orthogonal WT
 and the curvelet transform are used. The number of iterations is defaulted
 10. In general, the algorithm converges with less than six iterations.
If the ``-T'' option is set, the Total Variation is minimized instead of 
the $l_1$ norm of the multiscale coefficients.
A deconvolution can also be performed using the ``-P'' option. In this
case, a division is first done in Fourier space between the 
input image and the point spread function. All Fourier components with
a norm lower than $\epsilon$ (default value is $10^{-3}$) are set to zero.
Then the deconvolved image is filtered by the combined filtering method
using the new noise properties (still Gaussian, but not white).

{\bf
\begin{center}
 USAGE: cb\_filter options image\_in trans\_out
\end{center}}
where options are:
\begin{itemize}
\baselineskip=0.4truecm
\itemsep=0.1truecm
 \item {\bf [-t TransformSelection]}
\begin{enumerate}
\baselineskip=0.4truecm
\itemsep=0.1truecm
\item A trous algorithm
\item Bi-orthogonal WT with 7/9 filters
\item Ridgelet transform
\item Curvelet transform
\item Mirror Basis WT
% \item Multiscale Ridgelet
%\item Cosinus transform
%\item Pyramidal Median transform
\end{enumerate}

\item {\bf [-n number\_of\_scales]} \\
Number of scales used in the \`a trous wavelet transform 
%, the PMT and
 and the curvelet transform. 
% Number of ridgelet in the multi-ridgelet transform.
Default is 4.

\item {\bf [-b BlockSize]}  \\
 Block Size in the ridgelet transform.
Default is image size.  
% Starting Block Size in the multi-ridgelet transform. Default is 8. 

\item {\bf [-i NbrIter]}  \\
Number of iterations. Default is 10.

\item {\bf [-F FirstDetectionScale]} \\
First detection scale in the ridgelet transform.
Default is 1. 

\item {\bf [-k]} \\
Kill the last scale in ridgelet. % , and multiscale ridgelet transform.
Default no.

\item {\bf  [-K]} \\
Kill last scale in the \`a trous algorithm and the curvelet. % and the PMT
Default no.

\item {\bf  [-L FirstSoftThreshold]} \\
First soft thresholding value. Default is 0.5.

\item {\bf  [-l LastSoftThreshold]} \\
Last soft thresholding value. Default is 0.5.

\item {\bf  [-u]} \\
 Number of undecimated scales in the WT.
 Default is 1. 

\item {\bf [-s Nsigma]} \\
False detection rate. Default is 4.

\item {\bf [-g sigma]} \\
Gaussian noise: sigma = noise standard deviation.  \\
 Default is automatically estimated.

% \item {\bf [-p]} \\
% Poisson Noise. Default is no (Gaussian).

\item {\bf [-O]}  \\
No block overlapping. Default is no.

\item {\bf [-T]}  \\
 Minimize the Total Variation instead of the L1 norm. 

\item {\bf [-P PsfFile]}  \\
Apply a deconvolution using the PSF in the file PsfFile. 

\item {\bf [-e Eps]}  \\
 Remove frequencies with $|P P^*| < \epsilon $. 
 Default is $10^{-3}$.

\item {\bf [-C TolCoef]}  \\
 Default is 0.5. 

\item {\bf [-v]} \\
Verbose. Default is no.
\end{itemize}

\subsubsection*{Example:}
\begin{itemize}
\item cb\_filter image.fits sol\\
Image filtering by the combined filtering method, using both
the curvelet and wavelet transform.
\end{itemize}





 
\newpage
\bibliographystyle{plain}
\bibliography{starck,wave,restore,compress,ima,edge,curvelet,color,candes,mc}

\clearpage
\newpage


\section*{Annex A: Wavelet transform  using the Fourier Transform}
\addcontentsline{toc}{section}{Appendix A: Wavelet transform  using the Fourier Transform}

We start with the set of scalar products $c_0(k)=<f(x),\phi(x-k)>$. If
$\phi(x)$ has a cut-off frequency $\nu_c\le {1\over 2}$
(\cite{starck:sta94_3,starck:sta94_4,starck:book98}), 
the data are
correctly sampled. The data at resolution $j=1$ are:
\begin{eqnarray}
c_1(k)=<f(x),\frac{1}{2}\phi(\frac{x}{2}-k)>
\end{eqnarray}
and we can  compute the set $c_{1}(k)$ from $c_0(k)$ with a discrete 
filter $\hat h(\nu)$:
\begin{eqnarray}
\hat h(\nu)= \left\{
  \begin{array}{ll}
  {\hat{\phi}(2\nu)\over \hat{\phi}(\nu)} & \mbox{if } \mid \nu \mid < \nu_c \\
0 & \mbox{if } \nu_c  \leq \mid \nu \mid < {1\over 2} 
  \end{array}
  \right.
\end{eqnarray}
and
\begin{eqnarray}
\forall \nu, \forall n \mbox{    } & \hat h(\nu + n) = \hat h(\nu)
\end{eqnarray}
where $n$ is an integer.
So:
\begin{eqnarray}
\hat{c}_{j+1}(\nu)=\hat{c}_{j}(\nu)\hat{h}(2^{j}\nu)
\end{eqnarray}
The cut-off frequency is reduced by a factor $2$ at each step, allowing  a
reduction of the number of samples by this factor.

The wavelet coefficients at scale $j+1$ are:
\begin{eqnarray}
w_{j+1}(k)=<f(x),2^{-(j+1)}\psi(2^{-(j+1)}x-k)>
\end{eqnarray}
and they can be computed directly from $c_j(k)$ by:
\begin{eqnarray}
\hat{w}_{j+1}(\nu)=\hat{c}_{j}(\nu)\hat g(2^{j}\nu)
\end{eqnarray}
where $g$ is the following  discrete filter:
\begin{eqnarray}
\hat g(\nu)= \left\{
  \begin{array}{ll}
  {\hat{\psi}(2\nu)\over \hat{\phi}(\nu)} & \mbox{if } \mid \nu \mid < \nu_c \\
1 & \mbox{if } \nu_c  \leq \mid \nu \mid < {1\over 2} 
  \end{array}
  \right.
\end{eqnarray}
and
\begin{eqnarray}
\forall \nu, \forall n \mbox{    } & \hat g(\nu + n) = \hat g(\nu)
\end{eqnarray}

The frequency band is also reduced by a factor $2$ at each step.
Applying the sampling theorem, we can build a pyramid  of
\index{pyramid}
 $N+{N\over 2}+\ldots+1=2N$ elements.
For an image analysis the number of elements is ${4\over 3}N^2$. The
overdetermination is not very high.

The B-spline functions are compact in direct space. They
correspond to the autoconvolution of a square function. In
Fourier space we have:
\index{Fourier transform}
\begin{eqnarray}
\hat B_l(\nu)=({\sin\pi\nu\over\pi\nu})^{l+1}
\end{eqnarray}
$B_3(x)$ is a set of $4$ polynomials of degree $3$.
We choose the scaling function $\phi(\nu)$ which has a
$B_3(x)$ profile in Fourier space:
\begin{eqnarray}
\hat{\phi}(\nu)={3\over 2}B_3(4\nu)
\end{eqnarray}
In direct space we get:
\begin{eqnarray}
\phi(x)={3\over 8}[{\sin{\pi x\over 4}\over {\pi x\over
4}}]^4
\end{eqnarray}
This function is quite similar to a Gaussian and converges
rapidly to $0$. For 2-dimensions the scaling function is defined by
$\hat \phi(u,v) = {3\over 2}B_3(4r)$, with $r = \sqrt(u^2+v^2)$.
This is an isotropic function.

The wavelet transform algorithm with $n_p$ scales is the following:
\index{wavelet transform}
\begin{enumerate}
\item Start with a $B_3$-spline scaling function and derive $\psi$, $h$ and
$g$ numerically.
\item Compute the corresponding FFT image. 
Name the resulting complex array $T_0$.
\item Set $j$ to $0$. Iterate:
\item Multiply  $T_j$ by $\hat g(2^ju,2^jv)$. We get the complex array
$W_{j+1}$. The inverse FFT
gives the wavelet coefficients at scale $2^j$;
\item Multiply  $T_j$ by $\hat h(2^ju,2^jv)$. We get the array
$T_{j+1}$. Its inverse FFT gives the image at scale $2^{j+1}$.
The frequency band is reduced by a factor $2$.
\item Increment $j$.
\item If $j \leq  n_p$, go back to 4.
\item The set $\{w_1, w_2, \dots, w_{n_p}, c_{n_p}\}$ describes the
wavelet transform.
\end{enumerate}
If the wavelet is the difference between two resolutions, i.e.
\begin{eqnarray}
\hat \psi(2\nu) = \hat \phi(\nu) - \hat \phi(2\nu)
\end{eqnarray}
and:
\begin{eqnarray}
\hat g(\nu) = 1 - \hat h(\nu)
\end{eqnarray}
then the wavelet coefficients $\hat w_j(\nu)$ can be computed by 
$\hat c_{j-1}(\nu) - \hat c_j(\nu)$.

\subsubsection*{Reconstruction.}
If the wavelet is the difference between two resolutions,
an evident reconstruction for a wavelet transform 
${\cal W} = \{w_1,\dots, w_{n_p}, c_{n_p}\}$ is:
\begin{eqnarray}
\hat c_0(\nu) = \hat c_{n_p}(\nu) + \sum_j \hat w_j(\nu)
\end{eqnarray}
But this is a particular case, and other alternative wavelet functions can be
chosen. The reconstruction can be made step-by-step, starting from
the lowest resolution. At each scale, we have the relations:
\begin{eqnarray}
\hat c_{j+1} = \hat h(2^j \nu) \hat c_j(\nu) \\
\hat w_{j+1} = \hat g(2^j \nu) \hat c_j(\nu) 
\end{eqnarray}
We look for $c_j$ knowing $c_{j+1}$, $w_{j+1}$, $h$ and $g$.
We restore $\hat c_j(\nu)$ based on a least mean square estimator:
\begin{eqnarray}
\hat p_h(2^j\nu) \mid \hat c_{j+1}(\nu)-\hat h(2^j\nu)\hat c_j(\nu) \mid^2 + 
\nonumber \\
\hat p_g(2^j\nu) \mid \hat w_{j+1}(\nu)-\hat g(2^j\nu)\hat c_j(\nu) \mid^2
\end{eqnarray}
is to be minimum. $\hat p_h(\nu)$ and $\hat p_g(\nu)$ are weight
functions which permit a general solution to the
restoration of $\hat c_j(\nu)$. From the derivation of $\hat c_j(\nu)$  we get:
\begin{eqnarray}
\hat{c}_{j}(\nu)=\hat{c}_{j+1}(\nu) \hat{\tilde h}(2^{j}\nu)
                +\hat{w}_{j+1}(\nu) \hat{\tilde g}(2^{j}\nu)
\label{restauration}
\end{eqnarray} 
where the conjugate filters have the expression:
\begin{eqnarray}
\hat{\tilde h}(\nu) & = {\hat{p}_h(\nu) \hat{h}^*(\nu)\over \hat{p}_h(\nu)
\mid \hat{h}(\nu)\mid^2 + \hat{p}_g(\nu)\mid \hat{g}(\nu)\mid^2} \label{eqnht} \\ 
\hat{\tilde g}(\nu) & = {\hat{p}_g(\nu) \hat{g}^*(\nu)\over \hat p_h(\nu)
\mid \hat{h}(\nu)\mid^2 + \hat{p}_g(\nu)\mid \hat{g}(\nu)\mid^2}
\label{eqngt}
\end{eqnarray}

In this analysis, the
Shannon sampling condition is always respected and no aliasing
exists.

The denominator is reduced if we choose:
\[\hat{g}(\nu) = \sqrt{1 - \mid\hat{h}(\nu)\mid^2}\]
This corresponds to the case where the wavelet is the difference between
the square of two resolutions:
\begin{eqnarray}
\mid \hat \psi(2\nu)\mid^2  \ = \ \mid \hat \phi(\nu)\mid^2  - \mid  \hat
\phi(2\nu)\mid^2 
\end{eqnarray}

% \begin{figure}[htb]
% \centerline{
% \hbox{
% \psfig{figure=ch1_diff_uv_phi_psi.ps,bbllx=0.5cm,bblly=13.5cm,bburx=20.5cm,bbury=27cm,height=6cm,width=14.cm,clip=}
% }}
% \caption{On the left, the interpolation function $\hat{\phi}$ and, on the 
% right, the wavelet  $\hat{\psi}$.}
% \label{fig_diff_uv_phi_psi}
% \end{figure}

% \begin{figure*}[htb]
% \centerline{
% \hbox{
% \psfig{figure=ch1_diff_uv_ht_gt.ps,bbllx=0.5cm,bblly=13.5cm,bburx=20.5cm,bbury=27cm,height=5cm,width=14.5cm,clip=}
% }}
% \caption{On the left, the filter $\hat{\tilde{h}}$, and on the right the 
% filter $\hat{\tilde{g}}$.}
% \label{fig_diff_uv_ht_gt}
% \end{figure*}

% In Fig.\ \ref{fig_diff_uv_phi_psi} the chosen scaling function 
% derived from a B-spline of degree 
% 3, and its resulting wavelet function, are plotted in frequency space.
 
The reconstruction algorithm is:
\begin{enumerate}
\item Compute  the FFT of the image at the low resolution.
\item Set $j$ to $n_p$. Iterate:
\item Compute the FFT of the wavelet coefficients at scale $j$.
\item Multiply  the wavelet coefficients $\hat{w}_j$ by $\hat{\tilde{g}}$.
\item Multiply   the image at the lower resolution $\hat{c}_j$ by 
$\hat{\tilde{h}}$.
\item The inverse Fourier transform of the addition of  
$\hat{w}_j\hat{\tilde{g}}$ and $\hat{c}_j\hat{\tilde{h}}$ gives the 
image $c_{j-1}$.
\item Set $j = j - 1$ and return to 3.
\end{enumerate}
\index{Fourier transform}

The use of a scaling function with a cut-off frequency
allows a reduction of sampling at each scale, and limits the  
computing time and the memory size. 



\section*{Appendix B: The `` \`A Trous'' Wavelet Transform Algorithm}
\addcontentsline{toc}{section}{Appendix B: The `` \`A Trous'' Wavelet Transform Algorithm}

In a wavelet transform, a series of transformations of an image is 
generated, providing a resolution-related set of  ``views'' of the image.  
The properties satisfied by a wavelet transform, and in particular the
{\it \`a trous} wavelet transform (``with holes'', so called because of the 
interlaced
convolution used in successive levels: see step 2 of the algorithm below) 
are further discussed in \cite{starck:book98}. 

We consider sampled data, 
$\{c_0(k)\}$, defined as  the scalar product at 
pixels $k$ of the function $f(x)$ with a scaling function $\phi(x)$
which corresponds to a low pass band filter:
\begin{eqnarray}
c_0(k) = < f(x), \phi(x-k)>
\end{eqnarray}
The scaling function is chosen to satisfy the dilation equation:
\begin{eqnarray}
\frac{1}{2}\phi(\frac{x}{2}) = \sum_k h(k)\phi(x-k)
\end{eqnarray}
 
$h$ is a discrete low pass filter associated with the scaling function
$\phi$.  This means that a low-pass filtering
of the image is, by definition, closely linked to another resolution level
of the image.   
 
The smoothed data $c_j(k)$ at a given resolution $j$ and at a position
$k$  is the scalar product 
 
\begin{eqnarray}
c_j(k)= \frac{1}{2^j}< f(x), \phi(\frac{x-k}{2^j})>
\end{eqnarray}
 
This is consequently obtained by the convolution:
\begin{eqnarray}
c_j(k) = \sum_l h(l) \ \ c_{j-1} (k+2^{j-1}l)
\end{eqnarray}
The signal difference $w_j$ between two consecutive resolutions is:
\begin{eqnarray}
w_j(k) = c_{j-1}(k) - c_j(k) 
\end{eqnarray}
or:
\begin{eqnarray}
w_j(k) = \frac{1}{2^j}< f(x), \psi(\frac{x-k}{2^j})>  
\end{eqnarray}
Here, the wavelet function $\psi$ is defined by:
\begin{eqnarray}
\frac{1}{2}\psi(\frac{x}{2})  = \phi(x) - \frac{1}{2}\phi(\frac{x}{2})
\end{eqnarray}
 
For the scaling function, $\phi(x)$, the B-spline of degree 3 
was used in our calculations.     
We have derived a simple algorithm in order to compute the 
associated wavelet transform:
\begin{enumerate}
\item We initialize $j$ to 0 and we start with the data $c_j(k)$.
\item We increment $j$, and we carry out a discrete convolution of the data
$c_{j-1}(k)$ using  the filter $h$. The distance between the central pixel
and the adjacent ones is $2^{j-1}$.
\item After this smoothing, we obtain the discrete wavelet transform
from the difference $c_{j-1}(k) - c_j(k)$.
\item If $j$ is less than the number $p$ of resolutions we want to
compute, then go to step 2.
\item The set ${\cal W} = \{ w_1, ..., w_p, c_p \}$ represents the
wavelet transform of the data.
\end{enumerate}


The most general way to handle the boundaries is to consider
that $c(k + N) = c(N - k)$. But other methods can be used
such as periodicity ($c(k + N) = c(k)$), or continuity
 ($c(k + N) = c(N)$).  Choosing one of these methods has little influence 
on our general restoration strategy.  We used continuity.  
 
A series expansion of the original signal, $c_0$, 
in terms of
the wavelet coefficients is now given as follows. 
The final smoothed array $c_{p}(x)$ is added to all the differences $w_j$:
\begin{eqnarray}
c_0(k) = c_{p}(k) + \sum_{j=1}^{p} w_j(k)
\end{eqnarray}
 This equation provides a reconstruction formula for the original signal.
 
At each scale $j$, we obtain a set $\{w_j\}$.  This has 
the same number of pixels as the input signal. 

The above {\em \`a trous} algorithm has been discussed in terms of a single
index, $x$, but is easily extendable to 
two-dimensional space.  The use of the B$_3$ spline leads to a 
convolution with a mask of $5 \times 5$:
$$ 
\left(    \begin{array}{ccccc}
\frac{1}{256} & \frac{1}{64} & \frac{3}{128} & \frac{1}{64} & \frac{1}{256} \\
\frac{1}{64}  & \frac{1}{16} & \frac{3}{32}  & \frac{1}{16} & \frac{1}{64}  \\
\frac{3}{128} & \frac{3}{32} & \frac{9}{64}  & \frac{3}{32} & \frac{3}{128} \\
\frac{1}{64}  & \frac{1}{16} & \frac{3}{32}  & \frac{1}{16} & \frac{1}{64}  \\
\frac{1}{256} & \frac{1}{64} & \frac{3}{128} & \frac{1}{64} & \frac{1}{256} 
\end{array} \right)
$$
In one dimension, this mask is:
$ ( \frac{1}{16}, \frac{1}{4}, \frac{3}{8},
\frac{1}{4}, \frac{1}{16} ) $.
 
To facilitate computation, a simplification of this wavelet is to assume
separability in the 2-dimensional case.  In the case of the B$_3$ spline, this
leads to a row by row convolution with $(\frac{1}{16}, \frac{1}{4}, 
\frac{3}{8}, \frac{1}{4}, \frac{1}{16})$; followed by column by column 
convolution. 

As for the one dimensional case, an exact reconstruction is obtained
by adding all the scales and the final smoothed array:
\begin{eqnarray}
c_0(x,y) = c_{p}(x,y) + \sum_{j=1}^{p} w_j(x,y)
\label{eqn_rec}
\end{eqnarray}

% \bibliographystyle{plain}
% \bibliography{starck,restore,wave,ima,astro,compress}
 


\end{document}


\subsubsection{Elongated - point like object separation in astronomical images.}
 
\begin{figure}[htb]
\centerline{
\vbox{
\hbox{
\psfig{figure=fig_cb2_ngc2997.ps,bbllx=2cm,bblly=13.5cm,bburx=13cm,bbury=24.5cm,width=7cm,height=7cm,clip=}
\psfig{figure=fig_cb2_ngcatrou.ps,bbllx=2cm,bblly=13.5cm,bburx=13cm,bbury=24.5cm,width=7cm,height=7cm,clip=}
}
\hbox{
\psfig{figure=fig_cb2_ngcrid.ps,bbllx=2cm,bblly=13.5cm,bburx=13cm,bbury=24.5cm,width=7cm,height=7cm,clip=}
\psfig{figure=fig_cb2_ngc_cb.ps,bbllx=2cm,bblly=13.5cm,bburx=13cm,bbury=24.5cm,width=7cm,height=7cm,clip=}
}}}
\caption{Top left, galaxy NGC2997, top right  reconstructed image from the
\`a trous wavelet coefficients, bottom
left, reconstruction from the ridgelet coefficients,   
and bottom right addition of both reconstructed images.}
\label{fig_cb2_ngc}
\end{figure}
Figure~\ref{fig_cb2_ngc} shows the result of a decomposition 
of a spiral galaxy (NGC2997).
This image (figure~\ref{fig_cb2_ngc} top left) contains
many compact structures (stars and HII region), more
or less isotropic, and large scale elongated features (NGC2997 spiral part).
Compact objects are well represented by isotropic wavelets, and the  
elongated features are better represented by a ridgelet basis. 
In order to benefit of the optimal data representation of both transforms,
 the image has been decomposed on both the \`a trous wavelet transform and on 
the ridgelet transform by using the same method as described in
section~\ref{sect_l0}. When the functional is minimized, we get two 
images, and their coaddition is the filtered version of the original image.
The reconstructions from the \`a trous coefficients, and from  
the ridgelet coefficients can be seen in figure~\ref{fig_cb2_ngc} top right
and bottom left. The addition of both images is presented 
in figure~\ref{fig_cb2_ngc} bottom right.
