
\chapter{Filtering}
\label{sec:filtering}
\section{Curvelet Coefficient Thresholding}
We now apply our digital transforms for
removing noise from image data. The methodology is standard and is
outlined mainly for the sake of clarity and self-containedness.

Suppose that one is given noisy data of the form
\[
x_{i,j} = f(i,j) + \sigma z_{i,j},
\]
where $f$ is the image to be recovered and $z$ is white noise, i.e.
$z_{i,j} \stackrel{i.i.d.}{\sim} N(0,1)$. Unlike FFTs or FWTs, our
discrete ridgelet (resp. curvelet) transform is not norm-preserving
and, therefore, the variance of the noisy ridgelet (resp. curvelet)
coefficients will depend on the ridgelet (resp. curvelet) index
$\lambda$.  For instance, letting $F$ denote the discrete curvelet
transform matrix, we have $F z \stackrel{i.i.d.}{\sim} N(0,F F^T)$
where $T$ denotes transpose.
Because the computation of $F F^T$ is prohibitively expensive, we
calculated an approximate value $\tilde{\sigma}^2_\lambda$ of the
individual variances using Monte Carlo simulations where the diagonal
elements of $F F^T$ are simply estimated by evaluating the curvelet
transforms of a few standard white noise images.

Let $y_\lambda$ be the noisy curvelet coefficients
($y = F x$). We use the following hard-thresholding rule
for estimating the unknown curvelet coefficients:
\begin{eqnarray}
\hat y_\lambda = y_\lambda  & \mbox{ if }  &
|y_\lambda|/\sigma \geq k \tilde{\sigma}_\lambda\\
\hat y_\lambda = 0 &  \mbox{ if } & |y_\lambda|/\sigma
<  k \tilde{\sigma}_\lambda.
\end{eqnarray}
In our experiments, we actually chose a scale-dependent value for $k$;
we have $k = 4$ for the first scale $(j = 1)$ while $k = 3$
for the others $(j > 1)$.

\subsection*{Poisson Observations}
Assume now that we have Poisson data $x_{i,j}$ with unknown mean
$f(i,j)$.  The Anscombe transformation \cite{rest:anscombe48}
\begin{equation}
\tilde{x} = 2\sqrt{x + \frac{3}{8}}
\end{equation}
stabilizes the variance and we have $\tilde{x} = 2 \sqrt{f} +
\epsilon$ where $\epsilon$ is a vector with independent and
approximately standard normal components. In practice, this is a good
approximation whenever the number of counts is large enough, greater
than 30 per pixel, say.
 
For a small number of counts, a possibility is to compute the Radon
transform of the image, and then to apply the Anscombe transformation
to the Radon data. The rationale is that, roughly speaking, the
Radon transform corresponds to a summation of pixel values over lines
and  the sum of independent Poisson random variables is a Poisson
random variable with intensity equal to the sum of the individual
intensities. Hence, the intensity of the sum may be quite large (hence
validating the Gaussian approximation) even though the individual
intensities may be small. This might be viewed as an interesting
feature  since,  unlike wavelet transforms, the ridgelet and curvelet
transforms tend to average data over elongated and rather large
neighborhoods.

\section{Filtering Experiments}

\subsubsection{Recovery of Linear Features}

\begin{figure}[htb]
\centerline{
\vbox{
\hbox{
\psfig{figure=fig_line.ps,bbllx=1cm,bblly=12cm,bburx=15cm,bbury=26cm,width=7.5cm,height=7.5cm,clip=}
\psfig{figure=fig_line_g0p5.ps,bbllx=1cm,bblly=12cm,bburx=15cm,bbury=26cm,width=7.5cm,height=7.5cm,clip=}
}
\hbox{
\psfig{figure=fig_f24_line_g0p5.ps,bbllx=1cm,bblly=12cm,bburx=15cm,bbury=26cm,width=7.5cm,height=7.5cm,clip=}
\psfig{figure=fig_cur_line_g0p5.ps,bbllx=1cm,bblly=12cm,bburx=15cm,bbury=26cm,width=7.5cm,height=7.5cm,clip=}
}}}
\caption{The top panels display a geometric image and that same image
      contaminated with Gaussian white noise. The bottom left and right
      panels display the restored images using the undecimated wavelet
      transform and the curvelet transform, respectively.}
\label{fig_cur_line}
\end{figure}
The next experiment (Figure~\ref{fig_cur_line}) consists of an
artificial image containing a few bars, lines and a square.  The
intensity is constant along each individual bar; from left to right,
the intensities of the ten vertical bars (these are in fact thin
rectangles which are 4 pixels wide and 170 pixels long) are equal to
${32 \over 2^i}$, $i = 0, \ldots 9$.  The intensity along all the
other lines is equal to 1, and the noise standard deviation is $1/2$.
Displayed images have been log-transformed in order to better see
the results at low signal to noise ratio.

The curvelet reconstruction of the nonvertical lines is obviously
sharper than that obtained using wavelets.  The curvelet transform
also seems to go one step further as far as the reconstruction of the
vertical lines is concerned. Roughly speaking, for those templates,
the wavelet transform stops detecting signal at a SNR equal to 1
(we defined here the SNR as the intensity level of the pixels
on the line, divided by the noise standard deviation of the noise)
while the cut-off value equals 0.5 for the curvelet approach.  It is
important to note that the horizontal and vertical lines correspond to
privileged directions for the wavelet transform, because the underlying
basis functions are direct products of functions varying solely in
the horizontal
and vertical directions.  Wavelet methods will
give even poorer results on lines
of the same intensity but tilting substantially
away from the Cartesian axes.
Cf.\ the reconstructions of the faint diagonal
lines in the image.

\subsubsection{Recovery of Curves}
In this experiment (Figure~\ref{fig_cur_picasso}), we have added 
Gaussian noise to ``War and Peace,'' a drawing from Picasso which
contains many curved features.  Figure \ref{fig_cur_picasso} bottom left
and right show respectively the restored images by the undecimated
wavelet transform and the curvelet transform. Curves are more sharply
recovered with the curvelet transform.

The authors are working on new methods (some of which will be based on
the curvelet transform) to extract and recover curves from noisy data
with greater accuracy and, therefore, this example is merely to be
taken for illustrative purposes.

\begin{figure}[htb]
\centerline{
\vbox{
\hbox{
\psfig{figure=fig_picasso.ps,bbllx=2cm,bblly=13cm,bburx=18cm,bbury=25cm,width=8cm,height=6cm,clip=}
\psfig{figure=fig_picasso_g50.ps,bbllx=2cm,bblly=13cm,bburx=18cm,bbury=25cm,width=8cm,height=6cm,clip=}
}
\hbox{
\psfig{figure=fig_picasso_wt.ps,bbllx=2cm,bblly=13cm,bburx=18cm,bbury=25cm,width=8cm,height=6cm,clip=}
\psfig{figure=fig_picasso_cur.ps,bbllx=2cm,bblly=13cm,bburx=18cm,bbury=25cm,width=8cm,height=6cm,clip=}
}}}
\caption{The top panels display a Picasso picture (War and Peace) 
      and that same image
      contaminated with Gaussian white noise. The bottom left and right
      panels display the restored images using the undecimated wavelet
      transform and the curvelet transform respectively.}
\label{fig_cur_picasso}
\end{figure}


\subsubsection{Denoising of a Color Image}

\begin{figure}[htb]
\centerline{
\vbox{
\hbox{
\psfig{figure=fig_cmp_lena_rgb.ps,bbllx=2cm,bblly=13cm,bburx=20cm,bbury=25cm,width=6cm,height=4.cm,clip=}
\psfig{figure=fig_cmp_pepper_rgb.ps,bbllx=2cm,bblly=13cm,bburx=20cm,bbury=25cm,width=6cm,height=4.cm,clip=}
}
\hbox{
\psfig{figure=fig_cmp_mandrill_rgb.ps,bbllx=2cm,bblly=13cm,bburx=20cm,bbury=25cm,width=6cm,height=4.cm,clip=}
\psfig{figure=fig_cmp_barbara_rgb.ps,bbllx=2cm,bblly=13cm,bburx=20cm,bbury=25cm,width=6cm,height=4.cm,clip=}
}
}}
\caption{PSNR versus noise standard deviation using different filtering
      methods. YUV and curvelet, YUV and undecimated wavelet, and YUV and
      decimated wavelet transforms are represented respectively with a
      continuous, dashed, and dotted line.  The upper left panel
      corresponds to {\tt Lena} (RGB), the upper right to {\tt pepper}
      (RGB), the bottom left to {\tt Baboon} (RGB), and the bottom right
      to {\tt Barbara} (RGB).}
\label{fig_exp_rgb_curv}
\end{figure}

In a wavelet based denoising scenario, color RGB images are generally
mapped into the YUV space, and each YUV band is then filtered
independently from the others. The goal here is to see whether the
curvelet transform would give improved results. We used four of the
classical color images, namely {\tt Lena}, {\tt Peppers}, {\tt
      Baboon}, and {\tt Barbara} (all images except perhaps {\tt Barbara}
are available from the USC-SIPI Image Database \cite{SIPI}).
We performed
a series of experiments and
summarized our findings in Figure~\ref{fig_exp_rgb_curv} which again
displays the PSNR (peak signal-to-noise ratio) 
versus the noise standard deviation for the four
images.

In all cases, the curvelet transform outperforms the wavelet
transforms in terms of PSNR -- at least for moderate and large values
of the noise level.  In addition, the curvelet transform outputs
images that are visually more pleasant.
 
\subsubsection{Saturn Rings}

\begin{figure}[htb]
\centerline{
\vbox{
\hbox{
\psfig{figure=SATURN/fig_sat_sub_g20.ps,bbllx=1.8cm,bblly=12.7cm,bburx=14.5cm,bbury=25.4cm,width=7cm,height=7cm,clip=}
\psfig{figure=SATURN/fig_sat_sub_fil_owt.ps,bbllx=1.8cm,bblly=12.7cm,bburx=14.5cm,bbury=25.4cm,width=7cm,height=7cm,clip=}
}
\hbox{
\psfig{figure=SATURN/fig_sat_sub_fil_atrou.ps,bbllx=1.8cm,bblly=12.7cm,bburx=14.5cm,bbury=25.4cm,width=7cm,height=7cm,clip=}
\psfig{figure=SATURN/fig_sat_sub_fil_cur.ps,bbllx=1.8cm,bblly=12.7cm,bburx=14.5cm,bbury=25.4cm,width=7cm,height=7cm,clip=}
}}
}
\caption{Top left, part of Saturn image with Gaussian noise. Top right,
filtered image using the undecimated bi-orthogonal wavelet transform.
Bottom left and right, filtered image by the \`a trous wavelet 
transform algorithm and the curvelet transform.}
\label{fig_sub_saturn_cur_filter}
\end{figure}

Gaussian white noise with a standard deviation
fixed to 20 was added to the {\tt Saturn} image.
We employed several methods to filter the noisy image:
\begin{enumerate}
\item Thresholding of the Curvelet transform.
\item Bi-orthogonal undecimated wavelet de-noising methods using  the
      Dau\-che\-chies-An\-to\-ni\-ni 7/9 fil\-ters (FWT-7/9) and hard thresholding.
\item A trous wavelet transform algorithm and hard thresholding.
\end{enumerate}
Our experiments are reported in Figure~\ref{fig_sub_saturn_cur_filter}.  The
curvelet reconstruction does not contain the quantity of disturbing artifacts
along edges that one sees in wavelet reconstructions. An examination
of the details of the restored images % (Figure~\ref{fig_sub_saturn_cur_filter})
is instructive. One notices that the decimated wavelet transform
exhibits distortions of the boundaries and suffers substantial loss of
important detail.  The \`a trous  wavelet transform gives better
boundaries, but completely omits to reconstruct certain ridges. 
In addition, it exhibits numerous small-scale embedded
blemishes; setting higher thresholds to avoid these blemishes would
cause even more of the intrinsic structure to be missed.

% Further results are visible at the following URL: {\tt
%  http://www-stat.stanford.edu/$\sim$jstarck}.

\subsubsection{Supernova with Poisson noise}

\begin{figure}[htb]
\centerline{  
\hbox{
\psfig{figure=fig_kepler1604.ps,bbllx=1.8cm,bblly=12.7cm,bburx=14.5cm,bbury=25.4cm,width=7cm,height=7cm,clip=}
\psfig{figure=fig_kepler1604_fil_rid.ps,bbllx=1.8cm,bblly=12.7cm,bburx=14.5cm,bbury=25.4cm,width=7cm,height=7cm,clip=}
}}
\caption{Left, XMM/Newton image of the Kepler SN1604 supernova.
Right, ridgelet filtered image.}
\label{fig_sn1604}
\end{figure}
Figure~\ref{fig_sn1604} shows an example of an X-ray image filtering by the
ridgelet transform using such an approach. Figure~\ref{fig_sn1604} left
and right 
show respectively the  XMM/Newton image of the Kepler SN1604 supernova and
the ridgelet filtered image (using a five sigma hard thresholding). 


% \clearpage
% \newpage
