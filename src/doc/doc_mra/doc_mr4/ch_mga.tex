\chapter{Morphological Component Analysis}
\label{sect_ctm}
\section{Introduction}
The content of an image is often complex, and there is not a single
transform which is optimal to represent all the contained features. For
example, the Fourier transform better represents some textures, while
the wavelet transform better represents singularities. Even if we limit
our class of transforms to the wavelet one, decision have to
be taken between an isotropic wavelet transform which produce good 
results for isotropic objects (such stars and galaxies in astronomical 
images, cells in biological images, etc), or an orthogonal wavelet 
transform, which is better for images with edges. 
This has motivated
the development of different methods \cite{wave:donoho98,wave:meyer98,cur:huo99}, 
and the two most frequently discussed approaches are the Matching Pursuit (MP)
\cite{wave:mallat93} and the Basis pursuit (BP) \cite{wave:donoho98}.
A dictionary ${\cal D}$ being defined as a collection of waveforms 
$(\varphi_{\gamma})_{\gamma \in \Gamma}$,
the general principe consists in representing a signal $s$ as a ``sparse''
linear combination  of a small number of basis such that:
\begin{eqnarray}
 s = \sum_{\gamma} a_{\gamma} \varphi_{\gamma}
\end{eqnarray}
or an approximate decomposition
\begin{eqnarray}
 s = \sum_{i=1}^m a_{\gamma_i} \varphi_{\gamma_i} + R^{(m)} .
\end{eqnarray}
 
Matching pursuit \cite{wave:mallat93,ima:mallat98} method (MP) uses a greedy
algorithm which adaptively refines the signal approximation with an
iterative procedure:
\begin{itemize}
\item Set $s^0 = 0$ and $R^0 = 0$.
\item Find the element $\alpha_k \varphi_{\gamma_k}$ which best correlates with the 
residual.
\item Update $s$ and $R$:
\begin{eqnarray}
s^{k+1} & = & s^k + \alpha_k \varphi_{\gamma_k} \nonumber \\
R^{k+1} & = & s -  s^k .
\end{eqnarray}
\end{itemize}
In case of non orthogonal dictionaries, it has been shown \cite{wave:donoho98}
 that MP may
spend most of the time correcting mistakes made in the first few terms,
and therefore is suboptimal in term of sparsity.

\bigskip
Basis pursuit method  \cite{wave:donoho98} (BP) is a global procedure which
synthesizes an approximation $\tilde{s}$ to $s$ by minimizing a
functional of the type
\begin{equation}
\|s - \tilde{s}\|_{\ell_2} ^2 + \lambda \cdot \|\alpha\|_{\ell_1}, 
\quad \tilde{s} = \Phi \alpha.  
\label{eqn_mp}
\end{equation}
Between all possible solutions, the chosen one has the minimum $l^1$ norm.
This choice of $l_1$ norm is very important. A $l_2$ norm, as 
used in the method of frames \cite{wave:daube88b},
 does not preserve the sparsity \cite{wave:donoho98}. 
 
In many cases, BP or MP synthesis algorithms are computationally very
expensive. We present in the following an alternative approach, that we call 
{\em Morphological Component Analysis} (MCA), which combines  
 the different available transforms 
in order to benefit of the advantages of each of them. 

\section{The Combined Transformation}

Depending on the content of the data, several transforms 
can be combined in order to get an optimal representation of all 
features contained in our data set. In addition to the ridgelet
and the curvelet transform, we may want to use the \`a trous algorithm
which is very well suited to astronomical data, or the undecimated
wavelet transform which is commonly used in the signal processing domain.

Other transform such wavelet packets, the Fourier transform, 
the Pyramidal median transform \cite{starck:book98}, 
or other multiscale morphological transforms,
could also be considered. However, we found that in practice,
these four transforms (i.e. curvelet, ridgelet, \`a trous algorithm,
and undecimated wavelet transform)
furnishes a very large panel of waveforms
which is generally large enough to well 
represents all features contained in the data.

In general, suppose that we are given $K$ linear transforms $T_1,
\ldots, T_K$ and let $\alpha_k$ be the coefficient sequence of an
object $x$ after applying the transform $T_k$, i.e. $\alpha_k = T_k
x$. We will suppose that for each transform $T_k$ we have available a
reconstruction rule that we will denote by $T^{-1}_k$ although this is
clearly an abuse of notations.

Therefore, we search a vector  
${\bf \alpha} = {\alpha_1, \dots, \alpha_{K}}$ such that
\begin{equation}
s =  \Phi {\bf \alpha}
\end{equation}
where $\Phi {\bf \alpha} = \sum_{k=1}^K T_k^{-1} \alpha_k$.
As our dictionary is overcomplete, there is an infinity of vectors
verifing this condition, and we need to solve the following optimization 
problem:
\begin{equation}
min \parallel s - \phi {\bf \alpha} \parallel^2 + {\cal C}(\bf \alpha)  
\end{equation}
where ${\cal C}$ is a penalty term. We easily see that chosing
${\cal C}({\bf \alpha}) =  \parallel {\bf \alpha} \parallel_{l_1}$ leads to the
BP method, where the dictionary ${\cal D}$ is only composed of the
basis elements of the chosen transforms.
 
Two iterative methods, {\em soft-MCA} and {\em hard-MCA}, allowing us 
to realize such a combined transform, are described in this section.

\section{Soft-MCA}
\label{sect_l1}
% $l^1$ optimization by soft thresholding

Noting $T_1, ..., T_{K}$ the $K$ transform operators, 
a solution ${\bf \alpha}$  is obtained by minimizing a functional of the form:
\begin{eqnarray}
J({\bf \alpha}) = \parallel s - \sum_{k=1}^{K} T_k^{-1} \alpha_k  \parallel_2^2 + \lambda \sum_k \parallel \alpha_k \parallel_1
\label{eqn_min_l1}
\end{eqnarray}
where $s$ is the original signal, and $\alpha_k$ are the coefficients 
obtained with the transform $T_k$.

An simple algorithm to achieve such an solution is \cite{starck:spie01b,starck:sta02_3}:
\begin{enumerate}
\item Initialize $L_{\max}$, the number of iterations $N_i$, 
$\lambda = L_{\max}$, and $\delta_{\lambda} = \frac{L_{\max}}{N_i}$.
\item While $\lambda >= 0$ do
\item For k = 1, .., $K$ do
\begin{itemize}
\item Calculate the residual $R = s - \sum_k T_k^{-1} \alpha_k$.
\item Calculate the  transform $T_k$ of the residual:
$r_k = T_k R$.
\item Add the residual to  $\alpha_k$:  $\alpha_k = \alpha_k + r_k$.
\item Soft threshold the coefficient $\alpha_k$ with the $\lambda$ threshold.
\end{itemize}
\item $\lambda = \lambda - \delta$, and goto 2.
\end{enumerate}

Figure~\ref{fig_cb1_synt} illustrates the result in the case where the
input image contains only lines and Gaussians. In this experiment, we have 
initialized  $L_{\max}$ to $20$, and $\delta$ to $2$ (10 iterations).
Two transform operators
were used, the \`a trous wavelet transform and the ridgelet transform. The 
first is well adapted to the detection of Gaussian due to the isotropy of
the wavelet function~\cite{starck:book98}, while the second is optimal
to represent lines \cite{cur:candes99_1}. Figure~\ref{fig_cb1_synt} top,
bottom left, and bottom right represents respectively the 
original image, the reconstructed image from the \`a trous wavelet
coefficient, and the reconstructed image from the ridgelet
coefficient. The addition of both reconstructed images reproduces the
original one. 

\begin{figure}[htb]
\vbox{
\centerline{
\hbox{
\psfig{figure=fig_cb2_line_g.ps,bbllx=2cm,bblly=13.5cm,bburx=13cm,bbury=24.5cm,width=5.5cm,height=5.5cm,clip=}}
}
\centerline{
\hbox{
\psfig{figure=fig_cb2_line_g_atrou.ps,bbllx=2cm,bblly=13.5cm,bburx=13cm,bbury=24.5cm,width=5.5cm,height=5.5cm,clip=}
\psfig{figure=fig_cb2_line_g_rid.ps,bbllx=2cm,bblly=13.5cm,bburx=13cm,bbury=24.5cm,width=5.5cm,height=5.5cm,clip=}
}}}
\caption{Top, original image containing lines and gaussians. Botton left,
reconstructed image for the \`a trous wavelet coefficient, bottom right,
reconstructed image from the ridgelet coefficients.}
\label{fig_cb1_synt}
\end{figure}

In some specific cases where the data are sparse in all bases, 
it has been shown \cite{cur:huo99,cur:donoho01}  
that the solution is identical to the solution when using a 
$\parallel . \parallel_0$ penalty term. This is however generally not the 
case. 
The problem we met in image restoration applications,
when minimizing equation~\ref{eqn_min_l1}, is 
that both the signal and noise are split into the bases. The way the noise
is distributed in the coefficients $\alpha_k$ is not known, and leads to the problem
that we do not know at which level we should threshold the coefficients. Using
the threshold we would have used with a single transform 
makes a strong over-filtering of the data. Using the $l^1$ optimization 
for data restoration implies to first study how the noise is 
distributed in the coefficients. The hard-MCA method does not present
this drawback.

\section{Hard-MCA}
\label{sect_l0}
%  $l^0$ optimization by hard thresholding
The following algorithm consists in hard thresholding the residual 
successively on the different bases \cite{starck:spie01b,starck:sta02_3}. 
\begin{enumerate}
\item For noise filtering, estimate the noise standard deviation $\sigma$,
and set $L_{\min} = k_{\sigma}$. 
Otherwise, set $\sigma=1$ and $L_{\min} = 0$.
\item Initialize $L_{\max}$, the number of iterations $N_i$, 
$\lambda = L_{\max}$ and $\delta_{\lambda} = \frac{L_{\max} - L_{\min}}{N_i}$.
\item Set all coefficients $\alpha_k$ to 0.
\item While $\lambda >= L_{\min}$ do
\item for k = 1, .., $K$ do
\begin{itemize}
\item Calculate the residual $R = s - \sum_k T_k^{-1} \alpha_k$.
\item Calculate the transform $T_k$ of the residual:
$r_k = T_k R$.
\item For all coefficients $\alpha_{k,i}$ do
\begin{itemize}
\item  Update the coefficients:  
 if $\alpha_{k,i} \ne 0$ or $\mid r_{k,i}  \mid > \lambda \sigma$ 
then $\alpha_{k,i}  =  \alpha_{k,i} + r_{k,i}$.
\end{itemize}
\end{itemize}
\item $\lambda = \lambda - \delta_{\lambda} $, and goto 5.
\end{enumerate}
For an exact representation of the data, $k_{\sigma}$ must be set to 0.
Choosing $k_{\sigma} > 0$ introduces a filtering. If a single transform is used,
it corresponds to the standard  $k$-sigma hard thresholding.

It seems that starting with a high enough $L_{\max}$ and a high
number of iterations would lead to the $l^0$ optimization solution,
but this remains to be proved.


\section{Experiments}

\subsubsection{Experiment 1: Infrared Gemini Data}
\begin{figure}[htb]
\centerline{
\vbox{
\hbox{
\psfig{figure=fig_gemini.ps,bbllx=1.8cm,bblly=12.5cm,bburx=14.8cm,bbury=25.5cm,width=5cm,height=5cm,clip=}
\psfig{figure=fig_gem_cb_rid.ps,bbllx=1.8cm,bblly=12.5cm,bburx=14.8cm,bbury=25.5cm,width=5cm,height=5cm,clip=}
\psfig{figure=fig_gem_cb_cur.ps,bbllx=1.8cm,bblly=12.5cm,bburx=14.8cm,bbury=25.5cm,width=5cm,height=5cm,clip=}
}
\hbox{
\psfig{figure=fig_gem_cb_atrou.ps,bbllx=1.8cm,bblly=12.5cm,bburx=14.8cm,bbury=25.5cm,width=5cm,height=5cm,clip=}
\psfig{figure=fig_gem_cb_resi_cur_rid_atrou.ps,bbllx=1.8cm,bblly=12.5cm,bburx=14.8cm,bbury=25.5cm,width=5cm,height=5cm,clip=}
\psfig{figure=fig_gem_cb_resi_cur_rid.ps,bbllx=1.8cm,bblly=12.5cm,bburx=14.8cm,bbury=25.5cm,width=5cm,height=5cm,clip=}
}
}}
\caption{Upper left, galaxy SBS 0335-052 (10 $\mu$m), upper middle, upper middle,
and bottom left, reconstruction respectively from the ridgelet, the curvelet
and wavelet coefficients. Bottom middle, residual image. Bottom right, 
artifact free image.}
\label{fig_ctm_gemini1}
\end{figure}
$ $ 
\begin{figure}[htb]
\centerline{
\vbox{
\hbox{
\psfig{figure=fig_gemini2_bw.ps,bbllx=1.8cm,bblly=12.5cm,bburx=14.8cm,bbury=25.5cm,width=7cm,height=7cm,clip=}
\psfig{figure=fig_gem2_cb_cur_rid_bw.ps,bbllx=1.8cm,bblly=12.5cm,bburx=14.8cm,bbury=25.5cm,width=7cm,height=7cm,clip=}
}
\hbox{
\psfig{figure=fig_gem2_cb_atrou_bw.ps,bbllx=1.8cm,bblly=12.5cm,bburx=14.8cm,bbury=25.5cm,width=7cm,height=7cm,clip=}
\psfig{figure=fig_gem2_cb_resi_cur_rid_atrou_bw.ps,bbllx=1.8cm,bblly=12.5cm,bburx=14.8cm,bbury=25.5cm,width=7cm,height=7cm,clip=}
}
}}
\caption{Upper left, galaxy SBS 0335-052 (20 $\mu$m), upper right, addition
of the reconstructed images from both the ridgelet and the curvelet coefficients,
bottom left, reconstruction from the wavelet coefficients, and bottom right,
residual image.}
\label{fig_ctm_gemini2}
\end{figure}
\clearpage
Fig.~\ref{fig_ctm_gemini1} upper left shows
a compact blue galaxy located at 53 Mpc. The data have been
obtained on ground with the GEMINI-OSCIR instrument at $10$ $\mu$m. 
The pixel field of
view is $0.089^{\prime\prime}$/pix, and the  source was observed during 1500s.
The data are contaminated by a noise and a stripping artifact due to the 
instrument electronic. The same kind of artifact pattern were observed with
the ISOCAM instrument \cite{starck:sta99_1}.

This image, noted $D_{10}$, has been decomposed using wavelets, ridgelets, and curvelets.
Fig.~\ref{fig_ctm_gemini1} upper middle,  upper  right, and bottom left
show the three images $R_{10}$, $C_{10}$, $W_{10}$
reconstructed respectively from the ridgelets, the curvelets, and the wavelets.
Image in Fig.~\ref{fig_ctm_gemini1} bottom middle shows the residual, i.e.
$e_{10} = D_{10} - (R_{10} + C_{10} + W_{10})$. Another interesting
image is the artifact free one, obtained by subtracting $R_{10}$ and
$C_{10}$ from the input data (see Fig.~\ref{fig_ctm_gemini1} bottom right).
The galaxy has well been detected in the wavelet space, while all stripping
artifact have been capted by the ridgelets and curvelets.

Fig.~\ref{fig_ctm_gemini2} upper left shows the same galaxy, but at
$20$ $\mu$m. We have applied the same decomposition on $D_{20}$. 
Fig.~\ref{fig_ctm_gemini2} upper right shows the coadded 
image $R_{20} + C_{20}$, and we can see bottom left and right the 
wavelet reconstruction $W_{20}$ and the residudal
$e_{20} = D_{20} - (R_{20} + C_{20} + W_{20})$.

\subsubsection{Experiment 2: A370}
 
\begin{figure}[htb]
\centerline{
\vbox{
\hbox{
\psfig{figure=a370.ps,bbllx=1.5cm,bblly=6cm,bburx=19.5cm,bbury=24cm,width=7cm,height=7cm,clip=}
\psfig{figure=a370_ridcur.ps,bbllx=1.5cm,bblly=6cm,bburx=19.5cm,bbury=24cm,width=7cm,height=7cm,clip=}
}
\hbox{
\psfig{figure=a370_atrou.ps,bbllx=1.5cm,bblly=6cm,bburx=19.5cm,bbury=24cm,width=7cm,height=7cm,clip=}
\psfig{figure=a370_comb.ps,bbllx=1.5cm,bblly=6cm,bburx=19.5cm,bbury=24cm,width=7cm,height=7cm,clip=}
}
}}
\caption{Top left, HST image of A370, top right coadded image from the
reconstructions from the ridgelet and the curvelet coefficients, bottom
left  reconstruction from  the \`a trous wavelet coefficients, 
and bottom right addition of the three reconstructed images. }
\label{fig_a370}
\end{figure}

Figure~\ref{fig_a370} upper left shows the HST A370 image. It contains many 
anisotropic features such the gravitationnal arc, and the arclets. The
image has been decomposed using three transforms: the ridgelet transform, 
the curvelet transform, and the \`a trous wavelet transform. Three images
have then been reconstructed from the coefficients of the three basis.
Figure~\ref{fig_a370} upper right shows the coaddition of the ridgelet 
and curvelet reconstructed images. The \`a trous reconstructed image
is displayed in Figure~\ref{fig_a370} lower left, and the coaddition 
of the three images can be seen in Figure~\ref{fig_a370} lower right.
The gravitational arc and the arclets are all represented in the 
ridgelet and the curvelet basis, while all isotropic features are 
better represented in the wavelet basis.

We can see that this Morphological Component Analysis (MCA) allows
us to separate automatically features in an image which have different
morphological aspects. It is very different from other techniques such as
Principal Component Analysis or 
Independent Component Analysis \cite{mc:cardoso98} where the separation
is performed via statistical properties.


\clearpage
\newpage
