\chapter{\projcur Programs}
% \section{Program}
\section{Ridgelet Transform}
\subsection{im\_radon}
\index{im\_radon}
Program {\em im\_radon} makes an (inverse-) Radon transform of a square
$n \times n$ image.
The output file which contains the transformation has a 
suffix, .rad. If the output file name
given by the user does not contain this suffix, it is automatically
added. The ``.rad'' file is a FITS format file, and can be manipulated by
any package implementing the FITS format, or can be converted to another
format using the {\em im\_convert} program.
For the two first Radon transform methods, the user can change the
number of directions and the resolution. For other methods, the 
number of directions and the resolution are fixed, and the
x and y options are not valid. Options f, w and s allow the user to perform
a filtering-backprojection. They are valid only when the selected Radon
method is the second one. ``w'' fixes the width of the filter and the
``s'' is the sigma parameter of the Gaussian filter.
{\bf
\begin{center}
 USAGE:  im\_radon  options image\_in trans\_out
\end{center}}
where options are:
\begin{itemize}
\baselineskip=0.4truecm
\itemsep=0.1truecm
\item {\bf [-m type\_of\_radon\_method]}  
{\small 
\begin{enumerate}
\baselineskip=0.4truecm
\itemsep=0.1truecm
\item  Radon transform (resp.\ backprojection) in spatial domain. \\
       By default, the output image is a $2n \times n$ image.
\item  Radon projection in spatial domain and reconstruction in Fourier domain.
       By default, the output image is a $2n \times n$ image.
       The reconstruction is available only for image with
       a size $n$ being a power of 2.
\item  Radon transformation and reconstruction in Fourier space 
       (i.e.\ Linogram). \\
       The output is a 
       $2n \times n$ image. The number of rows is multiplied by two.
\item  Finite Radon Transform. \\
       The output image is a $(n+1) \times n$ image. The input image size $n$
       must be a prime number.
\item  Slant Stack Radon transform. \\
       The output image has twice the number of rows and twice the number
       of columns of the input image.
       The output is a $2n \times 2n$ image.
       The reconstruction is not available with this transform.
\end{enumerate}}
Default is Radon transformation and reconstruction in Fourier space.

\item {\bf [-y OutputLineNumber]} \\
For the RADON transform, OutputLineNumber = number of projection,
 and default is twice the number of input image rows. Only valid for Radon
methods 1 and 2.\\
% For the inverse RADON transform, OutputLineNumber = number of lines,
% and default is the input image column number.

\item {\bf [-x OutputColumnNumber]} \\
For the RADON transform, OutputLineNumber = number of pixels per projection,
and default is the input image column number. Only valid for Radon
methods 1 and 2.\\
% For the inv. RADON transform, OutputLineNumber = number of column,
% and default is the input image column number.

\item {\bf [-r]} \\
Inverse Radon transform.
\item {\bf [-f]} \\
Filter each scan of the Radon transform. Only valid for Radon
method 2.
\item {\bf [-w FilterWidth]} \\
Filter width. Only valid for Radon
method 2. Default is 100. 
\item {\bf [-s SigmaParameter} \\
 Sigma parameter for the filtering. Only valid for Radon
method 2. Default is 10.
\item {\bf [-v]} \\
Verbose. Default is no
\end{itemize}

\subsubsection*{Examples:}
\begin{itemize}
\item im\_radon image.fits trans\\
Apply the Radon transform to an image.
\item im\_info -r trans.rad rec\\
 Reconstruct an image from its Radon transform.
\end{itemize}

\subsection{rid\_trans}
\index{rid\_trans}
Program {\em rid\_trans} makes the  ridgelet transform 
(and the inverse when -r option is set).  
The output file which contains the transformation has a 
suffix, .rid. If the output file name
given by the user does not contain this suffix, it is automatically
added. The ``.rid'' file is a FITS format file, and can be manipulated by
any package implementing the FITS format, or can be converted to another
format using the {\em im\_convert} program.
The default transform is the second one 
(RectoPolar Ridgelets using a  FFT based WT). 
The two first transform are based on the RectoPolar
(i.e. linogram) radon transform, but the first applies a standard 
bi-orthogonal wavelet transform (WT) on the Radon image rows, while the
second uses the Fourier-based WT which introduces a redundancy of 2.
For an $n \times n$ image, the output has $2n$ lines and $n$ column 
with the first transform, and is a $2n \times 2n$ image for the second.
When the overlapping is set, the size is doubled in each direction.

{\bf
\begin{center}
 USAGE: rid\_trans options image\_in trans\_out
\end{center}}
where options are:

\begin{itemize}
\baselineskip=0.4truecm
\itemsep=0.1truecm
\item {\bf [-t type\_of\_ridgelet]}  
\begin{enumerate}
\baselineskip=0.4truecm
\itemsep=0.1truecm
\item RectoPolar Ridgelet Transform using a standard bi-orthogonal WT.
\item RectoPolar Ridgelet Transform using a FFT based Pyramidal WT.
\item Finite ridgelet transform.
\end{enumerate}
Default is 2.
\item {\bf [-n number\_of\_scale]} \\
 Number of scales used in the wavelet transform.
 Default is automatically calculated.
\item {\bf [-b BlockSize]} \\
Block Size. Default is image size.
\item {\bf [-i]} \\
Print statistical information about each band. Default is no. 
\item {\bf [-O]} \\
 No block overlapping. Default is no. When this option is set, the 
 number of rows and columns is multiplied by two. 
\item {\bf [-r]} \\
Inverse Ridgelet transform.
\item {\bf [-x]} \\
 Write all bands separately as images in the FITS format with prefix 'band\_j' 
(j being the band number).
\item {\bf [-v]} \\
Verbose. Default is no
\end{itemize}

\subsubsection*{Examples:}
\begin{itemize}
\item rid\_transform image.fits trans\\
Apply the Ridgelet transform to an image.
\item rid\_transform -r trans.rid rec\\
 Reconstruct an image from its Ridgelet transform.
\end{itemize}

\subsection{rid\_stat}
\index{rid\_stat}
Program {\em rid\_stat} makes the  ridgelet transform, and 
gives statistical information on the ridgelet coefficients.
At each scale, it caculates the standard deviation, the skewness,
the kurtosis, the minimum, and the maximum. The output file is a 
fits file containing a two-dimensional array $T[J-1,5]$ ($J$ being the
number of scales), with the following syntax:
\begin{itemize}
\baselineskip=0.4truecm
\itemsep=0.1truecm
\item $T[j,0] = $ standard deviation of the jth ridgelet band.
\item $T[j,1] = $ skewness of the jth ridgelet band.
\item $T[j,2] = $ kurtosis of the jth ridgelet band.
\item $T[j,3] = $ minimum of the jth ridgelet band.
\item $T[j,4] = $ maximum of the jth ridgelet band.
\end{itemize}
The last ridgelet scale is not used.
If the ``-A'' option is set, these statistics are calculated only for 
the ridgelet coefficients relative the specified angle.
{\bf
\begin{center}
 USAGE: rid\_stat options image\_in trans\_out
\end{center}}
where options are:

\begin{itemize}
\baselineskip=0.4truecm
\itemsep=0.1truecm
\item {\bf [-t type\_of\_ridgelet]}  
\begin{enumerate}
\baselineskip=0.4truecm
\itemsep=0.1truecm
\item RectoPolar Ridgelet Transform using a standard bi-orthogonal WT.
\item RectoPolar Ridgelet Transform using a FFT based Pyramidal WT.
\item Finite ridgelet transform.
\end{enumerate}
Default is 2.
\item {\bf [-n number\_of\_scale]} \\
 Number of scales used in the wavelet transform.
 Default is automatically calculated.
\item {\bf [-b BlockSize]} \\
Block Size. Default is 16.
\item {\bf [-O]} \\
   Use overlapping block. Default is no.
\item {\bf [-A Angle]} \\
 Statistics for a given angle. The value must be given  in degrees.
 Default is no, statistics are calculated from all coefficients.
\item {\bf [-v]} \\
Verbose. Default is no
\end{itemize}

\subsubsection*{Example:}
\begin{itemize}
\item rid\_stat -v  image.fits tabstat\\
\end{itemize}

\subsection{rid\_filter}

Program {\em rid\_filter} filters an image using the ridgelet transform.
\begin{center}
 USAGE:  rid\_filter options image\_in imag\_out
\end{center}
where options are 
\begin{itemize}
\baselineskip=0.4truecm
\itemsep=0.1truecm
\item {\bf [-t type\_of\_ridgelet]}  
\begin{enumerate}
\baselineskip=0.4truecm
\itemsep=0.1truecm
\item RectoPolar Ridgelet Transform using a standard bi-orthogonal WT.
\item RectoPolar Ridgelet Transform using a FFT based pyramidal WT.
\item Finite ridgelet transform.
\end{enumerate}
Default is 2.
\item {\bf [-n number\_of\_scale]} \\
 Number of scales used in the wavelet transform.
 Default is automatically calculated.
\item {\bf [-h]} \\
Apply the ridgelet transform only on the high frequencies.
Default is no.

\item {\bf [-b BlockSize]} \\
Block Size. Default is image size.

\item {\bf [-F FirstDetectionScale]} \\
 First detection scale. Default is 1. 

% \item {\bf [-i NbrIter]}  \\
%  Number of iteration for the constraint reconstruction.
%   Default is no. 
% \item {\bf [-G RegulParam]}  \\
%   Regularization parameter for the constraint reconstruction.
%  Default is 0.2.
% \item {\bf [-C ConvergParam]}  \\
%  Convergence parameter. Default is 1.

\item {\bf [-s Nsigma]} \\
False detection rate. The false detection rate for a detection is given
\begin{eqnarray}
\epsilon =  \mbox{erfc}( NSigma / \sqrt{2})
\end{eqnarray}
{\em Nsigma} parameter allows us to express the false detection rate
even if it is not Gaussian noise. \\
Default is 3.

\item {\bf [-g sigma]} \\
Gaussian noise: sigma = noise standard deviation.  \\
 Default is automatically estimated.

\item {\bf [-p]} \\
Poisson noise.

\item {\bf [-O]}  \\
Do not apply block overlapping. By default, block overlapping is used.

\item {\bf [-v]} \\
Verbose.
\end{itemize}
\subsubsection*{Examples:}
\begin{itemize}
\item rid\_filter  image.fits fima\\
Filter an image using all default options.
\item rid\_filter -h -s5 image.fits fima\\
Five sigma filtering, filtering only the high frequencies.
\end{itemize}


\section{Curvelet Transform}
\subsection{cur\_trans}
\index{cur\_trans}
Program {\em cur\_trans} determines the  curvelet transform 
(and the inverse when -r option is set).  
The output file which contains the transformation has a 
suffix, .cur. If the output file name
given by the user does not contain this suffix, it is automatically
added. The ``.cur'' file is a 3D FITS format file, and can be manipulated by
any package implementing the FITS format.
The curvelet transform uses the ridgelet transform, and the default 
ridgelet transform is the RectoPolar one with a FFT based pyramidal WT.

{\bf
\begin{center}
 USAGE: cur\_trans options image\_in trans\_out
\end{center}}
where options are:
\begin{itemize}
\baselineskip=0.4truecm
\itemsep=0.1truecm
\item {\bf [-t type\_of\_ridgelet]}  
\begin{enumerate}
\baselineskip=0.4truecm
\item RectoPolar Ridgelet Transform using a standard bi-orthogonal WT.
\item RectoPolar Ridgelet Transform using a FFT based pyramidal WT.
\item Finite ridgelet transform.
\end{enumerate}
Default is 2.
\item {\bf [-n number\_of\_scale]} \\
 Number of scales used in the 2D wavelet transform.
 Default is 4. 
\item {\bf [-N number\_of\_scale]} \\
 Number of scales used in the ridgelet transform.
 Default is automatically calculated.
\item {\bf [-b BlockSize]}  \\
Block Size. Default is 16.
\item {\bf [-r]}  \\
Inverse Curvelet transform.
\item {\bf [-i]}  \\
Print statistical information about each band. Default is no. 
\item {\bf [-O]}  \\
 Block overlapping. Default is no. 
\item {\bf [-x]} \\
 Write all bands separately as images in the FITS format with prefix 'band\_j' 
(j being the band number).
\item {\bf [-v]} \\
Verbose. Default is no.
\end{itemize}

\subsubsection*{Examples:}
\begin{itemize}
\item cur\_trans -i image.fits trans\\
Curvelet transform of an image.
\item cur\_trans -r  trans.cur rec\\
Image reconstruction from its curvelet transform.
\end{itemize}


\subsection{cur\_stat}
\index{cur\_stat}
Program {\em cur\_stat} determines the  curvelet transform, and 
gives statistical information on the curvelet coefficients.
At each scale, it caculates the standard deviation, the skewness,
the kurtosis, the minimum, and the maximum. The output file is a 
FITS file containing a two-dimensional array $T[J-1,5]$ ($J$ being the
number of bands), with the following syntax:
\begin{itemize}
\baselineskip=0.4truecm
\itemsep=0.1truecm
\item $T[j,0] = $ standard deviation of the jth ridgelet band.
\item $T[j,1] = $ skewness of the jth ridgelet band.
\item $T[j,2] = $ kurtosis of the jth ridgelet band.
\item $T[j,3] = $ minimum of the jth ridgelet band.
\item $T[j,4] = $ maximum of the jth ridgelet band.
\end{itemize}
{\bf
\begin{center}
 USAGE: cur\_stat options image\_in trans\_out
\end{center}}
where options are:
\begin{itemize}
\baselineskip=0.4truecm
\item {\bf [-n number\_of\_scale]} \\
 Number of scales used in the wavelet transform.
 Default is automatically calculated.
\item {\bf [-b BlockSize]} \\
Block Size. Default is 16.
\item {\bf [-O]} \\
   Use overlapping block. Default is no.
\item {\bf [-v]} \\
Verbose. Default is no
\end{itemize}

\subsubsection*{Example:}
\begin{itemize}
\item cur\_stat -v  image.fits tabstat\\
\end{itemize}


\subsection{cur\_filter}
\index{cur\_filter}

Program {\em cur\_filter} filters an image using the curvelet transform.
\begin{center}
 USAGE:  cur\_filter options image\_in imag\_out
\end{center}
where options are 
\begin{itemize}
\baselineskip=0.4truecm
\itemsep=0.1truecm
\item {\bf [-t type\_of\_ridgelet]} 
\begin{enumerate}
\baselineskip=0.4truecm
\itemsep=0.1truecm
\item RectoPolar Ridgelet Transform using a standard bi-orthogonal WT.
\item RectoPolar Ridgelet Transform using a FFT based Pyramidal WT.
\item Finite ridgelet transform.
\end{enumerate}
Default is 2.
\item {\bf [-n number\_of\_scale]} \\
 Number of scales used in the 2D wavelet transform.
 Default is 4. 

\item {\bf [-N number\_of\_scale]} \\
 Number of scales used in the ridgelet transform.
 Default is automatically calculated.

\item {\bf [-b BlockSize]}  \\
Block Size. Default is 16.

\item {\bf [-g sigma]} \\
Gaussian noise: sigma = noise standard deviation.  \\
 Default is automatically estimated.

\item {\bf [-s Nsigma]} \\
False detection rate. \\
Default is 3.

\item {\bf [-O]}  \\
Do not apply block overlapping. By default, block overlapping is used.

\item {\bf [-P]}  \\
 Supress the positivity constraint. Default is no. 

\item {\bf [-I NoiseFileName]}  \\
If the noise is stationary, the program can estimate the correct 
thresholds from a realization of the noise.

\item {\bf [-v]} \\
Verbose
\end{itemize}

\subsubsection*{Examples:}
\begin{itemize}
\item cur\_filter image.fits sol\\
Curvelet filtering of an image.
\item cur\_filter -n 5 -s4  image.fits sol\\
Curvelet filtering of an image, using five resolution levels, and
a 4-sigma detection.
\end{itemize}


\subsection{cur\_colfilter}
\index{cur\_colfilter}

Program {\em cur\_colfilter} filters a
color image using the curvelet transform.
\begin{center}
 USAGE:  cur\_colfilter options image\_in imag\_out
\end{center}
where options are 
\begin{itemize}
\baselineskip=0.4truecm
\itemsep=0.1truecm
\item {\bf [-n number\_of\_scale]} \\
 Number of scales used in the 2D wavelet transform.
 Default is 4. 

\item {\bf [-N number\_of\_scale]} \\
 Number of scales used in the ridgelet transform.
 Default is automatically calculated.

\item {\bf [-b BlockSize]}  \\
Block Size. Default is 16.

\item {\bf [-g sigma]} \\
Gaussian noise: sigma = noise standard deviation.  \\
 Default is automatically estimated.

\item {\bf [-s Nsigma]} \\
False detection rate. \\
Default is 3.

\item {\bf [-O]}  \\
Do not apply block overlapping. By default, block overlapping is used.

\item {\bf [-v]} \\
Verbose
\end{itemize}

\subsubsection*{Example:}
\begin{itemize}
\item cur\_colfilter image.fits sol\\
Curvelet transform of a color image.
\end{itemize}


\subsection{cur\_contrast}
\index{cur\_contrast}

Program {\em cur\_contrast} filters a color image using the curvelet transform.
\begin{center}
 USAGE:  cur\_contrast options image\_in imag\_out
\end{center}
where options are 
\begin{itemize}
\baselineskip=0.4truecm
\itemsep=0.1truecm
\item {\bf [-n number\_of\_scales]} \\
Number of scales used in the wavelet transform.
Default is 4. 
\item {\bf [-N number\_of\_scales]} \\
Number of scales used in the ridgelet transform.
Default is automatically calculated.
\item {\bf [-b BlockSize]} \\
Block size used by the curvelet transform. Default is 16.
\item {\bf [-O]} \\
Use overlapping block. Default is no.
\item {\bf [-g sigma]} \\
Noise standard deviation. Only used when filtering is performed.
Default is automatically estimated.
\item {\bf [-s NSigmalLow]} \\
 Coefficient $<$ NSigmalLow*SigmaNoise is not modified.
 Default is   5.
\item {\bf [-S NSigmalUp]} \\
 Coefficient $>$ NSigmalUp*SigmaNoise is not modified.
 Default is  20.
\item {\bf [-M MaxCoeff]} \\
If MaxBandCoef is the maximum coefficient in a given curvelet band,
 Coefficient $>$ MaxBandCoef*MaxCoeff is not modified.
 Default is 0.5.
\item {\bf  [-P P\_parameter]} \\
Determine the degree on non-linearity. P must be in ]0,1[.  
Default is 0.5.
\item {\bf [-T P\_parameter]} \\
 Curvelet coefficent saturation parameter. T must be in [0,1].  
Default is 0.
\item {\bf [-c]} \\
By default a sigma clipping is performed. When this option is set, no
sigma clipping is performed.
\item {\bf [-K ClippingValue]} \\
Clipping value. Default is 3.
\item {\bf [-L Saturation]} \\
Saturate the reconstructed image.
A coefficient larger than Saturation*MaxData is set to Saturation*MaxData.
Default is  1. If L is set to 0, then no saturation is applyied.
\end{itemize}

\subsubsection*{Examples:}
\begin{itemize}
\baselineskip=0.4truecm
\itemsep=0.1truecm
\item cur\_contrast image.fits image\_out.fits\\
Enhance the contrast using the curvelet transform.
\item cur\_contrast -O image.fits image\_out.fits\\
Enhance the contrast using the curvelet transform and block overlapping.
\item cur\_contrast -M 0.8 image.fits image\_out.fits\\
Enhance more the contrast.
\end{itemize}

\subsection{cur\_colcontrast}
\index{col\_colcontrast}
The program {\em col\_colcontrast} enhances the contrast of a color image
using the curvelet transform.
The command line is:
{\bf
\begin{center}
 USAGE: cur\_colcontrast option in\_image out\_image
\end{center}}
where options are:
\begin{itemize}
\baselineskip=0.4truecm
\itemsep=0.1truecm
\item {\bf [-n number\_of\_scales]} \\
Number of scales used in the wavelet transform.
Default is 4. 
\item {\bf [-N number\_of\_scales]} \\
Number of scales used in the ridgelet transform.
Default is automatically calculated.
\item {\bf [-b BlockSize]} \\
Block size used by the curvelet transform. Default is 16.
\item {\bf [-O]} \\
Use overlapping block. Default is no.
\item {\bf [-g sigma]} \\
Noise standard deviation.  
Default is automatically estimated.
\item {\bf [-s NSigmalLow]} \\
 Coefficient $<$ NSigmalLow*SigmaNoise is not modified.
 Default is   5.
\item {\bf [-S NSigmalUp]} \\
 Coefficient $>$ NSigmalUp*SigmaNoise is not modified.
 Default is  20.
\item {\bf [-M MaxCoeff]} \\
If MaxBandCoef is the maximum coefficient in a given curvelet band,
 Coefficient $>$ MaxBandCoef*MaxCoeff is not modified.
 Default is 0.5.
 \item {\bf  [-P P\_parameter]} \\
Determine the degree of non-linearity. P must be in ]0,1[.  
Default is 0.5.
\item {\bf [-T P\_parameter]} \\
 Curvelet coefficent saturation parameter. T must be in [0,1].  
Default is 0.
\item {\bf [-c]} \\
By default a sigma clipping is performed. When this option is set, no
sigma clipping is performed.
\item {\bf [-K ClippingValue]} \\
Clipping value. Default is 3.
\item {\bf [-L Luminance\_Saturation]} \\
Values in the luminance map which are 
larger than Saturation*MaxData are set to Saturation*MaxData.
Default is  1. 
\end{itemize}

\subsubsection*{Examples:}
\begin{itemize}
\baselineskip=0.4truecm
\itemsep=0.1truecm
\item cur\_colcontrast image.tiff image\_out.tiff\\
Enhance the contrast using the curvelet transform.
\item cur\_colcontrast -n 5 image.tiff image\_out.tiff\\
Ditto, but use five scales instead of four.
\item cur\_colcontrast -M 0.9 image.tiff image\_out.tiff\\
Enhance more the contrast.
\end{itemize}


\section{Combined Filtering}
\subsection{cb\_filter}
\index{cb\_filter}
Program {\em cb\_filter} filters an image corrupted by Gaussian noise by
 the combined filtering method. By default, the undecimated  bi-orthogonal WT
 and the curvelet transform are used. The number of iterations is defaulted
 10. In general, the algorithm converges with less than six iterations.
If the ``-T'' option is set, the Total Variation is minimized instead of 
the $l_1$ norm of the multiscale coefficients.
A deconvolution can also be performed using the ``-P'' option. In this
case, a division is first done in Fourier space between the 
input image and the point spread function. All Fourier components with
a norm lower than $\epsilon$ (default value is $10^{-3}$) are set to zero.
Then the deconvolved image is filtered by the combined filtering method
using the new noise properties (still Gaussian, but not white).

{\bf
\begin{center}
 USAGE: cb\_filter options image\_in trans\_out
\end{center}}
where options are:
\begin{itemize}
\baselineskip=0.4truecm
\itemsep=0.1truecm
 \item {\bf [-t TransformSelection]}
\begin{enumerate}
\baselineskip=0.4truecm
\itemsep=0.1truecm
\item A trous algorithm
\item Bi-orthogonal WT with 7/9 filters
\item Ridgelet transform
\item Curvelet transform
\item Mirror Basis WT
% \item Multiscale Ridgelet
%\item Cosinus transform
%\item Pyramidal Median transform
\end{enumerate}

\item {\bf [-n number\_of\_scales]} \\
Number of scales used in the \`a trous wavelet transform 
%, the PMT and
 and the curvelet transform. 
% Number of ridgelet in the multi-ridgelet transform.
Default is 4.

\item {\bf [-b BlockSize]}  \\
 Block Size in the ridgelet transform.
Default is image size.  
% Starting Block Size in the multi-ridgelet transform. Default is 8. 

\item {\bf [-i NbrIter]}  \\
Number of iterations. Default is 10.

\item {\bf [-F FirstDetectionScale]} \\
First detection scale in the ridgelet transform.
Default is 1. 

\item {\bf [-k]} \\
Kill the last scale in ridgelet. % , and multiscale ridgelet transform.
Default no.

\item {\bf  [-K]} \\
Kill last scale in the \`a trous algorithm and the curvelet. % and the PMT
Default no.

\item {\bf  [-L FirstSoftThreshold]} \\
First soft thresholding value. Default is 0.5.

\item {\bf  [-l LastSoftThreshold]} \\
Last soft thresholding value. Default is 0.5.

\item {\bf  [-u]} \\
 Number of undecimated scales in the WT.
 Default is 1. 

\item {\bf [-s Nsigma]} \\
False detection rate. Default is 4.

\item {\bf [-g sigma]} \\
Gaussian noise: sigma = noise standard deviation.  \\
 Default is automatically estimated.

% \item {\bf [-p]} \\
% Poisson Noise. Default is no (Gaussian).

\item {\bf [-O]}  \\
No block overlapping. Default is no.

\item {\bf [-T]}  \\
 Minimize the Total Variation instead of the L1 norm. 

\item {\bf [-P PsfFile]}  \\
Apply a deconvolution using the PSF in the file PsfFile. 

\item {\bf [-e Eps]}  \\
 Remove frequencies with $|P P^*| < \epsilon $. 
 Default is $10^{-3}$.

\item {\bf [-C TolCoef]}  \\
 Default is 0.5. 

\item {\bf [-v]} \\
Verbose. Default is no.
\end{itemize}

\subsubsection*{Example:}
\begin{itemize}
\item cb\_filter image.fits sol\\
Image filtering by the combined filtering method, using both
the curvelet and wavelet transform.
\end{itemize}



