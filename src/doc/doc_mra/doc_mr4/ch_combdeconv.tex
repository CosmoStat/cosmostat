\chapter{Deconvolution}
\section{Wavelet and Deconvolution}
\label{wdec}
Consider an image characterized by its intensity
distribution $I$, corresponding to the observation of a
``real image'' $O$ through an optical system. If the
imaging system is linear and shift-invariant, the relation between
the data and the image in the same coordinate frame is a
convolution:
$I(x,y) = (P * O)(x,y) + N(x,y)$,
where
$P$ is the point spread function (PSF) of the imaging system, and $N$
is additive noise. We want to determine $O(x,y)$ knowing $I$ and $P$. This
inverse problem has led to a large amount of work, the main difficulties 
being the existence of: (i) a cut-off frequency of the 
PSF, 
and (ii) the additive noise (see for example \cite{ima:bertero98}).

The wavelet based non-iterative algorithm, 
the wavelet-vaguelette decomposition \cite{rest:donoho95b},
consists of first applying an inverse filtering
($F = P^{-1} * I  +  P^{-1} * N = O + Z$
where $\hat{P}^{-1}(\nu) = \frac{1}{\hat{P}(\nu)}$). 
The noise $Z =  P^{-1} * N$ is not white but remains 
Gaussian. It is amplified when the deconvolution
problem is unstable. 
Then, a wavelet transform is applied on $F$, the wavelet coefficients
are soft or hard thresholded \cite{rest:donoho93_2}, 
and the inverse wavelet transform 
furnishes the solution. 

The method has been refined by 
adapting the wavelet basis to the frequency response of the inverse of $P$
\cite{rest:kalifa99}. This leads to a special basis,
the {\em Mirror Wavelet Basis}.  This basis has a 
time-frequency tiling structure different from the conventional wavelets one.
It isolates the frequency $\nu_s$ where $\hat{P}$ is close to zero, because 
a singularity in $\hat{P}^{-1}(\nu_s)$ influences the noise variance in
the wavelet scale corresponding to the frequency band which includes $\nu_s$.
Because it may not be possible to isolate all singularities, Neelamani
\cite{rest:neelamani99} has advocated a hybrid approach,
and proposes to still use the Fourier domain to restrict excessive noise
amplification. These approaches are fast and competitive compared to linear methods, and
the wavelet thresholding removes the Gibbs oscillations. 
This  presents however several drawbacks:
(i) the first step (division in the Fourier space by the PSF) 
cannot always be done properly,
(ii) the positivity a priori is not used, and
(iii) it is not trivial to consider non-Gaussian noise.

As an alternative, 
several wavelet-based iterative algorithms have been proposed 
 \cite{starck:book98}, especially in the astronomical
domain where the positivity a priori is known to improve 
significantly the result. The simplest method consists of first 
estimating the multiresolution support $M$ (i.e. $M(j,x,y)= 1$ if   
the wavelet transform of the data presents a significant coefficient 
at band $j$ and at pixel position $(x,y)$, and $0$ otherwise), and to
apply the following iterative scheme:
\begin{equation}
O^{n+1} = O^{n} + P^* * \cW^{-1}[M.\cW (I - P * O^n)]
\end{equation}
where $\cW$ is the wavelet transform operator.
At each iteration, information is extracted from the 
residual only at scales and positions defined by the multiresolution support.
$M$ is estimated from the input data and the correct noise modeling
can easily be considered.

 
\section{The Combined Deconvolution Method}
\label{cbdec}
 Similar to the filtering, we expect that the combination 
of different transforms can improve the quality of the result.
The combined approach for the deconvolution leads to two different 
methods. 

If the noise is Gaussian and if the division by the PSF in the Fourier space 
can be carried out properly, then the deconvolution problem becomes a 
filtering problem where the noise is still Gaussian, but not white.
The Combined Filtering Algorithm can then be applied using the curvelet 
transform and the wavelet transform, but by estimating first the 
correct thresholds in the different bands of both transforms.  
Since the mirror wavelet basis is known to produce better results than
the wavelet basis, it is recommended to use it instead of the standard
undecimated wavelet transform.

An iterative deconvolution method 
 is more general and can always be applied. 
Furthermore,
the correct noise modeling can much more easily be taken into account.
This approach consists of detecting, first, all the significant coefficients
with all multiscale transforms used. If we use $K$ transforms
$T_1, \dots, T_K$, we derive $K$ multiresolution supports $M_1, \dots, M_K$
from the input image $I$ using noise modeling.

For instance, in the case of Poisson noise, we apply the Anscombe transform
to the data (i.e. $\cA(I) = 2 \sqrt{I + \frac{3}{8}}$). Then
we detect the significant coefficients with the kth transform $T_k$, 
assuming Gaussian noise with standard deviation equal to 1,
in $T_k \cA(I)$ instead of $T_k I$.  $M_k(j,x,y) = 1$ if a coefficient
in band $j$ at pixel position $(x,y)$ is detected , and  $M_k(j,x,y) = 0$
otherwise. For the band $J$ which corresponds to the smooth array
in transforms such as the wavelet or the curvelet transform, 
we force $M_k(J,x,y) = 1$ for all $(x,y)$.

Following determination of a set of multiresolution supports, 
we propose to solve
the following optimization problem:
\begin{equation}
  \label{eq:dec-min}
  \min \cS(\tilde O), \quad \mbox{subject to} \quad \tilde O \in C,  
\end{equation}
where $\cS$ is an edge preservation penalization term defined by:
\[ \cS(\tilde O) = \int \parallel \nabla \tilde O \parallel_p, \] 
with $p=1.1$.
 $C$ is the set of images $\tilde{O}$ 
which obey the two constraints:
\begin{enumerate}
\item $\tilde{O} \ge 0$ (positivity).
\item $ M_k T_k I = M_k T_k [P *\tilde{O}]$, for all $k$. 
\end{enumerate}
The second constraint
imposes fidelity to the data, or more exactly, 
to the significant coefficients of the data, 
obtained by the different transforms.
Non-significant (i.e. noisy) coefficients are not taken into account, 
preventing any noise amplification in the final algorithm.

The solution is computed by using the 
projected Landweber method \cite{ima:bertero98}:
\begin{eqnarray}
\tilde O^{n+1} = {\cP}_c \left[ 
\tilde O^n + \alpha( P^* * {\bar R}^n - \lambda \frac{\partial \cS(\tilde O)}{\partial O}) \right]
\end{eqnarray}
where ${\cP}_c$ is the projection operator which enforces the positivity
(i.e. set to 0 all negative values).
${\bar R}^n$ is the significant residual which
 is obtained using the following algorithm:
\begin{itemize}
\item Set $I^n_{0} = I^n = P * \tilde O^n$.
\item For $k=1,\dots,K$ do 
$
I^n_k = I^n_{k-1} + T_k^{-1} \left[ M_k (T_k I - T_k I^n_{k-1}) \right] \nonumber
$
\item The significant residual ${\bar R}^n$ is obtained by:
  $ {\bar R}^n = I^n_K - I^n $.
\end{itemize}

$\alpha$ is a convergence parameter and $\lambda$ is the 
regularization hyperparameter. Since the noise is controlled by the
multiscale transforms, the regularization parameter does not   
have the same importance as in standard deconvolution methods.
A much lower value is enough to remove the artifacts relative to the
use of the wavelets and the curvelets. The positivity constraint can
be applied at each iteration.

\begin{figure}[htb]
\vbox{
\centerline{
\hbox{
\psfig{figure=phantom.ps,bbllx=8.2cm,bblly=12.6cm,bburx=12.7cm,bbury=17.1cm,width=7cm,height=7cm,clip=}
\psfig{figure=phantom_p.ps,bbllx=8.2cm,bblly=12.6cm,bburx=12.7cm,bbury=17.1cm,width=7cm,height=7cm,clip=}
}}
\centerline{
\hbox{
\psfig{figure=dec_mr_t24_p_n5_i400_s5.ps,bbllx=8.2cm,bblly=12.6cm,bburx=12.7cm,bbury=17.1cm,width=7cm,height=7cm,clip=}
\psfig{figure=dec_cb_s5_i400.ps,bbllx=8.2cm,bblly=12.6cm,bburx=12.7cm,bbury=17.1cm,width=7cm,height=7cm,clip=}
}}
}
\caption{Top, original image (phantom) and simulated data
(i.e. convolved image plus Poisson noise). Bottom, deconvolved image
by the wavelet based method and the combined approach.}
\label{fig_dec_phantom}
\end{figure}
Figure~\ref{fig_dec_phantom}, top, shows the Logan-Shepp Phantom 
and the simulated data, i.e. original image convolved by a Gaussian 
PSF (full width at half maximum,
FWHM=3.2) and Poisson noise. Figure~\ref{fig_dec_phantom}, 
bottom, shows the deconvolution with  (left) a pure wavelet deconvolution
method (no penalization term) and (right) the combined deconvolution method
(parameter $\lambda = 0.4$).


\clearpage
\newpage
