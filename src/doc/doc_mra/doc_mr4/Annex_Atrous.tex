\section*{Appendix B: The `` \`A Trous'' Wavelet Transform Algorithm}
\addcontentsline{toc}{section}{Appendix B: The `` \`A Trous'' Wavelet Transform Algorithm}

In a wavelet transform, a series of transformations of an image is 
generated, providing a resolution-related set of  ``views'' of the image.  
The properties satisfied by a wavelet transform, and in particular the
{\it \`a trous} wavelet transform (``with holes'', so called because of the 
interlaced
convolution used in successive levels: see step 2 of the algorithm below) 
are further discussed in \cite{starck:book98}. 

We consider sampled data, 
$\{c_0(k)\}$, defined as  the scalar product at 
pixels $k$ of the function $f(x)$ with a scaling function $\phi(x)$
which corresponds to a low pass band filter:
\begin{eqnarray}
c_0(k) = < f(x), \phi(x-k)>
\end{eqnarray}
The scaling function is chosen to satisfy the dilation equation:
\begin{eqnarray}
\frac{1}{2}\phi(\frac{x}{2}) = \sum_k h(k)\phi(x-k)
\end{eqnarray}
 
$h$ is a discrete low pass filter associated with the scaling function
$\phi$.  This means that a low-pass filtering
of the image is, by definition, closely linked to another resolution level
of the image.   
 
The smoothed data $c_j(k)$ at a given resolution $j$ and at a position
$k$  is the scalar product 
 
\begin{eqnarray}
c_j(k)= \frac{1}{2^j}< f(x), \phi(\frac{x-k}{2^j})>
\end{eqnarray}
 
This is consequently obtained by the convolution:
\begin{eqnarray}
c_j(k) = \sum_l h(l) \ \ c_{j-1} (k+2^{j-1}l)
\end{eqnarray}
The signal difference $w_j$ between two consecutive resolutions is:
\begin{eqnarray}
w_j(k) = c_{j-1}(k) - c_j(k) 
\end{eqnarray}
or:
\begin{eqnarray}
w_j(k) = \frac{1}{2^j}< f(x), \psi(\frac{x-k}{2^j})>  
\end{eqnarray}
Here, the wavelet function $\psi$ is defined by:
\begin{eqnarray}
\frac{1}{2}\psi(\frac{x}{2})  = \phi(x) - \frac{1}{2}\phi(\frac{x}{2})
\end{eqnarray}
 
For the scaling function, $\phi(x)$, the B-spline of degree 3 
was used in our calculations.     
We have derived a simple algorithm in order to compute the 
associated wavelet transform:
\begin{enumerate}
\item We initialize $j$ to 0 and we start with the data $c_j(k)$.
\item We increment $j$, and we carry out a discrete convolution of the data
$c_{j-1}(k)$ using  the filter $h$. The distance between the central pixel
and the adjacent ones is $2^{j-1}$.
\item After this smoothing, we obtain the discrete wavelet transform
from the difference $c_{j-1}(k) - c_j(k)$.
\item If $j$ is less than the number $p$ of resolutions we want to
compute, then go to step 2.
\item The set ${\cal W} = \{ w_1, ..., w_p, c_p \}$ represents the
wavelet transform of the data.
\end{enumerate}


The most general way to handle the boundaries is to consider
that $c(k + N) = c(N - k)$. But other methods can be used
such as periodicity ($c(k + N) = c(k)$), or continuity
 ($c(k + N) = c(N)$).  Choosing one of these methods has little influence 
on our general restoration strategy.  We used continuity.  
 
A series expansion of the original signal, $c_0$, 
in terms of
the wavelet coefficients is now given as follows. 
The final smoothed array $c_{p}(x)$ is added to all the differences $w_j$:
\begin{eqnarray}
c_0(k) = c_{p}(k) + \sum_{j=1}^{p} w_j(k)
\end{eqnarray}
 This equation provides a reconstruction formula for the original signal.
 
At each scale $j$, we obtain a set $\{w_j\}$.  This has 
the same number of pixels as the input signal. 

The above {\em \`a trous} algorithm has been discussed in terms of a single
index, $x$, but is easily extendable to 
two-dimensional space.  The use of the B$_3$ spline leads to a 
convolution with a mask of $5 \times 5$:
$$ 
\left(    \begin{array}{ccccc}
\frac{1}{256} & \frac{1}{64} & \frac{3}{128} & \frac{1}{64} & \frac{1}{256} \\
\frac{1}{64}  & \frac{1}{16} & \frac{3}{32}  & \frac{1}{16} & \frac{1}{64}  \\
\frac{3}{128} & \frac{3}{32} & \frac{9}{64}  & \frac{3}{32} & \frac{3}{128} \\
\frac{1}{64}  & \frac{1}{16} & \frac{3}{32}  & \frac{1}{16} & \frac{1}{64}  \\
\frac{1}{256} & \frac{1}{64} & \frac{3}{128} & \frac{1}{64} & \frac{1}{256} 
\end{array} \right)
$$
In one dimension, this mask is:
$ ( \frac{1}{16}, \frac{1}{4}, \frac{3}{8},
\frac{1}{4}, \frac{1}{16} ) $.
 
To facilitate computation, a simplification of this wavelet is to assume
separability in the 2-dimensional case.  In the case of the B$_3$ spline, this
leads to a row by row convolution with $(\frac{1}{16}, \frac{1}{4}, 
\frac{3}{8}, \frac{1}{4}, \frac{1}{16})$; followed by column by column 
convolution. 

As for the one dimensional case, an exact reconstruction is obtained
by adding all the scales and the final smoothed array:
\begin{eqnarray}
c_0(x,y) = c_{p}(x,y) + \sum_{j=1}^{p} w_j(x,y)
\label{eqn_rec}
\end{eqnarray}

% \bibliographystyle{plain}
% \bibliography{starck,restore,wave,ima,astro,compress}
 
