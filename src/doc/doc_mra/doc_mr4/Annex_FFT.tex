\section*{Annex A: Wavelet transform  using the Fourier Transform}
\addcontentsline{toc}{section}{Appendix A: Wavelet transform  using the Fourier Transform}

We start with the set of scalar products $c_0(k)=<f(x),\phi(x-k)>$. If
$\phi(x)$ has a cut-off frequency $\nu_c\le {1\over 2}$
(\cite{starck:sta94_3,starck:sta94_4,starck:book98}), 
the data are
correctly sampled. The data at resolution $j=1$ are:
\begin{eqnarray}
c_1(k)=<f(x),\frac{1}{2}\phi(\frac{x}{2}-k)>
\end{eqnarray}
and we can  compute the set $c_{1}(k)$ from $c_0(k)$ with a discrete 
filter $\hat h(\nu)$:
\begin{eqnarray}
\hat h(\nu)= \left\{
  \begin{array}{ll}
  {\hat{\phi}(2\nu)\over \hat{\phi}(\nu)} & \mbox{if } \mid \nu \mid < \nu_c \\
0 & \mbox{if } \nu_c  \leq \mid \nu \mid < {1\over 2} 
  \end{array}
  \right.
\end{eqnarray}
and
\begin{eqnarray}
\forall \nu, \forall n \mbox{    } & \hat h(\nu + n) = \hat h(\nu)
\end{eqnarray}
where $n$ is an integer.
So:
\begin{eqnarray}
\hat{c}_{j+1}(\nu)=\hat{c}_{j}(\nu)\hat{h}(2^{j}\nu)
\end{eqnarray}
The cut-off frequency is reduced by a factor $2$ at each step, allowing  a
reduction of the number of samples by this factor.

The wavelet coefficients at scale $j+1$ are:
\begin{eqnarray}
w_{j+1}(k)=<f(x),2^{-(j+1)}\psi(2^{-(j+1)}x-k)>
\end{eqnarray}
and they can be computed directly from $c_j(k)$ by:
\begin{eqnarray}
\hat{w}_{j+1}(\nu)=\hat{c}_{j}(\nu)\hat g(2^{j}\nu)
\end{eqnarray}
where $g$ is the following  discrete filter:
\begin{eqnarray}
\hat g(\nu)= \left\{
  \begin{array}{ll}
  {\hat{\psi}(2\nu)\over \hat{\phi}(\nu)} & \mbox{if } \mid \nu \mid < \nu_c \\
1 & \mbox{if } \nu_c  \leq \mid \nu \mid < {1\over 2} 
  \end{array}
  \right.
\end{eqnarray}
and
\begin{eqnarray}
\forall \nu, \forall n \mbox{    } & \hat g(\nu + n) = \hat g(\nu)
\end{eqnarray}

The frequency band is also reduced by a factor $2$ at each step.
Applying the sampling theorem, we can build a pyramid  of
\index{pyramid}
 $N+{N\over 2}+\ldots+1=2N$ elements.
For an image analysis the number of elements is ${4\over 3}N^2$. The
overdetermination is not very high.

The B-spline functions are compact in direct space. They
correspond to the autoconvolution of a square function. In
Fourier space we have:
\index{Fourier transform}
\begin{eqnarray}
\hat B_l(\nu)=({\sin\pi\nu\over\pi\nu})^{l+1}
\end{eqnarray}
$B_3(x)$ is a set of $4$ polynomials of degree $3$.
We choose the scaling function $\phi(\nu)$ which has a
$B_3(x)$ profile in Fourier space:
\begin{eqnarray}
\hat{\phi}(\nu)={3\over 2}B_3(4\nu)
\end{eqnarray}
In direct space we get:
\begin{eqnarray}
\phi(x)={3\over 8}[{\sin{\pi x\over 4}\over {\pi x\over
4}}]^4
\end{eqnarray}
This function is quite similar to a Gaussian and converges
rapidly to $0$. For 2-dimensions the scaling function is defined by
$\hat \phi(u,v) = {3\over 2}B_3(4r)$, with $r = \sqrt(u^2+v^2)$.
This is an isotropic function.

The wavelet transform algorithm with $n_p$ scales is the following:
\index{wavelet transform}
\begin{enumerate}
\item Start with a $B_3$-spline scaling function and derive $\psi$, $h$ and
$g$ numerically.
\item Compute the corresponding FFT image. 
Name the resulting complex array $T_0$.
\item Set $j$ to $0$. Iterate:
\item Multiply  $T_j$ by $\hat g(2^ju,2^jv)$. We get the complex array
$W_{j+1}$. The inverse FFT
gives the wavelet coefficients at scale $2^j$;
\item Multiply  $T_j$ by $\hat h(2^ju,2^jv)$. We get the array
$T_{j+1}$. Its inverse FFT gives the image at scale $2^{j+1}$.
The frequency band is reduced by a factor $2$.
\item Increment $j$.
\item If $j \leq  n_p$, go back to 4.
\item The set $\{w_1, w_2, \dots, w_{n_p}, c_{n_p}\}$ describes the
wavelet transform.
\end{enumerate}
If the wavelet is the difference between two resolutions, i.e.
\begin{eqnarray}
\hat \psi(2\nu) = \hat \phi(\nu) - \hat \phi(2\nu)
\end{eqnarray}
and:
\begin{eqnarray}
\hat g(\nu) = 1 - \hat h(\nu)
\end{eqnarray}
then the wavelet coefficients $\hat w_j(\nu)$ can be computed by 
$\hat c_{j-1}(\nu) - \hat c_j(\nu)$.

\subsubsection*{Reconstruction.}
If the wavelet is the difference between two resolutions,
an evident reconstruction for a wavelet transform 
${\cal W} = \{w_1,\dots, w_{n_p}, c_{n_p}\}$ is:
\begin{eqnarray}
\hat c_0(\nu) = \hat c_{n_p}(\nu) + \sum_j \hat w_j(\nu)
\end{eqnarray}
But this is a particular case, and other alternative wavelet functions can be
chosen. The reconstruction can be made step-by-step, starting from
the lowest resolution. At each scale, we have the relations:
\begin{eqnarray}
\hat c_{j+1} = \hat h(2^j \nu) \hat c_j(\nu) \\
\hat w_{j+1} = \hat g(2^j \nu) \hat c_j(\nu) 
\end{eqnarray}
We look for $c_j$ knowing $c_{j+1}$, $w_{j+1}$, $h$ and $g$.
We restore $\hat c_j(\nu)$ based on a least mean square estimator:
\begin{eqnarray}
\hat p_h(2^j\nu) \mid \hat c_{j+1}(\nu)-\hat h(2^j\nu)\hat c_j(\nu) \mid^2 + 
\nonumber \\
\hat p_g(2^j\nu) \mid \hat w_{j+1}(\nu)-\hat g(2^j\nu)\hat c_j(\nu) \mid^2
\end{eqnarray}
is to be minimum. $\hat p_h(\nu)$ and $\hat p_g(\nu)$ are weight
functions which permit a general solution to the
restoration of $\hat c_j(\nu)$. From the derivation of $\hat c_j(\nu)$  we get:
\begin{eqnarray}
\hat{c}_{j}(\nu)=\hat{c}_{j+1}(\nu) \hat{\tilde h}(2^{j}\nu)
                +\hat{w}_{j+1}(\nu) \hat{\tilde g}(2^{j}\nu)
\label{restauration}
\end{eqnarray} 
where the conjugate filters have the expression:
\begin{eqnarray}
\hat{\tilde h}(\nu) & = {\hat{p}_h(\nu) \hat{h}^*(\nu)\over \hat{p}_h(\nu)
\mid \hat{h}(\nu)\mid^2 + \hat{p}_g(\nu)\mid \hat{g}(\nu)\mid^2} \label{eqnht} \\ 
\hat{\tilde g}(\nu) & = {\hat{p}_g(\nu) \hat{g}^*(\nu)\over \hat p_h(\nu)
\mid \hat{h}(\nu)\mid^2 + \hat{p}_g(\nu)\mid \hat{g}(\nu)\mid^2}
\label{eqngt}
\end{eqnarray}

In this analysis, the
Shannon sampling condition is always respected and no aliasing
exists.

The denominator is reduced if we choose:
\[\hat{g}(\nu) = \sqrt{1 - \mid\hat{h}(\nu)\mid^2}\]
This corresponds to the case where the wavelet is the difference between
the square of two resolutions:
\begin{eqnarray}
\mid \hat \psi(2\nu)\mid^2  \ = \ \mid \hat \phi(\nu)\mid^2  - \mid  \hat
\phi(2\nu)\mid^2 
\end{eqnarray}

% \begin{figure}[htb]
% \centerline{
% \hbox{
% \psfig{figure=ch1_diff_uv_phi_psi.ps,bbllx=0.5cm,bblly=13.5cm,bburx=20.5cm,bbury=27cm,height=6cm,width=14.cm,clip=}
% }}
% \caption{On the left, the interpolation function $\hat{\phi}$ and, on the 
% right, the wavelet  $\hat{\psi}$.}
% \label{fig_diff_uv_phi_psi}
% \end{figure}

% \begin{figure*}[htb]
% \centerline{
% \hbox{
% \psfig{figure=ch1_diff_uv_ht_gt.ps,bbllx=0.5cm,bblly=13.5cm,bburx=20.5cm,bbury=27cm,height=5cm,width=14.5cm,clip=}
% }}
% \caption{On the left, the filter $\hat{\tilde{h}}$, and on the right the 
% filter $\hat{\tilde{g}}$.}
% \label{fig_diff_uv_ht_gt}
% \end{figure*}

% In Fig.\ \ref{fig_diff_uv_phi_psi} the chosen scaling function 
% derived from a B-spline of degree 
% 3, and its resulting wavelet function, are plotted in frequency space.
 
The reconstruction algorithm is:
\begin{enumerate}
\item Compute  the FFT of the image at the low resolution.
\item Set $j$ to $n_p$. Iterate:
\item Compute the FFT of the wavelet coefficients at scale $j$.
\item Multiply  the wavelet coefficients $\hat{w}_j$ by $\hat{\tilde{g}}$.
\item Multiply   the image at the lower resolution $\hat{c}_j$ by 
$\hat{\tilde{h}}$.
\item The inverse Fourier transform of the addition of  
$\hat{w}_j\hat{\tilde{g}}$ and $\hat{c}_j\hat{\tilde{h}}$ gives the 
image $c_{j-1}$.
\item Set $j = j - 1$ and return to 3.
\end{enumerate}
\index{Fourier transform}

The use of a scaling function with a cut-off frequency
allows a reduction of sampling at each scale, and limits the  
computing time and the memory size. 

