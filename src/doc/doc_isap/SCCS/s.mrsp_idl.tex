h28486
s 00009/00009/00506
d D 1.2 10/05/12 19:24:01 starck 2 1
c update olivier
e
s 00515/00000/00000
d D 1.1 10/02/18 09:30:30 starck 1 0
c date and time created 10/02/18 09:30:30 by starck
e
u
U
f e 0
t
T
I 1

\chapter{IDL Routines}
D 2
\label{ch_mrs_idl}
E 2
I 2
\label{ch_mrsp_idl}
E 2
%\chapterhead{IDL Routines}
\markright{IDL Routines}


\section{Introduction}
A set of routines has been developed in IDL. Starting IDL using the script program {\em mrs.pro} allows the user 
to get the multiresolution environment on the sphere, and all routines described in the following can be called. 
An online help facility is also available by invoking the {\em mrsh} program under IDL.
  

\section{Functions for polarized spherical maps}
\index{polar}

\subsection{Reading a polarized spherical map from a file : mrsp\_read}
\index{IDL routines!mrsp\_read}
\index{polar!read polarized map}
Read a polarized spherical map in Healpix format.
{\bf
\begin{center}
     USAGE: map = mrs\_read( file, noverb=noverb )
\end{center}}
where
\begin{itemize}
\item {\em file} : Input string, name of the file to be read. The pathname can be included in the string, by default 'file.fits' is equivalemt to './file.fits'
\item {\em noverb} : scalar, prevent the printing on the screen of the format (RING or NESTED) of the read map and the number of pixels.
\item {\em map} : Output 3D IDL array of Healpix map read. The map is setted to the NESTED format after reading.
\end{itemize}

\subsubsection*{Examples:} 
\begin{itemize}
\item map = mrsp\_read( 'my\_file\_healpix\_pola.fits', noverb=noverb ) \\
Read the map stored into the file 'my\_file\_healpix\_pola.fits' and load it into map.
\end{itemize}



\subsection{Writing a polarized spherical map into a file : mrsp\_write}
\index{IDL routines!mrsp\_write}
\index{polar!write polarized map}
D 2
Write a polarized spherical mapin Healpix format.
E 2
I 2
Write a polarized spherical map in Healpix format.
E 2
{\bf
\begin{center}
     USAGE: mrsp\_write, file, map, ring=ring
\end{center}}
where
\begin{itemize}
\item {\em file} : Input string, name of the file to be writen. The pathname can be included in the string, by default 'file.fits' is equivalemt to './file.fits'
\item {\em map} : Input 3D IDL array of Healpix map to be writen. The map is assumed to be in the NESTED format.
\item {\em ring} : scalar, if set convert the Healpix map data to the RING format for the writing.
\end{itemize}



\subsection{Conversion of a polarized spherical map from TQU scheme to TEB scheme : mrsp\_tqu2teb}
\index{IDL routines!mrsp\_tqu2teb}
\index{polar!conversion tqu2teb}
Convert a polarized map in Healpix nested format from TQU scheme to TEB scheme.
{\bf
\begin{center}
     USAGE: mrsp\_tqu2teb, map\_tqu, map\_teb
\end{center}}
where
\begin{itemize}
\item {\em map\_tqu} : Input 3D IDL array of healpix polarized map in TQU scheme.
\item {\em map\_teb} : Output 3D IDL array of healpix polarized map in TEB scheme.
\end{itemize}



\subsection{Conversion of a polarized spherical map from TEB scheme to TQU scheme : mrsp\_teb2tqu}
\index{IDL routines!mrsp\_teb2tqu}
\index{polar!conversion tqu2teb}
Convert a polarized map in Healpix nested format from TEB scheme to TQU scheme.
{\bf
\begin{center}
     USAGE: mrsp\_teb2tqu, map\_teb, map\_tqu
\end{center}}
where
\begin{itemize}
\item {\em map\_teb} : Input 3D IDL array of healpix polarized map in TEB scheme.
\item {\em map\_tqu} : Output 3D IDL array of healpix polarized map in TQU scheme.
\end{itemize}



\subsection{Resizing a polarized spherical map: mrsp\_resize}
\index{IDL routines!mrsp\_resize}
\index{polar!polarized map resizing}
Resize a polarized map in Healpix nested format.
{\bf
\begin{center}
     USAGE: resize\_map = mrsp\_resize( map, nside=nside, ViaAlm=ViaAlm, teb=teb )
\end{center}}
where
\begin{itemize}
\item {\em map} : Input 3D IDL array of healpix polarized map in TQU scheme to be transformed.
\item {\em resize\_map} : Output 3D IDL array of healpix polarized map in TQU scheme. Healpix input map and output resized map are in nested format.
\item {\em nside} : int, the new nside parameter of the output healpix resized map.
\item {\em ViaAlm} : scalar, if set use alm transform for the resizing, otherwise, use interpolation. 
Ignored if nside keyword value is lower than imag nside.
\item {\em teb} : scalar, if set specifies that the input and output images are in TEB scheme.
\end{itemize}

\subsubsection*{Examples:} 
\begin{itemize}
\item map2 = mrsp\_resize( map, nside = 256, /ViaAlm ) \\
resize an Healpix map.
\end{itemize}



\subsection{ALM transform of a polarized spherical map : mrsp\_almtrans}
\index{IDL routines!mrsp\_almtrans}
\index{polar!polar ALM transform}
Computes the spherical harmonic transform of a polarized TQU map using the Healpix representation (nested data).
{\bf
\begin{center}
     USAGE: mrsp\_almtrans, Imag, Trans, lmax=lmax, tab=tab, complex=complex, norm=norm, fast=fast
\end{center}}
where
\begin{itemize}
\item {\em Imag} : Input 3D IDL array of healpix polarized map in TQU scheme to be transformed.
\item {\em Trans} : Output IDL structure with the following fields:
\begin{itemize}
\item {\em ALM} : array of the ALM coefficients
\begin{center}
ALM = fltarray[*,2,3] list of the real part (ALM[*,0,*]) and imaginary part (ALM[*,1,*]) of the ALM. 
This is the default storage, ALM[*,*,0] is ALM T, ALM[*,*,1] is ALM E and ALM[*,*,2] is ALM B\\
ALM = cfarr[*,3] list of the ALM in complex values format if the keyword complex is set. 
ALM[*,0] is ALM T, ALM[*,1] is ALM E and ALM[*,2] is ALM B\\
ALM = fltarray[NbrMaxM, NbrMaxL, 2, 3] table of the real part (ALM[*,*,0,*]) and imaginary part (ALM[*,*,1,*]) 
of the ALM if the keyword tab is set. ALM[*,*,*,0] is ALM T, ALM[*,*,*,1] is ALM E and ALM[*,*,*,2] is ALM B\\
ALM = cfarr[NbrMaxM, NbrMaxL, 3] table of the ALM in complex values format if the keywords complex and tab 
are both setted. ALM[*,*,0] is ALM T, ALM[*,*,1] is ALM E and ALM[*,*,2] is ALM B\\
By default, NbrMaxM = NbrMaxL = lmax+1
\end{center}
\item {\em complex\_alm} : int, 0 (default value) if ALM array contains real and imaginary part separated. 
1 if ALM is a complex array.
\item {\em PixelType} : int, 0 for a Healpix input map (1 for GLESP but not used).
\item {\em tab} : int, 0 for default ALM representation as a list (i.e. 1D IDL array) and 1 for 2D 
representation as a table (i.e. l for the first dimension and m for the second).
\item {\em nside} : int, Healpix nside parameter.
\item {\em lmax} : int, maximum l value in the Spherical Harmonic Space.
\item {\em npix} : long, number of pixels of the input image.
\item {\em TabNbrM} : int array[NbrMaxL], max number of m value for a given l, only used if keyword tab is set otherwise, 0.
\item {\em index} : long array, indicies of the ALM coefficients, used only if keyword tab is not set.
\item {\em NormVal} : float, normalization value applied to the alm coefficients (only if keyword norm used).
\item {\em norm} : int, 0 if no normalization has been aplied, else 1.
\end{itemize}
\item {\em lmax} : int, Number of spherical harmonics computed in the decomposition. For a Healpix map, default is 
3*nside and should be between 2*nside and 4*nside.
\item {\em tab} : scalar, if set, ALM coefficients in Trans.alm are stored in a 2D array: Trans.alm[m,l] where m = 0..Trans.TabNbrM[l]-1  and l = 0..lmax-1
\item {\em complex} : scalar, if set Trans.alm will contain complex values instead of the real and imaginary parts.
\item {\em norm} : scalar, if set, a normalization is performed to the alm coefficients.
\end{itemize}

\subsubsection*{Examples:} 
\begin{itemize}
\item mrsp\_almtrans, Imag, Output \\
Compute the spherical harmonics transform of a polarized image, the result is stored in Output.
\end{itemize}



\subsection{ALM inverse transform of a polarized spherical map : mrsp\_almrec}
\index{IDL routines!mrsp\_almrec}
\index{polar!polar ALM inverse transform}
Computes the inverse spherical harmonic transform of a polarized TQU map using using the Healpix representation (nested data).
{\bf
\begin{center}
     USAGE: mrsp\_almrec, Trans, imag, pixel\_window=pixel\_window
\end{center}}
where
\begin{itemize}
\item {\em Trans} : Input IDL structure of ALM coefficients, see mrsp\_almtrans above for details.
\item {\em Imag} : Output 3D IDL array of healpix polarized map in TQU scheme. Image reconstructed in Healpix nested representation.
\item {\em pixel\_window} : scalar, if set the image is convolved by the healpix pixel window (only for Healpix map).
\end{itemize}

\subsubsection*{Examples:} 
\begin{itemize}
\item mrsp\_almtrans, PolaImag, Output \\
Compute the spherical harmonics transform of a polarized image, the result is stored in Output.
\item mrsp\_almrec, Output, PolaRec \\
Reconstruct the image.
\end{itemize}



D 2
\subsection{Power spectrum and cross spectrum exctraction from polarized ALM : mrsp\_alm2spec}
E 2
I 2
\subsection{Power spectrum and cross spectrum exctraction from polarized ALM~: mrsp\_alm2spec}
E 2
\index{IDL routines!mrsp\_alm2spec}
\index{polar!spectrum exctraction from polarized ALM}
Computes the power spectrums and cross spectrums of a polarized map from the polarized ALM coefficients.
{\bf
\begin{center}
     USAGE: spec = mrsp\_alm2spec( ALM, StdPS=StdPS )
\end{center}}
where
\begin{itemize}
\item {\em ALM} : Input IDL structure of ALM polarized coefficients, see mrsp\_almtrans above for details.
\item {\em spec} : Output 2D IDL float array[ALM.lmax+1,6], the TT, EE, BB, TE, TB, EB spectrums. P[k,i] = Mean( SPECTRUM[*,l,i] ) \quad i=0...5.
\item {\em StdPS} : Output 2D IDL float array[ALM.lmax+1,6]: estimated standard deviation of the spectrums coefficients.
\end{itemize}

\subsubsection*{Examples:} 
\begin{itemize}
\item mrsp\_almtrans, Imag, Output \\
Compute the spherical harmonics transform of a polarized image, the result is stored in Output.
\item spec = mrsp\_alm2spec( Output, StdPS=StdPS ) \\
Compute the spectrums of the image and it's associated standard deviation.
\end{itemize}



\subsection{Power spectrum and cross spectrum exctraction from a polarized image : mrsp\_spec}
\index{IDL routines!mrsp\_spec}
\index{polar!spectrum exctraction from a polarized image}
Computes the power spectrums and cross spectrums of a polarized map, using the HEALPix representation (nested data representation by default). 
By default a normalisation is applied on the ALM coefficents.
{\bf
\begin{center}
     USAGE: spec = mrsp\_spec( Imag, nonorm=nonorm, teb=teb, NormVal=NormVal, StdPS=StdPS, lmax=lmax )
\end{center}}
where
\begin{itemize}
\item {\em Imag} : Input 3D IDL array of healpix polarized map in TQU scheme. Input image whose power spectrum will be extracted.
\item {\em spec} : Output 2D IDL float array[ALM.lmax+1,6], the TT, EE, BB, TE, TB, EB spectrums. P[k,i] = Mean( SPECTRUM[*,l,i] ) \quad i=0...5.
\item {\em Lmax} : int, number of spherical harmonics computed in the decomposition and size of the computed spectrum (Lmax+1). Default is 3*nside and should be between 2*nside and 4*nside.
\item {\em nonorm} : scalar, if set no normalisation is applied on the ALM computed.
\item {\em StdPS} : Output 2D IDL float array[ALM.lmax+1,6]: estimated standard deviation of the spectrums coefficients.
\item {\em NormVal} : float, normalization value applied to the alm coefficients.
\item {\em teb} : scalar, if set specifies that the input map is in TEB scheme.
\end{itemize}

\subsubsection*{Examples:} 
\begin{itemize}
\item P = mrsp\_spec( Imag ) \\
Compute the spectrum of the polarized image.
\end{itemize}



\subsection{Undecimated Isotropic Wavelet Transform of a polarized spherical map : mrsp\_wttrans}
\index{IDL routines!mrsp\_wttrans}
\index{polar!undecimated wavelet transform of polarized image}
Computes the undecimated isotropic wavelet transform of polarized maps on the sphere in TQU scheme, 
using the Healpix representation (nested data representation). The wavelet function is zonal and its 
spherical harmonics coefficients $a_{l,0}$ follow a cubic box-spline profile. If the keyword DifInSH is 
set, the wavelet coefficients are derived in the Spherical Harmonic Space, otherwise (default) they 
are derived in the direct space.
{\bf
\begin{center}
     USAGE: mrsp\_wttrans, Imag, Trans, NbrScale=NbrScale, lmax=lmax, DifInSH=DifInSH, MeyerWave=MeyerWave
\end{center}}
where
\begin{itemize}
\item {\em Imag} : Input 3D IDL array of healpix polarized map in TQU scheme. Input image to be transformed.
\item {\em Trans} : Output IDL structure with the following fields:
\begin{itemize}
\item {\em NbrScale} : int, number of scales.
\item {\em nside} : int, Healpix nside parameter.
\item {\em lmax} : int, maximum l value in the Spherical Harmonic Space.
\item {\em npix} : long, number of pixels of the input image.
\item {\em MeyerWave} : int, 1 if the keyword MeyerWave used, otherwise 0
\item {\em DifInSH} : int, 1 if the keyword DifInSH used, otherwise 0
\item {\em Coef} : fltarr[npix,NbrScale,3] wavelet transform of the data. Coef[*,*,0] = wavelet transform on T, Coef[*,*,1] = wavelet transform on E, Coef[*,*,2] = wavelet transform on B
\begin{center}
Coef[*,0,*] = wavelet coefficients of the finest scale (highest frequencies).\\
Coef[*,NbrScale-1,*] = coarsest scale (lowest frequencies). 
\end{center}
\end{itemize}
\item {\em NbrScale} : int, optional input parameter specifying the number of scales (default is 4).
\item {\em Lmax} : int, optional input parameter specifying the maximum multipole number $l$ in the spherical harmonics decomposition 
(default is $3\times \textrm{nside}$, should be between $2\times \textrm{nside}$ and $4\times \textrm{nside}$).
\item {\em DifInSH} : Input keyword parameter. If set, the wavelet coefficients are computed as the difference between two resolutions in the spherical harmonics representation. 
Otherwise, the wavelet coefficients are computed as the difference between two resolutions in direct space.
\item {\em MeyerWave} : If set, use Meyer wavelets and set the keyword DifInSH.
\end{itemize}

\subsubsection*{Examples:} 
\begin{itemize}
\item mrsp\_wttrans, Imag, Output, NbrScale=5 \\
Compute the isotropic wavelet transform of the map Imag with five scales. The result is stored in Output.
\end{itemize}



\subsection{Undecimated Isotropic Wavelet Reconstruction of a polarized spherical map : mrsp\_wtrec}
\index{IDL routines!mrsp\_wtrec}
\index{polar!undecimated wavelet reconstruction of polarized image}
Reconstructs a polarized maps on the sphere in TQU scheme using the Healpix representation (nested data representation) 
from its wavelet coefficients obtained with the undecimated isotropic wavelet transform on the sphere, described right above.
{\bf
\begin{center}
     USAGE: mrsp\_wtrec, Trans, Rec, filter=filter
\end{center}}
where
\begin{itemize}
\item {\em Trans}: Input IDL structures with the following fields:  
\begin{itemize}
\item {\em NbrScale} : int, number of scales.
\item {\em nside} : int, Healpix nside parameter.
\item {\em lmax} : int, maximum l value in the Spherical Harmonic Space.
\item {\em npix} : long, number of pixels of the input image.
\item {\em MeyerWave} : int, 1 if the keyword MeyerWave used, otherwise 0
\item {\em DifInSH} : int, 1 if the keyword DifInSH used, otherwise 0
\item {\em Coef} : fltarr[npix,NbrScale,3] wavelet transform of the data. Coef[*,*,0] = wavelet transform on T, Coef[*,*,1] = wavelet transform on E, Coef[*,*,2] = wavelet transform on B
\begin{center}
Coef[*,0,*] = wavelet coefficients of the finest scale (highest frequencies).\\
Coef[*,NbrScale-1,*] = coarsest scale (lowest frequencies). 
\end{center}
\end{itemize}
\item {\em Rec} : Output 3D IDL array of healpix polarized map in TQU scheme. Reconstructed image from the wavelet coefficients. 
\item {\em filter} : Input keyword parameter. Use filters for the reconstructions. If this keyword is not set, the reconstructed image is obtained 
by a simple addition of all wavelet scales. Automaticaly applied if keyword MeyerWave or DifInSH were set at the wavelet decomposition.
\end{itemize}

\subsubsection*{Examples:} 
\begin{itemize}
\item mrsp\_wttrans, Imag, Output, NbrScale=5 \\
Compute the isotropic wavelet transform of the map Imag with five scales. The result is stored in Output.
\item mrsp\_wtrec, Output, map \\
Reconstruct the map. 
\end{itemize}



\subsection{Transformations of a polarized spherical map : mrsp\_trans}
\index{IDL routines!mrsp\_trans}
\index{polar!transforms of polarized image}
Compute a transform (wavelet, curvelet\ldots) on a polarized map on the sphere in the Healpix representation (nested data representation) in TQU scheme.\\ \\
The transoform can be:
\begin{enumerate}
\item A E/B decomposition using the spin\_2 transform.
\item An orthogonal wavelet on each T,Q,U component.
\item A pyramidal isotropic wavelet on each T,Q,U component.
\item An undecimated wavelet transform on each component.
\item A decimated modulus-Phase wavelet transform.
\item A undecimated modulus-Phase wavelet transform.
\item A curvelet transform.
\end{enumerate}
{\bf
\begin{center}
D 2
     USAGE: mrsp\_trans, Imag, Trans, ebdec=ebdec, NbrScale=NbrScale, Cur=Cur, uwt=uwt, PyrWT=PyrWT, owt=owt, mpdwt=mpdwt, mpuwt=mpuwt, 
     lmax=lmax, DifInSH=DifInSH, MeyerWave=MeyerWave, Overlap=Overlap, FirstBlockSize=FirstBlockSize
E 2
I 2
     USAGE: mrsp\_trans, Imag, Trans, NbrScale=NbrScale, lmax=lmax, MeyerWave=MeyerWave, ebdec=ebdec, Cur=Cur, uwt=uwt, owt=owt, mpdwt=mpdwt, mpuwt=mpuwt, PyrWT=PyrWT, 
     DifInSH=DifInSH, Overlap=Overlap, FirstBlockSize=FirstBlockSize
E 2
\end{center}}
where
\begin{itemize}
\item {\em Imag} : Input 3D IDL array of healpix polarized map in TQU scheme. Input image to be transformed.
\item {\em Trans} : Output IDL structure with the following fields:
\begin{itemize}
\item {\em NbrScale} : int, number of scales.
\item {\em nside} : int, Healpix nside parameter.
\item {\em lmax} : int, maximum l value in the Spherical Harmonic Space.
\item {\em npix} : long, number of pixels of the input image.
\item {\em MeyerWave} : int, 1 if the keyword MeyerWave used, otherwise 0.
\item {\em DifInSH} : int, 1 if the keyword DifInSH used, otherwise 0.
\item {\em pyrtrans} : int, 1 if a pyramidal decomposition has been applied, otherwise 0.
\item {\em ebdec} : int, 1 if an EB decomposiiton has been applied, otherwise 0.
\item {\em DEC1} : IDL structure, first component transformation (depends on the chosen transform).
\item {\em DEC2} : IDL structure, second component transformation (depends on the chosen transform).
\item {\em DEC3} : IDL structure, third component transformation (depends on the chosen transform).
\item {\em TransChoice} : string, code of the chosen transform.
\item {\em TabCodeTransform} : string array, array of transforms codes. TabCodeTransform = ['T\_EBDEC', 'T\_OWT', 'T\_PyrWT', 'T\_UWT', 'T\_MPDWT', 'T\_MPUWT', 'T\_CUR']
\item {\em TransName} : string, transform's name.
\item {\em TransTypeName} : string array, array of transforms names. 
TransTypeName = ['EBDEC','Bi-Orthogonal WT', 'Pyramidal WT', 'Undecimated WT', 'Module-Phase Decimated Transform', 'Module-Phase Undecimated Transform', 'Curvelet']
\end{itemize}
\item {\em NbrScale} : int, number of scales of the wavelet transforms.
\item {\em ebdec} : scalar, if set an E/B decomposition is applied before the chosen multiscale decomposition. 
If no transform is selected, it will be the default transformation.
\item {\em Cur} : scalar, if set perform a curvelet transform.
\item {\em uwt} : scalar, if set perform an undecimated isotropic wavelet transform.
\item {\em PyrWT} : scalar, if set perform a pyramidal isotropic wavelet transform.
\item {\em owt} : scalar, if set perform a bi-orthogonal wavelet transform on each face.
\item {\em mpdwt} : scalar, if set perform a decimated module-phase wavelet transform.
\item {\em mpuwt} : scalar, if set perform a undecimated module-phase wavelet transform.
\item {\em Overlap} : int, if equal to 1 if blocks are overlapping, only used with curvelet transform.
\item {\em FirstBlockSize} : int, block size in the ridgelet transform at the finest scale (default is 16), only used with curvelet transform.
\item {\em lmax} : int, maximum l value in the Spherical Harmonic Space (for isoptropic wavelet transform only).
\item {\em DifInSH} : Input keyword parameter. If set, the wavelet coefficients are computed as the difference between two resolutions in the spherical harmonics representation. 
Otherwise, the wavelet coefficients are computed as the difference between two resolutions in direct space. Only used with keyword uwt or PyrWT.
\item {\em MeyerWave} : If set, use Meyer wavelets and set the keyword DifInSH. Only used with keyword uwt or PyrWT.
\end{itemize}

\subsubsection*{Examples:} 
\begin{itemize}
\item mrsp\_trans, Imag, WT, NbrScale=5, /uwt \\
Compute the undecimated wavelet transform of the map Imag with five scales. The result is stored in WT.
\end{itemize}



\subsection{Reconstructions of a polarized spherical map : mrsp\_rec}
\index{IDL routines!mrsp\_rec}
\index{polar!inverse transform of polarized image}
Compute a inverse transform (wavelet, curvelet\ldots) to get a polarized map on the sphere in the Healpix representation 
(nested data representation) in TQU scheme from its decomposition obtained by mrsp\_trans.\\ \\
The transoform can be:
\begin{enumerate}
\item A E/B decomposition using the spin\_2 transform.
\item An orthogonal wavelet on each T,Q,U component.
\item A pyramidal isotropic wavelet on each T,Q,U component.
\item An undecimated wavelet transform on each component.
\item A decimated modulus-Phase wavelet transform.
\item A undecimated modulus-Phase wavelet transform.
\item A curvelet transform.
\end{enumerate}
{\bf
\begin{center}
D 2
     USAGE: mrsp\_trans, Trans, Rec
E 2
I 2
     USAGE: mrsp\_rec, Trans, Rec
E 2
\end{center}}
where
\begin{itemize}
\item {\em Trans} : Input IDL structure, see mrsp\_trans for more details.
\item {\em Rec} : Output 3D IDL array of healpix polarized map in TQU scheme. Reconstructed image.
\end{itemize}

\subsubsection*{Examples:} 
\begin{itemize}
\item mrsp\_trans, Imag, WT, NbrScale=5, /uwt \\
Compute the undecimated wavelet transform of the map Imag with five scales. The result is stored in WT.
\item mrsp\_rec, WT, RecIma \\
Reconstruct the image.
\end{itemize}



\subsection{Wavelet filtering of a polarized spherical map : mrsp\_wtfilter}
\index{IDL routines!mrsp\_wtfilter}
\index{polar!polar wavelet filtering}
Wavelet denoising of a polarized image on the sphere using Healpix representation in TQU scheme (nested pixel representation). 
By default Gaussian noise is considered. If the keyword SigmaNoise is not set, then the noise standard deviation is automatically 
estimated. If the keyword MAD is set, then a correlated Gaussian noise is considered and the noise level at each scale is derived 
from the Median Absolution Deviation (MAD) method. If the keyword KillLastScale is set, the coarsest resolution is set to zero. 
The thresholded wavelet coefficients can be obtained using the keyword Trans. If the input keyword niter is set, then an iterative 
algorithm is applied and if the pos keyword is also set, then a positivity constraint is added.
{\bf
\begin{center}
D 2
     USAGE:  mrsp\_wtfilter, Imag, Filter, NbrScale=NbrScale, NSigma=NSigma, SigmaNoise=SigmaNoise, mad=mad, 
     KillLastScale=KillLastScale, Trans=Trans, niter=niter, pos=pos, FirstScale=FirstScale, soft=soft, fdr=fdr, 
     Use\_FdrAll=Use\_FdrAll, lmax=lmax, FilterLast=FilterLast, mask=mask 
E 2
I 2
     USAGE:  mrsp\_wtfilter, Imag, Filter, NbrScale=NbrScale, NSigma=NSigma, SigmaNoise=SigmaNoise, KillLastScale=KillLastScale, 
     pos=pos, mad=mad, Trans=Trans, niter=niter, FirstScale=FirstScale, Use\_FdrAll=Use\_FdrAll, soft=soft, fdr=fdr, lmax=lmax, 
     FilterLast=FilterLast, mask=mask 
E 2
\end{center}}
where
\begin{itemize}
\item {\em Imag} : Input 3D IDL array of healpix polarized map in TQU scheme. Input image to befiltered.
\item {\em Filter} : Output 3D IDL array of healpix polarized map in TQU scheme containing the filtered map.
\item {\em NbrScale} : int, number of scales (default is 4).
\item {\em NSigma} : float, level of thresholding (default is 3).
\item {\em SigmaNoise} : float, noise standard deviation. Default is automatically estimated.
\item {\em mad} : scalar, if set the noise level is derived at each scale using the MAD of the wavelet coefficient.
\item {\em KillLastScale} : scalar, if set the last scale is set to zero.
\item {\em niter} : int, number of iterations used in the reconstruction.
\item {\em pos} : scalar, if set the solution is assumed to be positive.
\item {\em FirstScale} : int, consider only scales larger than FirstScale. Default is 1 (i.e. all scales are used).
\item {\em Soft} : scalar, if set use soft thresholding instead of hard thresholding.
\item {\em fdr} : float between 0 (default) and 1 (max, if greater or equal to 1, set to 0.05), used to estimate a threshold level 
instead of a NSigma threshold, threshold is applied from scale j=FirstScale to the last.
\item {\em Use\_FdrAll} : same as fdr but applied to all scales.
\item {\em FilterLast} : scalar, if set the last scale is filtered.
\item {\em mask} : IDL array of healpix map, input mask applied.
\item {\em lmax} : int, maximum l value in the Spherical Harmonic Space.
\item {\em Trans} : IDL structure: Thresholded wavelet decomposition of the input image.
\end{itemize}

\subsubsection*{Examples:} 
\begin{itemize}
\item mrsp\_wtfilter, Imag, Filter, NbrScale=5, Nsigma=5 \\
Wavelet filtering with five scales and a 5 sigma threshold.
\end{itemize}



\subsection{Extract a scale from a polar decomposition : mrsp\_wtget}
\index{IDL routines!mrsp\_wtget}
\index{polar!polar wavelet extraction}
Return a band of a transform for Healpix polarized map (wavelet, curvelet\ldots) obtained by the command mrsp\_trans.
{\bf
\begin{center}
     USAGE:  Scale = mrsp\_wtget( Trans, Component, ScaleNumber, BandNumber=BandNumber, NormVal=NormVal )
\end{center}}
where
\begin{itemize}
\item {\em Trans} : Input IDL structure, see mrsp\_trans for more details.
\item {\em ScaleNumber} : int, scale number of the band to be extracted. The scale number must be between 0 and Trans.NbrScale-1.
\item {\em Component} : int, choice of the component, 0 is for T, 1 for E and 2 for B.
\item {\em NormVal} : float, optional normalization value of the band (for isotropic wavelet transform).
\item {\em BandNumber} : int, ridgelet band number (for curvelet transform).
\item {\em Scale} : return value IDL array of the band extracted. See more details on the 1D versions of the functions. 
No band is extracted from Modulus-Phases transforms (return 0). For a E/B decomposition, it will be either the T map, the E map or the B map.
\end{itemize}



\subsection{Put a scale into a polar decomposition : mrsp\_wtput}
\index{IDL routines!mrsp\_wtput}
\index{polar!polar wavelet insertion}
Put a band into a transform for Healpix polarized map (wavelet, curvelet\ldots) obtained by the command mrsp\_trans.
{\bf
\begin{center}
     USAGE:   mrsp\_wtput, Trans, Scale, Component, ScaleNumber, BandNumber=BandNumber
\end{center}}
where
\begin{itemize}
\item {\em Trans} : Input IDL structure, see mrsp\_trans for more details.
\item {\em Scale} : Input IDL array of the band inserted. See more details on the 1D versions of the functions. 
No band is inserted into Modulus-Phases transforms. For a E/B decomposition, it will be either the T map, the E map or the B map.
\item {\em ScaleNumber} : int, scale number of the band to be extracted. The scale number must be between 0 and Trans.NbrScale-1.
\item {\em Component} : int, choice of the component, 0 is for T, 1 for E and 2 for B.
\item {\em BandNumber} : int, ridgelet band number (for curvelet transform).
\end{itemize}

E 1
