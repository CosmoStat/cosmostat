\chapter{Introduction}
\label{ch_msvst_intro}

Several techniques have been proposed in the literature to estimate Poisson intensity in 2D. A major class of methods adopt a multiscale bayesian framework specifically tailored for Poisson data~\citep{Nowak2000}, independently initiated by \citet{wave:timmermann99} and \citet{Kolaczyk1999}. \citet{Lefkimmiatis} proposed an improved bayesian framework for analyzing Poisson processes, based on a multiscale representation of the Poisson process in which the ratios of the underlying Poisson intensities in adjacent scales are modeled as mixtures of conjugate parametric distributions. Another approach includes preprocessing the count data by a variance stabilizing transform (VST) such as the Anscombe \citep{rest:anscombe48} and the Fisz \citep{rest:nason04} transforms, applied respectively 
in the spatial~\citep{rest:donoho93_2} or in the wavelet domain~\citep{Fryzlewicz2004}. The transform reforms the data so that the noise approximately becomes Gaussian with a constant variance. Standard techniques for independant identically distributed Gaussian noise are then used for denoising. \citet{starck:zhang07} proposed a powerful method called Multi-Scale Variance Stabilizing Tranform (MS-VST). It consists in combining a VST with a multiscale transform (wavelets, ridgelets or curvelets), yielding asymptotically normally distributed coefficients with known variances. The choice of the multi-scale method depends on the morphology of the data. Wavelets represent more efficiently regular structures and isotropic singularities, whereas ridgelets are designed to represent global lines in an image, and curvelets represent efficiently curvilinear contours. Significant coefficients are then detected with binary hypothesis testing, and the final estimate is reconstructed with an iterative scheme. In \citet{Starck09:fermi3d}, it was shown that sources can be detected in 3D FERMI LAT data (2D+time or 2D+energy) using a specific 3D extension of the MS-VST.
 \citet{Schmitt} proposed a method for Poisson intensity estimation on spherical data called Multi-Scale Variance Stabilizing Transform on the Sphere (MS-VSTS). This MS-VSTS (Multi-Scale Variance Stabilizing Transform on the Sphere) package offers a Poisson denoising method on the sphere, designed for Fermi photon counts maps.
This method is based on the MS-VST~\citep{starck:zhang07} and on multi-scale transforms on the sphere \citep{starck2006,inpainting:abrial06,starck:abrial08}.
 Chapter~\ref{ch_msvsts} introduces the MS-VSTS. 
 Chapter~\ref{ch_denoising} applies the MS-VSTS to spherical data restoration. Chapter~\ref{ch_inpainting} applies the MS-VSTS to inpainting. Chapter~\ref{ch_background} applies the MS-VSTS to background extraction. An accurate description of the IDL routines that makeup this package is given in Chapter~\ref{ch_idlproc}. An extension to multichannel denoising and deconvolution is given in Chapter~\ref{ch_multichannel} Conclusions are drawn in Chapter~\ref{ch_conclusion}. All experiments were performed on HEALPix maps with $nside=128$~\citep{pixel:healpix}, which corresponds to a good pixelisation choice for  data such as the GLAST/FERMI resolution. The performance of the method is not dependent on the nside parameter. For a given data set, if nside is small, it just means that we don't want to investigate the finest scales. If nside is large, the number of counts per pixel will be very small, and we may not have enough statistics to get any information at the finest resolution levels. But it will not have any bad effect on the solution. Indeed,  the finest scales
will be smoothed, since our algorithm will not detect any significant wavelet coefficients in the finest scales. Hence, starting with a fine pixelisation (i.e. large nside), our method will provide a kind of automatic binning, by thresholding wavelets coefficients at scales and at spatial positions where the number of counts is not sufficient.




% \newpage
