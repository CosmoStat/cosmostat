
\chapter{{\isap}  Introduction and Installation}
\label{ch_install}
\markright{Installation}

 
\section{Introduction to {\isap} }

{\isap} is a collection of packages, in IDL and C++, related to sparsity and its application in astronomical data analysis
(the IDL software (\texttt{http://www.idl-envi.com}) is analogous to Matlab and is very widely used in astrophysics and in medical imaging). 
 The C++ routines can be used independently of IDL.  The library is available via the web site:\\ \\
{ \centerline{\texttt{http://www.cosmostat.org/isap.html }}}\\
\\
It contains the following packages:
\begin{itemize}
 \item{\Large \bf  {Sparse2D V1.0: }} {  \bf  {Sparsity for 1D and 2D data set. }} \\
 IDL and C++ code, allowing sparse decomposition, denoising and deconvolution.

 \item{\Large \bf  {MSVST V1.0: }} {  \bf  {Multi-Scale Variance Stabilizing Transform (MSVST) for 1D and 2D data set.}} \\
 IDL and C++ code for Poisson noise removal.
 
 \item{\Large \bf  {MRS V3.2: }} {  \bf  {MultiResolution on the Sphere.}} \\
  IDL and C++ code for sparse representation on the sphere.

 \item {\Large \bf  {SparsePOL V1.1:}}   {  \bf  {Polarized Spherical Wavelets and Curvelets.}}  \\
 IDL code for sparse representation of polarized data on the sphere.
 
 \item {\Large \bf {MRS-MSVSTS   V1.1:}}  { \bf  {Multi-Scale Variance Stabilizing Transform on the Sphere.}} \\
  IDL code for Poisson noise removal and deconvolution on mono-channel and multichannel spherical data.

  \item{\Large \bf  {SparseGal V1.0: }} {  \bf  {Sparsity for galaxies survey analysis.}} \\
 IDL code, with two subpackages:
 \begin{itemize}
 \item ISW V1.0:  Integrated Sachs-Wolfe effect detection.
 \item DarthFader V1.0: Spectroscopic Redshift Estimation using sparsity.
 \end{itemize}
  \item{\Large \bf  {SparseCMB V1.0: }} {  \bf  {Sparsity for CMB data analysis.}}  \\
 IDL and C++ code.
 \end{itemize}

\section{IDL Installation}

A set of routines has been developed in IDL. Starting IDL using the script program {\em isap.pro} allows the user 
to get the sparse IDL astronomical data analysis environment, and all routines described in the following can be called. 
An online help facility is also available by invoking the {\em isaph} program under IDL.

Then, installing the {\isap} package simply requires adding some lines in your environment profile:
\begin{itemize}
\item[$\bullet$] {define the environment variable \textbf{ISAP}}  
\begin{verbatim}
setenv ISAP /home/user/ISAP
\end{verbatim}
 \item[$\bullet$]{define the alias \textbf{isap}}  
\begin{verbatim}
alias isap 'idl $ISAP/idl/isap.pro  $ISAP/idl/compile_healpixfile' 
\end{verbatim}
If the Healpix package is not installed, then replace the previous command with
\begin{verbatim}
alias isap 'idl $ISAP/idl/isap.pro' 
\end{verbatim}
In this case, packages MRS, MRSP, MRS-MSVST  will not be active.

\end{itemize}
Then the command "isap" will start the IDL session using the {\isap} environment. 


\section{\projmrs package}
 
The \mrs package, included in  {\isap}, requires IDL (version 6.0 or later) and HEALPix (version 2.0 or later) to be installed. The HEALPix environment 
variable \textbf{HEALPix} is expected to be defined. HEALPix is available at:\\ \\
{\centerline{\texttt{http://sourceforge.net/projects/healpix}}}\\

 HEALPix binaries must be in the user path.
  
 \projmrs  contains also C++ programs that can be used from IDL or directly from a terminal session. It has been tested using gcc-4.4.1. C++ is not necessary, but for some applications
 such as inpainting, C++ routines are much faster. To compile C++ routine, do the following commands:
\begin{itemize}
\item[$\bullet$] {go the \textbf{iSAP} directory }  
\begin{verbatim}
cd  $ISAP/cxx/mrs/build
\end{verbatim} 
 \item[$\bullet$]{build the makefile}  
\begin{verbatim}
./cmake .. 
\end{verbatim}
\item[$\bullet$]{run the makefile}  
\begin{verbatim}
make
\end{verbatim}
\item[$\bullet$]{copy the created binaries to the \textbf{iSAP} binary directory}
\begin{verbatim}
make install
\end{verbatim}
\item[$\bullet$]{set the IDL global variable ISAPCXX equal to 1 in the file  \textbf{\$iSAP}/isap.pro or on the user command line after being in the IDL-iSAP environment. }
\begin{verbatim}
ISAP>  ISAPCXX=1
\end{verbatim}
\end{itemize}

 Simular installation procedures must be done for each C++ package available in  \textbf{\$iSAP/cxx}.
 
\subsection{Input Data}
Most of the functions of \mrs package are working with spherical maps either as input or as output variables. The maps could be in 
two possible kind of formats: the first one which will be recognized by all functions is the HEALPix format. This format allows 
two kind of pixel order in the data array called NESTED and RING. Unless stated, \mrs functions will work only with NESTED maps. 
Conversions between NESTED and RING schemes could be done with the function "REORDER" from HEALPix package. 

The second format which  is recognized by some functions of \mrs package, but not all, is the GLESP format. In IDL, the variable for an image in GLESP format 
is an IDL structure. The \mrs package includes the file "mrs\_glesp.pro" which contains several functions for working with GLESP images 
especially the functions "healpix2glesp" and "glesp2healpix" which are used for the conversions between Healpix and GLESP format.
To use the GLESP image format, GLESP library, available at {\texttt{http://www.glesp.nbi.dk}, must be installed, and the environment variable \textbf{GLESP}
must be initialized to the correct path.

\subsection{Global IDL \mrs Variable}
Four global \mrs variables, defined in the file "isap.pro", are available, and can be changed by the user.
\begin{itemize}
\item HealpixCXX: default is 1. By default, \mrs uses HEALPix C++ programs to compute the spherical harmonic coefficients. 
If HealpixCXX is set to 0, \mrs will call the HEALPix  Fortran programs. We recommend to keep HealpixCXX equal to 1. Fortran option
will not be supported in the future.
\item DEF\_ALM\_FAST: default is 1. By default, spherical harmonic coefficients are calculated with floating values. This is faster and requires less
memory, but is not as accurate than using double. Set DEF\_ALM\_FAST to zero to make all calculation with double.
\item DEF\_ALM\_NITER: default is 10. This variable is only used when DEF\_ALM\_FAST is equal to zero. DEF\_ALM\_NITER is the number of iterations
used by HEALPix to compute the spherical harmonic coefficients. In the fast mode (default mode), there is no iteration.
\item DEF\_NORM\_POWSPEC: default is 0. If DEF\_NORM\_POWSPEC is set to 1, the command "mrs\_powspec" will return a normalized power spectrum, 
such that a map containing a Gaussian noise with variance 1 will have a power spectrum equal to 1. 
\item ISAPCXX: Default is 0:  If the  \textbf{iSAP}  C++ code has been compiled, it is recommended to set it to 1. When it is equal to 1, then spherical harmonic transform
and recontruction are done using the \textbf{MRScxx} binaries instead of Healpix binaries. 
\end{itemize}


\section{\projpol package}
The {\pol}  package, included in  {\isap}, requires IDL (version 6.0 or later), HEALPix (version 2.0 or later), and \projmrs to be installed.


 
